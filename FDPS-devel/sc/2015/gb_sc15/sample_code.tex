In this section, we present the complete working example of a
simulation code written using FDPS, to illustrate how a user actually
uses FDPS. As the target problem, we use the gravitational $N$-body
problem with an open boundary.  Within the terminology of FDPS, the
interaction between particles in the gravitational $N$-body problem is
of the ``long-range'' type. Therefore, we need to specify the function
to calculate interactions for both the ordinary particles and
superparticles. For the sake of brevity, we use the center-of-mass
approximation for superparticles, which means we can actually use the
same function for both types of particles.

The physical quantity vector $\myvec{u}_i$ and interaction functions
$\myvec{f}$, $\myvec{f'}$, and $\myvec{g}$ for the gravitational
$N$-body problem is now given by:
\begin{align}
  \myvec{u}_i &= (\myvec{r}_i,
  \myvec{v}_i,m_i) \label{eq:PhysicalVectorNbody} \\
%%  
  \myvec{f} (\myvec{u}_i, \myvec{u}_j) &= \frac{Gm_j \left(
    \myvec{r}_j - \myvec{r}_i \right)}{ \left( |\myvec{r}_j -
    \myvec{r}_i|^2 + \epsilon_i^2
    \right)^{3/2}} \label{eq:ParticleParticleNbody} \\
%%
  \myvec{f'} (\myvec{u}_i, \myvec{u'}_j) &= \frac{Gm_j' \left(
    \myvec{r}_j - \myvec{r'}_i \right)}{ \left( |\myvec{r}_j -
    \myvec{r'}_i|^2 + \epsilon_i^2
    \right)^{3/2}} \label{eq:ParticleSuperparticleNbody} \\
%%
  \myvec{g}(\myvec{F},\myvec{u}_i)  &= (\myvec{v}_i,\myvec{F},0),
\label{eq:ConversionNbody}
\end{align}
where $m_i$, $\myvec{r}_i$, $\myvec{v}_i$, and $\epsilon_i$ are, the
mass, position, velocity, and gravitational softening of particle $i$,
$m_j'$ and $\myvec{r'}_j$ are, the mass and position of a
superparticle $j$, and $G$ is the gravitational constant.  Note that
the shapes of the functions $\myvec{f}$ and $\myvec{f'}$ are the same.

Listing~\ref{code:samplecode} shows the complete code which can be
actually compiled and run, not only on a single-core machine but also
massively-parallel, distributed-memory machines such as the full-node
configuration of the K computer. The total number of lines is only
117.


\begin{lstlisting}[label=code:samplecode,numbers=left,numbersep=5pt,frame=single,basicstyle=\ttfamily,caption=A sample code of $N$-body simulation]
#include <particle_simulator.hpp>
using namespace PS;

class Nbody{
public:
    F64    mass, eps;
    F64vec pos, vel, acc;
    F64vec getPos() const {return pos;}
    F64 getCharge() const {return mass;}
    void copyFromFP(const Nbody &in){ 
        mass = in.mass;
        pos  = in.pos;
        eps  = in.eps;
    }
    void copyFromForce(const Nbody &out) {
        acc = out.acc;
    }    
    void clear() {
        acc = 0.0;
    }
    void readAscii(FILE *fp) {
        fscanf(fp,
               "%lf%lf%lf%lf%lf%lf%lf%lf",
               &mass, &eps,
               &pos.x, &pos.y, &pos.z,
               &vel.x, &vel.y, &vel.z);
    }
    void predict(F64 dt) {
        vel += (0.5 * dt) * acc;
        pos += dt * vel;
    }
    void correct(F64 dt) {
        vel += (0.5 * dt) * acc;
    }
};

template <class TPJ>
struct CalcGrav{
    void operator () (const Nbody * ip,
                      const S32 ni,
                      const TPJ * jp,
                      const S32 nj,
                      Nbody * force) {
        for(S32 i=0; i<ni; i++){
            F64vec xi  = ip[i].pos;
            F64    ep2 = ip[i].eps
                * ip[i].eps;
            F64vec ai = 0.0;
            for(S32 j=0; j<nj;j++){
                F64vec xj = jp[j].pos;
                F64vec dr = xi - xj;
                F64 mj  = jp[j].mass;
                F64 dr2 = dr * dr + ep2;
                F64 dri = 1.0 / sqrt(dr2);                
                ai -= (dri * dri * dri
                       * mj) * dr;
            }
            force[i].acc += ai;
        }
    }
};

template<class Tpsys>
void predict(Tpsys &p,
             const F64 dt) {
    S32 n = p.getNumberOfParticleLocal();
    for(S32 i = 0; i < n; i++)
        p[i].predict(dt);
}

template<class Tpsys>
void correct(Tpsys &p,
             const F64 dt) {
    S32 n = p.getNumberOfParticleLocal();
    for(S32 i = 0; i < n; i++)
        p[i].correct(dt);
}

template <class TDI, class TPS, class TTFF>
void calcGravAllAndWriteBack(TDI &dinfo,
                             TPS &ptcl,
                             TTFF &tree) {
    dinfo.decomposeDomainAll(ptcl);
    ptcl.exchangeParticle(dinfo);    
    tree.calcForceAllAndWriteBack
        (CalcGrav<Nbody>(),
         CalcGrav<SPJMonopole>(),
         ptcl, dinfo);    
}

int main(int argc, char *argv[]) {
    F32 time  = 0.0;
    const F32 tend  = 10.0;
    const F32 dtime = 1.0 / 128.0;
    PS::Initialize(argc, argv);
    PS::DomainInfo dinfo;
    dinfo.initialize();
    PS::ParticleSystem<Nbody> ptcl;
    ptcl.initialize();
    PS::TreeForForceLong<Nbody, Nbody,
        Nbody>::Monopole grav;
    grav.initialize(0);
    ptcl.readParticleAscii(argv[1]);
    calcGravAllAndWriteBack(dinfo,
                            ptcl,
                            grav);
    while(time < tend) {
        predict(ptcl, dtime);        
        calcGravAllAndWriteBack(dinfo,
                                ptcl,
                                grav);
        correct(ptcl, dtime);        
        time += dtime;
    }
    PS::Finalize();
    return 0;
}
\end{lstlisting}


Now let us explain how this sample code works. This code consists of
four parts: The declaration to use FDPS (lines 1 and 2), the
definition of the particle (the vector $\myvec{u}_i$) (lines 4 to 35),
the definition of the gravitational force (the functions $\myvec{f}$
and $\myvec{f'}$) (lines 37 to 61), and the actual user program,
comprising a user-defined main routine and user-defined functions from
which library functions of FDPS are called (lines 63 to line 117). In
the following, we explain them step by step.

In order to declare to use FDPS, the only thing the user program need
to do is to include the header file ``particle\_simulator.hpp''. This
file and other source library files of FDPS should be in the include
path of the compiler. Everything in the standard FDPS library is
provided as the header source library, since they are implemented as
template libraries which need to receive particle class and
interaction functions. Everything in FDPS is provided in the namespace
``PS''. Therefore in this sample program, we declare it as the default
namespace to simplify the code. (For simplicity's sake, we do not omit
the namespace ``PS'' of FDPS functions and class templates in the main
routine.)

Before going to the 2nd parts, let us list the data types and classes
defined in FDPS. \texttt{F32/F64} are data types of 32-bit and 64-bit
floating points. \texttt{S32} is a data type of 32-bit signed integer.
\texttt{F64vec} is a class of a vector consisting of three 64-bit
floating points. This class provides several operators, such as the
addition, subtraction and the inner product indicated by ``$*$''.
Users need not use these data types in their own program, but some of
the functions which users should define should return the values in
these data types.

In the 2nd part, we define the particle, i.e. the vector
$\myvec{u}_i$, as a class \texttt{Nbody}. This class has member
variables: \texttt{mass} ($m_i$), \texttt{eps}
($\epsilon_i$), \texttt{pos} ($\myvec{r}_i$), \texttt{vel}
($\myvec{v}_i$), and \texttt{acc} ($d\myvec{v}_i/dt$). Although the
member variable \texttt{acc} does not appear in
equation~(\ref{eq:PhysicalVectorNbody}) -- (\ref{eq:ConversionNbody}),
we need this variable to store the result of the gravitational force
calculation. A particle class for FDPS must provide public member
functions \texttt{getPos}, \texttt{getCharge}, \texttt{copyFromFP},
\texttt{copyFromForce}, \texttt{clear},
and \texttt{readAscii}, in these names, so that the internal functions
of FDPS can access the data within the particle class.  For the name
of the particle class itself and the names of the member variables, a
user can use whatever names allowed by the C++ syntax.  The member
functions \texttt{predict} and \texttt{correct} are used in the
user-defined part of the code to integrate the orbits of particles.
Note that since the interaction used here is of $1/r$ type, the
definition and construction method of the superparticle are given as
the default in FDPS and not shown here.

In the 3rd part, the interaction functions $\myvec{f}$ and
$\myvec{f'}$ are defined. Since the shapes of the functions
$\myvec{f}$ and $\myvec{f'}$ are the same, we give one as a template
function.  The interaction function used in FDPS should have the
following five arguments. The first argument \texttt{ip} is the
pointer to the array of variables of particle
class \texttt{Nbody}. This argument specifies $i$-particles which
receive the interaction. The second argument \texttt{ni} is the number
of $i$-particles. The third argument \texttt{jp} is the pointer to the
array of variable of a template data type \texttt{TPJ}. This argument
specifies $j$-particles or superparticles which exert the
interaction. The fourth argument \texttt{nj} is the number of
$j$-particles or super-particles. The fifth argument \texttt{force} is
the pointer to the array of a variable of a user-defined class to
which the calculated interaction on an $i$-particle can be stored. In
this example, we used the particle class itself, but this can be
another class or a simple array.

%
The interaction function should be defined as a function object, so
that it can be passed to other functions as argument. Thus, it is
declared as a \texttt{struct}, with the only member
function \texttt{operator ()}.  In this example, the interaction is
calculated through a simple double loop. In order to make full
advantage of the SIMD unit in modern processors,
architecture-dependent tuning may be necessary, but only to this
single function.

In the 4th part, we give the main routine and functions called from
the main routine. In the following, we describe the main routine in
detail, and briefly discuss other functions. The main routine consists
of the following seven steps:
\begin{enumerate}
\item Set simulation time and timestep (lines 92 to 94). \label{proc:literal}
\item Initialize FDPS (line 95). \label{proc:init}
\item Create and initialize objects of FDPS classes (lines 96 to 102). \label{proc:construct}
\item Read in particle data from a file (line 103). \label{proc:input}
\item Calculate the gravitational forces of all the particles at the
  initial time (lines 104 to 106). \label{proc:calcinteraction}
\item Integrate the orbits of all the particles with Leap-Frog method
  (lines 107 to 114). \label{proc:integration}
\item Finish the use of  FDPS (line 115). \label{proc:fin}
\end{enumerate}

In the following, we describe  steps~\ref{proc:init},
\ref{proc:construct}, \ref{proc:input}, \ref{proc:calcinteraction},
and \ref{proc:fin}, and skip steps~\ref{proc:literal}
and \ref{proc:integration}.  In step~\ref{proc:literal}, we do not
call FDPS libraries.  Although we call FDPS libraries in
step~\ref{proc:integration}, the usage is the same as in
step~\ref{proc:calcinteraction}.

In step~\ref{proc:init}, the FDPS function \texttt{Initialize} is
called. In this function, MPI and OpenMP libraries are initialized. If
neither of them are used, this function does nothing.  All functions
of FDPS must be called between this function and the
function \texttt{Finalize}.

In step~\ref{proc:construct}, we create and initialize three objects
of the FDPS classes:
\begin{itemize}
\item \texttt{dinfo}: An object of class \texttt{DomainInfo}. It is
  used for domain decomposition.
\item \texttt{ptcl}: An object of class template \texttt{ParticleSystem}.
It takes the user-defined particle class (in this
example, \texttt{Nbody}) as the template argument. From the user
program, this object looks as an array of $i$-particles.
\item \texttt{grav}: An object of a data type \texttt{Monopole} defined in
a class template \texttt{TreeForForceLong}. This object is used for
the calculation of long-range interaction using the tree algorithm.
It receives three user-defined classes template arguments: the class
to store the calculated interaction, the class for $i$-particles and
the class for $j$-particles. In this example, all three are the same
as the original class of particles.  It is possible to define classes
with minimal data for these purposes and use them here, in order to
optimize the cache usage. The data type \texttt{Monopole} indicates
that the center-of-mass approximation is used for superparticles.
\end{itemize}

In step~\ref{proc:input}, the data of particles are read from a file
into the object \texttt{ptcl}, using the FDPS
function \texttt{readParticleAscii}. In the function, a member
function of class \texttt{Nbody}, \texttt{readAscii}, is called.

In step~\ref{proc:calcinteraction}, the forces on all particles are
calculated through the function \texttt{calcGravAllAndWriteBack}, which
is defined in lines 79 to 89. In this function,
steps~\ref{proc:decompose}, \ref{proc:exchange}, and
\ref{proc:interaction} in section~\ref{sec:view} are performed. In
other words, all of the actual work of FDPS libraries to calculate
interaction between particles takes place here. For
step~\ref{proc:decompose}, \texttt{decomposeDomainAll}, a member function
of class \texttt{DomainInfo} is called. This function takes the object
\texttt{ptcl} as an argument to use the positions of particles to
determine the domain decomposition.  Step~\ref{proc:exchange} was
performed in \texttt{exchangeParticle}, a member function of
class \texttt{ParticleSystem}. This function takes the
object \texttt{dinfo} as an argument and redistributes particles among
MPI processes.  Step~\ref{proc:interaction} was performed
in \texttt{calcForceAllAndWriteBack}, a member function of
class \texttt{TreeForForceLong}. This function takes the user-defined
function object \texttt{CalcGrav} as the first and second arguments,
and calculates particle-particle and particle-superparticle
interactions using them.

In step~\ref{proc:fin}, the FDPS function \texttt{Finalize} is
called. It calls the \texttt{MPI\_finalize} function.

In this section, we have described in detail how a user program
written using FDPS looks like. As we stated earlier, this program can
be compiled with or without parallelization using MPI and/or OpenMP,
without any change in the user program. The executable parallelized
with MPI is generated by using an appropriate compiler with MPI
support and a compile-time flag.  Thus, a user need not worry about
complicated bookkeeping necessary for parallelization using MPI.
%
In the next section, we describe how FDPS provides a generic
framework which takes care of parallelization
and bookkeeping for particle-based simulations. 

% LocalWords:  monopole superparticle FDPS hpp namespace nd th vec Nbody eps dt
% LocalWords:  pos vel acc getPos getCharge copyFromFP copyFromForce readAscii
% LocalWords:  ip const ni jp TPJ nj MPI OpenMP DomainInfo dinfo subdomains
% LocalWords:  subdomain ParticleSystem ptcl TreeForForceLong readParticleAscii
% LocalWords:  calcGravAllAndWriteBack decomposeDomainAll exchangeParticle SIMD
% LocalWords:  calcForceAllAndWriteBack CalcGrav superparticles struct grav
% LocalWords:  parallelization parallelized timestep
