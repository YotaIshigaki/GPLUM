In this section, we present the overview of Framework for Developing Particle Simulator (FDPS)\describeForIF{and FDPS \progLangName interface}. FDPS is an application-development framework which helps the application programmers and researchers to develop simulation codes for particle systems. What FDPS does are calculation of the particle-particle interactions and all of the necessary  works to parallelize that part on distributed-memory parallel computers wit near-ideal load balancing, using hybrid parallel programming model (uses both MPI and OpenMP). Low-cost part of the simulation program, such as the integration of the orbits of particles using the calculated interaction, is taken care by the user-written part of the code.

FDPS support two- and three-dimensional Cartesian coordinates. Supported boundary conditions are open and periodic. For each coordinate, the user can select  open or periodic boundary.

The user should specify the functional form of the particle-particle interaction. FDPS divides the interactions into two categories: long-range and short-range. The difference between two categories is that if the grouping of distant particles is used to speedup calculation (long-range) or not (short range).

The long-range force is further divided into two subcategories: with and without a cutoff scale. The long range force without cutoff is what is used for gravitational $N$-body simulations with open boundary. For periodic boundary, one would usually use TreePM, $\rm P^3M$, PME or other variant, for which the long-range force with cutoff can be used.

The short-range force is divided to four subcategories. By definition, the short-range force has some cutoff length. If the cutoff length is a constant which does not depend on the identity of particles, the force belongs to ``constant'' class. If the cutoff depends on the source or receiver of the force, it is of ``scatter'' or ``gather'' classes. Finally,  if the cutoff depends on both the source and receiver in the symmetric way, its class is ``symmetric''. Example of a ``constant'' interaction is the Lennard-Jones potential. Other interactions appear, for example, SPH calculation with adaptive kernel size.

The user writes the code for particle-particle interaction kernel and orbital integration using \describeForEach{C++ language}{Fortran 2003}{C language}. 
