\subsection{概要}

FDPSはいくつかのエラーメッセージを用意している。1つはコンパイル時のエ
ラーメッセージであり、もう1つは実行時のエラーメッセージである。以下、
この順に記述する。

\subsection{コンパイル時のエラーメッセージ}

\subsection{実行時のエラーメッセージ}

標準エラー出力に以下のような書式でメッセージが出力される。
\begin{screen}
  PS\_ERROR: \textit{ERROR MESSAGE}
  
  function: \textit{FUNCTION NAME}, line: \textit{LINE NUMBER}, file:
  \textit{FILE NAME}
\end{screen}

\begin{itemize}
\item \textit{ERROR MESSAGE}

  エラーメッセージ

\item \textit{FUNCTION NAME}

  エラーが起こった関数の名前

\item \textit{LINE NUMBER}

  エラーが起こった行番号
  
\item \textit{FILE NAME}

  エラーが起こったファイルの名前
    
\end{itemize}


%%\subsection{概要}

%%\subsection{入力ファイルがない場合}

%%\subsection{メモリ確保に失敗した場合}

%%\subsection{規定より大きな配列確保をしようとした場合}

%%\subsection{粒子がツリーのルートセルからはみでている場合}

%%\subsection{不適切な初期設定}

