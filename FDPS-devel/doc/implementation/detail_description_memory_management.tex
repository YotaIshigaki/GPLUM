\subsubsection{ReallocatableArrayクラス}

この節では, {\tt ReallocatableArray}クラスの詳細記述を行う. 

{\tt ReallocatableArray}クラスは以下の様に記述される。メンバ{\tt
data\_}はT型の配列で、格納最大要素数を{\tt capacity\_}、実際の要素数を
{\tt size\_}に格納する。

\begin{lstlisting}[caption=DomainInfo]
namespace  ParticleSimulator{
    template<class T>
    class ReallocatableArray{
    public:
        T * data_;
        int size_;
        int capacity_;
        ReallocatableArray();
        ReallocatableArray(int capa);
        ReallocatableArray(int size, const T & val);
        ~ReallocatableArray();
        void reserve(const int n);
        int size() const;
        int capacity() const;
        const T & operator [] (const int i) const;
        T & operator [] (const int i);
        T & front(){ return data_[0]; }
        T & back(){ return data_[size_-1]; }
        T * data(){ return data_; }
        const T * data() const { return data_; }
        void push_back(const T & val);
        void resizeNoInitialize(const int n)
        T * getPointer(const int i=0) const;
        void pushBackNoCheck(const T & val){
        long getMemSize() const;
        void dump(const std::string str="");

        }
    };
}
\end{lstlisting}

コンストラクタは以下の3種類。

\begin{screen}
\begin{verbatim}
ReallocatableArray();
\end{verbatim}
\end{screen}

空のクラスを作成する。{\tt std::vector}と同じ振る舞いをする。

\begin{screen}
\begin{verbatim}
ReallocatableArray(int capa);
\end{verbatim}
\end{screen}

{\tt capacity\_}に引数{\tt capa}を代入する。この際、{\tt size\_}は0であ
る。{\tt std::vector}と振る舞いが異なる。

\begin{screen}
\begin{verbatim}
ReallocatableArray(int size, const T & val)
\end{verbatim}
\end{screen}

以下はデストラクタである。

\begin{screen}
\begin{verbatim}
~ReallocatableArray()
\end{verbatim}
\end{screen}

{\tt data\_}の領域が開放される。


{\tt capacity\_}と{\tt size\_}に引数{\tt size}を代入する。またsize個の
要素を引数{¥tt val}で初期化する。{\tt std::vector}と同じ振る舞いをする。

メソッド{\tt reserve()},{\tt size()},{\tt capacity()},{\tt
front()},{\tt back()},{\tt data()},{\tt push\_back()},演算子{\tt []}は
std::vectorのそれとほぼ同じである。

以下には、{\tt AllocatableArray}クラス特有のメソッドを記述する。

\begin{screen}
\begin{verbatim}
public:
void PS::AllocatableArray::resizeNoinitialize(const int n);
\end{verbatim}
\end{screen}

{\tt size\_}に引数{\tt n}を代入する。nが{\tt capacity\_}を超えた場合は
{\tt capacity\_}をnの2倍に変更する。{std::vector::resize}との違いは付け
加えた要素の初期化はされない。


\begin{screen}
\begin{verbatim}
public:
void PS::AllocatableArray::puchBackNoCheck(const T & val);
\end{verbatim}
\end{screen}

{\tt val}を{\tt data\_}の末尾に付け加え、{\tt size\_}を一つインクリメン
トする。{\tt std::vector::push\_back}と違い、配列の領域({\tt
capacity\_})のチェックを行わない。

\begin{screen}
\begin{verbatim}
public:
T * PS::AllocatableArray::getPointer(const int i=0);
\end{verbatim}
\end{screen}
メンバ{\tt data\_}のi番目のポインタを返す。

\begin{screen}
\begin{verbatim}
public:
long getMemSize();
\end{verbatim}
\end{screen}
{\tt data\_}が使用しているバイト数を返す。

\begin{screen}
\begin{verbatim}
public:
void dump(const std::string str="");
\end{verbatim}
\end{screen}

標準出力に引数{\tt str},{\tt size\_},{\tt capacity\_}を表示する。
