FDPSでは簡易的な時間計測クラス({\tt Timer})を用意した。

{\tt Timer}クラスは以下のように記述されている。

\begin{lstlisting}[caption=Timer]
namespace ParticleSimulator{
    class Timer{
    private:
        enum{
            SPLIT_CAPACITY = 128,
            STRING_SIZE = 1024,
        };
        double begin_time_;
        double end_time_;
        double split_time_[SPLIT_CAPACITY];
        char func_name_[SPLIT_CAPACITY][STRING_SIZE];
        size_t cnt_;
    public:
        void reset();
        void start();
        void stop(const char * str='\0');
        void restart(const char * str='\0');
        void dump(std::ostream & fout);
    };
}
\end{lstlisting}

メソッド、{\tt reset}でクラスの初期化を行い、{\tt start}で現在の時刻を
メンバ{\tt begin\_time\_}に入れる。{\tt stop}で時間計測を止め、現在時
刻と{\tt begin\_time\_}との差をメンバ{\tt split\_time\_[cnt\_]}に格納
する。続けて測定を行いたいときは{\tt stop}ではなく{\tt restart}を使う。
この関数はそこまでの時間を計測し、{\tt split\_time\_[cnt\_]}にその値を
入れる。{\tt restart}するたびにカウンター変数{\tt cnt\_}はインクリメン
トされる。{\tt dump}は全スプリットタイムや最もそのスプリットタイムのか
かったプロセスのランクとその時間を引数{\tt fout}に表示する。

{\tt stop}や{\tt restart}にある引数でそのスプリットタイムを識別するた
めの文字列をメンバ{\tt func\_name\_}入れる事が出来る。この名前は{\tt
dump}を行った時に出力される。
