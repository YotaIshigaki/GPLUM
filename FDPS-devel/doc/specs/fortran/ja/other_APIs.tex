本節ではFortran/C言語 インターフェースに用意されている他のAPIについて記述する。本節で説明するAPIの名称の一覧を以下に示す:
\begin{screen}
\begin{spverbatim}
(fdps_)create_mtts
(fdps_)delete_mtts
(fdps_)mtts_init_genrand
(fdps_)mtts_genrand_int31
(fdps_)mtts_genrand_real1
(fdps_)mtts_genrand_real2
(fdps_)mtts_genrand_res53
(fdps_)mt_init_genrand
(fdps_)mt_genrand_int31
(fdps_)mt_genrand_real1
(fdps_)mt_genrand_real2
(fdps_)mt_genrand_res53
\end{spverbatim}  
\end{screen}
ここに示されたAPIの内、名称にmtが含まれるAPIは擬似乱数列生成器メルセンヌ・ツイスタ(Mersenne twister)を操作・使用するためのAPIである。

以下、順に各APIの仕様について記述していく。

\clearpage


%=============================================================
\subsection{create\_mtts}
\subsubsection*{Fortran 構文}
\begin{screen}
\begin{spverbatim}  
subroutine fdps_ctrl%create_mtts(mtts_num)
\end{spverbatim}
\end{screen}

\subsubsection*{C言語 構文}
\begin{screen}
\begin{spverbatim}  
void fdps_create_mtts(int * mtts_num);
\end{spverbatim}
\end{screen}

\subsubsection*{仮引数仕様}
\begin{table}[h]
\begin{tabularx}{\linewidth}{cccX}
\toprule
\rowcolor{Snow2}
仮引数名 & データ型 & 入出力属性 & 定義 \\
\midrule
\verb|mtts_num| & integer(kind=c\_int) & 入出力 & 疑似乱数生成用オブジェクトの識別番号を受け取る変数。{\setnoko\uc{C言語では変数のアドレスを引数に指定する必要があることに注意}}。\\
\bottomrule
\end{tabularx}
\end{table}

\subsubsection*{返り値}
なし。

\subsubsection*{機能}
メモリ上にメルセンヌ・ツイスタ(Mersenne twister)を使って疑似乱数を生成するオブジェクトを1つ生成し、そのオブジェクトの識別番号を返す。
\clearpage

%=============================================================
\subsection{delete\_mtts}
\subsubsection*{Fortran 構文}
\begin{screen}
\begin{spverbatim}
subroutine fdps_ctrl%delete_mtts(mtts_num)
\end{spverbatim}
\end{screen}

\subsubsection*{C言語 構文}
\begin{screen}
\begin{spverbatim}
void fdps_delete_mtts(const int mtts_num);
\end{spverbatim}
\end{screen}

\subsubsection*{仮引数仕様}
\begin{table}[h]
\begin{tabularx}{\linewidth}{cccX}
\toprule
\rowcolor{Snow2}
仮引数名 & データ型 & 入出力属性 & 定義 \\
\midrule
\verb|mtts_num| & integer(kind=c\_int) & 入力 & 疑似乱数生成用オブジェクトの識別番号を格納した変数。\\
\bottomrule
\end{tabularx}
\end{table}

\subsubsection*{返り値}
なし。

\subsubsection*{機能}
識別番号\texttt{mtts\_num}の疑似乱数生成用オブジェクトをメモリ上から削除する。
\clearpage

%=============================================================
\subsection{mtts\_init\_genrand}
\subsubsection*{Fortran 構文}
\begin{screen}
\begin{spverbatim}  
subroutine fdps_ctrl%mtts_init_genrand(mtts_num,s)
\end{spverbatim}
\end{screen}

\subsubsection*{C言語 構文}
\begin{screen}
\begin{spverbatim}  
void fdps_mtts_init_genrand(const int mtts_num,
                            const int s);
\end{spverbatim}
\end{screen}

\subsubsection*{仮引数仕様}
\begin{table}[h]
\begin{tabularx}{\linewidth}{cccX}
\toprule
\rowcolor{Snow2}
仮引数名 & データ型 & 入出力属性 & 定義 \\
\midrule
\verb|mtts_num| & integer(kind=c\_int) & 入力 & 疑似乱数生成用オブジェクトの識別番号を格納した変数。\\
\verb|s| & integer(kind=c\_int) & 入力 & 疑似乱数生成に使用するシード。\\
\bottomrule
\end{tabularx}
\end{table}

\subsubsection*{返り値}
なし

\subsubsection*{機能}
識別番号\texttt{mtts\_num}の疑似乱数生成用オブジェクトを初期化する。
\clearpage

%=============================================================
\subsection{mtts\_genrand\_int31}
\subsubsection*{Fortran 構文}
\begin{screen}
\begin{spverbatim}  
function fdps_ctrl%mtts_genrand_int31(mtts_num)
\end{spverbatim}
\end{screen}

\subsubsection*{C言語 構文}
\begin{screen}
\begin{spverbatim}  
int fdps_mtts_genrand_int31(const int mtts_num);
\end{spverbatim}
\end{screen}

\subsubsection*{仮引数仕様}
\begin{table}[h]
\begin{tabularx}{\linewidth}{cccX}
\toprule
\rowcolor{Snow2}
仮引数名 & データ型 & 入出力属性 & 定義 \\
\midrule
\verb|mtts_num| & integer(kind=c\_int) & 入力 & 疑似乱数生成用オブジェクトの識別番号を格納した変数。\\
\bottomrule
\end{tabularx}
\end{table}

\subsubsection*{返り値}
integer(kind=c\_int)型スカラー値

\subsubsection*{機能}
識別番号\texttt{mtts\_num}の疑似乱数生成用オブジェクトを使って、[0,0x7fffffff]の範囲で一様な整数乱数を生成する。
\clearpage

%=============================================================
\subsection{mtts\_genrand\_real1}
\subsubsection*{Fortran 構文}
\begin{screen}
\begin{spverbatim}  
function fdps_ctrl%mtts_genrand_real1(mtts_num)
\end{spverbatim}
\end{screen}

\subsubsection*{C言語 構文}
\begin{screen}
\begin{spverbatim}  
double fdps_mtts_genrand_real1(const int mtts_num);
\end{spverbatim}
\end{screen}

\subsubsection*{仮引数仕様}
\begin{table}[h]
\begin{tabularx}{\linewidth}{cccX}
\toprule
\rowcolor{Snow2}
仮引数名 & データ型 & 入出力属性 & 定義 \\
\midrule
\verb|mtts_num| & integer(kind=c\_int) & 入力 & 疑似乱数生成用オブジェクトの識別番号を格納した変数。\\
\bottomrule
\end{tabularx}
\end{table}

\subsubsection*{返り値}
real(kind=c double) 型スカラー値。

\subsubsection*{機能}
識別番号\texttt{mtts\_num}の疑似乱数生成用オブジェクトを使って、[0.0,1.0]の範囲で一様な浮動小数点数乱数を生成する。
\clearpage

%=============================================================
\subsection{mtts\_genrand\_real2}
\subsubsection*{Fortran 構文}
\begin{screen}
\begin{spverbatim}  
function fdps_ctrl%mtts_genrand_real2(mtts_num)
\end{spverbatim}
\end{screen}

\subsubsection*{C言語 構文}
\begin{screen}
\begin{spverbatim}  
double fdps_mtts_genrand_real2(const int mtts_num);
\end{spverbatim}
\end{screen}

\subsubsection*{仮引数仕様}
\begin{table}[h]
\begin{tabularx}{\linewidth}{cccX}
\toprule
\rowcolor{Snow2}
仮引数名 & データ型 & 入出力属性 & 定義 \\
\midrule
\verb|mtts_num| & integer(kind=c\_int) & 入力 & 疑似乱数生成用オブジェクトの識別番号を格納した変数。\\
\bottomrule
\end{tabularx}
\end{table}

\subsubsection*{返り値}
real(kind=c double) 型スカラー値。

\subsubsection*{機能}
識別番号\texttt{mtts\_num}の疑似乱数生成用オブジェクトを使って、[0.0,1.0)の範囲で一様な浮動小数点数乱数を生成する。
\clearpage

%=============================================================
\subsection{mtts\_genrand\_real3}
\subsubsection*{Fortran 構文}
\begin{screen}
\begin{spverbatim}  
function fdps_ctrl%mtts_genrand_real3(mtts_num)
\end{spverbatim}
\end{screen}

\subsubsection*{C言語 構文}
\begin{screen}
\begin{spverbatim}  
double fdps_mtts_genrand_real3(const int mtts_num);
\end{spverbatim}
\end{screen}

\subsubsection*{仮引数仕様}
\begin{table}[h]
\begin{tabularx}{\linewidth}{cccX}
\toprule
\rowcolor{Snow2}
仮引数名 & データ型 & 入出力属性 & 定義 \\
\midrule
\verb|mtts_num| & integer(kind=c\_int) & 入力 & 疑似乱数生成用オブジェクトの識別番号を格納した変数。\\
\bottomrule
\end{tabularx}
\end{table}

\subsubsection*{返り値}
real(kind=c double) 型スカラー値。

\subsubsection*{機能}
識別番号\texttt{mtts\_num}の疑似乱数生成用オブジェクトを使って、(0.0,1.0)の範囲で一様な浮動小数点数乱数を生成する。
\clearpage

%=============================================================
\subsection{mtts\_genrand\_res53}
\subsubsection*{Fortran 構文}
\begin{screen}
\begin{spverbatim}  
function fdps_ctrl%mtts_genrand_res53(mtts_num)
\end{spverbatim}
\end{screen}

\subsubsection*{C言語 構文}
\begin{screen}
\begin{spverbatim}  
double fdps_mtts_genrand_res53(const int mtts_num);
\end{spverbatim}
\end{screen}

\subsubsection*{仮引数仕様}
\begin{table}[h]
\begin{tabularx}{\linewidth}{cccX}
\toprule
\rowcolor{Snow2}
仮引数名 & データ型 & 入出力属性 & 定義 \\
\midrule
\verb|mtts_num| & integer(kind=c\_int) & 入力 & 疑似乱数生成用オブジェクトの識別番号を格納した変数。\\
\bottomrule
\end{tabularx}
\end{table}

\subsubsection*{返り値}
real(kind=c double) 型スカラー値。

\subsubsection*{機能}
識別番号\texttt{mtts\_num}の疑似乱数生成用オブジェクトを使って、[0.0,1.0) の範囲で一様な浮動小数点乱数を生成する。前述した mtts\_genrand\_real$x$ ($x$=1-3) は浮動小数点数へ変換するのに32ビット整数乱数を使用しているのに対し、本APIでは53ビット整数乱数を使用している。
\clearpage

%=============================================================
% API name::MT_init_genrand()
\subsection{mt\_init\_genrand}
\subsubsection*{Fortran 構文}
\begin{screen}
\begin{spverbatim}
subroutine fdps_ctrl%mt_init_genrand(s)
\end{spverbatim}
\end{screen}

\subsubsection*{C言語 構文}
\begin{screen}
\begin{spverbatim}
void fdps_mt_init_genrand(const int s);
\end{spverbatim}
\end{screen}

\subsubsection*{仮引数仕様}
\begin{table}[h]
\begin{tabularx}{\linewidth}{cccX}
\toprule
\rowcolor{Snow2}
仮引数名 & データ型 & 入出力属性 & 定義 \\
\midrule
\texttt{s} & integer(kind=c\_int) & 入出力 & 擬似乱数生成に使用するシード。\\
\bottomrule
\end{tabularx}
\end{table}

\subsubsection*{返り値}
なし。

\subsubsection*{機能}
擬似乱数列生成器メルセンヌ・ツイスタ(Mersenne twister)のオブジェクトを生成し初期化を行う。
\clearpage

%=============================================================
% API名::MT_genrand_int31()
\subsection{mt\_genrand\_int31}
\subsubsection*{Fortran 構文}
\begin{screen}
\begin{spverbatim}
function fdps_ctrl%mt_genrand_int31()
\end{spverbatim}
\end{screen}

\subsubsection*{C言語 構文}
\begin{screen}
\begin{spverbatim}
int fdps_mt_genrand_int31();
\end{spverbatim}
\end{screen}

\subsubsection*{仮引数仕様}
なし。

\subsubsection*{返り値}
integer(kind=c\_int)型スカラー値。

\subsubsection*{機能}
擬似乱数列生成器メルセンヌ・ツイスタ(Mersenne twister)を使って、[0,0x7fffffff]の範囲で一様な整数乱数を生成する。
\clearpage

%=============================================================
% API名::MT_genrand_real1()
\subsection{mt\_genrand\_real1}
\subsubsection*{Fortran 構文}
\begin{screen}
\begin{spverbatim}
function fdps_ctrl%mt_genrand_real1()
\end{spverbatim}
\end{screen}

\subsubsection*{C言語 構文}
\begin{screen}
\begin{spverbatim}
double fdps_mt_genrand_real1();
\end{spverbatim}
\end{screen}


\subsubsection*{仮引数仕様}
なし

\subsubsection*{返り値}
real(kind=c\_double)型スカラー値。

\subsubsection*{機能}
擬似乱数列生成器メルセンヌ・ツイスタ(Mersenne twister)を使って、[0.0,1.0]の範囲で一様な浮動小数点数乱数を生成する。
\clearpage

%=============================================================
% API名::MT_genrand_real2()
\subsection{mt\_genrand\_real2}
\subsubsection*{Fortran 構文}
\begin{screen}
\begin{spverbatim}
function fdps_ctrl%mt_genrand_real2()
\end{spverbatim}
\end{screen}

\subsubsection*{C言語 構文}
\begin{screen}
\begin{spverbatim}
double fdps_mt_genrand_real2();
\end{spverbatim}
\end{screen}


\subsubsection*{仮引数仕様}
なし

\subsubsection*{返り値}
real(kind=c\_double)型スカラー値。

\subsubsection*{機能}
擬似乱数列生成器メルセンヌ・ツイスタ(Mersenne twister)を使って、[0.0,1.0)の範囲で一様な浮動小数点数乱数を生成する。
\clearpage

%=============================================================
% API名::MT_genrand_real3()
\subsection{mt\_genrand\_real3}
\subsubsection*{Fortran 構文}
\begin{screen}
\begin{spverbatim}
function fdps_ctrl%MT_genrand_real3()
\end{spverbatim}
\end{screen}

\subsubsection*{C言語 構文}
\begin{screen}
\begin{spverbatim}
double fdps_mt_genrand_real3();
\end{spverbatim}
\end{screen}

\subsubsection*{仮引数仕様}
なし。

\subsubsection*{返り値}
real(kind=c\_double)型スカラー値。

\subsubsection*{機能}
擬似乱数列生成器メルセンヌ・ツイスタ(Mersenne twister)を使って、(0.0,1.0)の範囲で一様な浮動小数点乱数を生成する。
\clearpage

%=============================================================
% API名::MT_genrand_res53()
\subsection{mt\_genrand\_res53}
\subsubsection*{Fortran 構文}
\begin{screen}
\begin{spverbatim}
function fdps_ctrl%MT_genrand_res53()
\end{spverbatim}
\end{screen}

\subsubsection*{C言語 構文}
\begin{screen}
\begin{spverbatim}
double fdps_mt_genrand_res53();
\end{spverbatim}
\end{screen}


\subsubsection*{仮引数仕様}
なし。

\subsubsection*{返り値}
real(kind=c\_double)型スカラー値。

\subsubsection*{機能}
擬似乱数列生成器メルセンヌ・ツイスタ(Mersenne twister)を使って、[0.0,1.0)の範囲で一様な浮動小数点乱数を生成する。前述したmt\_genrand\_real$x$ ($x$=1-3)は浮動小数点数へ変換するのに32ビット整数乱数を使用しているのに対し、本APIでは53ビット整数乱数を使用している。
\clearpage





