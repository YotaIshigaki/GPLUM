本章では、FDPS Fortran/C言語 インターフェース生成スクリプトの動作条件と使用方法について記述する。

%%%%%%%%%%%%%%%%%%%%%%%%%%%%%%%%%%%%%%%%%%%%%%%%%%%%%
\section{スクリプトの動作条件}
本節では、インターフェース生成スクリプト\path{gen_ftn_if.py} (Fortranインターフェース生成用) 及び \path{gen_c_if.py} (C言語インターフェス生成用)の動作条件について記述する。2つのインターフェース生成スクリプトはディレクトリ\path{scripts}の直下に配置されている。本スクリプトは、プログラミング言語 Python で実装されており、正常な動作のためには、Python 2.7.5 以上、或いは、Python 3.4 以上が必要である。ユーザの環境に合わせて、スクリプト第1行目の
\begin{Verbatim}[commandchars=\\\{\}]
#!/usr/bin/env python
\end{Verbatim}
の部分を適宜修正して使用されたい。ここで、確認すべき点は\texttt{env}コマンドのPATHとPythonインタープリタの名称である(利用する計算機システムによっては\texttt{python}が存在せず、\texttt{python2.7}や\texttt{python3.4}のように名称にバージョン名が付いたもののみが用意されている場合がある)。もしPythonインタープリタにPATHが通っていない場合には、PATHを通すか、以下のように、絶対PATHでPythonインタープリタを指定する:
\begin{Verbatim}[commandchars=\\\{\}]
#!/path/to/python
\end{Verbatim}

上記に加え、正常な動作のためには、以下の条件を満たす必要がある:
\begin{itemize}[leftmargin=*,itemsep=-1ex]
\item スクリプト\path{gen_ftn_if.py}に入力されるすべてのFortranコードがFortran 2003標準 (ISO/IEC 1539-1:2004(E))の文法に従って記述されていること。本スクリプトに言語の自動判別機能や詳細な文法チェック機能は実装されておらず、誤った文法が使用された場合の動作は不定である。
\item スクリプト\path{gen_c_if.py}に入力されるすべてのC言語コードはC99 (ISO/IEC 9899:1999(E))、または、それよりも新しい規格に従って記述されていること。本スクリプトに言語の自動判別機能や詳細な文法チェック機能は実装されておらず、誤った文法が使用された場合の動作は不定である。
\end{itemize}

%%%%%%%%%%%%%%%%%%%%%%%%%%%%%%%%%%%%%%%%%%%%%%%%%%%%%
\section{スクリプトの使用方法}
本スクリプトを使ってインターフェースプログラムを生成するためには、コマンドライン上で、以下のようにしてスクリプトを実行すればよい。Fortranインターフェースを生成する場合には
\begin{screen}
\begin{Verbatim}[commandchars=\\\{\}]
$ gen_ftn_if.py -o output_directory user1.F90 user2.F90... 
\end{Verbatim}
\end{screen}

C言語インターフェースを生成する場合には
\begin{screen}
\begin{Verbatim}[commandchars=\\\{\}]
$ gen_c_if.py -o output_directory user1.h user2.h... 
\end{Verbatim}
\end{screen}

ここで、環境変数\path{PATH}にディレクトリ\path{scripts}が追加されていると仮定している。PATHを通さずにスクリプトを使用する場合には、絶対PATHか相対PATHでスクリプトを実行する必要がある。スクリプトの引数には、ユーザ定義型が記述されたFortranファイル {\small (\path{gen_ftn_if.py}を使用する場合)} 或いは C言語ヘッダーフェイル {\small (\path{gen_c_if.py}を使用する場合)}を指定する。複数のファイルを指定する場合には、1個以上の半角スペースを空けて、ファイル名を並べる。この際、並べる順番は任意でよい。

オプション「\path{-o}」でインターフェースプログラムを出力するディレクトリを指定することができる。オプション「\path{-o}」の代わりに、「\path{--output}」あるいは「\path{--output_dir}」を使用することもできる。指定がない場合には、カレントディレクトリに出力される。

オプション「\path{-DPARTICLE_SIMULATOR_TWO_DIMENSION}」を指定した場合、シミュレーションの空間次元数が2と仮定される。このオプションに引数はない。本オプションが無指定の場合、3が仮定される。空間次元数は、ユーザ定義型の位置と速度に対応するメンバ変数のデータ型のチェックに使用される。ユーザコードのコンパイル時にマクロ\path{PARTICLE_SIMULATOR_TWO_DIMENSION}が定義される場合には、必ずこのオプションを指定しなければならない(無指定の場合、インターフェースプログラムの正常な動作は保証されない)。

本スクリプトの使用方法はオプション「\path{-h}」あるいは「\path{--help}」でも確認することできる。以下は、\path{gen_ftn.if.py}の場合の例である。
\begin{screen}
\begin{spverbatim}
[user@hostname somedir]$ gen_ftn_if.py -h
[namekata@jenever0 scripts]$ ./gen_ftn_if.py --help
usage: gen_ftn_if.py [-h] [-o DIRECTORY] [-DPARTICLE_SIMULATOR_TWO_DIMENSION]
                     FILE [FILE ...]

Analyze user's Fortran codes and generate C++/Fortran source files required to use FDPS from the user's Fortran code.

positional arguments:
  FILE                  The PATHs of input Fortran files

optional arguments:
  -h, --help            show this help message and exit
  -o DIRECTORY, --output DIRECTORY, --output_dir DIRECTORY
                        The PATH of output directory
  -DPARTICLE_SIMULATOR_TWO_DIMENSION
                        Indicate that simulation is performed
                        in the 2-dimensional space (equivalent
                        to define the macro 
                        PARTICLE_SIMULATOR_TWO_DIMENSION)
\end{spverbatim}
\end{screen}


生成されたインターフェースプログラムをユーザプログラムと一緒にコンパイルすることで、実行ファイルが得られる。コンパイルの仕方に関しては、次の第\ref{chap:compile_and_macro}章を参照されたい。

