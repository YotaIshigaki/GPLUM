In this chapter, we describe the limitation and the restriction of FDPS and FDPS Fortran/C interface. Because FDPS Fortran interface are unconditionally restricted by the specification of FDPS, we first summarize the limitation of FDPS. Then, we describe the limitation specific to FDPS Fortran/C interface.

%%%%%%%%%%%%%%%%%%%%%%%%%%%%%%%%%%%%%%%%%%%%%%%%%%%%
\section{FDPS}
\label{sec:limitation:FDPS}
\begin{itemize}
\item FDPS独自の整数型を用いる場合、GCCコンパイラとKコンパイラでのみ正
  常に動作することが保証されている。
\end{itemize}



%%%%%%%%%%%%%%%%%%%%%%%%%%%%%%%%%%%%%%%%%%%%%%%%%%%%
\section{FDPS Fortran/C interface}
\label{sec:limitation:FDPS_ftn_if}

As of writing this, FDPS Fortran/C interface has the following limitations and restrictions:
\begin{itemize}[leftmargin=*,itemsep=-1ex]
\item Some low-level APIs and APIs about I/O implemented in FDPS are not supported in Fortran/C interface.
\item Execution on GPUs (Graphics Processing Units) is not supported yet.
\item When user uses FDPS from C++ codes, the user can freely customize or design the moment information of superparticle, where the moment information is defined as the quantities that are required in the calculation of particle-superparticle interaction and that are calculated from physical quantities of the particles that consist of that superparticle. Examples of the moment information include monopole moment, dipole moment, and high-order multipole moments, etc. FDPS Fortran/C interface only supports the moment informations prepared by FDPS (see Chap.~\ref{chap:data_types} \S~\ref{sec:super_particle_types} and Chap.~\ref{chap:API_spec_list} \S~\ref{sec:tree_APIs}).
\end{itemize}