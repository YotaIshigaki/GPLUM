FDPS Fortran/C interface defines several data types including vector types, symmetric matrix types, superparticle types, time profile type, enum types. In addition, basic data types are also defined in C. These data types are required to be used in user-defined types and user-defined functions (Chap.~\ref{chap:user_defined}) or are used as returned values in some APIs (Chap.~\ref{chap:API_spec_list}).

%%%%%%%%%%%%%%%%%%%%%%%%%%%%%%%%%%%%%%%%%%%%%%%%%%%%%%
\section{Basic data types {\small (C only)}}
\label{sec:basic_types}
FDPS C interface defines six basic data types: \path{fdps_s32}, \path{fdps_u32}, \path{fdps_f32}, \path{fdps_s64}, \path{fdps_u64}, \path{fdps_f64}. They are defined in the file \path{src/c_interface/headers/FDPS_basic.h} as follows (Listing~\ref{listing:basic_types}). These data types are used to defined other data types provided by FDPS.

\begin{lstlisting}[language=C,caption=Basic data types (C only),label=listing:basic_types]
#pragma once

/* 32 bit data types */
typedef int          fdps_s32;
typedef unsigned int fdps_u32;
#ifdef PARTICLE_SIMULATOR_ALL_64BIT_PRECISION
typedef double       fdps_f32;
#else
typedef float        fdps_f32;
#endif

/* 64 bit data types */
typedef long long int          fdps_s64;
typedef unsigned long long int fdps_u64;
typedef double                 fdps_f64;
\end{lstlisting}

\redtext{Note that the macro \texttt{PARTICLE\_SIMULATOR\_ALL\_64BIT\_PRECISION} is just introduced for future and C interface programs work correctly only if this macro is undefined.}

%%%%%%%%%%%%%%%%%%%%%%%%%%%%%%%%%%%%%%%%%%%%%%%%%%%%%%
\section{Vector types}
\label{sec:vector_types}
FDPS Fortran and C interface define two types of vector: \path{fdps_f32vec} and \path{fdps_f64vec}. They are defined in the files \path{src/fortran_interface/modules/FDPS_vector.F90} and \path{src/c_interface/headers/FDPS_vector.h} as follows (Listings~\ref{listing:vector_types} and \ref{listing:vector_types_in_C}). Each vector has 2 or 3 member variables depending on the spatial dimension of the simulation. By default, the spatial dimension of the simulation is assumed to be 3, but it is 2 if the macro \path{PARTICLE_SIMULATOR_TWO_DIMENSION} is defined at the compilation. The data type of the member variables is either 32-bit or 64-bit floating point numbers.

\begin{lstlisting}[caption=Vector types (Fortran),label=listing:vector_types]
module fdps_vector
   use, intrinsic :: iso_c_binding
   implicit none
   
   type, public, bind(c) :: fdps_f32vec
#ifdef PARTICLE_SIMULATOR_TWO_DIMENSION 
      real(kind=c_float) :: x,y
#else
      real(kind=c_float) :: x,y,z
#endif
   end type fdps_f32vec

   type, public, bind(c) :: fdps_f64vec
#ifdef PARTICLE_SIMULATOR_TWO_DIMENSION 
      real(kind=c_double) :: x,y
#else
      real(kind=c_double) :: x,y,z
#endif
   end type fdps_f64vec

end module fdps_vector
\end{lstlisting}

\begin{lstlisting}[language=C,caption=Vector types (C), label=listing:vector_types_in_C]
#pragma once
#include "FDPS_basic.h"

//**** PS::F32vec
typedef struct  {
#ifdef PARTICLE_SIMULATOR_TWO_DIMENSION
   fdps_f32 x,y;
#else
   fdps_f32 x,y,z;
#endif
} fdps_f32vec;

//**** PS::F64vec
typedef struct  {
#ifdef PARTICLE_SIMULATOR_TWO_DIMENSION
   fdps_f64 x,y;
#else
   fdps_f64 x,y,z;
#endif
} fdps_f64vec;
\end{lstlisting}

In Fortran, for the vector types, the definitions of the assignment (\texttt{=}) and the arithmetic operators (\texttt{+},\texttt{-},\texttt{*},\texttt{/}) are extended as shown in Table.~\ref{tbl:op_ext:fdps_vector}. For more details, see \path{FDPS_vector.F90}.

\begin{table}[H]
\begin{tabularx}{\linewidth}{|c|c|c|X|}
\toprule
\rowcolor{Snow2}
Symbol & Left-hand side & Right-hand side & Definition \\
\midrule
% 代入記号(=)
\multirow{3}{*}{\texttt{=}} & vector & scalar$^{\dagger}$ & \multirow{3}{\hsize}{\footnotesize Assign RHS to LHS$^{\sharp}$. When RHS is scalar, it is assigned to all the components of LHS. When RHS is an array, each component of LHS is assigned by array element according to the memory ordering.} \\[2pt]
\cmidrule(r){2-3}
 & vector & array of scalars$^{\ddagger}$ &  \\[2pt]
\cmidrule(r){2-3}
 & vector & vector &  \\[2pt]
\midrule
% + 演算子
\multirow{4}{*}{\texttt{+}} & vector & array of scalars & \multirow{3}{\hsize}{\footnotesize Addition of LHS and RHS. When one of the operands is an array, it is assumed that each element of the array corresponds to the component of the vector according to the memory ordering.} \\[2pt]
\cmidrule(r){2-3}
 & array of scalars & vector & \\[2pt]
\cmidrule(r){2-3}
 & vector & vector & \\[2pt]
\cmidrule(r){2-4}
 & none & vector & Do nothing \\
\midrule
% - 演算子
\multirow{4}{*}{\texttt{-}} & vector & array of scalars & \multirow{3}{\hsize}{\footnotesize Subtraction RHS from LHS. When one of the operands is an array, it is assumed that each element of the array corresponds to the component of the vector according to the memory ordering.} \\[2pt]
\cmidrule(r){2-3}
& array of scalars & vector &  \\[2pt]
\cmidrule(r){2-3}
& vector & vector & \\[2pt]
\cmidrule(r){2-4}
& none & vector & Inversion of the sign of the vector \\
\midrule
% * 演算子
\multirow{5}{*}{\texttt{*}} & vector & scalar & \multirow{2}{*}{Scalar-vector product} \\
\cmidrule(r){2-3}
 & scalar & vector & \\
\cmidrule(r){2-4}
 & vector & array of scalars & \multirow{3}{\hsize}{\footnotesize Inner product. When one of the operands is an array, it is assumed that each element of the array corresponds to the component of the vector according to the memory ordering.} \\
\cmidrule(r){2-3}
 & array of scalars & vector &  \\
\cmidrule(r){2-3}
 & vector & vector &  \\
\midrule
\texttt{/} & vector & scalar & Division LHS by RHS. \\
\bottomrule
\end{tabularx}
\begin{flushleft}
$^{\dagger}$ The data type of scalar must be one of intrinsic data types in Fortran and be numeric. \\
$^{\ddagger}$ The size of array must be 2 if the macro \path{PARTICLE_SIMULATOR_TWO_DIMENSION} is defined at the compilation. Otherwise, it must be 3. \\
$^{\sharp}$ LHS and RHS stand for the left- and the right-hand sides, respectively.
\end{flushleft}
\caption{The definitions of the assignment and the arithmetic operators extended for the vector types}
\label{tbl:op_ext:fdps_vector}
\end{table}


%%%%%%%%%%%%%%%%%%%%%%%%%%%%%%%%%%%%%%%%%%%%%%%%%%%%%%
\section{Symmetric Matrix types}
\label{sec:symmetric_matrix_types}
There are two types of symmetric matrix: \path{fdps_f32mat} and \path{fdps_f64mat}. These are defined in the files \path{src/fortran_interface/modules/FDPS_matrix.F90} and \path{src/c_interface/headers/FDPS_matrix.h} as follows (Listings~\ref{listing:symmetric_matrix_types} and \ref{listing:symmetric_matrix_types_in_C}). Each symmetric matrix type has 3 or 6 member variables depending on the spatial dimension of the simulation. By default, the spatial dimension of the simulation is assumed to be 3, but it is 2 if the macro \path{PARTICLE_SIMULATOR_TWO_DIMENSION} is defined at the compilation. The data type of the member variables is either 32-bit or 64-bit floating point numbers.

\begin{lstlisting}[caption=Symmetric Matrix types (Fortran),label=listing:symmetric_matrix_types]
module fdps_matrix
   use, intrinsic :: iso_c_binding
   implicit none

   !**** PS::F32mat
   type, public, bind(c) :: fdps_f32mat
#ifndef PARTICLE_SIMULATOR_TWO_DIMENSION
      real(kind=c_float) :: xx,yy,zz,xy,xz,yz
#else
      real(kind=c_float) :: xx,yy,xy
#endif
   end type fdps_f32mat

   !**** PS::F64mat
   type, public, bind(c) :: fdps_f64mat
#ifndef PARTICLE_SIMULATOR_TWO_DIMENSION
      real(kind=c_double) :: xx,yy,zz,xy,xz,yz
#else
      real(kind=c_double) :: xx,yy,xy
#endif
   end type fdps_f64mat

end module fdps_matrix  
\end{lstlisting}

\begin{lstlisting}[language=C,caption=Symmetric Matrix types (C), label=listing:symmetric_matrix_types_in_C]
#pragma once
#include "FDPS_basic.h"

//**** PS::F32mat
typedef struct {
#ifndef PARTICLE_SIMULATOR_TWO_DIMENSION
   fdps_f32 xx,yy,zz,xy,xz,yz;
#else
   fdps_f32 xx,yy,xy;
#endif
} fdps_f32mat;

//**** PS::F64mat
typedef struct {
#ifndef PARTICLE_SIMULATOR_TWO_DIMENSION
   fdps_f64 xx,yy,zz,xy,xz,yz;
#else
   fdps_f64 xx,yy,xy;
#endif
} fdps_f64mat;
\end{lstlisting}

In Fortran, for the symmetric matrix types, the definitions of the assignment (\texttt{=}) and the arithmetic operators (\texttt{+},\texttt{-},\texttt{*},\texttt{-}) are extended as shown in Table.~\ref{tbl:op_ext:fdps_matrix}. For more details, see \path{FDPS_matrix.F90}.

\begin{table}[H]
\begin{tabularx}{\linewidth}{|c|c|c|X|}
\toprule
\rowcolor{Snow2}
Symbol & Left-hand side & Right-hand side & Definition \\
\midrule
% = 記号
\multirow{2}{*}{\texttt{=}}& matrix & scalar$^{\dagger}$ & \multirow{2}{\hsize}{\footnotesize Assign RHS to LHS. When RHS is scalar, it is assigned to all the components of LHS.} \\
\cmidrule(r){2-3}
& matrix & matrix & \\
\midrule
% + 演算子
\multirow{2}{*}{\texttt{+}} & matrix & matrix & Addition of LHS and RHS. \\
\cmidrule(r){2-4}
& none & matrix & Do nothing \\
\midrule
% - 演算子
\multirow{2}{*}{\texttt{-}} & matrix & matrix & Subtraction RHS from LHS. \\
\cmidrule(r){2-4}
& none & matrix & {\footnotesize Sign inversion of all the components of the matrix} \\
\midrule
% * 演算子
\multirow{3}{*}{\texttt{*}} & matrix & scalar & \multirow{2}{*}{Scalar-matrix product} \\
\cmidrule(r){2-3}
 & scalar & matrix &  \\
\cmidrule(r){2-4}
 & matrix & matrix & Matrix product \\
\midrule
% / 演算子
\texttt{/} & matrix & scalar & Division LHS by RHS. \\
\bottomrule
\end{tabularx}
\begin{flushleft}
$^{\dagger}$ The data type of scalar must be one of intrinsic data types in Fortran and be numeric.
\end{flushleft}
\caption{The definitions of the assignment and the arithmetic operators extended for the symmetric matrix types}
\label{tbl:op_ext:fdps_matrix}
\end{table}



%%%%%%%%%%%%%%%%%%%%%%%%%%%%%%%%%%%%%%%%%%%%%%%%%%%%%%
\section{Superparticle types}
\label{sec:super_particle_types}
A superparticle is a virtual particle that represents a group of real particles, which are located far from the $i$-particle (a particle which we want to calculate the force acting on). Its properties are computed from the properties of these real particles via some way. Superparticle types are derived data types required to describe the interaction between a particle and a superparticle. They are used in user-defined functions to receive data of superparticles.

The superparticle types include \path{fdps_spj_monopole},\path{fdps_spj_quadrupole},\path{fdps_spj_monopole_geomcen},\path{fdps_spj_dipole_geomcen},\path{fdps_spj_quadrupole_geomcen},\path{fdps_spj_monopole_scatter},\path{fdps_spj_quadrupole_scatter}, \path{fdps_spj_monopole_symmetry}, \path{fdps_spj_quadrupole_symmetry},  \path{fdps_spj_monopole_cutoff}. They are defined in the files \path{src/fortran_interface/modules/FDPS_super_particle.F90} and \path{src/c_interface/headers/FDPS_super_particle.h} as follows (Listings~\ref{listing:superparticle_types} and \ref{listing:superparticle_types_in_C}). Note that the vector types and the symmetric matrix types are used to define the superparticle types.

\begin{lstlisting}[caption=Superparticle types (Fortran),label=listing:superparticle_types]
module fdps_super_particle
   use, intrinsic :: iso_c_binding
   use fdps_vector
   use fdps_matrix
   implicit none

   !**** PS::SPJMonopole
   type, public, bind(c) :: fdps_spj_monopole
      real(kind=c_double) :: mass
      type(fdps_f64vec) :: pos
   end type fdps_spj_monopole

   !**** PS::SPJQuadrupole
   type, public, bind(c) :: fdps_spj_quadrupole
      real(kind=c_double) :: mass
      type(fdps_f64vec)  :: pos
      type(fdps_f64mat)  :: quad
   end type fdps_spj_quadrupole

   !**** PS::SPJMonopoleGeometricCenter
   type, public, bind(c) :: fdps_spj_monopole_geomcen
      integer(kind=c_long_long) :: n_ptcl
      real(kind=c_double) :: charge
      type(fdps_f64vec) :: pos
   end type fdps_spj_monopole_geomcen

   !**** PS::SPJDipoleGeometricCenter
   type, public, bind(c) :: fdps_spj_dipole_geomcen
      integer(kind=c_long_long) :: n_ptcl
      real(kind=c_double) :: charge
      type(fdps_f64vec) :: pos
      type(fdps_f64vec) :: dipole
   end type fdps_spj_dipole_geomcen

   !**** PS::SPJQuadrupoleGeometricCenter
   type, public, bind(c) :: fdps_spj_quadrupole_geomcen
      integer(kind=c_long_long) :: n_ptcl
      real(kind=c_double) :: charge
      type(fdps_f64vec) :: pos
      type(fdps_f64vec) :: dipole
      type(fdps_f64mat) :: quadrupole
   end type fdps_spj_quadrupole_geomcen

   !**** PS::SPJMonopoleScatter
   type, public, bind(c) :: fdps_spj_monopole_scatter
      real(kind=c_double) :: mass
      type(fdps_f64vec) :: pos
   end type fdps_spj_monopole_scatter

   !**** PS::SPJQuadrupoleScatter
   type, public, bind(c) :: fdps_spj_quadrupole_scatter
      real(kind=c_double) :: mass
      type(fdps_f64vec)  :: pos
      type(fdps_f64mat)  :: quad
   end type fdps_spj_quadrupole_scatter

   !**** PS::SPJMonopoleSymmetry
   type, public, bind(c) :: fdps_spj_monopole_symmetry
      real(kind=c_double) :: mass
      type(fdps_f64vec) :: pos
   end type fdps_spj_monopole_symmetry

   !**** PS::SPJQuadrupoleSymmetry
   type, public, bind(c) :: fdps_spj_quadrupole_symmetry
      real(kind=c_double) :: mass
      type(fdps_f64vec) :: pos
      type(fdps_f64mat) :: quad
   end type fdps_spj_quadrupole_symmetry

   !**** PS::SPJMonopoleCutoff
   type, public, bind(c) :: fdps_spj_monopole_cutoff
      real(kind=c_double) :: mass
      type(fdps_f64vec) :: pos
   end type fdps_spj_monopole_cutoff
   
end module fdps_super_particle   
\end{lstlisting}

\begin{lstlisting}[language=C,caption=Superparticle types (C),label=listing:superparticle_types_in_C]
#pragma once
#include "FDPS_basic.h"
#include "FDPS_vector.h"
#include "FDPS_matrix.h"

#ifdef PARTICLE_SIMULATOR_SPMOM_F32
typedef fdps_s32    fdps_sSP;
typedef fdps_f32    fdps_fSP;
typedef fdps_f32vec fdps_fSPvec;
typedef fdps_f32mat fdps_fSPmat;
#else
typedef fdps_s64    fdps_sSP;
typedef fdps_f64    fdps_fSP;
typedef fdps_f64vec fdps_fSPvec;
typedef fdps_f64mat fdps_fSPmat;
#endif

//**** PS::SPJMonopole
typedef struct {
   fdps_fSP mass;
   fdps_fSPvec pos;
} fdps_spj_monopole;

//**** PS::SPJQuadrupole
typedef struct {
   fdps_fSP mass;
   fdps_fSPvec pos;
   fdps_fSPmat quad;
} fdps_spj_quadrupole;

//**** PS::SPJMonopoleGeometricCenter
typedef struct {
   fdps_sSP n_ptcl;
   fdps_fSP charge;
   fdps_fSPvec pos;
} fdps_spj_monopole_geomcen;

//**** PS::SPJDipoleGeometricCenter
typedef struct {
   fdps_sSP n_ptcl;
   fdps_fSP charge;
   fdps_fSPvec pos;
   fdps_fSPvec dipole;
} fdps_spj_dipole_geomcen;

//**** PS::SPJQuadrupoleGeometricCenter
typedef struct {
   fdps_sSP n_ptcl;
   fdps_fSP charge;
   fdps_fSPvec pos;
   fdps_fSPvec dipole;
   fdps_fSPmat quadrupole;
} fdps_spj_quadrupole_geomcen;

//**** PS::SPJMonopoleScatter
typedef struct {
   fdps_fSP mass;
   fdps_fSPvec pos;
} fdps_spj_monopole_scatter;

//**** PS::SPJQuadrupoleScatter
typedef struct {
   fdps_fSP mass;
   fdps_fSPvec pos;
   fdps_fSPmat quad;
} fdps_spj_quadrupole_scatter;

//**** PS::SPJMonopoleSymmetry
typedef struct {
   fdps_fSP mass;
   fdps_fSPvec pos;
} fdps_spj_monopole_symmetry;

//**** PS::SPJQuadrupoleSymmetry
typedef struct {
   fdps_fSP mass;
   fdps_fSPvec pos;
   fdps_fSPmat quad;
} fdps_spj_quadrupole_symmetry;

//**** PS::SPJMonopoleCutoff
typedef struct {
   fdps_fSP mass;
   fdps_fSPvec pos;
} fdps_spj_monopole_cutoff;
\end{lstlisting}

There is an one-to-one correspondence between the superparticle types and the types of tree objects. Accordingly, users must use an appropriate superparticle type corresponding to the type of created tree object. The correspondence relationship between the types of tree objects and the superparticle types is shown in Table~\ref{tbl:tree_and_super_particle}. Note that the types of tree object for short-range force are not shown in this table, because superparticle is only used in the case of long-range force. For the other types of tree object and the way to create tree objects, see Chap.~\ref{chap:API_spec_list} \S~\ref{sec:tree_APIs}.

\begin{landscape}
\begin{table}[h]
\begin{tabularx}{\hsize}{lp{5cm}Xl}
\toprule
\rowcolor{Snow2}
Tree type & Highest-order of multipole moments$^{\dagger}$ & Range of interaction & Superparticle type \\
\midrule
Long-Monopole type & monopole(CoM) & Entire region & \path{fdps_spj_monopole} \\
\midrule
Long-Quadrupole type & quadrupole(CoM) & Entire region & \path{fdps_spj_quadrupole} \\
\midrule
Long-MonopoleGeometricCenter type & monopole(GC) & Entire region & \path{fdps_spj_monopole_geomcen} \\
\midrule
Long-DipoleGeometricCenter type & dipole(GC) & Entire region & \path{fdps_spj_dipole_geomcen} \\
\midrule
Long-QuadrupoleGeometricCenter type & quadrupole(GC) & Entire region & \path{fdps_spj_quadrupole_geomcen} \\
\midrule
Long-MonopoleWithScatterSearch type$^{\ddagger}$ & monopole(CoM) & Entire region & \path{fdps_spj_monopole_scatter} \\
\midrule
Long-QuadrupoleWithScatterSearch type$^{\ddagger}$ & quadrupole(CoM) & Entire region & \path{fdps_spj_quadrupole_scatter} \\
\midrule
Long-MonopoleWithSymmetrySearch type$^{\ddagger}$ & monopole(CoM) & Entire region & \path{fdps_spj_monopole_symmetry} \\
\midrule
Long-QuadrupoleWithSymmetrySearch type$^{\ddagger}$ & quadrupole(CoM) & Entire region & \path{fdps_spj_quadrupole_symmetry} \\
\midrule
Long-MonopoleWithCutoff type & monopole(CoM) & Within cutoff radius & \path{fdps_spj_monopole_cutoff} \\
\bottomrule
\end{tabularx}
\begin{flushleft}
$^{\dagger}$ CoM indicates that multipole moments are calculated assuming that the center-of-mass is the center of the expansion. GC indicates that multipole moments are calculated assuming that the geometric center is the center of the expansion. \\
$^{\ddagger}$ Scatter- or Symmetry-mode neighbor search is possible. In the calculation of interaction, neighbor particles are forcibly treated as normal particles. In other words, superparticles do not have neighbor particles as members.
\end{flushleft}
\caption{The correspondence relationship between the types of tree for long-range force and the superparticle types}
\label{tbl:tree_and_super_particle}
\end{table}
\end{landscape}


%%%%%%%%%%%%%%%%%%%%%%%%%%%%%%%%%%%%%%%%%%%%%%%%%%%%%%
\section{Time profile types}
\label{sec:time_profile_types}
Time profile types are used to obtain the elapsed times of various types of calculations performed in FDPS. At present, there is only one time profile type \path{fdps_time_profile}. It is defined in the files \path{src/fortran_interface/modules/FDPS_time_profile.F90} and \path{src/c_interface/headers/FDPS_time_profile.h} as follows (Listings~\ref{listing:time_profile_types} and \ref{listing:time_profile_types_in_C}). This data type is exclusively used by the APIs for time measurement (for detail, see Chap.~\ref{chap:API_spec_list}).

\begin{lstlisting}[caption=Time profile types (Fortran),label=listing:time_profile_types]
module fdps_time_profile
   use, intrinsic :: iso_c_binding
   implicit none

   !**** PS::TimeProfile
   type, public, bind(c) :: fdps_time_prof
      real(kind=c_double) :: collect_sample_particle
      real(kind=c_double) :: decompose_domain
      real(kind=c_double) :: exchange_particle
      real(kind=c_double) :: set_particle_local_tree
      real(kind=c_double) :: set_particle_global_tree
      real(kind=c_double) :: make_local_tree
      real(kind=c_double) :: make_global_tree
      real(kind=c_double) :: set_root_cell
      real(kind=c_double) :: calc_force
      real(kind=c_double) :: calc_moment_local_tree
      real(kind=c_double) :: calc_moment_global_tree
      real(kind=c_double) :: make_LET_1st
      real(kind=c_double) :: make_LET_2nd
      real(kind=c_double) :: exchange_LET_1st
      real(kind=c_double) :: exchange_LET_2nd
                                                                                                   
      real(kind=c_double) :: morton_sort_local_tree                                                
      real(kind=c_double) :: link_cell_local_tree                                                  
      real(kind=c_double) :: morton_sort_global_tree                                               
      real(kind=c_double) :: link_cell_global_tree                                                 
                                                                                                   
      real(kind=c_double) :: make_local_tree_tot                                                   
      ! = make_local_tree + calc_moment_local_tree                                                 
      real(kind=c_double) :: make_global_tree_tot                                                  
      real(kind=c_double) :: exchange_LET_tot                                                      
      ! = make_LET_1st + make_LET_2nd + exchange_LET_1st + exchange_LET_2nd                        
                                                                                                   
      real(kind=c_double) :: calc_force__core__walk_tree                                           
                                                                                                   
      real(kind=c_double) :: calc_force__make_ipgroup                                              
      real(kind=c_double) :: calc_force__core                                                      
      real(kind=c_double) :: calc_force__copy_original_order                                       
                                                                                                   
      real(kind=c_double) :: exchange_particle__find_particle                                       
      real(kind=c_double) :: exchange_particle__exchange_particle

      real(kind=c_double) :: decompose_domain__sort_particle_1st
      real(kind=c_double) :: decompose_domain__sort_particle_2nd
      real(kind=c_double) :: decompose_domain__sort_particle_3rd
      real(kind=c_double) :: decompose_domain__gather_particle

      real(kind=c_double) :: decompose_domain__setup
      real(kind=c_double) :: decompose_domain__determine_coord_1st
      real(kind=c_double) :: decompose_domain__migrae_particle_1st
      real(kind=c_double) :: decompose_domain__determine_coord_2nd
      real(kind=c_double) :: decompose_domain__determine_coord_3rd
      real(kind=c_double) :: decompose_domain__exchange_pos_domain

      real(kind=c_double) :: exchange_LET_1st__a2a_n
      real(kind=c_double) :: exchange_LET_1st__icomm_sp
      real(kind=c_double) :: exchange_LET_1st__a2a_sp
      real(kind=c_double) :: exchange_LET_1st__icomm_ep
      real(kind=c_double) :: exchange_LET_1st__a2a_ep
   end type fdps_time_prof

end module fdps_time_profile  
\end{lstlisting}

\begin{lstlisting}[language=C,caption=Time profile types (C),label=listing:time_profile_types_in_C]
//**** PS::TimeProfile
typedef struct {
   double collect_sample_particle;
   double decompose_domain;
   double exchange_particle;
   double set_particle_local_tree;
   double set_particle_global_tree;
   double make_local_tree;
   double make_global_tree;
   double set_root_cell;
   double calc_force;
   double calc_moment_local_tree;
   double calc_moment_global_tree;
   double make_LET_1st;
   double make_LET_2nd;
   double exchange_LET_1st;
   double exchange_LET_2nd;
   double write_back;

   double morton_sort_local_tree;
   double link_cell_local_tree;
   double morton_sort_global_tree;
   double link_cell_global_tree;

   double make_local_tree_tot; // = make_local_tree + calc_moment_local_tree
   double make_global_tree_tot;
   double exchange_LET_tot; // = make_LET_1st + make_LET_2nd + exchange_LET_1st + exchange_LET_2nd

   double calc_force__core__walk_tree;
   double calc_force__core__keep_list;
   double calc_force__core__copy_ep;
   double calc_force__core__dispatch;
   double calc_force__core__retrieve;

   double calc_force__make_ipgroup;
   double calc_force__core;
   double calc_force__copy_original_order;

   double exchange_particle__find_particle;
   double exchange_particle__exchange_particle;
   
   double decompose_domain__sort_particle_1st;
   double decompose_domain__sort_particle_2nd;
   double decompose_domain__sort_particle_3rd;
   double decompose_domain__gather_particle;

   double decompose_domain__setup;
   double decompose_domain__determine_coord_1st;
   double decompose_domain__migrae_particle_1st;
   double decompose_domain__determine_coord_2nd;
   double decompose_domain__determine_coord_3rd;
   double decompose_domain__exchange_pos_domain;

   double exchange_LET_1st__a2a_n;
   double exchange_LET_1st__icomm_sp;
   double exchange_LET_1st__a2a_sp;
   double exchange_LET_1st__icomm_ep;
   double exchange_LET_1st__a2a_ep;

   double add_moment_as_sp_local;
   double add_moment_as_sp_global;
} fdps_time_prof;
\end{lstlisting}

%%%%%%%%%%%%%%%%%%%%%%%%%%%%%%%%%%%%%%%%%%%%%%%%%%%%%%
\section{Enumerated types}
\label{sec:enum_types}
In this section, we describe enumerated types defined in FDPS Fortran/C interface.

%%%%%%%%%%%%%%%%%%%%%%%%%%%
\subsection{Boundary condition types}
\label{subsec:enum_bc}
Boundary condition types are used in the APIs \path{set_boundary_condition} {\small (Fortran)} and \path{fdps_set_boundary_condition} {\small (C)} to specify the boundary condition of simulation (see \S~\ref{sec:dinfo_APIs} ``APIs for DomainInfo object" in Chap.~\ref{chap:API_spec_list}). It is defined in the files \path{FDPS_module.F90} and \path{src/c_interface/headers/FDPS_enum.h} as follows.

\begin{lstlisting}[caption=Boundary condition types (Fortran)]
module FDPS_module
   use, intrinsic :: iso_c_binding
   implicit none
   
   !* Enum types
   !**** PS::BOUNDARY_CONDITION
   enum, bind(c)
      enumerator :: fdps_bc_open
      enumerator :: fdps_bc_periodic_x
      enumerator :: fdps_bc_periodic_y
      enumerator :: fdps_bc_periodic_z
      enumerator :: fdps_bc_periodic_xy
      enumerator :: fdps_bc_periodic_xz
      enumerator :: fdps_bc_periodic_yz
      enumerator :: fdps_bc_periodic_xyz
      enumerator :: fdps_bc_shearing_box
      enumerator :: fdps_bc_user_defined
   end enum
   
end module FDPS_module
\end{lstlisting}

\begin{lstlisting}[language=C,caption=Boundary condition types (C)]
typedef enum {
   FDPS_BC_OPEN,
   FDPS_BC_PERIODIC_X,
   FDPS_BC_PERIODIC_Y,
   FDPS_BC_PERIODIC_Z,
   FDPS_BC_PERIODIC_XY,
   FDPS_BC_PERIODIC_XZ,
   FDPS_BC_PERIODIC_YZ,
   FDPS_BC_PERIODIC_XYZ,
   FDPS_BC_SHEARING_BOX,
   FDPS_BC_USER_DEFINED,
} FDPS_BOUNDARY_CONDITION;
\end{lstlisting}


Table~\ref{tbl:boundary_conditions} shows the relations between the enumerators of the boundary condition types in Fortran and boundary conditions. Users can obtain the same kind of the table for the boundary condition types in C if capitalizing the names of the enumerators in the table.

\begin{table}[H]
\begin{tabularx}{\linewidth}{cX}
\toprule
\rowcolor{Snow2}
Enumerator & Boundary condition \\
\midrule
\path{fdps_bc_open} &  The open boundary condition. \textbf{This is the default boundary condition in FDPS}. \\
\midrule
\path{fdps_bc_periodic_x} & The periodic boundary condition in the direction of $x$-axis, and the open boundary condition in other directions. FDPS assume that the lower and upper boundaries of the computational box (or the interval) along $x$-axis are closed and open, respectively (i.e. \texttt{[)}). This is assumed for all periodic boundary conditions. \\
\midrule
\path{fdps_bc_periodic_y} & The periodic boundary condition in the direction of $y$-axis, and the open boundary condition in other directions. \\
\midrule
\path{fdps_bc_periodic_z} & The periodic boundary condition in the direction of $z$-axis, and the open boundary condition in other directions. \\
\midrule
\path{fdps_bc_periodic_xy} & The periodic boundary condition in the directions of $x$- and $y$-axes, and the open boundary condition in the direction of $z$-axis. \\
\midrule
\path{fdps_bc_periodic_xz} & The periodic boundary condition in the directions of $x$- and $z$-axes, and the open boundary condition in the direction of $y$-axis. \\
\midrule
\path{fdps_bc_periodic_yz} & The periodic boundary condition in the directions of $y$- and $z$-axes, and the open boundary condition in the direction of $x$-axis. \\
\midrule
\path{fdps_bc_periodic_xyz} & The periodic boundary condition in all three directions. \\
\midrule
\path{fdps_bc_shearing_box} & The shearing-box boundary condition (\textcolor{red}{Not implemented yet}). \\
\midrule
\path{fdps_bc_user_defined} & User-defined boundary condition (\textcolor{red}{Not implemented yet}). \\
\bottomrule
\end{tabularx}
\caption{The correspondence relation between the enumerator of the boundary condition types and the boundary conditions}
\label{tbl:boundary_conditions}
\end{table}

%%%%%%%%%%%%%%%%%%%%%%%%%%%
\subsection{Interaction list mode types}
\label{subsec:enum_list_mode}
Interaction list mode types are used to determine whether we reuse interaction lists at interaction calculations. These types are used as an argument in the APIs \path{calc_force_all_and_write_back} and \path{calc_force_all} in Fortran or \path{fdps_calc_force_all_and_write_back} and \path{fdps_calc_force_all} in C (see \S~\ref{sec:tree_APIs} `APIs for Tree object`" in Chap.~\ref{chap:API_spec_list}) and are defined in the files \path{FDPS_module.F90} and \path{src/c_interface/headers/FDPS_enum.h} as follows.

\begin{lstlisting}[caption=Interaction list mode types (Fortran)]
module FDPS_module
   use, intrinsic :: iso_c_binding
   implicit none
   
   !* Enum types
   !**** PS::INTERACTION_LIST_MODE
   enum, bind(c)
      enumerator :: fdps_make_list
      enumerator :: fdps_make_list_for_reuse
      enumerator :: fdps_reuse_list
   end enum
   
end module FDPS_module
\end{lstlisting}

\begin{lstlisting}[language=C,caption=Interaction list mode types (C)]
typedef enum {
   FDPS_MAKE_LIST,
   FDPS_MAKE_LIST_FOR_REUSE,
   FDPS_REUSE_LIST,
} FDPS_INTERACTION_LIST_MODE;  
\end{lstlisting}

Table \ref{tbl:interaction_list_mode} shows the relations between the enumerators of the interaction list mode types in Fortran and operation mode of the APIs described above. Users can obtain the same kind of the table for the interaction list mode types in C if capitalizing the name of the enumerators in the table.

\begin{table}[h]
\begin{tabularx}{\linewidth}{cX}
\toprule
\rowcolor{Snow2}
Enumerator & Operation mode \\
\midrule
\path{fdps_make_list} & FDPS (re)makes interaction lists for each interaction calculation (each call of the APIs described above). In this case, we cannot reuse interaction list in the next interaction calculation because FDPS does not store the information of interaction list. \textbf{This is the default operation mode in FDPS}.\\
\midrule
\path{fdps_make_list_for_reuse} & FDPS (re)makes interaction lists and stores them internally. Then, it performs interaction calculation. In this case, we can reuse these interaction lists in the next interaction calculation if we call the APIs with the flag \path{fdps_reuse_list}. The interaction lists memorized in FDPS are destroyed if we perform the interaction calculation with the flags \path{fdps_make_list_for_reuse} or \path{fdps_make_list}.\\
\midrule
\path{fdps_reuse_list} & FDPS performs interaction calculation using the previously-created interaction lists, which are the lists that are created at the previous call of the APIs with the flag \path{fdps_make_list_for_reuse}. In this case, moment information in superparticles are automatically updated using the latest particle information.\\
\bottomrule
\end{tabularx}
\caption{The correspondence relation between the enumerator of the interaction list mode types and the operation modes}
\label{tbl:interaction_list_mode}
\end{table}



