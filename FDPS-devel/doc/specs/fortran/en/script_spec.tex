In this chapter, we describe the system requirements and the usage of the Fortran interface-generating script.

%%%%%%%%%%%%%%%%%%%%%%%%%%%%%%%%%%%%%%%%%%%%%%%%%%%%%
\section{System requirements for the script}
This section describes the system requirements for the interface-generating scripts \path{gen_ftn_if.py} (for generation of Fortran interface) and \path{gen_c_if.py} (for generation of C interface). These scripts are placed in the directory \path{scripts} and are implemented by the programing language \textsc{Python}. In order for the scripts to work, \textsc{Python} 2.7.5 or later, or, \textsc{Python} 3.4 or later must be installed in your system. Before using the scripts, please modify the following first line of the scripts according to your computational environment:
\begin{Verbatim}[commandchars=\\\{\}]
#!/usr/bin/env python
\end{Verbatim}
Things to be checked are the PATH of the \texttt{env} command and the name of a \textsc{Python} interpreter (there is a case where the name of an interpreter is not \texttt{python}, but \texttt{python2.7} or \texttt{python3.4} [the numbers indicate the versions of the interpreters], depending on the system). If a \textsc{Python} interpreter is not in the environment variable \texttt{PATH}, please add the PATH of the interpreter to  \texttt{PATH} or specify the interpreter by its absolute PATH as follows:
\begin{Verbatim}[commandchars=\\\{\}]
#!/path/to/python
\end{Verbatim}

In addition, user's codes must satisfy the following conditions for the script to work:
\begin{itemize}[leftmargin=*,itemsep=-1ex]
\item All the user's codes input to the script \path{gen_ftn_if.py} must be written by Fortran 2003 standard (ISO/IEC 1539-1:2004(E)). This script does not have function to identify the programing language used in user's codes and an advanced syntax checker. Hence, the script might show an unexpected behavior if there are syntax errors in users codes.
\item \item All the user's codes input to the script \path{gen_c_if.py} must be written by C99 (ISO/IEC9899:1999(E)). This script does not have function to identify the programing language used in user's codes and an advanced syntax checker. Hence, the script might show an unexpected behavior if there are syntax errors in users codes.
\end{itemize}

%%%%%%%%%%%%%%%%%%%%%%%%%%%%%%%%%%%%%%%%%%%%%%%%%%%%%
\section{Usage of the script}
To generate FDPS Fortran interface programs, run the script \path{gen_ftn_if.py} as follows.
\begin{screen}
\begin{Verbatim}[commandchars=\\\{\}]
\$ gen_ftn_if.py -o output_directory user1.F90 user2.F90... 
\end{Verbatim}
\end{screen}
To generate FDPS C interface programs, run the script \path{gen_c_if.py} as follows.
\begin{screen}
\begin{Verbatim}[commandchars=\\\{\}]
\$ gen_c_if.py -o output_directory user1.c user2.c... 
\end{Verbatim}
\end{screen}
where we assumed that the directory \path{scripts} is added to the environment variable \path{PATH}. Otherwise, you must run the script by its relative PATH or its absolute PATH. The script \path{gen_ftn_if.py} accepts Fortran source codes as arguments and it requires that the user-defined types must be defined in one of the input files. Similarly, the script \path{gen_c_if.py} accepts C source codes as arguments and it requires that the user-defined types must be defined in one of the input files. If the number of the input files is more than 1, you must separate them by at least one space in no particular order.

The option \path{-o} can be used to specify the directory where the interface programs are output. You can use the options \path{--output} or \path{--output_dir} instead of \path{-o}. If not specified, the interface programs are output in the current directory.

The option \path{-DPARTICLE_SIMULATOR_TWO_DIMENSION} is used to indicate that the spatial dimension of the simulation is 2. This option does not have an argument. If this option is not specified, the script assume that the spatial dimension of the simulation is 3. The spatial dimension of the simulation is used to check the data types of the member variables that correspond to position and velocity. If the macro \path{PARTICLE_SIMULATOR_TWO_DIMENSION} is defined at the compilation of user's codes, you must specify this option when generating the interface programs (otherwise, the interface programs do not work as expected).

The usage of the script can be checked by executing the script with the options \path{-h} or \path{--help}. The following is an example for the script \path{gen_ftn_if.py}.
\begin{screen}
\begin{spverbatim}
[user@hostname somedir]$ gen_ftn_if.py -h
[namekata@jenever0 scripts]$ ./gen_ftn_if.py --help
usage: gen_ftn_if.py [-h] [-o DIRECTORY] [-DPARTICLE_SIMULATOR_TWO_DIMENSION]
                     FILE [FILE ...]

Analyze user's Fortran codes and generate C++/Fortran source files required to use FDPS from the user's Fortran code.

positional arguments:
  FILE                  The PATHs of input Fortran files

optional arguments:
  -h, --help            show this help message and exit
  -o DIRECTORY, --output DIRECTORY, --output_dir DIRECTORY
                        The PATH of output directory
  -DPARTICLE_SIMULATOR_TWO_DIMENSION
                        Indicate that simulation is performed
                        in the 2-dimensional space (equivalent
                        to define the macro 
                        PARTICLE_SIMULATOR_TWO_DIMENSION)
\end{spverbatim}
\end{screen}

You can obtain the execution file by compiling interface programs generated and user's codes. Next chapter (Chap.~\ref{chap:compile_and_macro}) explain how to compile your codes.

