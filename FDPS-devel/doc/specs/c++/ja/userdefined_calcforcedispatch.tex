
\subsubsection{概要}

関数calcForceDispatchは関数calcForceRetrieveと合わせて粒子同士の相互作
用を記述するものであり、calcForceSpEp やcalcForceEpEp の代わりに相互作
用の定義(節\ref{sec:overview_action}の手順0)に使うことができる。
calcForceSpEp やcalcForceEpEp との違いは、calcForceDispatch は複数の相
互作用リストと i粒子リストを受け取ることである。これにより、GPGPU 等の
アクセラレータを起動する回数を削減し、実行効率を向上させる。以下、これ
の書き方の規定を記述する。関数calcForceDispatchの名前はGravityDispatch
とする。これは変更自由である。また、EssentialParitlceIクラスのクラス名
をEPI, EssentialParitlceJクラスのクラス名をEPJ, SuperParitlceJクラスの
クラス名をSPJとする。

\subsubsection{短距離力の場合}

\begin{lstlisting}[caption=calcForceDispatch]
class EPI;
class EPJ;
PS::S32 HydroforceDispatch(const PS::S32  tag,
                           const PS::S32  nwalk,
                           const EPI**      epi,
                           const PS::S32*  ni,
                           const EPJ**      epj,
                           const PS::S32*  nj_ep;
};
\end{lstlisting}

\begin{itemize}

\item {\bf 引数}

  tag: 入力。const PS::S32 型。tagの番号。発行されるtagの番号は\\
  0から関数PS::TreeForForce::calcForceAllandWriteBackMultiWalk()の\\
  第三引数として設定された値から1引いた数までである。\\
  tagの番号はcalcForceRetrieve()で設定するtagの番号と対応させる必要がある。
    

  nwalk: 入力。const PS::S32 型。walkの数。walkの数の最大値は\\
  PS::TreeForForce::calcForceAllandWriteBackMultiWalk()の第六引数の値である。

  epi: 入力。const EPI** 型。i粒子情報を持つポインタのポインタ。

  ni: 入力。const PS::S32*型。i粒子数のポインタ。

  epj: 入力。const EPJ** 型。j粒子情報を持つポインタのポインタ。
  
  nj\_ep: 入力。const PS::S32* 型。j粒子数のポインタ。

\item {\bf 返値}

  PS::S32型。ユーザーは正常に実行された場合は0を、エラーが起こった場合
  は0以外の値を返すようにする。
  
\item {\bf 機能}

epi,epjの情報をアクセラレータに送り、相互作用カーネルを発行する。
  
\end{itemize}

\subsubsection{長距離力の場合}

\begin{lstlisting}[caption=calcForceDispatch]
class EPI;
class EPJ;
class SPJ;
PS::S32 GravityDispatch(const PS::S32   tag,
                        const PS::S32   nwalk,
                        const EPI**     epi,
                        const PS::S32*  ni,
                        const EPJ**     epj,
                        const PS::S32*  nj_ep,
                        const SPJ**     spj,
                        const PS::S32*  nj_sp);
};
\end{lstlisting}

\begin{itemize}

\item {\bf 引数}


  tag: 入力。const PS::S32 型。tagの番号。発行されるtagの番号は0から関
  数PS::TreeForForce::calcForceAllandWriteBackMultiWalk()の第三引数と
  して設定された値から1引いた数までである。tagの番号は
  calcForceRetrieve()で設定するtagの番号と対応させる必要がある。

  nwalk: 入力。const PS::S32 型。walkの数。walkの数の最大値は
  PS::TreeForForce::calcForceAllandWriteBackMultiWalk()の第六引数の値
  である。

  epi: 入力。const EPI** 型。i粒子情報を持つ配列の配列。

  ni: 入力。const PS::S32*型。i粒子数の配列。

  epj: 入力。const EPJ** 型。j粒子情報を持つ配列の配列。
  
  nj\_ep: 入力。const PS::S32* 型。j粒子数の配列。

  spj: 入力。const SPJ** 型。j粒子情報を持つ配列の配列。
  
  nj\_sp: 入力。const PS::S32* 型。j粒子数の配列。

\item {\bf 返値}

  PS::S32型。ユーザーは正常に実行された場合は0を、エラーが起こった場合
  は0以外の値を返すようにする。
  
\item {\bf 機能}

epi,epj,spjの情報をアクセラレータに送り、相互作用カーネルを発行する。
  
\end{itemize}

