本節では、通信用データクラスについて記述する。このクラスはノード間通信
のための情報の保持や実際の通信を行うモジュールである。このクラスはシン
グルトンパターンとして管理されており、オブジェクトの生成は必要としない。
ここではこのモジュールのAPIを記述する。

%%%%%%%%%%%%%%%%%%%%%%%%%%%%%%%%%%%%%%%%%%%%%%%%%%%%%%%%%%%%%%%%%
\subsubsubsection{API}

このモジュールのAPIの宣言は以下のようになっている。このあと各APIについ
て記述する。
\begin{lstlisting}[caption=Communication]
namespace ParticleSimulator {
    class Comm{
    public:
        static S32 getRank();
        static S32 getNumberOfProc();
        static S32 getRankMultiDim(const S32 id);
        static S32 getNumberOfProcMultiDim(const S32 id);
        static bool synchronizeConditionalBranchAND
                    (const bool local);
        static bool synchronizeConditionalBranchOR
                    (const bool local);
        template<class T>
        static T getMinValue(const T val);
        template<class Tfloat, class Tint>
        static void getMinValue(const Tfloat f_in,
                                const Tint i_in,
                                Tfloat & f_out,
                                Tint & i_out);
        template<class T>
        static T getMaxValue(const T val);
        template<class Tfloat, class Tint>
        static void getMaxValue(const Tfloat f_in,
                                const Tint i_in,
                                Tfloat & f_out,
                                Tint & i_out );
        template<class T>
        static T getSum(const T val);
        template<class T>
        static void broadcast(T * val,
                              const S32 n,
                              const S32 src=0);
    };
}
\end{lstlisting}

\subsubsubsubsection{PS::Comm::getRank}

\begin{screen}
\begin{verbatim}
static PS::S32 PS::Comm::getRank();
\end{verbatim}
\end{screen}

\begin{itemize}

\item{{\bf 引数}}

なし。

\item{{\bf 返り値}}

{PS::S32}型。全プロセス中でのランクを返す。

\end{itemize}

\subsubsubsubsection{PS::Comm::getNumberOfProc}

\begin{screen}
\begin{verbatim}
static PS::S32 PS::Comm::getNumberOfProc();
\end{verbatim}
\end{screen}

\begin{itemize}

\item{{\bf 引数}}

なし。

\item{{\bf 返り値}}

PS::S32型。全プロセス数を返す。

\end{itemize}

\subsubsubsubsection{PS::Comm::getRankMultiDim}

\begin{screen}
\begin{verbatim}
static PS::S32 PS::Comm::getRankMultiDim(const PS::S32 id);
\end{verbatim}
\end{screen}

\begin{itemize}

\item{{\bf 引数}}

id: 入力。const PS::S32型。軸の番号。x軸:0, y軸:1, z軸:2。

\item{{\bf 返り値}}

PS::S32型。id番目の軸でのランクを返す。2次元の場合、id=2は1を返す。

\end{itemize}

\subsubsubsubsection{PS::Comm::getNumberOfProcMultiDim}

\begin{screen}
\begin{verbatim}
static PS::S32 PS::Comm::getNumberOfProcMultiDim(const PS::S32 id);
\end{verbatim}
\end{screen}

\begin{itemize}

\item{{\bf 引数}}

id: 入力。const PS::S32型。軸の番号。x軸:0, y軸:1, z軸:2。

\item{{\bf 返り値}}

PS::S32型。id番目の軸のプロセス数を返す。2次元の場合、id=2は1を返す。

\end{itemize}

\subsubsubsubsection{PS::Comm::synchronizeConditionalBranchAND}

\begin{screen}
\begin{verbatim}
static bool PS::Comm::synchronizeConditionalBranchAND(const bool local)
\end{verbatim}
\end{screen}

\begin{itemize}

\item{{\bf 引数}}

local: 入力。const bool型。

\item{{\bf 返り値}}

bool型。全プロセスでlocalの論理積を取り、結果を返す。

\end{itemize}

\subsubsubsubsection{PS::Comm::synchronizeConditionalBranchOR}

\begin{screen}
\begin{verbatim}
static bool PS::Comm::synchronizeConditionalBranchOR(const bool local);
\end{verbatim}
\end{screen}

\begin{itemize}

\item{{\bf 引数}}

local: 入力。const bool型。

\item{{\bf 返り値}}

bool型。全プロセスでlocalの論理和を取り、結果を返す。

\end{itemize}

\subsubsubsubsection{PS::Comm::getMinValue}

\begin{screen}
\begin{verbatim}
template <class T>
static T PS::Comm::getMinValue(const T val);
\end{verbatim}
\end{screen}

\begin{itemize}

\item{{\bf 引数}}

val: 入力。const T型。

\item{{\bf 返り値}}

T型。全プロセスでvalの最小値を取り、結果を返す。

\end{itemize}

\begin{screen}
\begin{verbatim}
template <class Tfloat, class Tint>
static void PS::Comm::getMinValue(const Tfloat f_in,
                                  const Tint i_in,
                                  Tfloat & f_out,
                                  Tint & i_out);
\end{verbatim}
\end{screen}

\begin{itemize}

\item{{\bf 引数}}

f\_in: 入力。const Tfloat型。

i\_in: 入力。const Tint型。

f\_out: 出力。Tfloat型。全プロセスでf\_inの最小値を取
り、結果を返す。

i\_out: 出力。Tint型。f\_outに伴うIDを返す。

\item{{\bf 返り値}}

なし。

\end{itemize}

\subsubsubsubsection{PS::Comm::getMaxValue}

\begin{screen}
\begin{verbatim}
template <class T>
static T PS::Comm::getMaxValue(const T val);
\end{verbatim}
\end{screen}

\begin{itemize}

\item{{\bf 引数}}

val: 入力。const T型。

\item{{\bf 返り値}}

T型。全プロセスでvalの最大値を取り、結果を返す。

\end{itemize}

\begin{screen}
\begin{verbatim}
template <class Tfloat, class Tint>
static void PS::Comm::getMaxValue(const Tfloat f_in,
                                  const Tint i_in,
                                  Tfloat & f_out,
                                  Tint & i_out);
\end{verbatim}
\end{screen}

\begin{itemize}

\item{{\bf 引数}}

f\_in: 入力。const Tfloat型。

i\_in: 入力。{const Tint}型。

{f\_out}: 出力。{Tfloat}型。全プロセスで{f\_in}の最大値を取
り、結果を返す。

{i\_out}: 出力。{Tint}型。{f\_out}に伴うIDを返す。

\item{{\bf 返り値}}

なし。

\end{itemize}

\subsubsubsubsection{PS::Comm::getSum}

\begin{screen}
\begin{verbatim}
template <class T>
static T PS::Comm::getSum(const T val);
\end{verbatim}
\end{screen}

\begin{itemize}

\item{{\bf 引数}}

{val}: 入力。{const T}型。

\item{{\bf 返り値}}

{T}型。全プロセスで{val}の総和を取り、結果を返す。

\end{itemize}

\subsubsubsubsection{PS::Comm::broadcast}

\begin{screen}
\begin{verbatim}
template <class T>
static void PS::Comm::broadcast(T * val,
                                const PS::S32 n,
                                const PS::S32 src=0);
\end{verbatim}
\end{screen}

\begin{itemize}

\item{{\bf 引数}}

val: 入力。T *型。

n: 入力。const PS::S32型。T型変数の数。

src: 入力。const PS::S32型。放送するプロセスランク。デフォルトのランク
は0。

\item{{\bf 返り値}}

なし。

\item{{\bf 機能}}

プロセスランクsrcのプロセスがn個のT型変数を全プロセスに放送する。

\end{itemize}
