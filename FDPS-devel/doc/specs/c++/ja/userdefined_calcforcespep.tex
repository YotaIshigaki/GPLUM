\subsubsection{概要}

関数オブジェクトcalcForceSpEpは超粒子から粒子への作用を記述するもので
あり、相互作用の定義(節\ref{sec:overview_action}の手順0)に必要となる。
以下、これの書き方の規定を記述する。

\subsubsection{前提}

ここで示すのは重力N体シミュレーションにおける超粒子から粒子への作用の
記述の仕方である。超粒子は単極子までの情報で作られているものとする。関
数オブジェクトcalcForceSpEpの名前はgravitySpEpとする。これは変更自由で
ある。また、EssentialParitlceIクラスのクラス名をEPI, SuperParitlceJク
ラスのクラス名をSPJ, Forceクラスのクラス名をResultとする。

\subsubsection{gravitySpEp::operator ()}

\if 0
\begin{lstlisting}[caption=calcForceSpEp]
class Result {
public:
    PS::F32vec accfromspj;
};
class EPI {
public:
    PS::S32    id;
    PS::F32vec pos;
};
class SPJ {
public:
    PS::F32    mass;
    PS::F32vec pos;
};
struct gravitySpEp {
    static PS::F32 eps2;
    void operator () (const EPI *epi,
                      const PS::S32 ni,
                      const SPJ *spj,
                      const PS::S32 nj,
                      Result *result) {
                      
        for(PS::S32 i = 0; i < ni; i++) {
            PS::F32vec xi = epi[i].pos;
            PS::F32vec ai = 0.0;
            for(PS::S32 j = 0; j < nj; j++) {
                PS::F32    mj = spj[j].mass;
                PS::F32vec xj = spj[j].pos;

                PS::F32vec dx   = xi - xj;
                PS::F32    r2   = dx * dx + eps2;
                PS::F32    rinv = 1. / sqrt(r2);

                ai += mj * rinv * rinv * rinv * dx;
            }
            result.accfromspj = ai;
        }
    }
};
PS::F32 gravitySpEp::eps2 = 9.765625e-4;
\end{lstlisting}

\begin{itemize}

\item {\bf 前提}

  クラスResult, EPI, SPJに必要なメンバ関数は省略した。クラスResultのメ
  ンバ変数accfromspjはi粒子が超粒子から受ける重力加速度である。クラス
  EPIとSPJのメンバ変数posはそれぞれの粒子位置である。クラスSPJのメンバ
  変数massは超粒子の質量である。ファンクタgravitySpEpのメンバ変数eps2
  は重力ソフトニングの2乗である。ここの外側でスレッド並列になっている
  ため、ここでOpenMPを記述する必要はない。

\item {\bf 引数}

  epi: 入力。const EPI *型またはEPI *型。i粒子情報を持つ配列。

  ni: 入力。const PS::S32型またはPS::S32型。i粒子数。

  spj: 入力。const SPJ *型またはSPJ *型。超粒子情報を持つ配列。
  
  nj: 入力。const PS::S32型またはPS::S32型。超粒子数。

  result: 出力。Result *型。i粒子の相互作用結果を返す配列。

\item {\bf 返値}

  なし。
  
\item {\bf 機能}

  超粒子からi粒子への作用を計算する。
  
\item {\bf 備考}

  引数名すべて変更可能。関数オブジェクトの内容などはすべて変更可能。
  
\end{itemize}
\fi

\begin{lstlisting}[caption=calcForceSpEp]
class Result;
class EPI;
class SPJ;
struct gravitySpEp {
    void operator () (const EPI *epi,
                      const PS::S32 ni,
                      const SPJ *spj,
                      const PS::S32 nj,
                      Result *result);
};
\end{lstlisting}

\begin{itemize}

\item {\bf 引数}

  epi: 入力。const EPI *型またはEPI *型。i粒子情報を持つ配列。

  ni: 入力。const PS::S32型またはPS::S32型。i粒子数。

  spj: 入力。const SPJ *型またはSPJ *型。超粒子情報を持つ配列。
  
  nj: 入力。const PS::S32型またはPS::S32型。超粒子数。

  result: 出力。Result *型。i粒子の相互作用結果を返す配列。

\item {\bf 返値}

  なし。
  
\item {\bf 機能}

  超粒子からi粒子への作用を計算する。
  
\end{itemize}
