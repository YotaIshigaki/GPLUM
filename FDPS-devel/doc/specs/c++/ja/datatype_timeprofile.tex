\subsubsection{概要}

本節では、PS::TimeProfile型について記述する。PS::TimeProfile型はFDPSで
使われる3つのクラス、領域分割クラス、粒子群クラス、相互作用ツリークラ
ス、各メソッドの計算時間を格納するクラスである。これら3つのクラスには
{\tt PS::TimeProfile getTimeProfile()}というメソッドが存在し、このメソッ
ドをつかって、ユーザーは各メソッドの計算時間を取得出来る。

このクラスは以下のように記述されている。

\begin{lstlisting}[caption=TimeProfile]
namespace ParticleSimulator{
    class TimeProfile{
    publid:
        F64 collect_sample_particle;
        F64 decompose_domain;
        F64 exchange_particle;
        F64 make_local_tree;
        F64 make_global_tree;
        F64 calc_force;
        F64 calc_moment_local_tree;
        F64 calc_moment_global_tree;
        F64 make_LET_1st;
        F64 make_LET_2nd;        
        F64 exchange_LET_1st;
        F64 exchange_LET_2nd;
    };
}    
\end{lstlisting}

%%%%%%%%%%%%%%%%%%%%%%%%%%%%%
\subsubsubsection{加算}
\mbox{}
%%%%%%%%%%%%%%%%%%%%%%%%%%%%%

%%%%%%%%%%%%%%%%%%%%%%%%%%%%%
\begin{screen}
\begin{verbatim}
PS::TimeProfile PS::TimeProfile::operator + 
               (const PS::TimeProfile & rhs) const;
\end{verbatim}
\end{screen}

\begin{itemize}

\item{{\bf 引数}}

{rhs}: 入力。{const TimeProfile \&}型。

\item{{\bf 返り値}}

{PS::TimeProfile}型。{rhs}のすべてのメンバ変数の値と自身のメンバ変数の
値の和を取った値を返す。

\end{itemize}

%%%%%%%%%%%%%%%%%%%%%%%%%%%%%
\subsubsubsection{縮約}
\mbox{}
%%%%%%%%%%%%%%%%%%%%%%%%%%%%%

%%%%%%%%%%%%%%%%%%%%%%%%%%%%%
\begin{screen}
\begin{verbatim}
PS::F64 PS::TimeProfile::getTotalTime() const;
\end{verbatim}
\end{screen}

\begin{itemize}

\item{{\bf 引数}}

なし。

\item{{\bf 返り値}}

{PS::F64}型。すべてのメンバ変数の値の和を返す。

\end{itemize}

%%%%%%%%%%%%%%%%%%%%%%%%%%%%%
\subsubsubsection{初期化}
\mbox{}
%%%%%%%%%%%%%%%%%%%%%%%%%%%%%

%%%%%%%%%%%%%%%%%%%%%%%%%%%%%
\begin{screen}
\begin{verbatim}
void PS::TimeProfile::clear();
\end{verbatim}
\end{screen}

\begin{itemize}

\item{{\bf 引数}}

なし。

\item{{\bf 返り値}}

なし。

\item{{\bf 機能}}

すべてのメンバ変数に0を代入する。

\end{itemize}
