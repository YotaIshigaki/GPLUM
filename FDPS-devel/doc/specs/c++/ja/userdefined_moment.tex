\subsubsection{概要}

Momentクラスは近い粒子同士でまとまった複数の粒子のモーメント情報を持つ
クラスであり、相互作用の定義(節\ref{sec:overview_action}の手順0)に必要
となる。モーメント情報の例としては、複数粒子の単極子や双極子、さらにこ
れら粒子の持つ最大の大きさなど様々なものが考えられる。このクラスは、
EssentialParticleJクラスからSuperParticleJクラスを作るための中間変数の
ような役割を果す。従って、このクラスが持つメンバ関数は、
EssentialParticleJクラスから情報を読み出してモーメントを計算するメンバ
関数、少ない数の粒子のモーメントからそれらの粒子を含むより多くの粒子の
モーメントを計算するメンバ関数などがある。

このようなモーメント情報にはある程度決っているものが多いので、それらに
ついてはFDPS側で用意した。これら既存のクラスについてまず記述する。その
後にユーザーがモーメントクラスを自作する際に必ず必要なメンバ関数、場合
によっては必要になるメンバ関数について記述する。

%%%%%%%%%%%%%%%%%%%%%%%%%%%%%%%%%%%%%%%%%%%%%%%%%%%%%%
\subsubsection{既存のクラス}

\subsubsubsection{概要}

FDPSはいくつかのMomentクラスを用意している。これらは相互作用ツリークラ
スで特定のPS::SEARCH\_MODE型を選んだ場合に有効である。以下、各
PS::SEARCH\_MODE型において選ぶことのできるMoment型を記述する。
PS::SEARCH\_MODE\_GATHER, \\PS::SEARCH\_MODE\_SCATTER,
PS::SEARCH\_MODE\_SYMMETRYについてはMomentクラスを意識してコーディング
する必要がないので、これらについては記述しない。

\subsubsubsection{PS::SEARCH\_MODE\_LONG}

\subsubsubsubsection{PS::MomentMonopole}
\label{sec:MomentMonopole}

単極子までを情報として持つクラス。単極子を計算する際の座標系の中心には
粒子の重心や粒子電荷の重心を取る。以下、このクラスの概要を記述する。
\begin{screen}
\begin{verbatim}
namespace ParticleSimulator {
    class MomentMonopole {
    public:
        F64    mass;
        F64vec pos;
    };
}
\end{verbatim}
\end{screen}

\begin{itemize}
\item クラス名
  PS::MomentMonopole

\item メンバ変数とその情報

  mass: 近傍でまとめた粒子の全質量、または全電荷

  pos: 近傍でまとめた粒子の重心、または粒子電荷の重心

\item 使用条件

  EssentialParticleJクラス(節\ref{sec:essentialparticlej})がメンバ関数
  EssentialParticleJ::getChargeとEssentialParticleJ::getPosを持ち、そ
  れぞれが粒子質量(または粒子電荷)、粒子位置を返すこと。
  EssentialParticleJクラスのクラス名は変更自由。

\end{itemize}

\subsubsubsubsection{PS::MomentQuadrupole}
\label{sec:MomentQuadrupole}

単極子と四重極子を情報として持つクラス。これらのモーメントを計算する際
の座標系の中心には粒子の重心を取る。以下、このクラスの概要を記述する。
\begin{screen}
\begin{verbatim}
namespace ParticleSimulator {
    class MomentQuadrupole {
    public:
        F64    mass;    
        F64vec pos;
        F64mat quad;
    };
}
\end{verbatim}
\end{screen}

\begin{itemize}
\item クラス名
  PS::MomentQuadrupole

\item メンバ変数とその情報

  mass: 近傍でまとめた粒子の全質量

  pos: 近傍でまとめた粒子の重心

  quad: 近傍でまとめた粒子の四重極子

\item 使用条件

  EssentialParticleJクラス(節\ref{sec:essentialparticlej})がメンバ関数
  EssentialParticleJ::getChargeとEssentialParticleJ::getPosを持ち、そ
  れぞれが粒子質量(または粒子電荷)、粒子位置を返すこと。
  EssentialParticleJクラスのクラス名は変更自由。

\end{itemize}

\subsubsubsubsection{PS::MomentMonopoleGeometricCenter}
\label{sec:MomentMonopoleGeometricCenter}

単極子までを情報として持つクラス。これらのモーメントを計算する際の座標
系の中心には粒子の幾何中心を取る。以下、このクラスの概要を記述する。
\begin{screen}
\begin{verbatim}
namespace ParticleSimulator {
    class MomentMonopoleGeometricCenter {
    public:
        F64    charge;    
        F64vec pos;
    };
}
\end{verbatim}
\end{screen}

\begin{itemize}
\item クラス名
  PS::MomentMonopoleGeometricCenter

\item メンバ変数とその情報

  charge: 近傍でまとめた粒子の全質量、または全電荷

  pos: 近傍でまとめた粒子の幾何中心

\item 使用条件

  EssentialParticleJクラス(節\ref{sec:essentialparticlej})がメンバ関数
  EssentialParticleJ::getChargeとEssentialParticleJ::getPosを持ち、そ
  れぞれが粒子質量(または粒子電荷)、粒子位置を返すこと。
  EssentialParticleJクラスのクラス名は変更自由。

\end{itemize}

\subsubsubsubsection{PS::MomentDipoleGeometricCenter}
\label{sec:MomentDipoleGeometricCenter}

双極子までを情報として持つクラス。これらのモーメントを計算する際の座標
系の中心には粒子の幾何中心を取る。以下、このクラスの概要を記述する。
\begin{screen}
\begin{verbatim}
namespace ParticleSimulator {
    class MomentDipoleGeometricCenter {
    public:
        F64    charge;    
        F64vec pos;
        F64vec dipole;
    };
}
\end{verbatim}
\end{screen}

\begin{itemize}
\item クラス名
  PS::MomentDipoleGeometricCenter

\item メンバ変数とその情報

  charge: 近傍でまとめた粒子の全質量、または全電荷

  pos: 近傍でまとめた粒子の幾何中心

  dipole: 粒子の質量または電荷の双極子

\item 使用条件

  EssentialParticleJクラス(節\ref{sec:essentialparticlej})がメンバ関数
  EssentialParticleJ::getChargeとEssentialParticleJ::getPosを持ち、そ
  れぞれが粒子質量(または粒子電荷)、粒子位置を返すこと。
  EssentialParticleJクラスのクラス名は変更自由。

\end{itemize}

\subsubsubsubsection{PS::MomentQuadrupoleGeometricCenter}
\label{sec:MomentQuadrupoleGeometricCenter}

四重極子までを情報として持つクラス。これらのモーメントを計算する際の座標
系の中心には粒子の幾何中心を取る。以下、このクラスの概要を記述する。
\begin{screen}
\begin{verbatim}
namespace ParticleSimulator {
    class MomentQuadrupoleGeometricCenter {
    public:
        F64    charge;    
        F64vec pos;
        F64vec dipole;
        F64mat quadrupole;
    };
}
\end{verbatim}
\end{screen}

\begin{itemize}
\item クラス名
  PS::MomentQuadrupoleGeometricCenter

\item メンバ変数とその情報

  charge: 近傍でまとめた粒子の全質量、または全電荷

  pos: 近傍でまとめた粒子の幾何中心

  dipole: 粒子の質量または電荷の双極子

  quadrupole: 粒子の質量または電荷の四重極子

\item 使用条件

  EssentialParticleJクラス(節\ref{sec:essentialparticlej})がメンバ関数
  EssentialParticleJ::getChargeとEssentialParticleJ::getPosを持ち、そ
  れぞれが粒子質量(または粒子電荷)、粒子位置を返すこと。
  EssentialParticleJクラスのクラス名は変更自由。

\end{itemize}

\subsubsubsection{PS::SEARCH\_MODE\_LONG\_SCATTER}

\subsubsubsubsection{PS::MomentMonopoleScatter}
\label{sec:MomentMonopoleScatter}

単極子までを情報として持つクラス。単極子を計算する際の座標系の中心には
粒子の重心や粒子電荷の重心を取る。以下、このクラスの概要を記述する。
\begin{screen}
\begin{verbatim}
namespace ParticleSimulator {
    class MomentMonopoleScatter {
    public:
        F64    mass;
        F64vec pos;
        F64ort vertex_out_;
        F64ort vertex_in_;
    };
}
\end{verbatim}
\end{screen}

\begin{itemize}
\item クラス名
  PS::MomentMonopoleScatter

\item メンバ変数とその情報

  mass: 近傍でまとめた粒子の全質量、または全電荷

  pos: 近傍でまとめた粒子の重心、または粒子電荷の重心
  
  vertex\_out\_: 粒子をEssentialParticleJ::getRSearchの返り値を半径とする球とみなしたとき、近傍でまとめた粒子すべてを包含する最小の直方体の座標情報
  
  vertex\_in\_: 近傍でまとめた粒子すべての座標を包含する最小の直方体の座標情報

\item 使用条件

  EssentialParticleJクラス(節\ref{sec:essentialparticlej})がメンバ関数
  EssentialParticleJ::getCharge, EssentialParticleJ::getPos,
  EssentialParticleJ::getRSearchを持ち、それぞれが粒子質量(または粒子
  電荷)、粒子位置、粒子の力の到達距離を返すこと。EssentialParticleJク
  ラスのクラス名は変更自由。

\end{itemize}

\subsubsubsubsection{PS::MomentQuadrupoleScatter}
\label{sec:MomentQuadrupoleScatter}

単極子と四重極子を情報として持つクラス。単極子を計算する際の座標系の中心には
粒子の重心や粒子電荷の重心を取る。以下、このクラスの概要を記述する。
\begin{screen}
\begin{verbatim}
namespace ParticleSimulator {
    class MomentQuadrupoleScatter {
    public:
        F64    mass;
        F64vec pos;
        F64mat quad;
        F64ort vertex_out_;
        F64ort vertex_in_;
    };
}
\end{verbatim}
\end{screen}

\begin{itemize}
\item クラス名
  PS::MomentQuadrupoleScatter

\item メンバ変数とその情報

  mass: 近傍でまとめた粒子の全質量、または全電荷

  pos: 近傍でまとめた粒子の重心、または粒子電荷の重心
  
  quad: 近傍でまとめた粒子の四重極子
  
  vertex\_out\_: 粒子をEssentialParticleJ::getRSearchの返り値を半径とする球とみなしたとき、近傍でまとめた粒子すべてを包含する最小の直方体の座標情報
  
  vertex\_in\_: 近傍でまとめた粒子すべての座標を包含する最小の直方体の座標情報

\item 使用条件

  EssentialParticleJクラス(節\ref{sec:essentialparticlej})がメンバ関数
  EssentialParticleJ::getCharge, EssentialParticleJ::getPos,
  EssentialParticleJ::getRSearchを持ち、それぞれが粒子質量(または粒子
  電荷)、粒子位置、粒子の力の到達距離を返すこと。EssentialParticleJク
  ラスのクラス名は変更自由。

\end{itemize}

\subsubsubsection{PS::SEARCH\_MODE\_LONG\_SYMMETRY}

\subsubsubsubsection{PS::MomentMonopoleSymmetry}
\label{sec:MomentMonopoleSymmetry}

単極子までを情報として持つクラス。単極子を計算する際の座標系の中心には
粒子の重心や粒子電荷の重心を取る。以下、このクラスの概要を記述する。
\begin{screen}
\begin{verbatim}
namespace ParticleSimulator {
    class MomentMonopoleSymmetry {
    public:
        F64    mass;
        F64vec pos;
        F64ort vertex_out_;
        F64ort vertex_in_;
    };
}
\end{verbatim}
\end{screen}

\begin{itemize}
\item クラス名
  PS::MomentMonopoleSymmetry

\item メンバ変数とその情報

  mass: 近傍でまとめた粒子の全質量、または全電荷

  pos: 近傍でまとめた粒子の重心、または粒子電荷の重心
  
  vertex\_out\_: 粒子をEssentialParticleJ::getRSearchの返り値を半径とする球とみなしたとき、近傍でまとめた粒子すべてを包含する最小の直方体の座標情報
  
  vertex\_in\_: 近傍でまとめた粒子すべての座標を包含する最小の直方体の座標情報

\item 使用条件

  EssentialParticleJクラス(節\ref{sec:essentialparticlej})がメンバ関数
  EssentialParticleJ::getCharge, EssentialParticleJ::getPos,
  EssentialParticleJ::getRSearchを持ち、それぞれが粒子質量(または粒子
  電荷)、粒子位置、粒子の力の到達距離を返すこと。EssentialParticleJク
  ラスのクラス名は変更自由。

\end{itemize}

\subsubsubsubsection{PS::MomentQuadrupoleSymmetry}
\label{sec:MomentQuadrupoleSymmetry}

単極子と四重極子を情報として持つクラス。単極子を計算する際の座標系の中心には
粒子の重心や粒子電荷の重心を取る。以下、このクラスの概要を記述する。
\begin{screen}
\begin{verbatim}
namespace ParticleSimulator {
    class MomentQuadrupoleSymmetry {
    public:
        F64    mass;
        F64vec pos;
        F64mat quad;
        F64ort vertex_out_;
        F64ort vertex_in_;
    };
}
\end{verbatim}
\end{screen}

\begin{itemize}
\item クラス名
  PS::MomentQuadrupoleSymmetry

\item メンバ変数とその情報

  mass: 近傍でまとめた粒子の全質量、または全電荷

  pos: 近傍でまとめた粒子の重心、または粒子電荷の重心
  
  quad: 近傍でまとめた粒子の四重極子
  
  vertex\_out\_: 粒子をEssentialParticleJ::getRSearchの返り値を半径とする球とみなしたとき、近傍でまとめた粒子すべてを包含する最小の直方体の座標情報
  
  vertex\_in\_: 近傍でまとめた粒子すべての座標を包含する最小の直方体の座標情報

\item 使用条件

  EssentialParticleJクラス(節\ref{sec:essentialparticlej})がメンバ関数
  EssentialParticleJ::getCharge, EssentialParticleJ::getPos,
  EssentialParticleJ::getRSearchを持ち、それぞれが粒子質量(または粒子
  電荷)、粒子位置、粒子の力の到達距離を返すこと。EssentialParticleJク
  ラスのクラス名は変更自由。

\end{itemize}


\subsubsubsection{PS::SEARCH\_MODE\_LONG\_CUTOFF}

\subsubsubsubsection{PS::MomentMonopoleCutoff}
\label{sec:MomentMonopoleCutoff}

単極子までを情報として持つクラス。単極子を計算する際の座標系の中心には
粒子の重心や粒子電荷の重心を取る。以下、このクラスの概要を記述する。
\begin{screen}
\begin{verbatim}
namespace ParticleSimulator {
    class MomentMonopoleCutoff {
    public:
        F64    mass;
        F64vec pos;
    };
}
\end{verbatim}
\end{screen}

\begin{itemize}
\item クラス名
  PS::MomentMonopoleCutoff

\item メンバ変数とその情報

  mass: 近傍でまとめた粒子の全質量、または全電荷

  pos: 近傍でまとめた粒子の重心、または粒子電荷の重心

\item 使用条件

  EssentialParticleJクラス(節\ref{sec:essentialparticlej})がメンバ関数
  EssentialParticleJ::getCharge, EssentialParticleJ::getPos,
  EssentialParticleJ::getRSearchを持ち、それぞれが粒子質量(または粒子
  電荷)、粒子位置、粒子の力の到達距離を返すこと。EssentialParticleJク
  ラスのクラス名は変更自由。

\end{itemize}
  

%%%%%%%%%%%%%%%%%%%%%%%%%%%%%%%%%%%%%%%%%%%%%%%%%%%%%%
\subsubsection{必要なメンバ関数}

\subsubsubsection{概要}

以下ではMomentクラスを定義する際に、必要なメンバ関数を記述する。このと
きMomentクラスのクラス名をMomとする。これは変更自由である。

%--------------
\subsubsubsection{コンストラクタ}


\begin{screen}
\begin{verbatim}
class Mom {
public:
    Mom ();
};
\end{verbatim}
\end{screen}

\begin{itemize}

\item {\bf 引数}

  なし
  
\item {\bf 返値}

  なし

\item {\bf 機能}

  Momクラスのオブジェクトの初期化をする。
  
\end{itemize}

%--------------
\subsubsubsection{Mom::init}


\begin{screen}
\begin{verbatim}
class Mom {
public:
    void init();
};
\end{verbatim}
\end{screen}

\begin{itemize}

\item {\bf 引数}

  なし
  
\item {\bf 返値}

  なし

\item {\bf 機能}

  Momクラスのオブジェクトの初期化をする。
  
\end{itemize}

%--------------
\subsubsubsection{Mom::getPos}

\begin{screen}
\begin{verbatim}
class Mom {
public:
    PS::F32vec getPos() const;
};
\end{verbatim}
\end{screen}

\begin{itemize}

\item {\bf 引数}

  なし
  
\item {\bf 返値}

  PS::F32vecまたはPS::F64vec型。Momクラスのメンバ変数pos。

\item {\bf 機能}

  Momクラスのメンバ変数posを返す。
  
\end{itemize}

%--------------
\subsubsubsection{Mom::getCharge}

\begin{screen}
\begin{verbatim}
class Mom {
public:
    PS::F32 getCharge() const;
};
\end{verbatim}
\end{screen}

\begin{itemize}

\item {\bf 引数}

  なし
  
\item {\bf 返値}

  PS::F32またはPS::F64型。Momクラスのメンバ変数mass。

\item {\bf 機能}

  Momクラスのメンバ変数massを返す。
  
\end{itemize}

%--------------
\subsubsubsection{Mom::getVertexIn}

\begin{screen}
\begin{verbatim}
class Mom {
public:
    PS::F32ort getVertexIn() const;
};
\end{verbatim}
\end{screen}

\begin{itemize}

\item {\bf 引数}

  なし
  
\item {\bf 返値}

  PS::F32ortまたはPS::F64ort型。

\item {\bf 機能}

  Momクラスに対応する(複数の)粒子すべてを包含する直方体の
  
\end{itemize}

%--------------
\subsubsubsection{Mom::accumulateAtLeaf}


\begin{screen}
\begin{verbatim}
class Mom {
public:
    template <class Tepj>
    void accumulateAtLeaf(const Tepj & epj);
};
\end{verbatim}
\end{screen}

\begin{itemize}

\item {\bf 引数}

  epj: 入力。const Tepj \&型。Tepjのオブジェクト。
  
\item {\bf 返値}

  なし。

\item {\bf 機能}

  EssentialParticleJクラスのオブジェクトからモーメントを計算する。
  
\end{itemize}

%--------------
\subsubsubsection{Mom::accumulate}

\begin{screen}
\begin{verbatim}
class Mom {
public:
    void accumulate(const Mom & mom);
};
\end{verbatim}
\end{screen}

\begin{itemize}

\item {\bf 引数}

  mom: 入力。const Mom \&型。Momクラスのオブジェクト。
  
\item {\bf 返値}

  なし。

\item {\bf 機能}

  MomクラスのオブジェクトからさらにMomクラスの情報を計算する。
  
\end{itemize}

%--------------
\subsubsubsection{Mom::set}

\begin{screen}
\begin{verbatim}
class Mom {
public:
    void set();
};
\end{verbatim}
\end{screen}

\begin{itemize}

\item {\bf 引数}

  なし
  
\item {\bf 返値}

  なし

\item {\bf 機能}

  上記のメンバ関数Mom::accumulateAtLeaf, Mom::accumulateではモー
  メントの位置情報の規格化ができていない場合ので、ここで規格化する。
  
\end{itemize}

%--------------
\subsubsubsection{Mom::accumulateAtLeaf2}

\begin{screen}
\begin{verbatim}
class Mom {
public:
    template <class Tepj>
    void accumulateAtLeaf2(const Tepj & epj);
};
\end{verbatim}
\end{screen}

\begin{itemize}

\item {\bf 引数}

  epj: 入力。const Tepj \&型。Tepjのオブジェクト。
  
\item {\bf 返値}

  なし。

\item {\bf 機能}

  EssentialParticleJクラスのオブジェクトからモーメントを計算する。
  
\end{itemize}

%--------------
\subsubsubsection{Mom::accumulate2}

\begin{screen}
\begin{verbatim}
class Mom {
public:
    void accumulate2(const Mom & mom);
};
\end{verbatim}
\end{screen}

\begin{itemize}

\item {\bf 引数}

  mom: 入力。const Mom \&型。Momクラスのオブジェクト。
  
\item {\bf 返値}

  なし。

\item {\bf 機能}

  MomクラスのオブジェクトからさらにMomクラスの情報を計算する。
  
\end{itemize}

%%%%%%%%%%%%%%%%%%%%%%%%%%%%%%%%%%%%%%%%%%%%%%%%%%%%%%
\subsubsection{場合によっては必要なメンバ関数}

\subsubsubsection{概要}

以下ではMomentクラスを定義する際に、場合によっては必要なメンバ関数を記述する。
このときMomentクラスのクラス名をMomとする。これは変更自由である。

\subsubsubsection{相互作用ツリークラスのPS::SEARCH\_MODE型に\newline PS::SEARCH\_MODE\_LONG\_CUTOFF、\newline PS::SEARCH\_MODE\_LONG\_SCATTER、\newline PS::SEARCH\_MODE\_LONG\_SYMMETRY \newline のいずれかを用いる場合}
%--------------
\subsubsubsection{Mom::getVertexIn}

\begin{screen}
\begin{verbatim}
class Mom {
public:
    F64ort getVertexIn();
};
\end{verbatim}
\end{screen}

\begin{itemize}

\item {\bf 引数}

  なし。
  
\item {\bf 返値}

  PS::F32ortまたはPS::F64ort型。

\item {\bf 機能}

  Momクラスに対応する粒子すべてを包含する最小の直方体(2次元の場合には直方体)の座標を
  オルソトープ型で返す。

\item {\bf 備考}
  
上記の直方体の座標は、各ツリーセルで正しく計算される必要がある。そのためには、粒子情報からツリー構造の一番深いレベルのツリーセルのモーメント情報を計算するためのメンバ関数\texttt{accumulateAtLeaf}と、複数のツリーセルからその親セルのモーメント情報を計算するためのメンバ関数 \texttt{accumulate} に直方体の座標計算を実装しなければならない。以下に実装例を示す。 
  
\begin{screen}
\begin{verbatim}
class Mom {
public:
    F64ort vertex_in_; // 直方体の座標を格納するメンバ変数
    F64ort getVertexIn() const { return vertex_in_; }
    template<class Tepj>
    void accumulateAtLeaf(const Tepj & epj){ 
         // 他の演算は省略
        (this->vertex_in_).merge(epj.getPos());
    }
    void accumulate(const Mom & mom){
        // 他の演算は省略
        (this->vertex_in_).merge(mom.vertex_in_);
    }
};
\end{verbatim}
\end{screen}
上記実装例では、オルソトープ型のメンバ関数\texttt{merge}を使用している。
なお、メンバ変数\texttt{vertex\_in\_}の名称は自由に変更可能である。
  
\end{itemize}

%--------------
\subsubsubsection{Mom::getVertexOut}

\begin{screen}
\begin{verbatim}
class Mom {
public:
    F64ort getVertexOut();
};
\end{verbatim}
\end{screen}

\begin{itemize}

\item {\bf 引数}

  なし。
  
\item {\bf 返値}

  PS::F32ortまたはPS::F64ort型。

\item {\bf 機能}

Momクラスに対応する各粒子を、中心を粒子座標、半径をその粒子のgetRSearch()の返り値とする球(2次元の場合には円)に置き換えたとき、
それらすべてを包含する最小の直方体(2次元の場合には直方体)の座標をオルソトープ型で返す。

\item {\bf 備考}
  
メンバ関数 \texttt{getVertexIn} のときと同様、上記の直方体の座標は、各ツリーセルで正しく計算される必要がある。そのためには、メンバ関数\texttt{accumulateAtLeaf}とメンバ関数 \texttt{accumulate} に直方体の座標計算を実装しなければならない。以下に実装例を示す。 
  
\begin{screen}
\begin{verbatim}
class Mom {
public:
    F64ort vertex_out_; // 直方体の座標を格納するメンバ変数
    F64ort getVertexOut() const { return vertex_out_; }
    template<class Tepj>
    void accumulateAtLeaf(const Tepj & epj){ 
         // 他の演算は省略
         (this->vertex_out_).merge(epj.getPos(), epj.getRSearch());
    }
    void accumulate(const Mom & mom){
        // 他の演算は省略
        (this->vertex_out_).merge(mom.vertex_out_);
    }
};
\end{verbatim}
\end{screen}
上記実装例では、オルソトープ型のメンバ関数\texttt{merge}を使用している。
なお、メンバ変数\texttt{vertex\_out\_}の名称は自由に変更可能である。
  
\end{itemize}