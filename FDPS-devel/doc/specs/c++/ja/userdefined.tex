\subsection{概要}

本節では、ユーザーが定義するクラスとファンクタについて記述する。ユーザー
定義クラスとなるのは、FullParticleクラス、EssentialParticleIクラス、
EssentialParticleJクラス、Momentクラス、SuperParticleJクラス、Forceクラ
ス、ヘッダクラスである。またユーザー定義の関数オブジェクトには、関数オブジェ
クトcalcForceEpEp、calcForceSpEpがある。

FullParticleクラスは、ある1粒子の情報すべてを持つクラスであり、粒子群
クラスにテンプレート引数として渡されるものである(節
\ref{sec:overview_action}の手順0)。

関数オブジェクトcalcForceEpEpとcalcForceSpEpは、それぞれj粒子からi粒子
への作用を計算する関数オブジェクトと超粒子からi粒子への作用を計算する
関数オブジェクトである。これらは相互作用ツリークラスのAPIの引数として
渡されるものである(節\ref{sec:overview_action}の手順0)。超粒子を必要と
するPS::SEARCH\_MODE型(PS::SEARCH\_MODE\_LONGか
PS::SEARCH\_MODE\_LONG\_CUTOFF)以外を使用する場合には、関数オブジェク
トcalcForceSpEpを定義する必要はない。

EssentialParticleIクラス、EssentialParticleJクラス、Momentクラス、
SuperParticleJクラス、Forceクラスは粒子間の相互作用の定義を補助するも
のである。これらのクラスのうちEssentialParticleIクラス、
EssentialParticleJクラス、Forceクラスはそれぞれ相互作用を計算する際にi
粒子に必要な情報、相互作用を計算する際にj粒子に必要な情報、相互作用の
結果の情報を持つ。これらはFullParticleクラスのサブセットであるため、こ
れらをFullParticleクラスで代用することも可能である。しかし、
FullParticleクラスは相互作用の定義に必要のないデータを多く含む場合も考
えられるため、計算コストを軽減したいならば、これらのクラスを使用するこ
とを検討するべきである。MomentクラスとSuperParticleJクラスは、それぞれ
ツリーセルのモーメント情報を持つクラスと超粒子に必要な情報を持つ
SuperParticleJクラスである。ユーザーが定義する必要があるのは、超粒子を
使う必要がある場合、すなわちPS::SEARCH\_MODE型にPS::SEARCH\_MODE\_LONG
かPS::SEARCH\_MODE\_LONG\_CUTOFFを選んだ場合のみである。

ヘッダクラスは入出力ファイルのヘッダ情報を持つ。

この節で記述するのは、これらのクラスや関数オブジェクトを定義する際の規
定である。ユーザーはこれらの間でのデータのやりとりや、関数オブジェクト
内でのデータの加工についてコードに書く必要がある。これらは上に挙げたク
ラスのメンバ関数と関数オブジェクト内で行われる。以下、必要なメンバ関数
とその規定について記述する。

\subsection{FullParticleクラス}
\label{sec:fullparticle}

\subsubsection{Summary}

The \texttt{FullParticle} class contains all information of a particle and is one of the template parameters of FDPS-defined \texttt{ParticleSystem} (see step 0 in Sec. \ref{sec:overview_action}). Users can define arbitrary member variables and member functions, as far as required member functions are defined. Below, we describe the required member functions.

\subsubsection{Premise}

Let us take \texttt{FP} class as an example of \texttt{FullParticle} as below. Users can use an arbitrary name in place of \texttt{FP}.
\begin{screen}
\begin{verbatim}
class FP;
\end{verbatim}
\end{screen}

\subsubsection{Required member functions}

\subsubsubsection{Summary}

The member functions \texttt{FP::getPos} and \texttt{FP::copyFromForce} are required. \texttt{FP::getPos} returns the position of a particle. Function \texttt{FP::copyFromForce} copies the results of calculation back to \texttt{FullParticle}. The examples and descriptions for these member functions are listed below.

\subsubsubsection{FP::getPos}

\begin{screen}
\begin{verbatim}
class FP {
public:
    PS::F64vec getPos() const;
};
\end{verbatim}
\end{screen}

\begin{itemize}

\item {\bf Arguments}
  None.
  
\item {\bf Returns}

  \texttt{PS::F32vec} or \texttt{PS::F64vec}.
  Returns the position of \texttt{FP} class.
  
\item {\bf Behaviour}

  Returns the member variables which contains the position of a particle.

\end{itemize}

\subsubsubsection{FP::copyFromForce}

\begin{screen}
\begin{verbatim}
class Force;

class FP {
public:
    void copyFromForce(const Force & force);
};
\end{verbatim}
\end{screen}

\begin{itemize}

\item {\bf Arguments}

  \texttt{force}: Input. \texttt{const Force} type.
  
\item {\bf Returns}

  None.
  
\item {\bf Behaviour}

  Copies back the results of calculation to \texttt{FP} class.

\end{itemize}

\subsubsection{Required member functions for specific cases}

\subsubsubsection{Summary}

In this section we describe the member functions for specific cases listed below;
\begin{itemize}
\item{} Modes other than \texttt{PS::SEARCH\_MODE\_LONG} for \texttt{PS::SEARCH\_MODE} are used.
\item{} APIs for file I/O in \texttt{ParticleSystem} are used.
\item{} \texttt{ParticleSystem::adjustPositionIntoRootDomain} API is used.
\item{} Particle Mesh class, which is an extension of FDPS, is used.
\end{itemize}

\subsubsubsection{Modes other than \texttt{PS::SEARCH\_MODE\_LONG} for \texttt{PS::SEARCH\_MODE} are used}

\subsubsubsubsection{FP::getRSearch}

\begin{screen}
\begin{verbatim}
class FP {
public:
    PS::F64 getRSearch() const;
};
\end{verbatim}
\end{screen}

\begin{itemize}

\item {\bf Arguments}

  None.

\item {\bf Returns}

  \texttt{PS::F32vec} or \texttt{PS::F64vec}.
  Returns the value of the member variable which contains neighbor search radius in \texttt{FP}.

\item {\bf Behaviour}

  Returns the value of the member variable which contains neighbor search radius in \texttt{FP}.
  
\end{itemize}

\subsubsubsection{APIs for file I/O in \texttt{ParticleSystem} are used}
\label{sec:userdefined_fullparticle_io}


The member functions \texttt{readAscii}, \texttt{writeAscii}, \texttt{readBinary}, and \texttt{writeBinary} are necessary, if users use \texttt{ParticleSystem::readParticleAscii}, \texttt{ParticleSystem::writeParticleAscii}, \texttt{ParticleSystem::readParticleBinary}, and \texttt{ParticleSystem::writeParticleBinary}, respectively  (users can also use different names for these member functions. For more details, please see section \ref{sec:ParticleSystem:IO}).  In this section we describe the rules for defining \texttt{readAscii}, \texttt{writeAscii}, \texttt{readBinary}, and \texttt{writeBinary}.

\subsubsubsubsection{FP::readAscii}
\label{sec:FP_readAscii}

\begin{screen}
\begin{verbatim}
class FP {
public:
    void readAscii(FILE *fp);
};
\end{verbatim}
\end{screen}

\begin{itemize}

\item {\bf Arguments}

  \texttt{fp}: \texttt{FILE *} type. A file pointer of input file.

\item {\bf Returns}

  None.
  
\item {\bf Behavior}

  Set the values of member variables of an instance of class \texttt{FP} by reading data from a ASCII file specified by file pointer \texttt{fp}. Data of one particle must be stored in one line. In other words, Data of one particle must end with a single new-line character (\texttt{\textbackslash n}).
  
\end{itemize}

\subsubsubsubsection{FP::writeAscii}
\label{sec:FP_writeAscii}

\begin{screen}
\begin{verbatim}
class FP {
public:
    void writeAscii(FILE *fp);
};
\end{verbatim}
\end{screen}

\begin{itemize}

\item {\bf Arguments}

  \texttt{fp}: \texttt{FILE *} type. A file pointer of output file.
  
\item {\bf Returns}

  None.
  
\item {\bf Behavior}


  Output the data of an instance of class \texttt{FP} to a file specified by file pointer \texttt{fp} in a ASCII format. Data must be written as one line. In other words, data of one particle must end with a single newline character (\texttt{\textbackslash n}).
  
\end{itemize}

\subsubsubsubsection{FP::readBinary}
\label{sec:FP_readBinary}

\begin{screen}
\begin{verbatim}
class FP {
public:
    void readBinary(FILE *fp);
};
\end{verbatim}
\end{screen}

\begin{itemize}

\item {\bf Arguments}

  \texttt{fp}: \texttt{FILE *} type. A file pointer of input file.

\item {\bf Returns}

  None.
  
\item {\bf Behavior}

  Set the values of member variables of an instance of class \texttt{FP} by reading data from a binary file specified by file pointer \texttt{fp}.
  
\end{itemize}

\subsubsubsubsection{FP::writeBinary}
\label{sec:FP_writeBinary}

\begin{screen}
\begin{verbatim}
class FP {
public:
    void writeBinary(FILE *fp);
};
\end{verbatim}
\end{screen}

\begin{itemize}

\item {\bf Arguments}

  \texttt{fp}: \texttt{FILE *} type. A file pointer of output file.
  
\item {\bf Returns}

  None.
  
\item {\bf Behavior}


  Output the data of an instance of class \texttt{FP} to a file specified by file pointer \texttt{fp} in a binary format. 
  
\end{itemize}


\subsubsubsection{\texttt{ParticleSystem::adjustPositionIntoRootDomain} API is used}

\subsubsubsubsection{FP::setPos}

\begin{screen}
\begin{verbatim}
class FP {
public:
    void setPos(const PS::F64vec pos_new);
};
\end{verbatim}
\end{screen}

\begin{itemize}

\item {\bf Arguments}

  \texttt{pos\_new}: Input. \texttt{const PS::F32vec} or \texttt{const PS::F64vec}. Modified positions of particle by FDPS.

\item {\bf Returns}

  None.
  
\item {\bf Behaviour}

  Replaces the positions in \texttt{FP} class by those modified by FDPS.

\end{itemize}

\subsubsubsection{Particle Mesh class, which is an extension of FDPS, is used}

When \texttt{Particle Mesh} class is used, \texttt{FP::getChargeParticleMesh} and \texttt{FP::copyFromForceParticleMesh} must be defined. Below, the rules for these functions are described.

\subsubsubsubsection{FP::getChargeParticleMesh}

\begin{screen}
\begin{verbatim}
class FP {
public:
    PS::F64 getChargeParticleMesh() const;
};
\end{verbatim}
\end{screen}

\begin{itemize}

\item {\bf Arguments}

  None.

\item {\bf Returns}

  \texttt{PS::F32} or \texttt{PS::F64}.
  Returns the mass or the electric charge of a particle.

\end{itemize}

\subsubsubsubsection{FP::copyFromForceParticleMesh}

\begin{screen}
\begin{verbatim}
class FP {
public:
    void copyFromForceParticleMesh(const PS::F32vec & acc_pm);
};
\end{verbatim}
\end{screen}

\begin{itemize}

\item {\bf Arguments}

  \texttt{acc\_pm}: \texttt{const PS::F32vec} or \texttt{const PS::F64vec}. Returns the resulting force by Particle Mesh.

\item {\bf Returns}

  None.
  
\item {\bf Behaviour}

  Writes back the resulting force by Particle Mesh to a particle.
  
\end{itemize}

%%%%%%%%%%%%%%%%%%%%%%%%%%%%%%%%%%%%%%%%%%%%%%%%%%%%%%
%%%%%%%%%%%%%%%%%%%%%%%%%%%%%%%%%%%%%%%%%%%%%%%%%%%%%%
%%%%%%%%%%%%%%%%%%%%%%%%%%%%%%%%%%%%%%%%%%%%%%%%%%%%%%

\subsubsubsection{Serialize particle data when particle exchange}
\label{sec:FP:serialize}

Member functions \texttt{FP::pack} and \texttt{FP::unpack} are necessary, if users send the data of particles with serializing them during particle exchange. Below, we describe the specifications for these functions.

%%%%%%%%%%%%%%%%%%%%%%%%%%%%%%%%%%%%%%%%%%%%%%%%%%%%%%

\subsubsubsubsection{FP::pack}

\begin{screen}
\begin{verbatim}
class FP {
public:
    static PS::S32 pack(const PS::S32 n_ptcl, const FP *ptcl[], char *buf, 
                        size_t & packed_size, const size_t max_buf_size);
};
\end{verbatim}
\end{screen}

\begin{itemize}

\item {\bf Arguments}

  \texttt{n\_ptcl}: Number of particles to be sent when exchanging particles. 
  
  \texttt{ptcl}: Array of pointers to particles to be sent.
  
  \texttt{buf}: Beginning address of a send buffer. 
  
  \texttt{packed\_size}: Size to be written to the send buffer by the user (in bytes). 
  
  \texttt{max\_buf\_size}: Size of writable area of the send buffer (in bytes).

\item {\bf Returns}

  Type \texttt{PS::S32}. Return -1 if \texttt{packed\_size} is greater than \texttt{max\_buf\_size}. Otherwise, returns 0.
  
\item {\bf Behaviour}

This function serializes the data of particles to be sent when exchanging particles and writes it into a send buffer.

\end{itemize}

%%%%%%%%%%%%%%%%%%%%%%%%%%%%%%%%%%%%%%%%%%%%%%%%%%%%%%

\subsubsubsubsection{FP::unPack}

\begin{screen}
\begin{verbatim}
class FP {
public:
    static PS::S32 unPack(const PS::S32 n_ptcl, FP ptcl[],
                          const char *buf);
};
\end{verbatim}
\end{screen}

\begin{itemize}

\item {\bf Arguments}

  \texttt{n\_ptcl}: Number of particles received when exchanging particles. 
  
  \texttt{ptcl}: Array of particles to store the received particles. 
  
  \texttt{buf}: Beginning address of a receive buffer.

\item {\bf Returns}

  Type \texttt{PS::S32}. Returns -1 when failing to deserialize. Otherwise, returns 0.

\item {\bf Behaviour}

 This functions deserializes received particles when exchanging particles and writes them to an array of particles. For details, see \S~\ref{sec:particleSystem:exchangeParticle}. When failing to deserialize, it calls PS::Abort() and the programs is terminated.

\end{itemize}


\subsection{EssentialParticleIクラス}
\label{sec:essentialparticlei}

\subsubsection{概要}

EssentialParticleIクラスは相互作用の計算に必要なi粒子の情報を持つクラ
スであり、相互作用の定義(節\ref{sec:overview_action}の手順0)に必要とな
る。EssentialParticleIクラスはFullParticleクラス
(節\ref{sec:fullparticle})のサブセットである。FDPSは、このクラスのデー
タにアクセスする必要がある。そのため、EssentialParticleIクラスはいくつ
かのメンバ関数を持つ必要がある。以下、この節の前提、常に必要なメンバ関
数と、場合によっては必要なメンバ関数について記述する。

\subsubsection{前提}

この節の中では、EssentialParticleIクラスとしてEPIというクラスを一例と
して使う。また、FullParticleクラスの一例としてFPというクラスを使う。
EPI, FPというクラス名は変更可能である。

EPIとFPの宣言は以下の通りである。
\begin{screen}
\begin{verbatim}
class FP;
class EPI;
\end{verbatim}
\end{screen}

\subsubsection{必要なメンバ関数}

\subsubsubsection{概要}

常に必要なメンバ関数はEPI::getPosとEPI::copyfromFPである。EPI::getPos
はEPIクラスの位置情報をFDPSに読み込ませるための関数で、EPI::copyFromFP
はFPクラスの情報をEPIクラスに書きこむ関数である。これらのメンバ関数の
記述例と解説を以下に示す。

\subsubsubsection{EPI::getPos}

%%%%%%%%%%%%%%%%%%%%%%%%%%%%%%%%%%%%%%%%%%%%%%%%%%%%%%
%%%%%%%%%%%%%%%%%%%%%%%%%%%%%%%%%%%%%%%%%%%%%%%%%%%%%%
\if 0
\begin{screen}
\begin{verbatim}
class EPI {
public:
    PS::F64vec pos;
    PS::F64vec getPos() const {
        return this->pos;
    }
};
\end{verbatim}
\end{screen}

\begin{itemize}

\item {\bf 前提}
  
  EPIのメンバ変数posはある1つの粒子の位置情報。このposのデー
  タ型はPS::F64vec型。
  
\item {\bf 引数}

  なし
  
\item {\bf 返値}

  PS::F64vec型。EPIクラスの位置情報を保持したメンバ変数。
  
\item {\bf 機能}

  EPIクラスのオブジェクトの位置情報を保持したメンバ変数を返す。
  
\item {\bf 備考}

  EPIクラスのメンバ変数posの変数名は変更可能。

\end{itemize}
\fi
%%%%%%%%%%%%%%%%%%%%%%%%%%%%%%%%%%%%%%%%%%%%%%%%%%%%%%
%%%%%%%%%%%%%%%%%%%%%%%%%%%%%%%%%%%%%%%%%%%%%%%%%%%%%%

\begin{screen}
\begin{verbatim}
class EPI {
public:
    PS::F64vec getPos() const;
};
\end{verbatim}
\end{screen}

\begin{itemize}

\item {\bf 引数}

  なし
  
\item {\bf 返値}

  PS::F64vec型。EPIクラスの位置情報を保持したメンバ変数。
  
\item {\bf 機能}

  EPIクラスのオブジェクトの位置情報を保持したメンバ変数を返す。
  
\end{itemize}


\subsubsubsection{EPI::copyFromFP}

%%%%%%%%%%%%%%%%%%%%%%%%%%%%%%%%%%%%%%%%%%%%%%%%%%%%%%
%%%%%%%%%%%%%%%%%%%%%%%%%%%%%%%%%%%%%%%%%%%%%%%%%%%%%%
\if 0
\begin{screen}
\begin{verbatim}
class FP {
public:
    PS::S64    identity;
    PS::F64    mass;
    PS::F64vec position;
    PS::F64vec velocity;
    PS::F64vec acceleration;
    PS::F64    potential;
};
class EPI {
public:
    PS::S64    id;
    PS::F64vec pos;
    void copyFromFP(const FP & fp) {
        this->id  = fp.identity;
        this->pos = fp.position;
    }
};
\end{verbatim}
\end{screen}

\begin{itemize}

\item {\bf 前提}

  FPクラスのメンバ変数identity, positionとEPI
  クラスのメンバ変数id, posはそれぞれ対応する情報を持つ。

\item {\bf 引数}

  fp: 入力。const FP \&型。FPクラスの情報を持つ。
  
\item {\bf 返値}

  なし。
  
\item {\bf 機能}

  FPクラスの持つ1粒子の情報の一部をEssnetialParticleIクラス
  に書き込む。
  
\item {\bf 備考}

  FPクラスのメンバ変数の変数名、EPIクラスのメ
  ンバ変数の変数名は変更可能。メンバ関数EPI::copyFromFP
  の引数名は変更可能。EPIクラスの粒子情報はFP
  クラスの粒子情報のサブセット。対応する情報を持つメンバ変数同士のデー
  タ型が一致している必要はないが、実数型とベクトル型(または整数型とベ
  クトル型)という違いがある場合に正しく動作する保証はない。

\end{itemize}
\fi
%%%%%%%%%%%%%%%%%%%%%%%%%%%%%%%%%%%%%%%%%%%%%%%%%%%%%%
%%%%%%%%%%%%%%%%%%%%%%%%%%%%%%%%%%%%%%%%%%%%%%%%%%%%%%

\begin{screen}
\begin{verbatim}
class FP;
class EPI {
public:
    void copyFromFP(const FP & fp);
};
\end{verbatim}
\end{screen}

\begin{itemize}

\item {\bf 引数}

  fp: 入力。const FP \&型。FPクラスの情報を持つ。
  
\item {\bf 返値}

  なし。
  
\item {\bf 機能}

  FPクラスの持つ1粒子の情報の一部をEssnetialParticleIクラス
  に書き込む。
  
\end{itemize}

\subsubsection{場合によっては必要なメンバ関数}

\subsubsubsection{概要}

本節では、場合によっては必要なメンバ関数について記述する。相互作用ツリー
クラスのPS::SEARCH\_MODE型にPS::SEARCH\_MODE\_GATHERまたは
PS::SEARCH\_MODE\_SYMMETRYを用いる場合に必要となるメンバ関数ついて記述
する。

\subsubsubsection{相互作用ツリークラスのPS::SEARCH\_MODE型に\\PS::SEARCH\_MODE\_GATHERまたはPS::SEARCH\_MODE\_SYMMETRYを用いる場合}

\subsubsubsubsection{EPI::getRSearch}

%%%%%%%%%%%%%%%%%%%%%%%%%%%%%%%%%%%%%%%%%%%%%%%%%%%%%%
%%%%%%%%%%%%%%%%%%%%%%%%%%%%%%%%%%%%%%%%%%%%%%%%%%%%%%
\if 0
\begin{screen}
\begin{verbatim}
class EPI {
public:
    PS::F64 search_radius;
    PS::F64 getRSearch() const {
        return this->search_radius;
    }
};
\end{verbatim}
\end{screen}

\begin{itemize}

\item {\bf 前提}

  EPIクラスのメンバ変数search\_radiusはある1つの粒子の
  近傍粒子を探す半径の大きさ。このsearch\_radiusのデータ型はPS::F32型
  またはPS::F64型。
  
\item {\bf 引数}

  なし
  
\item {\bf 返値}

  PS::F32型またはPS::F64型。 EPIクラスの近傍粒子を探す
  半径の大きさを保持したメンバ変数。
  
\item {\bf 機能}

  EPIクラスの近傍粒子を探す半径の大きさを保持したメンバ
  変数を返す。

\item {\bf 備考}

  EPIクラスのメンバ変数search\_radiusの変数名は変更可能。
  
\end{itemize}
\fi
%%%%%%%%%%%%%%%%%%%%%%%%%%%%%%%%%%%%%%%%%%%%%%%%%%%%%%
%%%%%%%%%%%%%%%%%%%%%%%%%%%%%%%%%%%%%%%%%%%%%%%%%%%%%%

\begin{screen}
\begin{verbatim}
class EPI {
public:
    PS::F64 getRSearch() const;
};
\end{verbatim}
\end{screen}

\begin{itemize}

\item {\bf 引数}

  なし
  
\item {\bf 返値}

  PS::F32型またはPS::F64型。 EPIクラスの近傍粒子を探す
  半径の大きさを保持したメンバ変数。
  
\item {\bf 機能}

  EPIクラスの近傍粒子を探す半径の大きさを保持したメンバ
  変数を返す。

\end{itemize}

\subsection{EssentialParticleJクラス}
\label{sec:essentialparticlej}

\subsubsection{Summary}

%EssentialParticleJクラスは相互作用の計算に必要なj粒子の情報を持つクラ
%スであり、相互作用の定義(節\ref{sec:overview_action}の手順0)に必要とな
%る。EssentialParticleJクラスはFullParticleクラス
%(節\ref{sec:fullparticle})のサブセットである。FDPSは、このクラスのデー
%タにアクセスする必要がある。このために、EssentialParticleJクラスはいく
%つかのメンバ関数を持つ必要がある。以下、この節の前提、常に必要なメンバ
%関数と、場合によっては必要なメンバ関数について記述する。
The \texttt{EssentialParticleJ} class should contain all information of
a $j-$ particle which is necessary to calculate interaction
(see step 0 in Sec. \ref{sec:overview_action}).
This class is a subset of \texttt{FullParticle} (see, Sec. \ref{sec:fullparticle}).
Class \texttt{EssentialParticleJ} should have required member functions
with specific names, as described below.

\subsubsection{Premise}

%この節の中では、EssentialParticleJクラスとしてEPJというクラスを一例と
%して使う。また、FullParticleクラスの一例としてFPというクラスを使う。
%EPJ, FPというクラス名は変更可能である。
Let us take \texttt{EPJ} and \texttt{FP} classes as examples of
\texttt{EssentialParticleJ} and \texttt{FullParticle} as below.
Users can use arbitrary names in place of \texttt{EPJ} and \texttt{FP}.

%EPJとFPの宣言は以下の通りである。
\begin{screen}
\begin{verbatim}
class FP;
class EPJ;
\end{verbatim}
\end{screen}

\subsubsection{Required member functions}

\subsubsubsection{Summary}

%常に必要なメンバ関数はEPJ::getPosとEPJ::copyfromFPである。EPJ::getPos
%はEPJクラスの位置情報をFDPSに読み込ませるための関数で、EPJ::copyFromFP
%はFPクラスの情報をEPJクラスに書きこむ関数である。これら
%のメンバ関数の記述例と解説を以下に示す。
The member functions \texttt{EPJ::getPos} and \texttt{EPJ::copyFromForce} are required.
\texttt{EPJ::getPos} returns the position of a particle to FDPS.
\texttt{EPJ::copyFromFP} copies the information necessary for the interaction calculation from \texttt{FullParticle}.
The examples and descriptions for these member functions are listed below.

\subsubsubsection{EPJ::getPos}

\begin{screen}
\begin{verbatim}
class EPJ {
public:
    PS::F64vec getPos() const;
};
\end{verbatim}
\end{screen}

\begin{itemize}

\item {\bf Arguments}

  None.
  
\item {\bf Returns}

  %PS::F64vec型。EPJクラスの位置情報を保持したメンバ変数。
  \texttt{PS::F64vec}.
  Returns the position of a particle of \texttt{EPJ} class.

\end{itemize}

\subsubsubsection{EPJ::copyFromFP}

\begin{screen}
\begin{verbatim}
class FP;
class EPJ {
public:
    void copyFromFP(const FP & fp);
};
\end{verbatim}
\end{screen}

\begin{itemize}

\item {\bf Arguments}

  %fp: 入力。const FP \&型。FPクラスの情報を持つ。
  \texttt{fp}: Input. \texttt{const FP \&} type.

\item {\bf Returns}

  None.
  
\item {\bf Behaviour}

  %FPクラスの持つ1粒子の情報の一部をEPJクラスに書き込む。
  Copies the part of information of \texttt{FP} to \texttt{EPJ}.

\end{itemize}

\subsubsection{Required member functions for specific cases}

\subsubsubsection{Summary}

%本節では、場合によっては必要なメンバ関数について記述する。相互作用ツリー
%クラスのPS::SEARCH\_MODE型にPS::SEARCH\_MODE\_LONG以外を用いる場合に必
%要なメンバ関数、列挙型のBOUNDARY\_CONDITION型に
%PS::BOUNDARY\_CONDITION\_OPEN以外を選んだ場合に必要となるメンバ関数に
%ついて記述する。なお、既存のMomentクラスやSuperParticleJクラスを用いる
%際に必要となるメンバ変数はこれら既存のクラスの節を参照のこと。
In this section we describe the member functions in the case that the other
mode than \texttt{PS::SEARCH\_MODE\_LONG} as \texttt{PS::SEARCH\_MODE} or
\texttt{PS::BOUNDARY\_CONDITION\_OPEN} as \texttt{BOUNDARY\_CONDITION} are used.
For the detailed description about pre-defined \texttt{Moment} and
\texttt{SuperParticleJ} classes, see corresponding section.

\subsubsubsection{Modes other than \texttt{PS::SEARCH\_MODE\_LONG} as \texttt{PS::SEARCH\_MODE} are used}

\subsubsubsubsection{EPJ::getRSearch}

\begin{screen}
\begin{verbatim}
class EPJ {
public:
    PS::F64 getRSearch() const;
};
\end{verbatim}
\end{screen}

\begin{itemize}

\item {\bf Arguments}

  None.
  
\item {\bf Returns}

  %PS::F32型またはPS::F64型。 EPJクラスの近傍粒子を探す
  %半径の大きさを保持したメンバ変数。
  \texttt{PS::F32vec} or \texttt{PS::F64vec}.
  Returns the value of the member variable which contains neighbor search radius in \texttt{EPJ}. When using \texttt{PS::SEARCH\_MODE\_LONG\_CUTOFF}, all EPJ must return the same value as the cutoff radius.

\end{itemize}

\subsubsubsection{Modes other than \texttt{PS::BOUNDARY\_CONDITION\_OPEN} are used \texttt{BOUNDARY\_CONDITION}}

\subsubsubsubsection{EPJ::setPos}

\begin{screen}
\begin{verbatim}
class EPJ {
public:
    void setPos(const PS::F64vec pos_new);
};
\end{verbatim}
\end{screen}

\begin{itemize}

\item {\bf Arguments}

  %pos\_new: 入力。const PS::F32vecまたはconst PS::F64vec型。FDPS側で修
  %正した粒子の位置情報。
  \texttt{pos\_new}: Input. \texttt{const PS::F32vec} or \texttt{const PS::F64vec}.
  Modified positions of particle by FDPS.

\item {\bf Returns}

  None.
  
\item {\bf Behaviour}

  %FDPSが修正した粒子の位置情報をEPJクラスの位置情報に書き込む。
  Replaces the positions in \texttt{EPJ} class by those modified by FDPS.

\end{itemize}


%%%%%%%%%%%%%%%%%%%%%%%%%%%%%%%%%%%%%%%%%%%%%%%%%%%%%%
\subsubsubsection{Obtian EPJ from id of particle}
\subsubsubsubsection{EPJ::getId}
\label{sec:EPJ:getId}

\begin{screen}
\begin{verbatim}
class EPJ {
public:
    PS::S64 getId();
};
\end{verbatim}
\end{screen}

\begin{itemize}

\item {\bf arguments}

void.

\item {\bf returned value}

Type PS::S64.
  
\item {\bf function}

This function is needed when user program uses PS::TreeForForce::getEpjFromId().
For more information, please see \ref{sec:getEpjFromId}.

%PS::TreeForForce::getEpjFromId()を使用する場合に必要。詳しく
%は\ref{sec:getEpjFromId}を参照。


\end{itemize}

%%%%%%%%%%%%%%%%%%%%%%%%%%%%%%%%%%%%%%%%%%%%%%%%%%%%%%
%%%%%%%%%%%%%%%%%%%%%%%%%%%%%%%%%%%%%%%%%%%%%%%%%%%%%%
%%%%%%%%%%%%%%%%%%%%%%%%%%%%%%%%%%%%%%%%%%%%%%%%%%%%%%

\subsubsubsection{Serialize particle data when LET exchange}
\label{sec:EPJ:serialize}

Member functions \texttt{EPJ::pack} and \texttt{EPJ::unpack} are necessary, if serializing particle data when LET exchange. Below, we describe the specifications for these functions.

%%%%%%%%%%%%%%%%%%%%%%%%%%%%%%%%%%%%%%%%%%%%%%%%%%%%%%

\subsubsubsubsection{EPJ::pack}

\begin{screen}
\begin{verbatim}
class EPJ {
public:
    static PS::S32 pack(const PS::S32 n_ptcl, const EPJ *ptcl[], char *buf, 
                        size_t & packed_size, const size_t max_buf_size);
};
\end{verbatim}
\end{screen}

\begin{itemize}

\item {\bf Arguments}

  \texttt{n\_ptcl}: Number of particles to be sent when LET exchange. 
  

  \texttt{ptcl}: Array of pointers to particles to be sent. 

  \texttt{buf}: Beginning address of a send buffer. 

  \texttt{packed\_size}: Size to be written to the send buffer by the user (in bytes). 

  \texttt{max\_buf\_size}: Size of writable area of the send buffer (in bytes).

\item {\bf Returns}

  Type \texttt{PS::S32}. Returns -1 if \texttt{packed\_size} is greater than \texttt{max\_buf\_size}. Otherwise, returns 0.
  
\item {\bf Behaviour}

  This function serializes the data of particles to be sent when LET exchange and writes them to a send buffer.

\end{itemize}

%%%%%%%%%%%%%%%%%%%%%%%%%%%%%%%%%%%%%%%%%%%%%%%%%%%%%%

\subsubsubsubsection{EPJ::unPack}

\begin{screen}
\begin{verbatim}
class EPJ {
public:
    static void unPack(const PS::S32 n_ptcl, EPJ ptcl[],
                       const char *buf);
};
\end{verbatim}
\end{screen}

\begin{itemize}

\item {\bf Arguments}

  \texttt{n\_ptcl}: Number of particles received when LET exchange. 
                       
  \texttt{ptcl}: Array of particles to store the received particles.
                       
  \texttt{buf}: Beginning address of a receive buffer.

\item {\bf Returns}

  None.

\item {\bf Behaviour}

 This function deserializes received particle data when LET exchange and write them to an array of particles. For details, see \S~\ref{sec:treeForForceHighLevelAPI}. When failing to deserialize, it calls PS::Abort() and the program is terminated.

\end{itemize}

\subsection{Momentクラス}
\label{sec:moment}

\subsubsection{概要}

Momentクラスは近い粒子同士でまとまった複数の粒子のモーメント情報を持つ
クラスであり、相互作用の定義(節\ref{sec:overview_action}の手順0)に必要
となる。モーメント情報の例としては、複数粒子の単極子や双極子、さらにこ
れら粒子の持つ最大の大きさなど様々なものが考えられる。このクラスは、
EssentialParticleJクラスからSuperParticleJクラスを作るための中間変数の
ような役割を果す。従って、このクラスが持つメンバ関数は、
EssentialParticleJクラスから情報を読み出してモーメントを計算するメンバ
関数、少ない数の粒子のモーメントからそれらの粒子を含むより多くの粒子の
モーメントを計算するメンバ関数などがある。

このようなモーメント情報にはある程度決っているものが多いので、それらに
ついてはFDPS側で用意した。これら既存のクラスについてまず記述する。その
後にユーザーがモーメントクラスを自作する際に必ず必要なメンバ関数、場合
によっては必要になるメンバ関数について記述する。

%%%%%%%%%%%%%%%%%%%%%%%%%%%%%%%%%%%%%%%%%%%%%%%%%%%%%%
\subsubsection{既存のクラス}

\subsubsubsection{概要}

FDPSはいくつかのMomentクラスを用意している。これらは相互作用ツリークラ
スで特定のPS::SEARCH\_MODE型を選んだ場合に有効である。以下、各
PS::SEARCH\_MODE型において選ぶことのできるMoment型を記述する。
PS::SEARCH\_MODE\_GATHER, \\PS::SEARCH\_MODE\_SCATTER,
PS::SEARCH\_MODE\_SYMMETRYについてはMomentクラスを意識してコーディング
する必要がないので、これらについては記述しない。

\subsubsubsection{PS::SEARCH\_MODE\_LONG}

\subsubsubsubsection{PS::MomentMonopole}
\label{sec:MomentMonopole}

単極子までを情報として持つクラス。単極子を計算する際の座標系の中心には
粒子の重心や粒子電荷の重心を取る。以下、このクラスの概要を記述する。
\begin{screen}
\begin{verbatim}
namespace ParticleSimulator {
    class MomentMonopole {
    public:
        F64    mass;
        F64vec pos;
    };
}
\end{verbatim}
\end{screen}

\begin{itemize}
\item クラス名
  PS::MomentMonopole

\item メンバ変数とその情報

  mass: 近傍でまとめた粒子の全質量、または全電荷

  pos: 近傍でまとめた粒子の重心、または粒子電荷の重心

\item 使用条件

  EssentialParticleJクラス(節\ref{sec:essentialparticlej})がメンバ関数
  EssentialParticleJ::getChargeとEssentialParticleJ::getPosを持ち、そ
  れぞれが粒子質量(または粒子電荷)、粒子位置を返すこと。
  EssentialParticleJクラスのクラス名は変更自由。

\end{itemize}

\subsubsubsubsection{PS::MomentQuadrupole}
\label{sec:MomentQuadrupole}

単極子と四重極子を情報として持つクラス。これらのモーメントを計算する際
の座標系の中心には粒子の重心を取る。以下、このクラスの概要を記述する。
\begin{screen}
\begin{verbatim}
namespace ParticleSimulator {
    class MomentQuadrupole {
    public:
        F64    mass;    
        F64vec pos;
        F64mat quad;
    };
}
\end{verbatim}
\end{screen}

\begin{itemize}
\item クラス名
  PS::MomentQuadrupole

\item メンバ変数とその情報

  mass: 近傍でまとめた粒子の全質量

  pos: 近傍でまとめた粒子の重心

  quad: 近傍でまとめた粒子の四重極子

\item 使用条件

  EssentialParticleJクラス(節\ref{sec:essentialparticlej})がメンバ関数
  EssentialParticleJ::getChargeとEssentialParticleJ::getPosを持ち、そ
  れぞれが粒子質量(または粒子電荷)、粒子位置を返すこと。
  EssentialParticleJクラスのクラス名は変更自由。

\end{itemize}

\subsubsubsubsection{PS::MomentMonopoleGeometricCenter}
\label{sec:MomentMonopoleGeometricCenter}

単極子までを情報として持つクラス。これらのモーメントを計算する際の座標
系の中心には粒子の幾何中心を取る。以下、このクラスの概要を記述する。
\begin{screen}
\begin{verbatim}
namespace ParticleSimulator {
    class MomentMonopoleGeometricCenter {
    public:
        F64    charge;    
        F64vec pos;
    };
}
\end{verbatim}
\end{screen}

\begin{itemize}
\item クラス名
  PS::MomentMonopoleGeometricCenter

\item メンバ変数とその情報

  charge: 近傍でまとめた粒子の全質量、または全電荷

  pos: 近傍でまとめた粒子の幾何中心

\item 使用条件

  EssentialParticleJクラス(節\ref{sec:essentialparticlej})がメンバ関数
  EssentialParticleJ::getChargeとEssentialParticleJ::getPosを持ち、そ
  れぞれが粒子質量(または粒子電荷)、粒子位置を返すこと。
  EssentialParticleJクラスのクラス名は変更自由。

\end{itemize}

\subsubsubsubsection{PS::MomentDipoleGeometricCenter}
\label{sec:MomentDipoleGeometricCenter}

双極子までを情報として持つクラス。これらのモーメントを計算する際の座標
系の中心には粒子の幾何中心を取る。以下、このクラスの概要を記述する。
\begin{screen}
\begin{verbatim}
namespace ParticleSimulator {
    class MomentDipoleGeometricCenter {
    public:
        F64    charge;    
        F64vec pos;
        F64vec dipole;
    };
}
\end{verbatim}
\end{screen}

\begin{itemize}
\item クラス名
  PS::MomentDipoleGeometricCenter

\item メンバ変数とその情報

  charge: 近傍でまとめた粒子の全質量、または全電荷

  pos: 近傍でまとめた粒子の幾何中心

  dipole: 粒子の質量または電荷の双極子

\item 使用条件

  EssentialParticleJクラス(節\ref{sec:essentialparticlej})がメンバ関数
  EssentialParticleJ::getChargeとEssentialParticleJ::getPosを持ち、そ
  れぞれが粒子質量(または粒子電荷)、粒子位置を返すこと。
  EssentialParticleJクラスのクラス名は変更自由。

\end{itemize}

\subsubsubsubsection{PS::MomentQuadrupoleGeometricCenter}
\label{sec:MomentQuadrupoleGeometricCenter}

四重極子までを情報として持つクラス。これらのモーメントを計算する際の座標
系の中心には粒子の幾何中心を取る。以下、このクラスの概要を記述する。
\begin{screen}
\begin{verbatim}
namespace ParticleSimulator {
    class MomentQuadrupoleGeometricCenter {
    public:
        F64    charge;    
        F64vec pos;
        F64vec dipole;
        F64mat quadrupole;
    };
}
\end{verbatim}
\end{screen}

\begin{itemize}
\item クラス名
  PS::MomentQuadrupoleGeometricCenter

\item メンバ変数とその情報

  charge: 近傍でまとめた粒子の全質量、または全電荷

  pos: 近傍でまとめた粒子の幾何中心

  dipole: 粒子の質量または電荷の双極子

  quadrupole: 粒子の質量または電荷の四重極子

\item 使用条件

  EssentialParticleJクラス(節\ref{sec:essentialparticlej})がメンバ関数
  EssentialParticleJ::getChargeとEssentialParticleJ::getPosを持ち、そ
  れぞれが粒子質量(または粒子電荷)、粒子位置を返すこと。
  EssentialParticleJクラスのクラス名は変更自由。

\end{itemize}

\subsubsubsection{PS::SEARCH\_MODE\_LONG\_SCATTER}

\subsubsubsubsection{PS::MomentMonopoleScatter}
\label{sec:MomentMonopoleScatter}

単極子までを情報として持つクラス。単極子を計算する際の座標系の中心には
粒子の重心や粒子電荷の重心を取る。以下、このクラスの概要を記述する。
\begin{screen}
\begin{verbatim}
namespace ParticleSimulator {
    class MomentMonopoleScatter {
    public:
        F64    mass;
        F64vec pos;
        F64ort vertex_out_;
        F64ort vertex_in_;
    };
}
\end{verbatim}
\end{screen}

\begin{itemize}
\item クラス名
  PS::MomentMonopoleScatter

\item メンバ変数とその情報

  mass: 近傍でまとめた粒子の全質量、または全電荷

  pos: 近傍でまとめた粒子の重心、または粒子電荷の重心
  
  vertex\_out\_: 粒子をEssentialParticleJ::getRSearchの返り値を半径とする球とみなしたとき、近傍でまとめた粒子すべてを包含する最小の直方体の座標情報
  
  vertex\_in\_: 近傍でまとめた粒子すべての座標を包含する最小の直方体の座標情報

\item 使用条件

  EssentialParticleJクラス(節\ref{sec:essentialparticlej})がメンバ関数
  EssentialParticleJ::getCharge, EssentialParticleJ::getPos,
  EssentialParticleJ::getRSearchを持ち、それぞれが粒子質量(または粒子
  電荷)、粒子位置、粒子の力の到達距離を返すこと。EssentialParticleJク
  ラスのクラス名は変更自由。

\end{itemize}

\subsubsubsubsection{PS::MomentQuadrupoleScatter}
\label{sec:MomentQuadrupoleScatter}

単極子と四重極子を情報として持つクラス。単極子を計算する際の座標系の中心には
粒子の重心や粒子電荷の重心を取る。以下、このクラスの概要を記述する。
\begin{screen}
\begin{verbatim}
namespace ParticleSimulator {
    class MomentQuadrupoleScatter {
    public:
        F64    mass;
        F64vec pos;
        F64mat quad;
        F64ort vertex_out_;
        F64ort vertex_in_;
    };
}
\end{verbatim}
\end{screen}

\begin{itemize}
\item クラス名
  PS::MomentQuadrupoleScatter

\item メンバ変数とその情報

  mass: 近傍でまとめた粒子の全質量、または全電荷

  pos: 近傍でまとめた粒子の重心、または粒子電荷の重心
  
  quad: 近傍でまとめた粒子の四重極子
  
  vertex\_out\_: 粒子をEssentialParticleJ::getRSearchの返り値を半径とする球とみなしたとき、近傍でまとめた粒子すべてを包含する最小の直方体の座標情報
  
  vertex\_in\_: 近傍でまとめた粒子すべての座標を包含する最小の直方体の座標情報

\item 使用条件

  EssentialParticleJクラス(節\ref{sec:essentialparticlej})がメンバ関数
  EssentialParticleJ::getCharge, EssentialParticleJ::getPos,
  EssentialParticleJ::getRSearchを持ち、それぞれが粒子質量(または粒子
  電荷)、粒子位置、粒子の力の到達距離を返すこと。EssentialParticleJク
  ラスのクラス名は変更自由。

\end{itemize}

\subsubsubsection{PS::SEARCH\_MODE\_LONG\_SYMMETRY}

\subsubsubsubsection{PS::MomentMonopoleSymmetry}
\label{sec:MomentMonopoleSymmetry}

単極子までを情報として持つクラス。単極子を計算する際の座標系の中心には
粒子の重心や粒子電荷の重心を取る。以下、このクラスの概要を記述する。
\begin{screen}
\begin{verbatim}
namespace ParticleSimulator {
    class MomentMonopoleSymmetry {
    public:
        F64    mass;
        F64vec pos;
        F64ort vertex_out_;
        F64ort vertex_in_;
    };
}
\end{verbatim}
\end{screen}

\begin{itemize}
\item クラス名
  PS::MomentMonopoleSymmetry

\item メンバ変数とその情報

  mass: 近傍でまとめた粒子の全質量、または全電荷

  pos: 近傍でまとめた粒子の重心、または粒子電荷の重心
  
  vertex\_out\_: 粒子をEssentialParticleJ::getRSearchの返り値を半径とする球とみなしたとき、近傍でまとめた粒子すべてを包含する最小の直方体の座標情報
  
  vertex\_in\_: 近傍でまとめた粒子すべての座標を包含する最小の直方体の座標情報

\item 使用条件

  EssentialParticleJクラス(節\ref{sec:essentialparticlej})がメンバ関数
  EssentialParticleJ::getCharge, EssentialParticleJ::getPos,
  EssentialParticleJ::getRSearchを持ち、それぞれが粒子質量(または粒子
  電荷)、粒子位置、粒子の力の到達距離を返すこと。EssentialParticleJク
  ラスのクラス名は変更自由。

\end{itemize}

\subsubsubsubsection{PS::MomentQuadrupoleSymmetry}
\label{sec:MomentQuadrupoleSymmetry}

単極子と四重極子を情報として持つクラス。単極子を計算する際の座標系の中心には
粒子の重心や粒子電荷の重心を取る。以下、このクラスの概要を記述する。
\begin{screen}
\begin{verbatim}
namespace ParticleSimulator {
    class MomentQuadrupoleSymmetry {
    public:
        F64    mass;
        F64vec pos;
        F64mat quad;
        F64ort vertex_out_;
        F64ort vertex_in_;
    };
}
\end{verbatim}
\end{screen}

\begin{itemize}
\item クラス名
  PS::MomentQuadrupoleSymmetry

\item メンバ変数とその情報

  mass: 近傍でまとめた粒子の全質量、または全電荷

  pos: 近傍でまとめた粒子の重心、または粒子電荷の重心
  
  quad: 近傍でまとめた粒子の四重極子
  
  vertex\_out\_: 粒子をEssentialParticleJ::getRSearchの返り値を半径とする球とみなしたとき、近傍でまとめた粒子すべてを包含する最小の直方体の座標情報
  
  vertex\_in\_: 近傍でまとめた粒子すべての座標を包含する最小の直方体の座標情報

\item 使用条件

  EssentialParticleJクラス(節\ref{sec:essentialparticlej})がメンバ関数
  EssentialParticleJ::getCharge, EssentialParticleJ::getPos,
  EssentialParticleJ::getRSearchを持ち、それぞれが粒子質量(または粒子
  電荷)、粒子位置、粒子の力の到達距離を返すこと。EssentialParticleJク
  ラスのクラス名は変更自由。

\end{itemize}


\subsubsubsection{PS::SEARCH\_MODE\_LONG\_CUTOFF}

\subsubsubsubsection{PS::MomentMonopoleCutoff}
\label{sec:MomentMonopoleCutoff}

単極子までを情報として持つクラス。単極子を計算する際の座標系の中心には
粒子の重心や粒子電荷の重心を取る。以下、このクラスの概要を記述する。
\begin{screen}
\begin{verbatim}
namespace ParticleSimulator {
    class MomentMonopoleCutoff {
    public:
        F64    mass;
        F64vec pos;
    };
}
\end{verbatim}
\end{screen}

\begin{itemize}
\item クラス名
  PS::MomentMonopoleCutoff

\item メンバ変数とその情報

  mass: 近傍でまとめた粒子の全質量、または全電荷

  pos: 近傍でまとめた粒子の重心、または粒子電荷の重心

\item 使用条件

  EssentialParticleJクラス(節\ref{sec:essentialparticlej})がメンバ関数
  EssentialParticleJ::getCharge, EssentialParticleJ::getPos,
  EssentialParticleJ::getRSearchを持ち、それぞれが粒子質量(または粒子
  電荷)、粒子位置、粒子の力の到達距離を返すこと。EssentialParticleJク
  ラスのクラス名は変更自由。

\end{itemize}
  

%%%%%%%%%%%%%%%%%%%%%%%%%%%%%%%%%%%%%%%%%%%%%%%%%%%%%%
\subsubsection{必要なメンバ関数}

\subsubsubsection{概要}

以下ではMomentクラスを定義する際に、必要なメンバ関数を記述する。このと
きMomentクラスのクラス名をMomとする。これは変更自由である。

%--------------
\subsubsubsection{コンストラクタ}


\begin{screen}
\begin{verbatim}
class Mom {
public:
    Mom ();
};
\end{verbatim}
\end{screen}

\begin{itemize}

\item {\bf 引数}

  なし
  
\item {\bf 返値}

  なし

\item {\bf 機能}

  Momクラスのオブジェクトの初期化をする。
  
\end{itemize}

%--------------
\subsubsubsection{Mom::init}


\begin{screen}
\begin{verbatim}
class Mom {
public:
    void init();
};
\end{verbatim}
\end{screen}

\begin{itemize}

\item {\bf 引数}

  なし
  
\item {\bf 返値}

  なし

\item {\bf 機能}

  Momクラスのオブジェクトの初期化をする。
  
\end{itemize}

%--------------
\subsubsubsection{Mom::getPos}

\begin{screen}
\begin{verbatim}
class Mom {
public:
    PS::F32vec getPos() const;
};
\end{verbatim}
\end{screen}

\begin{itemize}

\item {\bf 引数}

  なし
  
\item {\bf 返値}

  PS::F32vecまたはPS::F64vec型。Momクラスのメンバ変数pos。

\item {\bf 機能}

  Momクラスのメンバ変数posを返す。
  
\end{itemize}

%--------------
\subsubsubsection{Mom::getCharge}

\begin{screen}
\begin{verbatim}
class Mom {
public:
    PS::F32 getCharge() const;
};
\end{verbatim}
\end{screen}

\begin{itemize}

\item {\bf 引数}

  なし
  
\item {\bf 返値}

  PS::F32またはPS::F64型。Momクラスのメンバ変数mass。

\item {\bf 機能}

  Momクラスのメンバ変数massを返す。
  
\end{itemize}

%--------------
\subsubsubsection{Mom::getVertexIn}

\begin{screen}
\begin{verbatim}
class Mom {
public:
    PS::F32ort getVertexIn() const;
};
\end{verbatim}
\end{screen}

\begin{itemize}

\item {\bf 引数}

  なし
  
\item {\bf 返値}

  PS::F32ortまたはPS::F64ort型。

\item {\bf 機能}

  Momクラスに対応する(複数の)粒子すべてを包含する直方体の
  
\end{itemize}

%--------------
\subsubsubsection{Mom::accumulateAtLeaf}


\begin{screen}
\begin{verbatim}
class Mom {
public:
    template <class Tepj>
    void accumulateAtLeaf(const Tepj & epj);
};
\end{verbatim}
\end{screen}

\begin{itemize}

\item {\bf 引数}

  epj: 入力。const Tepj \&型。Tepjのオブジェクト。
  
\item {\bf 返値}

  なし。

\item {\bf 機能}

  EssentialParticleJクラスのオブジェクトからモーメントを計算する。
  
\end{itemize}

%--------------
\subsubsubsection{Mom::accumulate}

\begin{screen}
\begin{verbatim}
class Mom {
public:
    void accumulate(const Mom & mom);
};
\end{verbatim}
\end{screen}

\begin{itemize}

\item {\bf 引数}

  mom: 入力。const Mom \&型。Momクラスのオブジェクト。
  
\item {\bf 返値}

  なし。

\item {\bf 機能}

  MomクラスのオブジェクトからさらにMomクラスの情報を計算する。
  
\end{itemize}

%--------------
\subsubsubsection{Mom::set}

\begin{screen}
\begin{verbatim}
class Mom {
public:
    void set();
};
\end{verbatim}
\end{screen}

\begin{itemize}

\item {\bf 引数}

  なし
  
\item {\bf 返値}

  なし

\item {\bf 機能}

  上記のメンバ関数Mom::accumulateAtLeaf, Mom::accumulateではモー
  メントの位置情報の規格化ができていない場合ので、ここで規格化する。
  
\end{itemize}

%--------------
\subsubsubsection{Mom::accumulateAtLeaf2}

\begin{screen}
\begin{verbatim}
class Mom {
public:
    template <class Tepj>
    void accumulateAtLeaf2(const Tepj & epj);
};
\end{verbatim}
\end{screen}

\begin{itemize}

\item {\bf 引数}

  epj: 入力。const Tepj \&型。Tepjのオブジェクト。
  
\item {\bf 返値}

  なし。

\item {\bf 機能}

  EssentialParticleJクラスのオブジェクトからモーメントを計算する。
  
\end{itemize}

%--------------
\subsubsubsection{Mom::accumulate2}

\begin{screen}
\begin{verbatim}
class Mom {
public:
    void accumulate2(const Mom & mom);
};
\end{verbatim}
\end{screen}

\begin{itemize}

\item {\bf 引数}

  mom: 入力。const Mom \&型。Momクラスのオブジェクト。
  
\item {\bf 返値}

  なし。

\item {\bf 機能}

  MomクラスのオブジェクトからさらにMomクラスの情報を計算する。
  
\end{itemize}

%%%%%%%%%%%%%%%%%%%%%%%%%%%%%%%%%%%%%%%%%%%%%%%%%%%%%%
\subsubsection{場合によっては必要なメンバ関数}

\subsubsubsection{概要}

以下ではMomentクラスを定義する際に、場合によっては必要なメンバ関数を記述する。
このときMomentクラスのクラス名をMomとする。これは変更自由である。

\subsubsubsection{相互作用ツリークラスのPS::SEARCH\_MODE型に\newline PS::SEARCH\_MODE\_LONG\_CUTOFF、\newline PS::SEARCH\_MODE\_LONG\_SCATTER、\newline PS::SEARCH\_MODE\_LONG\_SYMMETRY \newline のいずれかを用いる場合}
%--------------
\subsubsubsection{Mom::getVertexIn}

\begin{screen}
\begin{verbatim}
class Mom {
public:
    F64ort getVertexIn();
};
\end{verbatim}
\end{screen}

\begin{itemize}

\item {\bf 引数}

  なし。
  
\item {\bf 返値}

  PS::F32ortまたはPS::F64ort型。

\item {\bf 機能}

  Momクラスに対応する粒子すべてを包含する最小の直方体(2次元の場合には直方体)の座標を
  オルソトープ型で返す。

\item {\bf 備考}
  
上記の直方体の座標は、各ツリーセルで正しく計算される必要がある。そのためには、粒子情報からツリー構造の一番深いレベルのツリーセルのモーメント情報を計算するためのメンバ関数\texttt{accumulateAtLeaf}と、複数のツリーセルからその親セルのモーメント情報を計算するためのメンバ関数 \texttt{accumulate} に直方体の座標計算を実装しなければならない。以下に実装例を示す。 
  
\begin{screen}
\begin{verbatim}
class Mom {
public:
    F64ort vertex_in_; // 直方体の座標を格納するメンバ変数
    F64ort getVertexIn() const { return vertex_in_; }
    template<class Tepj>
    void accumulateAtLeaf(const Tepj & epj){ 
         // 他の演算は省略
        (this->vertex_in_).merge(epj.getPos());
    }
    void accumulate(const Mom & mom){
        // 他の演算は省略
        (this->vertex_in_).merge(mom.vertex_in_);
    }
};
\end{verbatim}
\end{screen}
上記実装例では、オルソトープ型のメンバ関数\texttt{merge}を使用している。
なお、メンバ変数\texttt{vertex\_in\_}の名称は自由に変更可能である。
  
\end{itemize}

%--------------
\subsubsubsection{Mom::getVertexOut}

\begin{screen}
\begin{verbatim}
class Mom {
public:
    F64ort getVertexOut();
};
\end{verbatim}
\end{screen}

\begin{itemize}

\item {\bf 引数}

  なし。
  
\item {\bf 返値}

  PS::F32ortまたはPS::F64ort型。

\item {\bf 機能}

Momクラスに対応する各粒子を、中心を粒子座標、半径をその粒子のgetRSearch()の返り値とする球(2次元の場合には円)に置き換えたとき、
それらすべてを包含する最小の直方体(2次元の場合には直方体)の座標をオルソトープ型で返す。

\item {\bf 備考}
  
メンバ関数 \texttt{getVertexIn} のときと同様、上記の直方体の座標は、各ツリーセルで正しく計算される必要がある。そのためには、メンバ関数\texttt{accumulateAtLeaf}とメンバ関数 \texttt{accumulate} に直方体の座標計算を実装しなければならない。以下に実装例を示す。 
  
\begin{screen}
\begin{verbatim}
class Mom {
public:
    F64ort vertex_out_; // 直方体の座標を格納するメンバ変数
    F64ort getVertexOut() const { return vertex_out_; }
    template<class Tepj>
    void accumulateAtLeaf(const Tepj & epj){ 
         // 他の演算は省略
         (this->vertex_out_).merge(epj.getPos(), epj.getRSearch());
    }
    void accumulate(const Mom & mom){
        // 他の演算は省略
        (this->vertex_out_).merge(mom.vertex_out_);
    }
};
\end{verbatim}
\end{screen}
上記実装例では、オルソトープ型のメンバ関数\texttt{merge}を使用している。
なお、メンバ変数\texttt{vertex\_out\_}の名称は自由に変更可能である。
  
\end{itemize}

\subsection{SuperParticleJクラス}
\label{sec:superparticlej}

\subsubsection{Summary}

The class \texttt{SuperParticleJ} encapsulates the informations of a grouped particles (see, step 0 in Sec. \ref{sec:overview_action}). This class is required if one of the followings are employed as \texttt{PS::SEARCH\_MODE}.
\begin{itemize}[itemsep=-1ex]
\item \texttt{PS::SEARCH\_MODE\_LONG}
\item \texttt{PS::SEARCH\_MODE\_LONG\_SCATTER}
\item \texttt{PS::SEARCH\_MODE\_LONG\_SYMMETRY}
\item \texttt{PS::SEARCH\_MODE\_LONG\_CUTOFF}  
\end{itemize}
This class has a member function which is used for data-transfer between FDPS. This class is related to \texttt{Moment} class. Thus, this class has member functions which cast the \texttt{Moment} class to this class (and vice versa).

Similar to \texttt{Moment} class, since this class has prescribed rules, FDPS has pre-defined classes.
This section aims to describe the specific member functions both always and in specific case required for \texttt{SuperParticleJ}.

\subsubsection{Pre-defined class}

FDPS \texttt{has} some pre-defined \texttt{SuperParticleJ} class. Below, the available classes for each \texttt{PS::SEARCH\_MODE} are described.

\subsubsubsection{PS::SEARCH\_MODE\_LONG}

\subsubsubsubsection{PS::SPJMonopole}
\label{sec:SPJMonopole}

A \texttt{SuperParticleJ} class which is related to \texttt{PS::MomentMonopole} class. It contains up to monopole moment.

\begin{screen}
\begin{verbatim}
namespace ParticleSimulator {
    class SPJMonopole {
    public:
        F64    mass;
        F64vec pos;
    };
}
\end{verbatim}
\end{screen}

\begin{itemize}

\item Name of the class

  \texttt{PS::SPJMonopole}


\item Members and their information

  \texttt{mass}: accumulated mass or electron charge.

  \texttt{pos}: center of mass or center of charge.

\item Terms of use

  The same as \texttt{PS::MomentMonopole} class.

\end{itemize}

\subsubsubsubsection{PS::SPJQuadrupole}
\label{sec:SPJQuadrupole}

A \texttt{SuperParticleJ} class which is related to \texttt{PS::MomentQuadrupole} class. It contains up to quadrupole moment.

\begin{screen}
\begin{verbatim}
namespace ParticleSimulator {
    class SPJQuadrupole {
    public:
        F64    mass;
        F64vec pos;
        F64mat quad;
    };
}
\end{verbatim}
\end{screen}

\begin{itemize}

\item Name of the class

  \texttt{PS::SPJQuadrupole}

\item Members and their information

  \texttt{mass}: accumulated mass.

  \texttt{pos}: center of mass.

  \texttt{quad}: accumulated quadrupole.

\item Terms of use

  The same as \texttt{PS::MomentQuadrupole} class.

\end{itemize}

\subsubsubsubsection{PS::SPJMonopoleGeometricCenter}
\label{sec:SPJMonopoleGeometricCenter}

A \texttt{SuperParticleJ} class which is related to \texttt{PS::MomentMonopoleGeometricCenter} class. It contains up to quadrupole moment (the reference point is set to geometric center of grouped particles).

\begin{screen}
\begin{verbatim}
namespace ParticleSimulator {
    class SPJMonopoleGeometricCenter {
    public:
        F64    charge;    
        F64vec pos;
    };
}
\end{verbatim}
\end{screen}

\begin{itemize}

\item Name of the class

  \texttt{PS::SPJMonopoleGeometricCenter}

\item Members and their information

  \texttt{charge}: accumulated mass/charge.

  \texttt{pos}: geometric center.

\item Terms of use

  The same as \texttt{PS::MomentMonopoleGeometricCenter} class.

\end{itemize}

\subsubsubsubsection{PS::SPJDipoleGeometricCenter}
\label{sec:SPJDipoleGeometricCenter}

A \texttt{SuperParticleJ} class which is related to \texttt{PS::MomentDipoleGeometricCenter} class. It contains up to dipole moment (the reference point is set to geometric center of grouped particles).

\begin{screen}
\begin{verbatim}
namespace ParticleSimulator {
    class SPJDipoleGeometricCenter {
    public:
        F64    charge;    
        F64vec pos;
        F64vec dipole;
    };
}
\end{verbatim}
\end{screen}

\begin{itemize}

\item Name of the class

  \texttt{PS::SPJDipoleGeometricCenter}


\item Members and their information

  \texttt{charge}: accumulated mass/charge.

  \texttt{pos}: geometric center.

  \texttt{dipole}: dipole of the particle mass or electron charge.

\item Terms of use

  The same as \texttt{PS::MomentDipoleGeometricCenter} class.

\end{itemize}

\subsubsubsubsection{PS::SPJQuadrupoleGeometricCenter}
\label{sec:SPJQuadrupoleGeometricCenter}

A \texttt{SuperParticleJ} class which is related to \texttt{PS::MomentQuadrupoleGeometricCenter} class. It contains up to quadrupole moment (the reference point is set to geometric center of grouped particles).

\begin{screen}
\begin{verbatim}
namespace ParticleSimulator {
    class SPJQuadrupoleGeometricCenter {
    public:
        F64    charge;
        F64vec pos;
        F64vec dipole;
        F64mat quadrupole;
    };
}
\end{verbatim}
\end{screen}

\begin{itemize}

\item Name of the class

  \texttt{PS::SPJQuadrupoleGeometricCenter}

\item Members and their information

  \texttt{charge}: accumulated mass/charge.

  \texttt{pos}: geometric center.

  \texttt{dipole}: dipole of the particle mass or electron charge.

  \texttt{quadrupole}: quadrupole of the particle mass or electron charge.

\item Terms of use

  The same as \texttt{PS::MomentQuadrupoleGeometricCenter} class.

\end{itemize}

\subsubsubsection{PS::SEARCH\_MODE\_LONG\_SCATTER}

\subsubsubsubsection{PS::SPJMonopoleScatter}
\label{sec:SPJMonopoleScatter}

A \texttt{SuperParticleJ} class which is related to \texttt{PS::MomentMonopoleScatter} class. It contains up to monopole moment.

\begin{screen}
\begin{verbatim}
namespace ParticleSimulator {
    class SPJMonopoleScatter {
    public:
        F64    mass;
        F64vec pos;
    };
}
\end{verbatim}
\end{screen}

\begin{itemize}
  
\item Name of the class

  \texttt{PS::SPJMonopoleScatter}

\item Members and their information

  \texttt{mass}: accumulated mass or electron charge.

  \texttt{pos}: center of mass or center of charge.

\item Terms of use

  The same as \texttt{PS::MomentMonopoleScatter} class.

\end{itemize}  

\subsubsubsubsection{PS::SPJQuadrupoleScatter}
\label{sec:SPJQuadrupoleScatter}

A \texttt{SuperParticleJ} class which is related to \texttt{PS::MomentQuadrupoleScatter} class. It contains up to quadrupole moment.

\begin{screen}
\begin{verbatim}
namespace ParticleSimulator {
    class SPJQuadrupoleScatter {
    public:
        F64    mass;
        F64vec pos;
        F64mat quad;
    };
}
\end{verbatim}
\end{screen}

\begin{itemize}
  
\item Name of the class

  \texttt{PS::SPJQuadpoleScatter}

\item Members and their information

  \texttt{mass}: accumulated mass or electron charge.

  \texttt{pos}: center of mass or center of charge.

  \texttt{quad}: accumulated quadrupole.

\item Terms of use

  The same as \texttt{PS::MomentQuadrupoleScatter} class.

\end{itemize}  

\subsubsubsection{PS::SEARCH\_MODE\_LONG\_SYMMETRY}

\subsubsubsubsection{PS::SPJMonopoleSymmetry}
\label{sec:SPJMonopoleSymmetry}

A \texttt{SuperParticleJ} class which is related to \texttt{PS::MomentMonopoleSymmetry} class. It contains up to monopole moment.

\begin{screen}
\begin{verbatim}
namespace ParticleSimulator {
    class SPJMonopoleSymmetry {
    public:
        F64    mass;
        F64vec pos;
    };
}
\end{verbatim}
\end{screen}

\begin{itemize}
  
\item Name of the class

  \texttt{PS::SPJMonopoleSymmetry}

\item Members and their information

  \texttt{mass}: accumulated mass or electron charge.

  \texttt{pos}: center of mass or center of charge.


\item Terms of use

  The same as \texttt{PS::MomentMonopoleSymmetry} class.

\end{itemize}  

\subsubsubsubsection{PS::SPJQuadrupoleSymmetry}
\label{sec:SPJQuadrupoleSymmetry}

A \texttt{SuperParticleJ} class which is related to \texttt{PS::MomentQuadrupoleSymmetry} class. It contains up to quadrupole moment.

\begin{screen}
\begin{verbatim}
namespace ParticleSimulator {
    class SPJQuadrupoleSymmetry {
    public:
        F64    mass;
        F64vec pos;
        F64mat quad;
    };
}
\end{verbatim}
\end{screen}

\begin{itemize}
  
\item Name of the class

  \texttt{PS::SPJQuadpoleSymmetry}

\item Members and their information

  \texttt{mass}: accumulated mass or electron charge.

  \texttt{pos}: center of mass or center of charge.

  \texttt{quad}: accumulated quadrupole.

\item Terms of use

  The same as \texttt{PS::MomentQuadrupoleSymmetry} class.

\end{itemize}  

\subsubsubsection{PS::SEARCH\_MODE\_LONG\_CUTOFF}

\subsubsubsubsection{PS::SPJMonopoleCutoff}
\label{sec:SPJMonopoleCutoff}

A \texttt{SuperParticleJ} class which is related to \texttt{PS::MomentMonopoleCutoff} class. It contains up to monopole moment.

\begin{screen}
\begin{verbatim}
namespace ParticleSimulator {
    class SPJMonopoleCutoff {
    public:
        F64    mass;
        F64vec pos;
    };
}
\end{verbatim}
\end{screen}

\begin{itemize}

\item Name of the class

  \texttt{PS::SPJMonopoleCutoff}

\item Members and their information

  \texttt{mass}: accumulated mass or electron charge.

  \texttt{pos}: center of mass or center of charge.

\item Terms of use

  The same as \texttt{PS::MomentMonopoleCutoff} class.

\end{itemize}

\subsubsection{Required member functions}

\subsubsubsection{Summary}

Below, the required member functions of \texttt{SuperParticleJ} class are described. In this section we use the name \texttt{SPJ} as a \texttt{SuperParticleJ} class.

\subsubsubsection{SPJ::getPos}

\begin{screen}
\begin{verbatim}
class SPJ {
public:
    PS::F64vec getPos() const;
};
\end{verbatim}
\end{screen}

\begin{itemize}

\item {\bf Arguments}

  None.
  
\item {\bf Returns}

  \texttt{PS::F32vec} or \texttt{PS::F64vec}.
  Returns the position pf a super-particle of class \texttt{SPJ}.
  
\end{itemize}

\subsubsubsection{SPJ::setPos}

\begin{screen}
\begin{verbatim}
class SPJ {
public:
    void setPos(const PS::F64vec pos_new);
};
\end{verbatim}
\end{screen}

\begin{itemize}

\item {\bf Arguments}

  \texttt{pos\_new}: Input. \texttt{const PS::F32vec} or \texttt{const PS::F64vec}.
  Modified positions of particle by FDPS.

\item {\bf Returns}

  None.
  
\item {\bf Behaviour}

  Replaces the positions in \texttt{SPJ} class by those modified by FDPS.

\end{itemize}

\subsubsubsection{SPJ::copyFromMoment}

\begin{screen}
\begin{verbatim}
class Mom;
class SPJ {
public:
    void copyFromMoment(const Mom & mom);
};
\end{verbatim}
\end{screen}

\begin{itemize}
  
\item {\bf Arguments}

  \texttt{mom}: Input. \texttt{const Mom \&} type.
  \texttt{Mom} can be both user defined and pre-defined \texttt{Moment} class.

\item {\bf Returns}

  None.
  
\item {\bf Behaviour}

  Copies the informations of \texttt{Mom} class to \texttt{SPJ}.

\end{itemize}

\subsubsubsection{SPJ::convertToMoment}

\begin{screen}
\begin{verbatim}
class Mom {
public:
    Mom(const PS::F32 m,
        const PS::F32vec & p);
}
class SPJ {
public:
    Mom convertToMoment() const;
};
\end{verbatim}
\end{screen}

\begin{itemize}
  
\item {\bf Arguments}

  None.

\item {\bf Returns}

  \texttt{Mom} type.
  The constructor of \texttt{Mom} class.

\item {\bf Behaviour}

  Returns the constructor of \texttt{Mom} class.

\end{itemize}

\subsubsubsection{SPJ::clear}

\begin{screen}
\begin{verbatim}
class SPJ {
public:
    void clear();
};
\end{verbatim}
\end{screen}

\begin{itemize}
  
\item {\bf Arguments}

  None.

\item {\bf Returns}

  None.
  
\item {\bf Behaviour}

  Clears the information of \texttt{SPJ} class.

\end{itemize}


\subsubsection{Required member functions for specific case}

\subsubsubsection{Serialize particle data when LET exchange}
\label{sec:SPJ:serialize}

Member functions \texttt{SPJ::pack} and \texttt{SPJ::unpack} are necessary, if serializing particle data when LET exchange. Below, we describe the specifications for these functions.

%%%%%%%%%%%%%%%%%%%%%%%%%%%%%%%%%%%%%%%%%%%%%%%%%%%%%%

\subsubsubsubsection{SPJ::pack}

\begin{screen}
\begin{verbatim}
class SPJ {
public:
    static PS::S32 pack(const PS::S32 n_ptcl, const SPJ *ptcl[], char *buf, 
                        size_t & packed_size, const size_t max_buf_size);
};
\end{verbatim}
\end{screen}

\begin{itemize}

\item {\bf Arguments}

  \texttt{n\_ptcl}: Number of superparticles to be sent when LET exchange.

  \texttt{ptcl}: Array of pointers to superparticles to be sent. 

  \texttt{buf}: Beginning address of a send buffer. 

  \texttt{packed\_size}: Size to be written to the send buffer by the user (in bytes). 
  
  \texttt{max\_buf\_size}: Size of writable area of the send buffer (in bytes).

\item {\bf Returns}

  Type \texttt{PS::S32}. Returns -1 if \texttt{packed\_size} is greater than \texttt{max\_buf\_size}. Otherwise, returns 0.
  
\item {\bf Behaviour}

 This function serializes the data of particles to be sent when LET exchange and writes them to a send buffer.

\end{itemize}

%%%%%%%%%%%%%%%%%%%%%%%%%%%%%%%%%%%%%%%%%%%%%%%%%%%%%%

\subsubsubsubsection{SPJ::unPack}

\begin{screen}
\begin{verbatim}
class SPJ {
public:
    static void unPack(const PS::S32 n_ptcl, SPJ[],
                       const char *buf);
};
\end{verbatim}
\end{screen}

\begin{itemize}

\item {\bf Arguments}

  \texttt{n\_ptcl}: Number of superparticles received when LET exchange. 
                       
  \texttt{ptcl}: Array of superparticles to store the received superparticles.
                       
  \texttt{buf}: Beginning address of a receive buffer.

\item {\bf Returns}

  None.

\item {\bf Behaviour}

 This function deserializes a received data of superparticles when LET exchange and writes them to an array of superparticles.

\end{itemize}

\subsection{Forceクラス}
\label{sec:force}

\subsubsection{概要}

Forceクラスは相互作用の結果を保持するクラスであり、相互作用の定義
(節\ref{sec:overview_action}の手順0)に必要となる。以下、この節の前提、
常に必要なメンバ関数について記述する。

\subsubsection{前提}

この節で用いる例としてForceクラスのクラス名をResultとする。このクラス名
は変更自由である。

\subsubsection{必要なメンバ関数}

常に必要なメンバ関数はResult::clearである。この関数は相互作用の計算結果を初期
化する。以下、Result::clearについて記述する。

\subsubsubsection{Result::clear}

\if 0
\begin{screen}
\begin{verbatim}
class Result {
public:
    PS::F32vec acc;
    PS::F32    pot;
    void clear() {
        acc = 0.0;
        pot = 0.0;
    }
};
\end{verbatim}
\end{screen}

\begin{itemize}

\item {\bf 前提}
  
  Resultクラスのメンバ変数はaccとpot。
  
\item {\bf 引数}

  なし
  
\item {\bf 返値}

  なし。
  
\item {\bf 機能}

  Resultクラスのメンバ変数を初期化する。
  
\item {\bf 備考}

  Resultクラスのメンバ変数acc, potの変数名は変更可能。他のメンバ変数を
  加えることも可能。

\end{itemize}
\fi

\begin{screen}
\begin{verbatim}
class Result {
public:
    void clear();
};
\end{verbatim}
\end{screen}

\begin{itemize}

\item {\bf 引数}

  なし
  
\item {\bf 返値}

  なし。
  
\item {\bf 機能}

  Resultクラスのメンバ変数を初期化する。
  
\end{itemize}

\subsection{ヘッダクラス}
\label{sec:userdefined_header}

\subsubsection{Summary}

\texttt{Header} class defines the format of header of input/output files. This class is required if users use API for file I/O provided by FDPS (\texttt{ParticleSystem::readParticleAscii},  \texttt{ParticleSystem::writeParticleAscii}, \texttt{ParticleSystem::readParticleBinary}, and \texttt{ParticleSystem::writeParticleBinary}) and want to add header information to input/output file. Below, we describe the rules for defining the member functions required from provided input/output API. There is no functions which is always required.

\subsubsection{Premise}

Let us take \texttt{Hdr} class as an example of \texttt{Header} as below. Users can use an arbitrary name in place of \texttt{Hdr}.

\subsubsection{Required member functions for specific case}
%%%%%%%%
\subsubsubsection{Hdr::readAscii}
\label{sec:Hdr_readAscii}

\begin{screen}
\begin{verbatim}
class Hdr {
public:
    PS::S32 readAscii(FILE *fp);
};
\end{verbatim}
\end{screen}

\begin{itemize}

\item {\bf Arguments}

  \texttt{fp}: Input. \texttt{FILE *} type. The file pointer of the input file.

\item {\bf Returns}

  \texttt{PS::S32} type. Returns the value of number of particles. Returns -1 if header does not contain the number of particle.
  
\item {\bf Behavior}

  Read the header data from the input file.

\end{itemize}

%%%%%%%%
\subsubsubsection{Hdr::writeAscii}
\label{sec:Hdr_writeAscii}

\begin{screen}
\begin{verbatim}
class Hdr {
public:
    void writeAscii(FILE *fp);
};
\end{verbatim}
\end{screen}

\begin{itemize}

\item {\bf Arguments}

  \texttt{fp}: Input. \texttt{FILE *} type. The file pointer of the output file.

\item {\bf Returns}

  None.
  
\item {\bf Behavior}

  Copies the header information to the output file.
    
\end{itemize}

%%%%%%%%  
\subsubsubsection{Hdr::readBinary}
\label{sec:Hdr_readBinary}

\begin{screen}
\begin{verbatim}
class Hdr {
public:
    PS::S32 readBinary(FILE *fp);
};
\end{verbatim}
\end{screen}

\begin{itemize}

\item {\bf Arguments}

  \texttt{fp}: Input. \texttt{FILE *} type. The file pointer of the input file.

\item {\bf Returns}

  \texttt{PS::S32} type. Returns the value of number of particles. Returns -1 if header does not contain the number of particle.
  
\item {\bf Behavior}

  Read the header data from the input file.

\end{itemize}
  
%%%%%%%%  
\subsubsubsection{Hdr::writeBinary}
\label{sec:Hdr_writeBinary}

\begin{screen}
\begin{verbatim}
class Hdr {
public:
    void writeBinary(FILE *fp);
};
\end{verbatim}
\end{screen}

\begin{itemize}

\item {\bf Arguments}

  \texttt{fp}: Input. \texttt{FILE *} type. The file pointer of the output file.

\item {\bf Returns}

  None.
  
\item {\bf Behavior}

  Copies the header information to the output file.
    
\end{itemize}


\subsection{関数オブジェクトcalcForceEpEp}
\label{sec:userdefined_calcForceEpEp}

\subsubsection{概要}

関数オブジェクトcalcForceEpEpは粒子同士の相互作用を記述するものであり、
相互作用の定義(節\ref{sec:overview_action}の手順0)に必要となる。以下、
これの書き方の規定を記述する。

\subsubsection{前提}

ここで示すのは重力N体シミュレーションの粒子間相互作用の記述の仕方であ
る。関数オブジェクトcalcForceEpEpの名前はgravityEpEpとする。これは変更
自由である。また、EssentialParitlceIクラスのクラス名をEPI,
EssentialParitlceJクラスのクラス名をEPJ, Forceクラスのクラス名をResult
とする。

\subsubsection{gravityEpEp::operator ()}

\if 0
\begin{lstlisting}[caption=calcForceEpEp]
class Result {
public:
    PS::F32vec acc;
};
class EPI {
public:
    PS::S32    id;
    PS::F32vec pos;
};
class EPJ {
public:
    PS::S32    id;
    PS::F32    mass;
    PS::F32vec pos;
};
struct gravityEpEp {
    static PS::F32 eps2;
    void operator () (const EPI *epi,
                      const PS::S32 ni,
                      const EPJ *epj,
                      const PS::S32 nj,
                      Result *result) {

        for(PS::S32 i = 0; i < ni; i++) {
            PS::S32    ii = epi[i].id;
            PS::F32vec xi = epi[i].pos;
            PS::F32vec ai = 0.0;
            for(PS::S32 j = 0; j < nj; j++) {
                PS::S32    jj = epj[j].id;
                PS::F32    mj = epj[j].mass;
                PS::F32vec xj = epj[j].pos;

                PS::F32vec dx   = xi - xj;
                PS::F32    r2   = dx * dx + eps2;
                PS::F32    rinv = (ii != jj) ? 1. / sqrt(r2)
                                             : 0.0;

                ai += mj * rinv * rinv * rinv * dx;
            }
            result.acc = ai;
        }
    }
};
PS::F32 gravityEpEp::eps2 = 9.765625e-4;
\end{lstlisting}

\begin{itemize}

\item {\bf 前提}

  クラスResult, EPI, EPJに必要なメンバ関数は省略した。クラスResultのメ
  ンバ変数accはi粒子がj粒子から受ける重力加速度である。クラスEPIとEPJ
  のメンバ変数idとposはそれぞれの粒子IDと粒子位置である。クラスEPJのメ
  ンバ変数massはj粒子の質量である。関数オブジェクトgravityEpEpのメンバ
  変数eps2は重力ソフトニングの2乗である。ここの外側でスレッド並列になっ
  ているため、ここでOpenMPを記述する必要はない。

\item {\bf 引数}

  epi: 入力。const EPI *型またはEPI *型。i粒子情報を持つ配列。

  ni: 入力。const PS::S32型またはPS::S32型。i粒子数。

  epj: 入力。const EPJ *型またはEPJ *型。j粒子情報を持つ配列。
  
  nj: 入力。const PS::S32型またはPS::S32型。j粒子数。

  result: 出力。Result *型。i粒子の相互作用結果を返す配列。

\item {\bf 返値}

  なし。
  
\item {\bf 機能}

  j粒子からi粒子への作用を計算する。
  
\item {\bf 備考}

  引数名すべて変更可能。関数オブジェクトの内容などはすべて変更可能。
  
\end{itemize}
\fi

\begin{lstlisting}[caption=calcForceEpEp]
class Result;
class EPI;
class EPJ;
struct gravityEpEp {
    void operator () (const EPI *epi,
                      const PS::S32 ni,
                      const EPJ *epj,
                      const PS::S32 nj,
                      Result *result);
};
\end{lstlisting}

\begin{itemize}

\item {\bf 引数}

  epi: 入力。const EPI *型またはEPI *型。i粒子情報を持つ配列。

  ni: 入力。const PS::S32型またはPS::S32型。i粒子数。

  epj: 入力。const EPJ *型またはEPJ *型。j粒子情報を持つ配列。
  
  nj: 入力。const PS::S32型またはPS::S32型。j粒子数。

  result: 出力。Result *型。i粒子の相互作用結果を返す配列。

\item {\bf 返値}

  なし。
  
\item {\bf 機能}

  j粒子からi粒子への作用を計算する。
  
\end{itemize}


\subsection{関数オブジェクトcalcForceSpEp}
\label{sec:userdefined_calcForceSpEp}

\subsubsection{概要}

関数オブジェクトcalcForceSpEpは超粒子から粒子への作用を記述するもので
あり、相互作用の定義(節\ref{sec:overview_action}の手順0)に必要となる。
以下、これの書き方の規定を記述する。

\subsubsection{前提}

ここで示すのは重力N体シミュレーションにおける超粒子から粒子への作用の
記述の仕方である。超粒子は単極子までの情報で作られているものとする。関
数オブジェクトcalcForceSpEpの名前はgravitySpEpとする。これは変更自由で
ある。また、EssentialParitlceIクラスのクラス名をEPI, SuperParitlceJク
ラスのクラス名をSPJ, Forceクラスのクラス名をResultとする。

\subsubsection{gravitySpEp::operator ()}

\if 0
\begin{lstlisting}[caption=calcForceSpEp]
class Result {
public:
    PS::F32vec accfromspj;
};
class EPI {
public:
    PS::S32    id;
    PS::F32vec pos;
};
class SPJ {
public:
    PS::F32    mass;
    PS::F32vec pos;
};
struct gravitySpEp {
    static PS::F32 eps2;
    void operator () (const EPI *epi,
                      const PS::S32 ni,
                      const SPJ *spj,
                      const PS::S32 nj,
                      Result *result) {
                      
        for(PS::S32 i = 0; i < ni; i++) {
            PS::F32vec xi = epi[i].pos;
            PS::F32vec ai = 0.0;
            for(PS::S32 j = 0; j < nj; j++) {
                PS::F32    mj = spj[j].mass;
                PS::F32vec xj = spj[j].pos;

                PS::F32vec dx   = xi - xj;
                PS::F32    r2   = dx * dx + eps2;
                PS::F32    rinv = 1. / sqrt(r2);

                ai += mj * rinv * rinv * rinv * dx;
            }
            result.accfromspj = ai;
        }
    }
};
PS::F32 gravitySpEp::eps2 = 9.765625e-4;
\end{lstlisting}

\begin{itemize}

\item {\bf 前提}

  クラスResult, EPI, SPJに必要なメンバ関数は省略した。クラスResultのメ
  ンバ変数accfromspjはi粒子が超粒子から受ける重力加速度である。クラス
  EPIとSPJのメンバ変数posはそれぞれの粒子位置である。クラスSPJのメンバ
  変数massは超粒子の質量である。ファンクタgravitySpEpのメンバ変数eps2
  は重力ソフトニングの2乗である。ここの外側でスレッド並列になっている
  ため、ここでOpenMPを記述する必要はない。

\item {\bf 引数}

  epi: 入力。const EPI *型またはEPI *型。i粒子情報を持つ配列。

  ni: 入力。const PS::S32型またはPS::S32型。i粒子数。

  spj: 入力。const SPJ *型またはSPJ *型。超粒子情報を持つ配列。
  
  nj: 入力。const PS::S32型またはPS::S32型。超粒子数。

  result: 出力。Result *型。i粒子の相互作用結果を返す配列。

\item {\bf 返値}

  なし。
  
\item {\bf 機能}

  超粒子からi粒子への作用を計算する。
  
\item {\bf 備考}

  引数名すべて変更可能。関数オブジェクトの内容などはすべて変更可能。
  
\end{itemize}
\fi

\begin{lstlisting}[caption=calcForceSpEp]
class Result;
class EPI;
class SPJ;
struct gravitySpEp {
    void operator () (const EPI *epi,
                      const PS::S32 ni,
                      const SPJ *spj,
                      const PS::S32 nj,
                      Result *result);
};
\end{lstlisting}

\begin{itemize}

\item {\bf 引数}

  epi: 入力。const EPI *型またはEPI *型。i粒子情報を持つ配列。

  ni: 入力。const PS::S32型またはPS::S32型。i粒子数。

  spj: 入力。const SPJ *型またはSPJ *型。超粒子情報を持つ配列。
  
  nj: 入力。const PS::S32型またはPS::S32型。超粒子数。

  result: 出力。Result *型。i粒子の相互作用結果を返す配列。

\item {\bf 返値}

  なし。
  
\item {\bf 機能}

  超粒子からi粒子への作用を計算する。
  
\end{itemize}

\subsection{関数オブジェクトcalcForceDispatch}
\label{sec:userdefined_calcForceDispatch}


\subsubsection{概要}

関数calcForceDispatchは関数calcForceRetrieveと合わせて粒子同士の相互作
用を記述するものであり、calcForceSpEp やcalcForceEpEp の代わりに相互作
用の定義(節\ref{sec:overview_action}の手順0)に使うことができる。
calcForceSpEp やcalcForceEpEp との違いは、calcForceDispatch は複数の相
互作用リストと i粒子リストを受け取ることである。これにより、GPGPU 等の
アクセラレータを起動する回数を削減し、実行効率を向上させる。以下、これ
の書き方の規定を記述する。関数calcForceDispatchの名前はGravityDispatch
とする。これは変更自由である。また、EssentialParitlceIクラスのクラス名
をEPI, EssentialParitlceJクラスのクラス名をEPJ, SuperParitlceJクラスの
クラス名をSPJとする。

\subsubsection{短距離力の場合}

\begin{lstlisting}[caption=calcForceDispatch]
class EPI;
class EPJ;
PS::S32 HydroforceDispatch(const PS::S32  tag,
                           const PS::S32  nwalk,
                           const EPI**      epi,
                           const PS::S32*  ni,
                           const EPJ**      epj,
                           const PS::S32*  nj_ep;
};
\end{lstlisting}

\begin{itemize}

\item {\bf 引数}

  tag: 入力。const PS::S32 型。tagの番号。発行されるtagの番号は\\
  0から関数PS::TreeForForce::calcForceAllandWriteBackMultiWalk()の\\
  第三引数として設定された値から1引いた数までである。\\
  tagの番号はcalcForceRetrieve()で設定するtagの番号と対応させる必要がある。
    

  nwalk: 入力。const PS::S32 型。walkの数。walkの数の最大値は\\
  PS::TreeForForce::calcForceAllandWriteBackMultiWalk()の第六引数の値である。

  epi: 入力。const EPI** 型。i粒子情報を持つポインタのポインタ。

  ni: 入力。const PS::S32*型。i粒子数のポインタ。

  epj: 入力。const EPJ** 型。j粒子情報を持つポインタのポインタ。
  
  nj\_ep: 入力。const PS::S32* 型。j粒子数のポインタ。

\item {\bf 返値}

  PS::S32型。ユーザーは正常に実行された場合は0を、エラーが起こった場合
  は0以外の値を返すようにする。
  
\item {\bf 機能}

epi,epjの情報をアクセラレータに送り、相互作用カーネルを発行する。
  
\end{itemize}

\subsubsection{長距離力の場合}

\begin{lstlisting}[caption=calcForceDispatch]
class EPI;
class EPJ;
class SPJ;
PS::S32 GravityDispatch(const PS::S32   tag,
                        const PS::S32   nwalk,
                        const EPI**     epi,
                        const PS::S32*  ni,
                        const EPJ**     epj,
                        const PS::S32*  nj_ep,
                        const SPJ**     spj,
                        const PS::S32*  nj_sp);
};
\end{lstlisting}

\begin{itemize}

\item {\bf 引数}


  tag: 入力。const PS::S32 型。tagの番号。発行されるtagの番号は0から関
  数PS::TreeForForce::calcForceAllandWriteBackMultiWalk()の第三引数と
  して設定された値から1引いた数までである。tagの番号は
  calcForceRetrieve()で設定するtagの番号と対応させる必要がある。

  nwalk: 入力。const PS::S32 型。walkの数。walkの数の最大値は
  PS::TreeForForce::calcForceAllandWriteBackMultiWalk()の第六引数の値
  である。

  epi: 入力。const EPI** 型。i粒子情報を持つ配列の配列。

  ni: 入力。const PS::S32*型。i粒子数の配列。

  epj: 入力。const EPJ** 型。j粒子情報を持つ配列の配列。
  
  nj\_ep: 入力。const PS::S32* 型。j粒子数の配列。

  spj: 入力。const SPJ** 型。j粒子情報を持つ配列の配列。
  
  nj\_sp: 入力。const PS::S32* 型。j粒子数の配列。

\item {\bf 返値}

  PS::S32型。ユーザーは正常に実行された場合は0を、エラーが起こった場合
  は0以外の値を返すようにする。
  
\item {\bf 機能}

epi,epj,spjの情報をアクセラレータに送り、相互作用カーネルを発行する。
  
\end{itemize}


\subsection{関数オブジェクトcalcForceRetrieve}
\label{sec:userdefined_calcForceRetrieve}


\subsubsection{概要}

関数calcForceRetrieveは関数calcForceDispatchで行った相互作用の結果を回
収する関数である。以下、これの書き方の規定を記述する。関数
calcForceRetrieveの名前はGravityRetrieveとする。これは変更自由である。
また、Forceクラスのクラス名をResultとする。

\begin{lstlisting}[caption=calcForceDispatch]
class EPI;
class EPJ;
class Result;
PS::S32 GravityRetrieve(const PS::S32  tag,
                        const PS::S32  nwalk,
                        const PS::S32  ni     [],
                        Result         result [][]);
};
\end{lstlisting}

\begin{itemize}

\item {\bf 引数}


  tag: 入力。const PS::S32 型。tagの番号。対応する calcForceDispatch の
  tag 番号と一致させる必要がある。

  nwalk: 入力。const PS::S32 型。walkの数。対応する calcForceDispatch
  に与えた nwalk の値と一致させる必要がある。

  ni: 入力。const PS::S32*型。i粒子数の配列。
  
  result: 出力。Result *型。i粒子の相互作用結果を返す配列の配列。

\item {\bf 返値}

  PS::S32型。ユーザーは正常に実行された場合は0を、エラーが起こった場合
  は0以外の値を返すようにする。
  

\item {\bf 機能}

同じtag番号を持つ関数calcForceDispatchで行った相互作用の結果を回収する。

\end{itemize}



%\subsection{関数オブジェクトcomp}
%\label{sec:userdefined_comp}
%\input{userdefined_comp.tex}
