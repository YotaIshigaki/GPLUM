\subsubsection{概要}

本節では、PS::SEARCH\_MODE型について記述する。PS::SEARCH\_MODE型は相互
作用ツリークラスのテンプレート引数としてのみ使用されるものである。この
型によって、相互作用ツリークラスで計算する相互作用のモードを決定する。
PS::SEARCH\_MODE型には以下がある:
\begin{itemize}[leftmargin=*,itemsep=-1ex]
\item PS::SEARCH\_MODE\_LONG
\item PS::SEARCH\_MODE\_LONG\_CUTOFF
\item PS::SEARCH\_MODE\_GATHER
\item PS::SEARCH\_MODE\_SCATTER
\item PS::SEARCH\_MODE\_SYMMETRY
\item PS::SEARCH\_MODE\_LONG\_SCATTER
\item PS::SEARCH\_MODE\_LONG\_SYMMETRY
\item PS::SEARCH\_MODE\_LONG\_CUTOFF\_SCATTER
\end{itemize}
以下で、それぞれが対応する相互作用のモードについて記述する。

\subsubsection{PS::SEARCH\_MODE\_LONG}

この型を使用するのは、遠くの粒子からの寄与を複数の粒子にまとめた超粒子
からの寄与として計算する場合である。開放境界条件における重力やクーロン
力に適用できる(周期境界条件では使用不可能)。

\subsubsection{PS::SEARCH\_MODE\_LONG\_CUTOFF}

この型を使用するのは、遠くの粒子からの寄与を複数の粒子にまとめた超粒子
からの寄与として計算し、かつ有限の距離までの寄与しか計算しない場合であ
る。周期境界条件における重力やクーロン力(Particle Mesh法の並用が必要)
などに適用できる。

\subsubsection{PS::SEARCH\_MODE\_GATHER}

この型を使用するのは、相互作用の到達距離が有限でかつ、その到達距離がi
粒子の大きさで決まる場合である。

\subsubsection{PS::SEARCH\_MODE\_SCATTER}

この型を使用するのは、相互作用の到達距離が有限でかつ、その到達距離がj
粒子の大きさで決まる場合である。

\subsubsection{PS::SEARCH\_MODE\_SYMMETRY}

この型を使用するのは、相互作用の到達距離が有限でかつ、その到達距離がi,
j粒子のうち大きいほうのサイズで決まる場合である。

\subsubsection{PS::SEARCH\_MODE\_LONG\_SCATTER}

基本的にはSEARCH\_MODE\_LONGと同じであるが、i粒子とj粒子の距離がj粒子の
探査半径よりも短い場合は、そのj粒子は超粒子に含めない(超粒子としてではなく粒子として扱う)。

\subsubsection{PS::SEARCH\_MODE\_LONG\_SYMMETRY}

基本的にはSEARCH\_MODE\_LONGと同じであるが、i粒子とj粒子の距離が、i粒子とj粒子の
探査半径のどちらか大きい方よりも短い場合は、そのj粒子は超粒子に含めない(超粒子としてではなく粒子として扱う)。


\subsubsection{PS::SEARCH\_MODE\_LONG\_CUTOFF\_SCATTER}

未実装。
