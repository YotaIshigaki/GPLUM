\begin{itemize}
\item 2015.03.13
  \begin{itemize}
  \item 様々なユーザー定義クラスのメンバ関数でgetRsearchとなっていたも
    のをgetRSearchと訂正(節\ref{sec:userdefined})
  \item 異常終了させる関数PS::Abortについての記述を追加(節
    \ref{sec:initfin})
  \end{itemize}
\item 2015.03.17
  \begin{itemize}
  \item バージョン1.0リリース
  \end{itemize}
\item 2015.03.18
  \begin{itemize}
  \item Particle Meshクラス関連のライセンス事項を修正。
  \end{itemize}
\item 2015.03.20
  \begin{itemize}
  \item PS::Comm::broadcastの記述が抜けていたので追加。
  \end{itemize}
\item 2015.04.01
  \begin{itemize}
  \item Particle Meshクラスはプロセス数が2以上でないと動かないという注
    意事項を追加。
  \end{itemize}
  \item 2015.10.07
  \begin{itemize}
  \item PM法についての記述を追加。セクション
    \ref{sec:module_extended_PM}。
  \end{itemize}
\item 2015.12.01
  \begin{itemize}
  \item Multi Walkについての記述を追加。セクション
    \ref{sec:example_userdefined_calcForceDispatch}、
    \ref{sec:example_userdefined_calcForceRetrieve}、
    \ref{sec:module_standard_treeforforce_calcforceallandwriteback}、
    \ref{sec:userdefined_calcForceDispatch}、
    \ref{sec:userdefined_calcForceRetrieve}。
  \end{itemize}
\item 2016.02.07
  \begin{itemize}
  \item getNeighborListOneParticle についての記述を追加。セクション
    \ref{sec:neighborlist}。
\item それに伴い、PS:: MomentMonopoleScatter、PS::MomentQuadrupoleScatter 
  の記述を追加。セクション \ref{sec:MomentForSearchModeLong}。
\item さらに、これらのラッパーである MonopoleWithScatter および
  QuadrupoleWithScatter の記述を
  セクション \ref{sec:module_treeforce_standard_search_mode_long} に追加。
  \end{itemize}

  \item 2016.09.13
  \begin{itemize}
  \item 粒子の追加、削除のAPIをセクション\ref{sec:addAndRemoveParticle}に追加。
  \item Vector型の範囲外アクセスについての記述をセクション
    \ref{sec:errormessage:vector_invalid_access}に追加。
%  \item Vector型の範囲外アクセスについての記述をセクション
%    \ref{sec:compile:vector_invalid_access}と
%    \ref{sec:errormessage:vector_invalid_access}に追加。
  \end{itemize}

    \item 2016.10.11
  \begin{itemize}
  \item Vector型の演算子''[]''の実装変更。それによる注意事項を追加セク
    ション\ref{sec:datatype_vector}に追加。
  \end{itemize}

  \item 2016.11.04
  \begin{itemize}
  \item ファイル読み込みAPIをセクション
    \ref{sec:readParticleAscii},\ref{sec:readParticleBinary},\ref{sec:writeParticleAscii},\ref{sec:writeParticleBinary}
    に追加
  \end{itemize}

    \item 2017.07.11
  \begin{itemize}
  \item 相互作用リストの再利用に関する記述をセクション
    \ref{sec:treeForForceHighLevelAPI}に追加。
  \end{itemize}

    \item 2017.09.06
  \begin{itemize}
  \item 粒子群クラスの粒子配列の並び替えに関する記述をセクション
    \ref{sec:ParticleSystem:sortParticle}に追加。
  \item 粒子のid番号からEPJの情報を取り出す関数に関する記述をセクショ
    ン\ref{sec:EPJ:getId},\ref{sec:getEpjFromId}に追加。
  \end{itemize}

  \item 2017.11.08
    \begin{itemize}
    \item FDPS 4.0 リリース
    \end{itemize}

  \item 2018.01.16
    \begin{itemize}
    \item FDPS 4.0b リリース
      \begin{itemize}
      \item メモリ割付に関するバグを修正。
      \item TreeForForceクラス等にデストラクタを追加。
      \end{itemize}
    \end{itemize}
    
  \item 2018.01.23
    \begin{itemize}
    \item 粒子交換時に粒子データをシリアライズする方法をセクション
          \ref{sec:FP:serialize}, \ref{sec:EPJ:serialize}, \ref{sec:SPJ:serialize},
          \ref{sec:particleSystem:exchangeParticle}, \ref{sec:treeForForceHighLevelAPI}
          に追加。
    \end{itemize}

  \item 2018.05.25
    \begin{itemize}
    \item FDPS 4.0c リリース
      \begin{itemize}
      \item Particle Mesh 拡張の実装からMPI の C++ biding を削除。
      \end{itemize}
    \end{itemize}

  \item 2018.06.13
    \begin{itemize}
    \item FDPS 4.1 リリース
      \begin{itemize}
      \item Tree法で$\theta=0$の場合にも超粒子が発生しうる場合があるバグを修正。
      \item オルソトープ型の記述を第\ref{sec:datatype_orthotope}節に追加。
      \item Momentクラスのセクションに、場合によっては必要なメンバ関数についての記述を追加。
      \end{itemize}
    \end{itemize}

  \item 2018.06.20
    \begin{itemize}
    \item FDPS 4.1a リリース
      \begin{itemize}
      \item 仕様書(本書)の変更履歴のセクションの記載もれを修正。
      \end{itemize}
    \end{itemize}

  \item 2018.08.2
    \begin{itemize}
    \item 新しいAPI \texttt{getBoundaryCondition}を追加
    \item API \texttt{collectSampleParticle}の引数\texttt{weight}の意味についての説明を追加
    \end{itemize}

  \item 2018.12.7
    \begin{itemize}
    \item FDPS 5.0a リリース
       \begin{itemize}
          \item Momentクラス 及び SuperParticleJクラスの節(第\ref{sec:moment},\ref{sec:superparticlej}節)に、\newline
          PS::SEARCH\_MODE\_LONG\_SCATTER \newline
          PS::SEARCH\_MODE\_LONG\_SYMMETRY \newline
          に対して用意されている既存のクラスの説明を追加。
          \item FullParticleクラス 及び ヘッダクラスの節(第\ref{sec:fullparticle}, \ref{sec:userdefined_header}節)に、ファイル入出力 API を使う場合に必要となるメンバ関数 readBinary 及び writeBinary についての説明を追加。同様、これらの実装例を、それぞれ第\ref{sec:example_fullparticle}, \ref{sec:example_userdefined_header}節に追加。
       \end{itemize}
    \end{itemize}

  \item 2018.12.9
    \begin{itemize}
    \item FDPS 5.0b リリース
       \begin{itemize}
          \item 仕様書のAPI getNeighborListOneParticleの記述が不正確であった問題を修正(ソースコードには変更ありません)。
       \end{itemize}
    \end{itemize}

  \item 2019.1.25
    \begin{itemize}
    \item FDPS 5.0c リリース
       \begin{itemize}
          \item 周期境界条件で、かつ、特定の粒子分布が与えられた場合にFDPS内部の関数LinkCellでエラーが発生する問題を修正。
       \end{itemize}
    \end{itemize}

  \item 2019.3.1
    \begin{itemize}
    \item FDPS 5.0d リリース
       \begin{itemize}
          \item 孤立境界条件及び全方向周期境界条件以外の境界条件で、ツリーのルートセルの設定が正しく行われない問題を修正。
          \item FDPS バージョン5.0から5.0cにおいて、一部の\texttt{SEARCH\_MODE}で、\texttt{PS::TimeProfile}のメンバ変数\texttt{make\_LET\_1st}, \texttt{make\_LET\_2nd}, \texttt{exchange\_LET\_1st}, \texttt{exchange\_LET\_2nd}に値が設定されない問題を修正。時間測定の範囲の変更のため、個々の変数の値はFDPS バージョン4.1以前のものと互換性がないが、これらの和については互換性がある。
          \item FDPSが内部で使用するメモリプールのサイズが不足していた場合、セグメンテーション違反によりコードが停止してしまう問題を修正。
          \item 本リリースからFDPS内でC++11の機能を使用している。\textcolor{red}{そのため、C++コンパイラ、及び、CUDAを利用する場合にはCUDAコンパイラに適切なオプション(GNU gccであれば\texttt{-std=c++11})をつける必要がある。}
       \end{itemize}
    \end{itemize}

  \item 2019.3.7
    \begin{itemize}
    \item TreeForForceLong$\langle$,,$\rangle$::MonopoleWithCutoff に関する記述を改善。
    \end{itemize}

  \item 2019.9.6
    \begin{itemize}
    \item FDPS 5.0f リリース
       \begin{itemize}
          \item コンパイル時にマクロ\texttt{PARTICLE\_SIMULATOR\_TWO\_DIMENSION}を定義した場合、コンパイルエラーになる問題を修正。
       \end{itemize}
    \end{itemize}

  \item 2019.9.10
    \begin{itemize}
    \item FDPS 5.0g リリース
       \begin{itemize}
          \item コンパイル時にマクロ\texttt{PARTICLE\_SIMULATOR\_TWO\_DIMENSION}を定義した場合、実行時エラーになる問題を修正。
       \end{itemize}
    \end{itemize}

  \item 2020.8.16
    \begin{itemize}
    \item FDPS 6.0 リリース
       \begin{itemize}
          \item PIKGを導入
       \end{itemize}
    \end{itemize}
    
  \item 2020.8.18
    \begin{itemize}
    \item FDPS 6.0a リリース
       \begin{itemize}
          \item 付属するPIKGのバージョンをv0.1bに更新
       \end{itemize}
    \end{itemize}

  \item 2020.8.19
    \begin{itemize}
    \item FDPS 6.0b リリース
       \begin{itemize}
          \item $N$体シミュレーションサンプルコード(\path{sample/*/nbody})の実装をPIKGで生成したカーネルを使った場合に性能が出るように改善
       \end{itemize}
    \end{itemize}

  \item 2020.8.28
    \begin{itemize}
    \item FDPS 6.0b1 リリース
       \begin{itemize}
          \item サンプルコードで使用可能な初期条件ファイルの配布先が変更になったため、チュートリアルを修正
       \end{itemize}
    \end{itemize}


  \item 2020.9.02
    \begin{itemize}
    \item FDPS 6.0b2 リリース
       \begin{itemize}
          \item サンプルコードのTreeForForceクラスのオブジェクトを初期化する関数の引数を修正
          \item 上記に合わせて、チュートリアルも修正
       \end{itemize}
    \end{itemize}

\end{itemize}
