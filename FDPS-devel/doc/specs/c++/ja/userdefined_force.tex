\subsubsection{概要}

Forceクラスは相互作用の結果を保持するクラスであり、相互作用の定義
(節\ref{sec:overview_action}の手順0)に必要となる。以下、この節の前提、
常に必要なメンバ関数について記述する。

\subsubsection{前提}

この節で用いる例としてForceクラスのクラス名をResultとする。このクラス名
は変更自由である。

\subsubsection{必要なメンバ関数}

常に必要なメンバ関数はResult::clearである。この関数は相互作用の計算結果を初期
化する。以下、Result::clearについて記述する。

\subsubsubsection{Result::clear}

\if 0
\begin{screen}
\begin{verbatim}
class Result {
public:
    PS::F32vec acc;
    PS::F32    pot;
    void clear() {
        acc = 0.0;
        pot = 0.0;
    }
};
\end{verbatim}
\end{screen}

\begin{itemize}

\item {\bf 前提}
  
  Resultクラスのメンバ変数はaccとpot。
  
\item {\bf 引数}

  なし
  
\item {\bf 返値}

  なし。
  
\item {\bf 機能}

  Resultクラスのメンバ変数を初期化する。
  
\item {\bf 備考}

  Resultクラスのメンバ変数acc, potの変数名は変更可能。他のメンバ変数を
  加えることも可能。

\end{itemize}
\fi

\begin{screen}
\begin{verbatim}
class Result {
public:
    void clear();
};
\end{verbatim}
\end{screen}

\begin{itemize}

\item {\bf 引数}

  なし
  
\item {\bf 返値}

  なし。
  
\item {\bf 機能}

  Resultクラスのメンバ変数を初期化する。
  
\end{itemize}
