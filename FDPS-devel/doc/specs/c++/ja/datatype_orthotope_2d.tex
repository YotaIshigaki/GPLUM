\texttt{PS::Orthotope2}は\texttt{PS::Vector2}型のメンバ変数\texttt{low\_}, \texttt{high\_}を持つ。
これらに対する様々なAPIや演算子を定義した。それらの宣言を以下に記述する。
この節ではこれらについて詳しく記述する。
\begin{lstlisting}[caption=Orthotope2]
namespace ParticleSimulator{
    template<class T>
    class Orthotope2{
    public:
        Vector2<T> low_;
        Vector2<T> high_;

        Orthotope2(): low_(9999.9), high_(-9999.9){}
        
        Orthotope2(const Vector2<T> & _low, const Vector2<T> & _high)
            : low_(_low), high_(_high){}
        
        Orthotope2(const Orthotope2 & src) : low_(src.low_), high_(src.high_){}

        Orthotope2(const Vector2<T> & center, const T length) :
            low_(center-(Vector2<T>)(length)), high_(center+(Vector2<T>)(length)) {
        }

        void initNegativeVolume(){
            low_ = std::numeric_limits<float>::max() / 128;
            high_ = -low_;
        }

        void init(){
            initNegativeVolume();
        }

        void merge( const Orthotope2 & ort ){
            this->high_.x = ( this->high_.x > ort.high_.x ) ? this->high_.x : ort.high_.x;
            this->high_.y = ( this->high_.y > ort.high_.y ) ? this->high_.y : ort.high_.y;
            this->low_.x = ( this->low_.x <= ort.low_.x ) ? this->low_.x : ort.low_.x;
            this->low_.y = ( this->low_.y <= ort.low_.y ) ? this->low_.y : ort.low_.y;
        }

        void merge( const Vector2<T> & vec ){
            this->high_.x = ( this->high_.x > vec.x ) ? this->high_.x : vec.x;
            this->high_.y = ( this->high_.y > vec.y ) ? this->high_.y : vec.y;
            this->low_.x = ( this->low_.x <= vec.x ) ? this->low_.x : vec.x;
            this->low_.y = ( this->low_.y <= vec.y ) ? this->low_.y : vec.y;
        }

        void merge( const Vector2<T> & vec, const T size){
            this->high_.x = ( this->high_.x > vec.x + size ) ? this->high_.x : vec.x + size;
            this->high_.y = ( this->high_.y > vec.y + size ) ? this->high_.y : vec.y + size;
            this->low_.x = ( this->low_.x <= vec.x - size) ? this->low_.x : vec.x - size;
            this->low_.y = ( this->low_.y <= vec.y - size) ? this->low_.y : vec.y - size;
        }
    };
}
namespace PS = ParticleSimulator;
\end{lstlisting}

%%%%%%%%%%%%%%%%%%%%%%%%%%%%%
\subsubsubsection{メンバ変数}

\begin{screen}
\begin{verbatim}
template<typename T>
PS::Vector2<T> PS::Orthotope2<T>::low_;

template<typename T>
PS::Vector2<T> PS::Orthotope2<T>::high_;
\end{verbatim}
\end{screen}

\begin{itemize}
  
\item{{\bf 機能}}
  
メンバ変数\texttt{low\_}、\texttt{high\_}を直接操作出来る。
  
\end{itemize}


%%%%%%%%%%%%%%%%%%%%%%%%%%%%%%%%
\subsubsubsection{コンストラクタ}
%---------------
\begin{screen}
\begin{verbatim}
template<typename T>
PS::Orthotope2<T>::Orthotope2();
\end{verbatim}
\end{screen}

\begin{itemize}

\item{{\bf 引数}}

なし。ただし、テンプレート引数Tは \texttt{PS::F32} または \texttt{PS::F64} でなければならない。

\item{{\bf 機能}}

デフォルトコンストラクタ。
メンバ変数\texttt{low\_}, \texttt{high\_}は、それぞれ、(9999.9, 9999.9), (-9999.9, -9999.9)で初期化される。

\end{itemize}
%---------------
\begin{screen}
\begin{verbatim}
template<typename T>
PS::Orthotope2<T>::Orthotope2(const Vector2<T> _low, const Vector2<T> _high);
\end{verbatim}
\end{screen}

\begin{itemize}

\item{{\bf 引数}}

\texttt{\_low}: 入力。\texttt{const Vector2$<$T$>$}型。

\texttt{\_high}: 入力。\texttt{const Vector2$<$T$>$}型。

ここで、\texttt{T} は \texttt{PS::F32} または \texttt{PS::F64} でなければならない。

\item{{\bf 機能}}

メンバ変数\texttt{low\_}、\texttt{high\_}を、それぞれ\texttt{\_low}、\texttt{\_high}で初期化する。

\end{itemize}
%---------------
\begin{screen}
\begin{verbatim}
template<typename T>
PS::Orthotope2<T>::Orthotope2(const Vector2<T> & center, const T length);
\end{verbatim}
\end{screen}

\begin{itemize}

\item{{\bf 引数}}

\texttt{center}: 入力。\texttt{const Vector2$<$T$>$ \&}

\texttt{length}: 入力。\texttt{const T}型。

\item{{\bf 機能}}

メンバ変数\texttt{low\_}、\texttt{high\_}を、それぞれ、\texttt{center-(Vector2$<$T$>$)(length)}、\texttt{center+(Vector2$<$T$>$)(length)}で初期化する。

\end{itemize}


%%%%%%%%%%%%%%%%%%%%%%%%%%%%%%%%
\subsubsubsection{コピーコンストラクタ}
%---------------
\begin{screen}
\begin{verbatim}
template<typename T>
PS::Orthotope2<T>::Orthotope2(const Orthotope2<T> & src);
\end{verbatim}
\end{screen}

\begin{itemize}

\item{{\bf 引数}}

\texttt{src}: 入力。\texttt{const Orthotope2$<$T$>$ \&}型。

\item{{\bf 機能}}

メンバ変数\texttt{low\_}、\texttt{high\_}を、それぞれ、\texttt{src.low\_}、\texttt{src.high\_}で初期化する。

\end{itemize}


%%%%%%%%%%%%%%%%%%%%%%%%%%%%%
\subsubsubsection{初期化}
%---------------
\begin{screen}
\begin{verbatim}
template<typename T>
PS::Orthotope2<T>::initNegativeVolume();
\end{verbatim}
\end{screen}

\begin{itemize}

\item{{\bf 引数}}

なし。

\item{{\bf 機能}}

メンバ変数\texttt{low\_}、\texttt{high\_}を、それぞれ、($a$, $a$)、(-$a$, -$a$)で初期化する。
ここで、$a$=\texttt{std::numeric\_limits$<$float$>$::max() / 128}である。

\end{itemize}

%---------------
\begin{screen}
\begin{verbatim}
template<typename T>
PS::Orthotope2<T>::init();
\end{verbatim}
\end{screen}

\begin{itemize}

\item{{\bf 引数}}

なし。

\item{{\bf 機能}}

メンバ関数 \texttt{initNegativeVolume} と同じ機能を提供する。

\end{itemize}

%%%%%%%%%%%%%%%%%%%%%%%%%%%%%
\subsubsubsection{結合操作}
%---------------
\begin{screen}
\begin{verbatim}
template<typename T>
void PS::Orthotope2<T>::merge( const Orthotope2 & ort )();
\end{verbatim}
\end{screen}

\begin{itemize}

\item{{\bf 引数}}

\texttt{ort}: 入力。\texttt{const Orthotope2 \&}型。

\item{{\bf 機能}}

メンバ変数\texttt{low\_}、\texttt{high\_}を、当該長方形と長方形\texttt{ort}を包含する最小の長方形を記述するように更新する。

\end{itemize}
%---------------
\begin{screen}
\begin{verbatim}
template<typename T>
void PS::Orthotope2<T>::merge( const Vector2<T> & vec )();
\end{verbatim}
\end{screen}

\begin{itemize}

\item{{\bf 引数}}

\texttt{vec}: 入力。\texttt{const Vector2$<$T$>$ \&}型。

\item{{\bf 機能}}

メンバ変数\texttt{low\_}、\texttt{high\_}を、点\texttt{vec}を包含するように更新する。

\end{itemize}
%---------------
\begin{screen}
\begin{verbatim}
template<typename T>
void PS::Orthotope2<T>::merge( const Vector2<T> & vec, const T size )();
\end{verbatim}
\end{screen}

\begin{itemize}

\item{{\bf 引数}}

\texttt{vec}: 入力。\texttt{const Vector2$<$T$>$ \&}型。

\texttt{size}: 入力。\texttt{const T}型。

\item{{\bf 機能}}

メンバ変数\texttt{low\_}、\texttt{high\_}を、中心\texttt{vec}、半径\texttt{size}の円を包含するように更新する。

\end{itemize}

