本節では、粒子群クラスについて記述する。このクラスは粒子情報の保持や
MPIプロセス間で粒子情報の交換を行うモジュールである。まずオブジェクト
の生成方法を記述し、その後APIを記述する。

%%%%%%%%%%%%%%%%%%%%%%%%%%%%%%%%%%%%%%%%%%%%%%%%%%%%%%%%%%%%%%%%%
\subsubsubsection{オブジェクトの生成}

粒子群クラスは以下のように宣言されている。
\begin{lstlisting}[caption=ParticleSystem0]
namespace ParticleSimulator {
    template<class Tptcl>
    class ParticleSystem;
}
\end{lstlisting}
テンプレート引数Tptclはユーザー定義のFullParticleクラスである。

粒子群クラスのオブジェクトの生成は以下のように行う。ここではsystemとい
うオブジェクトを生成している。
\begin{screen}
\begin{verbatim}
PS::ParticleSystem<FP> system;
\end{verbatim}
\end{screen}
テンプレート引数FPはユーザー定義のFullParticleクラスの1例であるFPクラ
スである。

%%%%%%%%%%%%%%%%%%%%%%%%%%%%%%%%%%%%%%%%%%%%%%%%%%%%%%%%%%%%%%%%%
\subsubsubsection{API}

このモジュールには初期設定関連のAPI、オブジェクト情報取得設定関連のAPI、
ファイル入出力関連のAPI、粒子交換関連のAPIがある。以下、各節に分けて記
述する。

%%%%%%%%%%%%%%%%%%%%%%%%%%%%%%%%%%%%%%%%%%%%%%%%%%%%%%
\subsubsubsubsection{初期設定}

初期設定関連のAPIの宣言は以下のようになっている。このあと各APIについて
記述する。
\begin{lstlisting}[caption=ParticleSystem1]
namespace ParticleSimulator {
    template<class Tptcl>
    class ParticleSystem{
    public:
        ParticleSystem();
        void initialize();
        void setAverageTargetNumberOfSampleParticlePerProcess
                        (const S32 & nsampleperprocess);
    };
}
\end{lstlisting}


%%%%%%%%%%%%%%%%%%%%%%%%%%%%%%%%%%%%%%%%%%%
\subsubsubsubsubsection{コンストラクタ}

\begin{screen}
\begin{verbatim}
template <class Tptcl>
void PS::ParticleSystem<Tptcl>::ParticleSystem();
\end{verbatim}
\end{screen}

\begin{itemize}

\item {\bf 引数}

なし

\item {\bf 返値}

なし

\item {\bf 機能}

粒子群クラスのオブジェクトを生成する。

\end{itemize}

%%%%%%%%%%%%%%%%%%%%%%%%%%%%%%%%%%%%%%%%%%%
\subsubsubsubsubsection{PS::ParticleSystem::initialize}

\begin{screen}
\begin{verbatim}
template <class Tptcl>
void PS::ParticleSystem<Tptcl>::initialize();
\end{verbatim}
\end{screen}

\begin{itemize}

\item {\bf 引数}

なし

\item {\bf 返値}

なし

\item {\bf 機能}

粒子群クラスのオブジェクトを初期化する。1度は呼ぶ必要がある。


\end{itemize}

%%%%%%%%%%%%%%%%%%%%%%%%%%%%%%%%%%%%%%%%%%%
\subsubsubsubsubsection{PS::ParticleSystem::\\setAverateTargetNumberOfSampleParticlePerProcess}

\begin{screen}
\begin{verbatim}
template <class Tptcl>
void PS::ParticleSystem<Tptcl>::setAverateTargetNumberOfSampleParticlePerProcess
                 (const PS::S32 & nsampleperprocess);
\end{verbatim}
\end{screen}

\begin{itemize}

\item {\bf 引数}

nsampleperprocess: 入力。const PS::S32 \&型。1つのMPIプロセスでサンプル
する粒子数目標。

\item {\bf 返値}

なし

\item {\bf 機能}

1つのMPIプロセスでサンプルする粒子数の目標を設定する。呼び出さなくて
もよいが、呼び出さないとこの目標数が30となる。

\end{itemize}

%%%%%%%%%%%%%%%%%%%%%%%%%%%%%%%%%%%%%%%%%%%%%%%%%%%%%%
\subsubsubsubsection{情報取得}

オブジェクト情報取得関連のAPIの宣言は以下のようになっている。このあと
各APIについて記述する。

\begin{lstlisting}[caption=ParticleSystem2]
namespace ParticleSimulator {
    template<class Tptcl>
    class ParticleSystem{
    public:
        Tptcl & operator [] (const S32 id);
        S32 getNumberOfParticleLocal() const;
        S32 getNumberOfParticleGlobal() const;
        S64 getUsedMemorySize() const;
    };
}
\end{lstlisting}

%%%%%%%%%%%%%%%%%%%%%%%%%%%%%%%%%%%%%%%%%%%
\subsubsubsubsubsection{PS::ParticleSystem::operator []}

\begin{screen}
\begin{verbatim}
template <class Tptcl>
Tptcl & PS::ParticleSystem<Tptcl>::operator []
             (const PS::S32 id);
\end{verbatim}
\end{screen}

\begin{itemize}

\item {\bf 引数}

id: 入力。const PS::S32型。粒子配列のインデックス。

\item {\bf 返値}

FullParticle \&型。Tptcl型のオブジェクト。

\item {\bf 機能}

id番目のFullParticle型のオブジェクトの参照を返す。

\end{itemize}



%%%%%%%%%%%%%%%%%%%%%%%%%%%%%%%%%%%%%%%%%%%
\subsubsubsubsubsection{PS::ParticleSystem::getNumberOfParticleLocal}

\begin{screen}
\begin{verbatim}
template <class Tptcl>
PS::S32 PS::ParticleSystem<Tptcl>::getNumberOfParticleLocal();
\end{verbatim}
\end{screen}

\begin{itemize}

\item {\bf 引数}

なし

\item {\bf 返値}

PS::S32型。1つのMPIプロセスの持つ粒子数。

\item {\bf 機能}

1つのMPIプロセスの持つ粒子数を返す。

\end{itemize}

%%%%%%%%%%%%%%%%%%%%%%%%%%%%%%%%%%%%%%%%%%%
\subsubsubsubsubsection{PS::ParticleSystem::getNumberOfParticleGlobal}

\begin{screen}
\begin{verbatim}
template <class Tptcl>
PS::S32 PS::ParticleSystem<Tptcl>::getNumberOfParticleGlobal();
\end{verbatim}
\end{screen}

\begin{itemize}

\item {\bf 引数}

なし

\item {\bf 返値}

PS::S32型。全MPIプロセスの持つ粒子数。

\item {\bf 機能}

全MPIプロセスの持つ粒子数を返す。

\end{itemize}

%%%%%%%%%%%%%%%%%%%%%%%%%%%%%%%%%%%%%%%%%%%
\subsubsubsubsubsection{PS::DomainInfo::getUsedMemorySize}
\begin{screen}
\begin{verbatim}
PS::S64 PS::DomainInfo::getUsedMemorySize() const;
\end{verbatim}
\end{screen}

\begin{itemize}

\item {\bf 引数}

なし。

\item {\bf 返値}

PS::S64。

\item {\bf 機能}

対象のオブジェクトが使用しているメモリー量をByte単位で返す。

\end{itemize}

%%%%%%%%%%%%%%%%%%%%%%%%%%%%%%%%%%%%%%%%%%%%%%%%%%%%%%
%%%%%%%%%%%%%%%%%%%%%%%%%%%%%%%%%%%%%%%%%%%%%%%%%%%%%%
%%%%%%%%%%%%%%%%%%%%%%%%%%%%%%%%%%%%%%%%%%%%%%%%%%%%%%

\subsubsubsubsection{ファイル入出力}
\label{sec:ParticleSystem:IO}

ファイル入出力関連の API の宣言は以下のようになっている。
\begin{lstlisting}[caption=ParticleSystem3]
namespace ParticleSimulator {
    template<class Tptcl>
    class ParticleSystem{
    public:
        template <class Theader>
        void readParticleAscii(const char * const filename,
                               const char * const format,
                               Theader & header);
        void readParticleAscii(const char * const filename,       
                               const char * const format);
        template <class Theader>
        void readParticleAscii(const char * const filename,
                               Theader & header);
        void readParticleAscii(const char * const filename);
        template <class Theader>
        void readParticleAscii(const char * const filename,
                               const char * const format,
                               Theader & header,
                               void (Tptcl::*pFunc)(FILE*));
        void readParticleAscii(const char * const filename,       
                               const char * const format,
                               void (Tptcl::*pFunc)(FILE*));
        template <class Theader>
        void readParticleAscii(const char * const filename,
                               Theader & header,
                               void (Tptcl::*pFunc)(FILE*));
        void readParticleAscii(const char * const filename,
                               void (Tptcl::*pFunc)(FILE*) );

        template <class Theader>
        void readParticleBinary(const char * const filename,
                               const char * const format,
                               Theader & header);
        void readParticleBinary(const char * const filename,       
                               const char * const format);
        template <class Theader>
        void readParticleBinary(const char * const filename,
                               Theader & header);
        void readParticleBinary(const char * const filename);
        template <class Theader>
        void readParticleBinary(const char * const filename,
                               const char * const format,
                               Theader & header,
                               void (Tptcl::*pFunc)(FILE*));
        void readParticleBinary(const char * const filename,       
                               const char * const format,
                               void (Tptcl::*pFunc)(FILE*));
        template <class Theader>
        void readParticleBinary(const char * const filename,
                               Theader & header,
                               void (Tptcl::*pFunc)(FILE*));
        void readParticleBinary(const char * const filename,
                               void (Tptcl::*pFunc)(FILE*) );

        template <class Theader>
        void writeParticleAscii(const char * const filename,
                                const char * const format,
                                const Theader & header);        
        void writeParticleAscii(const char * const filename,
                                const char * format);                                       
        template <class Theader>
        void writeParticleAscii(const char * const filename,
                                const Theader & header);
        void writeParticleAscii(const char * const filename);
        template <class Theader>
        void writeParticleAscii(const char * const filename,
                                const char * const format,
                                const Theader & header,
                                void (Tptcl::*pFunc)(FILE*)const);   
        void writeParticleAscii(const char * const filename,
                                const char * format,
                                void (Tptcl::*pFunc)(FILE*)const);
        template <class Theader>
        void writeParticleAscii(const char * const filename,
                                const Theader & header,
                                void (Tptcl::*pFunc)(FILE*)const);
        void writeParticleAscii(const char * const filename,
                                void (Tptcl::*pFunc)(FILE*)const);

        template <class Theader>
        void writeParticleBinary(const char * const filename,
                                const char * const format,
                                const Theader & header);        
        void writeParticleBinary(const char * const filename,
                                const char * format);                                       
        template <class Theader>
        void writeParticleBinary(const char * const filename,
                                const Theader & header);
        void writeParticleBinary(const char * const filename);
        template <class Theader>
        void writeParticleBinary(const char * const filename,
                                const char * const format,
                                const Theader & header,
                                void (Tptcl::*pFunc)(FILE*)const);   
        void writeParticleBinary(const char * const filename,
                                const char * format,
                                void (Tptcl::*pFunc)(FILE*)const);
        template <class Theader>
        void writeParticleBinary(const char * const filename,
                                const Theader & header,
                                void (Tptcl::*pFunc)(FILE*)const);
        void writeParticleBinary(const char * const filename,
                                void (Tptcl::*pFunc)(FILE*)const);
    };
}
\end{lstlisting}


上記に示すようにファイル入出力関連のAPIにはreadParticleAscii, readParticleBinary, writeParticleAscii, writeParticleBinaryが存在する。これらはオーバーロードされた関数であり、それぞれ8種類の異なる引数仕様を持つ。この数は下記に示す引数の有無の組み合わせに依るものである(引数の定義については後述する)。
\begin{enumerate}[itemsep=-1ex,label=\arabic*)]
\item \verb|template <class Theader> Theader & header|
\item \verb|const char * const format|
\item \verb|void (Tptcl::*pFunc)(FILE *)|
\end{enumerate}
このあと、各APIについて記述する。説明を簡潔にするため、オーバーロードされた関数群をまとめて記述する。

%%%%%%%%%%%%%%%%%%%%%%%%%%%%%%%%%%%%%%%%%%%
\subsubsubsubsubsection{PS::ParticleSystem::readParticleAscii}
\label{sec:readParticleAscii}

readParticleAsciiは下記の8つの異なる引数仕様を持つ。
\begin{breakitembox}<parindent=0pt>{}
\begin{verbatim}
template <class Tptcl>
template <class Theader>
void PS::ParticleSystem<Tptlc>::readParticleAscii
                 (const char * const filename,
                  const char * const format,
                  Theader & header);
                  
template <class Tptcl>
void PS::ParticleSystem<Tptcl>::readParticleAscii
            (const char * const filename,
             const char * const format);
             
template <class Tptcl>
template <class Theader>
void PS::ParticleSystem<Tptcl>::readParticleAscii
            (const char * const filename,
             Theader & header);
             
template <class Tptcl>
void PS::ParticleSystem<Tptcl>::readParticleAscii
            (const char * const filename);
            
template <class Tptcl>
template <class Theader>
void PS::ParticleSystem<Tptlc>::readParticleAscii
                 (const char * const filename,
                  const char * const format,
                  Theader & header,
                  void (Tptcl::*pFunc)(FILE*));
                  
template <class Tptcl>
void PS::ParticleSystem<Tptcl>::readParticleAscii
            (const char * const filename,
             const char * const format,
             void (Tptcl::*pFunc)(FILE*));
             
template <class Tptcl>
template <class Theader>
void PS::ParticleSystem<Tptcl>::readParticleAscii
            (const char * const filename,
             Theader & header,
             void (Tptcl::*pFunc)(FILE*));
             
template <class Tptcl>
void PS::ParticleSystem<Tptcl>::readParticleAscii
            (const char * const filename,
             void (Tptcl::*pFunc)(FILE*));
\end{verbatim}
\end{breakitembox}

\begin{itemize}

\item {\bf 引数}

filename: 入力。const char *型。分散ファイル読み込み時には入力ファイル名のベースとなる部分。単一ファイル読み込み時には入力ファイル名。

format: 入力。const char *型。分散ファイルから粒子データを読み込む際の
ファイルフォーマット。

header: 入力。Theader \&型。ファイルのヘッダ情報。

pFunc: 入力。 void (Tptcl::*)(FILE*) 型。FILEポインタを引数にとりvoidを返すTptclのメンバ関数ポインタ。

\item {\bf 返値}

なし

\item {\bf 機能}

引数formatが存在する場合、各プロセスがfilenameとformatの組によって指定される名前の入力ファイルから粒子データを読み出し、データをFullParticleクラスのオブジェクトに格納する(\textbf{分散ファイルの読み込み})。一方、引数formatが存在しない場合、ルートプロセスがfilenameで指定された入力ファイルから粒子データを読み出し、データをFullParticleクラスのオブジェクトに格納した後、各プロセスに分配する(\textbf{単一ファイルの読み込み})。いずれの場合においても、引数headerが存在しない場合、ファイルに保存されている粒子数を調べるため、1回ファイルを読み込んで行数を取得した後、もう一度ファイルを読み込み直し、粒子データを読み込む。すなわち、粒子数の推定は、1粒子のデータが1行に記述されているという仮定の下行う。


分散ファイルを読み込む場合、filenameで、分散しているファイルのベースとなる名前を指定する。formatでファイル名のフォーマットを指定する。フォーマットの指定方法は標準Cライブラリの関数printfの第1引数と同じである。ただし変換指定は必ず3つであり、その指定子は1つめは文字列、残りはどちらも整数である。2つ目の変換指定にはそのジョブの全プロセス数が、3つ目の変換指定にはプロセス番号が入る。例えば、filenameがnbody、formatが\%s\_\%03d\_\%03d.initならば、全プロセス数$64$のジョブのプロセス番号$12$のプロセスは、nbody\_064\_012.initというファイルを読み込む。

1粒子のデータを読み取る関数は引数pFuncが存在すればそれを使用し、そうでない場合にはFullParticleクラスのメンバ関数readAsciiを使用する。readAsciiはユーザが定義しなければならない。readAsciiの仕様については節\ref{sec:FP_readAscii}を、実装例については節\ref{sec:example_userdefined_fullparticle_io}を参照のこと。pFuncもreadAsciiと同じ仕様を満たす必要がある。

ファイルのヘッダのデータを読み取る関数はTheaderのメンバ関数readAsciiでユーザが定義する。仕様については節\ref{sec:Hdr_readAscii}を、実装例については節\ref{sec:example_userdefined_header}を参照のこと。

ファイルはアスキーモードで開く。

\end{itemize}


%%%%%%%%%%%%%%%%%%%%%%%%%%%%%%%%%%%%%%%%%%%
\subsubsubsubsubsection{PS::ParticleSystem::readParticleBinary}
\label{sec:readParticleBinary}

readParticleBinaryは下記の8つの異なる引数仕様を持つ。
\begin{breakitembox}<parindent=0pt>{}
\begin{verbatim}
template <class Tptcl>
template <class Theader>
void PS::ParticleSystem<Tptlc>::readParticleBinary
                 (const char * const filename,
                  const char * const format,
                  Theader & header);

template <class Tptcl>
void PS::ParticleSystem<Tptcl>::readParticleBinary
            (const char * const filename,
             const char * const format); 
             
template <class Tptcl>
template <class Theader>
void PS::ParticleSystem<Tptcl>::readParticleBinary
            (const char * const filename,
             Theader & header);

template <class Tptcl>
void PS::ParticleSystem<Tptcl>::readParticleBinary
            (const char * const filename);

template <class Tptcl>
template <class Theader>
void PS::ParticleSystem<Tptlc>::readParticleBinary
                 (const char * const filename,
                  const char * const format,
                  Theader & header,
                  void (Tptcl::*pFunc)(FILE*));
                 
template <class Tptcl>
void PS::ParticleSystem<Tptcl>::readParticleBinary
            (const char * const filename,
             const char * const format,
             void (Tptcl::*pFunc)(FILE*));

template <class Tptcl>
template <class Theader>
void PS::ParticleSystem<Tptcl>::readParticleBinary
            (const char * const filename,
             Theader & header,
             void (Tptcl::*pFunc)(FILE*));
             
template <class Tptcl>
void PS::ParticleSystem<Tptcl>::readParticleBinary
            (const char * const filename,
             void (Tptcl::*pFunc)(FILE*));

template <class Tptcl>
template <class Theader>
void PS::ParticleSystem<Tptcl>::writeParticleAscii
            (const char * const filename,
             const Theader & header,
             void (Tptcl::*pFunc)(FILE*)const);

\end{verbatim}
\end{breakitembox}

\begin{itemize}

\item {\bf 引数}

filename: 入力。const char *型。分散ファイル読み込み時には入力ファイル名のベースとなる部分。単一ファイル読み込み時にはファイル名。

format: 入力。const char *型。分散ファイルから粒子データを読み込む際の
ファイルフォーマット。

header: 入力。Theader \&型。ファイルのヘッダ情報。

pFunc: 入力。 void (Tptcl::*)(FILE*) 型。FILEポインタを引数にとりvoidを返すTptclのメンバ関数ポインタ。


\item {\bf 返値}

なし

\item {\bf 機能}

readParticleAsciiとほぼ同じ機能を提供するが、下記の点が異なる。
\begin{itemize}
\item ファイルはバイナリーモードで開く。
\item 引数 headerが存在しない場合の粒子数の推定は、実際に1粒子を読み込んだときのファイル位置指示子のバイト数を調べることによって行うため、1粒子のデータの最後が改行文字(\texttt{\textbackslash n})となっている必要はない。
\end{itemize}

\end{itemize}

%%%%%%%%%%%%%%%%%%%%%%%%%%%%%%%%%%%%%%%%%%%
\subsubsubsubsubsection{PS::ParticleSystem::writeParticleAscii}
\label{sec:writeParticleAscii}

writeParticleAsciiは下記の8つの異なる引数仕様を持つ。
\begin{breakitembox}<parindent=0pt>{}
\begin{verbatim}
template <class Tptcl>
template <class Theader>
void PS::ParticleSystem<Tptcl>::writeParticleAscii
            (const char * const filename,
             const char * const format,
             const Theader & header);
             
template <class Tptcl>
void PS::ParticleSystem<Tptcl>::writeParticleAscii
            (const char * const filename,
             const char * const format);

template <class Tptcl>
template <class Theader>
void PS::ParticleSystem<Tptcl>::writeParticleAscii
            (const char * const filename,
             const Theader & header);

template <class Tptcl>
void PS::ParticleSystem<Tptcl>::writeParticleAscii
            (const char * const filename);

template <class Tptcl>
template <class Theader>
void PS::ParticleSystem<Tptcl>::writeParticleAscii
            (const char * const filename,
             const char * const format,
             const Theader & header,
             void (Tptcl::*pFunc)(FILE*)const);

template <class Tptcl>
void PS::ParticleSystem<Tptcl>::writeParticleAscii
            (const char * const filename,
             const char * const format,
             void (Tptcl::*pFunc)(FILE*)const);

template <class Tptcl>             
template <class Theader>
void PS::ParticleSystem<Tptcl>::writeParticleAscii
            (const char * const filename,
             const Theader & header,
             void (Tptcl::*pFunc)(FILE*)const);
             
template <class Tptcl>                        
void PS::ParticleSystem<Tptcl>::writeParticleAscii
            (const char * const filename,
             void (Tptcl::*pFunc)(FILE*)const);
\end{verbatim}
\end{breakitembox}

\begin{itemize}

\item{{\bf 引数}}

  filename: 入力。const char * const型。出力ファイル名のベースとなる部
  分。

  format: 入力。const char * const型。分散ファイルに粒子データを書き込
  む際のファイルフォーマット。

  header: 入力。const Theader \&型。ファイルのヘッダ情報。
  
  pFunc: 入力。void (Tptcl::*)(FILE*)const 型。FILEポインタを引数にとりvoidを返すTptclのメンバ関数ポインタ。


\item{{\bf 返り値}}

  なし。

\item{{\bf 機能}}

引数formatが存在する場合、各プロセスがfilenameとformatの組によって指定される名前の出力ファイルにFullParticleクラスのオブジェクトのデータと、もし存在すればTheaderクラスのオブジェクトのデータを出力する(\textbf{分散ファイルへの書き出し})。一方、引数formatが存在しない場合、各プロセスのFullParticleクラスのオブジェクトのデータをルートプロセスに集め、ルートプロセスがfilenameで指定された出力ファイルに集めたデータと、もし存在すればTheaderクラスのオブジェクトデータを出力する(\textbf{単一ファイルへの書き出し})。


分散ファイル書き出し時の出力ファイル名のフォーマットはメンバ関数\\PS::ParticleSystem::readParticleAsciiの場合と同様である。
  
1粒子のデータを書き込む関数はpFuncが存在する場合にはそれが使用され、そうでない場合にはFullParticleクラスのメンバ関数writeAsciiが使用される。writeAsciiはユーザが定義しなければならない。writeAsciiの仕様については節\ref{sec:FP_writeAscii}を、実装例については節\ref{sec:example_userdefined_fullparticle_io}を参照のこと。

ファイルのヘッダのデータを書き込む関数はヘッダクラスのメンバ関数writeAsciiでユーザが定義する。仕様については節\ref{sec:Hdr_writeAscii}を、実装例については節\ref{sec:example_userdefined_header}を参照のこと。

ファイルはアスキーモードで開く。

\end{itemize}


%%%%%%%%%%%%%%%%%%%%%%%%%%%%%%%%%%%%%%%%%%%
\subsubsubsubsubsection{PS::ParticleSystem::writeParticleBinary}
\label{sec:writeParticleBinary}

writeParticleBinaryは下記の8つの異なる引数仕様を持つ。
\begin{breakitembox}<parindent=0pt>{}
\begin{verbatim}
template <class Tptcl>
template <class Theader>
void PS::ParticleSystem<Tptcl>::writeParticleBinary
            (const char * const filename,
             const char * const format,
             const Theader & header);
             
template <class Tptcl>
void PS::ParticleSystem<Tptcl>::writeParticleBinary
            (const char * const filename,
             const char * const format);

template <class Tptcl>
template <class Theader>
void PS::ParticleSystem<Tptcl>::writeParticleBinary
            (const char * const filename,
             const Theader & header);

template <class Tptcl>
void PS::ParticleSystem<Tptcl>::writeParticleBinary
            (const char * const filename);

template <class Tptcl>
template <class Theader>
void PS::ParticleSystem<Tptcl>::writeParticleBinary
            (const char * const filename,
             const char * const format,
             const Theader & header,
             void (Tptcl::*pFunc)(FILE*)const);

template <class Tptcl>
void PS::ParticleSystem<Tptcl>::writeParticleBinary
            (const char * const filename,
             const char * const format,
             void (Tptcl::*pFunc)(FILE*)const);

template <class Tptcl>
template <class Theader>
void PS::ParticleSystem<Tptcl>::writeParticleBinary
            (const char * const filename,
             const Theader & header,
             void (Tptcl::*pFunc)(FILE*)const);

template <class Tptcl>             
void PS::ParticleSystem<Tptcl>::writeParticleBinary
            (const char * const filename,
             void (Tptcl::*pFunc)(FILE*)const));
\end{verbatim}
\end{breakitembox}

\begin{itemize}

\item{{\bf 引数}}

  filename: 入力。const char * const型。分散ファイルへの書き込み時には出力ファイル名のベースとなる部分。単一ファイルへの書き込み時にはファイル名。

  format: 入力。const char * const型。分散ファイルに粒子データを書き込む際のファイルフォーマット。

  header: 入力。const Theader \&型。ファイルのヘッダ情報。
  
  pFunc: 入力。void (Tptcl::*)(FILE*)const 型。FILEポインタを引数にとりvoidを返すTptclのメンバ関数ポインタ。


\item{{\bf 返り値}}

  なし。

\item{{\bf 機能}}

writeParticleAsciiとほぼ同じ機能を提供するが、下記の点が異なる。
\begin{itemize}
\item ファイルはバイナリーモードで開く。
\end{itemize}

\end{itemize}


%%%%%%%%%%%%%%%%%%%%%%%%%%%%%%%%%%%%%%%%%%%%%%%%%%%%%%
\subsubsubsubsection{粒子交換}


粒子交換関連のAPIの宣言は以下のようになっている。このあと各APIについて
記述する。
\begin{lstlisting}[caption=ParticleSystem4]
namespace ParticleSimulator {
    template<class Tptcl>
    class ParticleSystem{
    public:
        template<class Tdinfo>
        void exchangeParticle(Tdinfo & dinfo, const bool flag_serialize=false);
    };
}
\end{lstlisting}

%%%%%%%%%%%%%%%%%%%%%%%%%%%%%%%%%%%%%%%%%%%
\subsubsubsubsubsection{PS::ParticleSystem::exchangeParticle}
\label{sec:particleSystem:exchangeParticle}

\begin{screen}
\begin{verbatim}
template <class Tptcl>
template <class Tdinfo>
void PS::ParticleSystem<Tptcl>::exchangeParticle
                 (Tdinfo & dinfo, const bool flag_serialize=false);
\end{verbatim}
\end{screen}

\begin{itemize}

\item {\bf 引数}

dinfo: 入力。DomainInfo \& 型。領域クラスのオブジェクト。

flag\_serialize: 入力。const bool 型。粒子情報をシリアライズして送信す
るかを決定するフラグ。trueで粒子をシリアライズする。デフォルトはfalse。

\item {\bf 返値}

なし

\item {\bf 機能}

粒子が適切なドメインに配置されるように、粒子の交換を行う。粒子をシリア
ライズして送信する場合はFPにメンバ関数packとunPackを定義し(詳しくはセ
クション\ref{sec:FP:serialize})、flag\_serializeをtrueにする (\textcolor{red}{シリアライズ送信の対応は未実装})。

\end{itemize}

%%%%%%%%%%%%%%%%%%%%%%%%%%%%%%%%%%%%%%%%%%%%%%%%%%%%%%
\subsubsubsubsection{粒子の追加、削除}
\label{sec:addAndRemoveParticle}

粒子の追加もしくは削除関連のAPIは以下の様に宣言されている。

%%%%%%%%%%%%%%%%%%%%%%%%%%%%%%%%%%%%%%%%%%%
\subsubsubsubsubsection{PS::ParticleSystem::addOneParticle()}
\begin{screen}
\begin{verbatim}
void PS::ParticleSystem::addOneParticle(const FullPartilce & fp);
\end{verbatim}
\end{screen}

\begin{itemize}

\item {\bf 引数}

fp: 入力。const FullParticle \&型。追加する粒子のFullParticle。

\item {\bf 返値}

なし。

\item {\bf 機能}

追加された粒子を粒子配列の末尾に追加する。

\end{itemize}


%%%%%%%%%%%%%%%%%%%%%%%%%%%%%%%%%%%%%%%%%%%
\subsubsubsubsubsection{PS::ParticleSystem::removeParticle()}
\begin{screen}
\begin{verbatim}
void PS::ParticleSystem::removeParticle(const PS::S32 * idx, 
                                        const PS::S32 n);
\end{verbatim}
\end{screen}

\begin{itemize}

\item {\bf 引数}

idx: 入力。const PS::S32 * 型。消去する粒子の配列インデックスの配列。

n: 入力。const PS::S32 型。配列idxのサイズ。

\item {\bf 返値}

なし。

\item {\bf 機能}

配列idx[]に格納されているインデックスの粒子を削除する。この関数を呼ぶ
前後で、粒子の配列インデックスが同じである事は保証されない。


\end{itemize}


%%%%%%%%%%%%%%%%%%%%%%%%%%%%%%%%%%%%%%%%%%%%%%%%%%%%%%
\subsubsubsubsection{時間計測}

クラス内の情報取得関連のAPIの宣言は以下のようになっている。自クラスの
主要なメソッドを呼び出すとそれにかかった時間をプライベートメンバの
time\_profile\_の該当メンバに書き込む。メソッドclearTimeProfile()を呼
ばない限り時間は足しあわされていく。

\begin{lstlisting}[caption=ParticleSystem3]
namespace ParticleSimulator {
    template<class Tptcl>
    class ParticleSystem{
    private:
        TimeProfile time_profile_;
    public:
        TimeProfile getTimeProfile();
        void clearTimeProfile();
    };
}
\end{lstlisting}

%%%%%%%%%%%%%%%%%%%%%%%%%%%%%%%%%%%%%%%%%%%
\subsubsubsubsubsection{PS::ParticleSystem::getTimeProfile}
\begin{screen}
\begin{verbatim}
PS::TimeProfile PS::ParticleSystem::getTimeProfile();
\end{verbatim}
\end{screen}

\begin{itemize}

\item {\bf 引数}

なし。

\item {\bf 返値}

PS::TimeProfile型。

\item {\bf 機能}

メンバ関数exchangeParticleにかかった時間(ミリ秒単位)をTimeProfile型
のメンバ変数exchange\_particles\_に格納し、返す。

\end{itemize}

%%%%%%%%%%%%%%%%%%%%%%%%%%%%%%%%%%%%%%%%%%%
\subsubsubsubsubsection{PS::ParticleSystem::clearTimeProfile}
\begin{screen}
\begin{verbatim}
void PS::ParticleSystem::clearTimeProfile();
\end{verbatim}
\end{screen}

\begin{itemize}

\item {\bf 引数}

なし。

\item {\bf 返値}

なし。

\item {\bf 機能}

領域情報クラスのTimeProfile型のプライベートメンバ変数のメンバ変数
exchange\_particles\_の値を0クリアする。

\end{itemize}




%%%%%%%%%%%%%%%%%%%%%%%%%%%%%%%%%%%%%%%%%%%%%%%%%%%%%%
\subsubsubsubsection{その他}

その他のAPIの宣言は以下のようになっている。このあと各APIについて記述す
る。
\begin{lstlisting}[caption=ParticleSystem4]
namespace ParticleSimulator {
    template<class Tptcl>
    class ParticleSystem{
    public:
        template<class Tdinfo>
        void adjustPositionIntoRootDomain
                    (const Tdinfo & dinfo);
        void setNumberOfParticleLocal(const S32 n);
        tamplate<class Tcomp>
        void sortParticle(Tcomp comp);
    };
}
\end{lstlisting}

%%%%%%%%%%%%%%%%%%%%%%%%%%%%%%%%%%%%%%%%%%%
\subsubsubsubsubsection{PS::ParticleSystem::adjustPositionIntoRootDomain}

\begin{screen}
\begin{verbatim}
template <class Tptcl>
template <class Tdinfo>
void ParticleSystem<Tptcl>::adjustPositionIntoRootDomain
            (const Tdinfo & dinfo);
\end{verbatim}
\end{screen}

\begin{itemize}

\item {\bf 引数}

  dinfo: 入力。Tdinfo型。領域クラスのオブジェクト。

\item {\bf 返値}

  なし

\item {\bf 機能}

  周期境界条件の場合に、計算領域からはみ出した粒子を計算領域に適切に戻
  す。

\end{itemize}


%%%%%%%%%%%%%%%%%%%%%%%%%%%%%%%%%%%%%%%%%%%
\subsubsubsubsubsection{PS::ParticleSystem::setNumberOfParticleLocal}

\begin{screen}
\begin{verbatim}
template <class Tptcl>
void PS::ParticleSystem<Tptcl>::setNumberOfParticleLocal
                 (const PS::S32 n);
\end{verbatim}
\end{screen}

\begin{itemize}

\item {\bf 引数}

n: 入力。const PS::S32型。粒子数。

\item {\bf 返値}

なし

\item {\bf 機能}

1つのMPIプロセスの持つ粒子数を設定する。

\end{itemize}


%%%%%%%%%%%%%%%%%%%%%%%%%%%%%%%%%%%%%%%%%%%
\subsubsubsubsubsection{PS::ParticleSystem::sortParticle}
\label{sec:ParticleSystem:sortParticle}

\begin{screen}
\begin{verbatim}
tamplate<class Tcomp>
void sortParticle(Tcomp comp);
\end{verbatim}
\end{screen}

\begin{itemize}

\item {\bf 引数}

comp: 入力。Tcomp型。比較関数。比較関数は返り値をbool型とし、引数は
const FullParticle \& を2つ。例として以下にFullParticleがメンバ変数id
を持っておりidの昇順に並べ替える場合を示す。

\begin{verbatim}
bool comp(const FP & left, const FP & right){
    return left.id < right.id;
}
\end{verbatim}

\item {\bf 返値}

なし

\item {\bf 機能}

粒子群クラスが保持するFullParticleの配列を引数compで指示したように並べ替える。

\end{itemize}
