\subsection{概要}

FDPSでは、座標系や並列処理の有無、浮動小数点数型の精度、エラー検出等を選択できる。この選択はコンパイル時のマクロの定義によってなされる。以下、選択の方法について座標系、並列処理の有無、浮動小数点型の精度の順に記述する。

\subsection{座標系}
\label{sec:compile_coordinate}

\subsubsection{概要}

座標系は直角座標系3次元と直角座標系2次元の選択ができる。以下、それら
の選択方法について述べる。

\subsubsection{直角座標系3次元}

デフォルトは直角座標系3次元である。なにも行わなくても直角座標系3次元
となる。

\subsubsection{直角座標系2次元}

コンパイル時にPARTICLE\_SIMULATOR\_TWO\_DIMENSIONをマクロ定義すると直
交座標系2次元となる。

\subsection{並列処理}

\subsubsection{概要}

並列処理に関しては、OpenMPの使用/不使用、MPIの使用/不使用を選択でき
る。以下、選択の仕方について記述する。

\subsubsection{OpenMPの使用}

デフォルトはOpenMP不使用である。使用する場合は、
\\ PARTICLE\_SIMULATOR\_THREAD\_PARALLELをマクロ定義すればよい。GCCコ
ンパイラの場合はコンパイラオプションに-fopenmpをつける必要がある。

\subsubsection{MPIの使用}

デフォルトはMPI不使用である。使用する場合は、
PARTICLE\_SIMULATOR\_THREAD\_PARALLELをマクロ定義すればよい。

\subsection{データ型の精度}
\subsubsection{概要}
FDPS側で用意したMomentクラス(第7.5.2節参照)とSuperParticleJクラス(第7.6.2節参照)のデータ型の精度を選択できる。以下、選択の仕方について記述する。

\subsubsection{既存のSuperParticleJクラスとMomentクラスの精度}
既存のSuperParticleJクラスとMomentクラスのメンバ変数の精度はデフォルトで64ビットである。32ビットにしたい場合、\\
PARTICLE\_SIMULATOR\_SPMOM\_F32 \\
をマクロ定義すればよい。


%\subsubsection{Vector型の範囲チェック}
%\label{sec:compile:vector_invalid_access}

%コンパイル時にPARTICLE\_SIMULATOR\_VECTOR\_RANGE\_CHECKをマクロ定義す
%るとVector型の範囲外の成分にアクセスするとエラーメッセージを出力するこ
%とが出来る。詳しくは節\ref{sec:errormessage:vector_invalid_access}を参
%照。
