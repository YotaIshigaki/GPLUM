%\subsection{概要}
\subsection{Summary}

%本節では、ユーザーが定義するクラスとファンクタについて記述する。ユーザー
%定義クラスとなるのは、FullParticleクラス、EssentialParticleIクラス、
%EssentialParticleJクラス、Momentクラス、SuperParticleJクラス、Forceクラ
%ス、ヘッダクラスである。またユーザー定義の関数オブジェクトには、関数オブジェ
%クトcalcForceEpEp、calcForceSpEpがある。
In this section we provide details of user-defined classes and functors.
The user-defined classes are \texttt{FullParticle}, \texttt{EssentialParticleI},
\texttt{EssentialParticleJ}, \texttt{Moment}, \texttt{SuperParticleJ},
\texttt{Force} and \texttt{Header} classes.
The functors are \texttt{calcForceEpEp} and \texttt{calcForceSpEp}.

%FullParticleクラスは、ある1粒子の情報すべてを持つクラスであり、粒子群
%クラスにテンプレート引数として渡されるものである(節
%\ref{sec:overview_action}の手順0)。
A \texttt{FullParticle} class contains all informations of a particle and is
a template argument of FDPS-defined classes (see step 0 in Sec. \ref{sec:overview_action}).

%関数オブジェクトcalcForceEpEpとcalcForceSpEpは、それぞれj粒子からi粒子
%への作用を計算する関数オブジェクトと超粒子からi粒子への作用を計算する
%関数オブジェクトである。これらは相互作用ツリークラスのAPIの引数として
%渡されるものである(節\ref{sec:overview_action}の手順0)。超粒子を必要と
%するPS::SEARCH\_MODE型(PS::SEARCH\_MODE\_LONGか
%PS::SEARCH\_MODE\_LONG\_CUTOFF)以外を使用する場合には、関数オブジェク
%トcalcForceSpEpを定義する必要はない。
Functors \texttt{calcForceEpEp} and \texttt{calcForceSpEp} define the interaction
between two \texttt{EssentialParticle} classes and \texttt{SuperParticles}
class and \texttt{EssentialParticle}, respectively.
These functors are argument of FDPS-defined \texttt{TreeForForce}
(see step 0 in Sec. \ref{sec:overview_action}).
If neither \texttt{PS::SEARCH\_MODE\_LONG} nor
\texttt{PS::SEARCH\_MODE\_LONG\_CUTOFF} are used,
users do not need to define \texttt{calcForceSpEp}.

%EssentialParticleIクラス、EssentialParticleJクラス、Momentクラス、
%SuperParticleJクラス、Forceクラスは粒子間の相互作用の定義を補助するも
%のである。これらのクラスのうちEssentialParticleIクラス、
%EssentialParticleJクラス、Forceクラスはそれぞれ相互作用を計算する際にi
%粒子に必要な情報、相互作用を計算する際にj粒子に必要な情報、相互作用の
%結果の情報を持つ。これらはFullParticleクラスのサブセットであるため、こ
%れらをFullParticleクラスで代用することも可能である。しかし、
%FullParticleクラスは相互作用の定義に必要のないデータを多く含む場合も考
%えられるため、計算コストを軽減したいならば、これらのクラスを使用するこ
%とを検討するべきである。MomentクラスとSuperParticleJクラスは、それぞれ
%ツリーセルのモーメント情報を持つクラスと超粒子に必要な情報を持つ
%SuperParticleJクラスである。ユーザーが定義する必要があるのは、超粒子を
%使う必要がある場合、すなわちPS::SEARCH\_MODE型にPS::SEARCH\_MODE\_LONG
%かPS::SEARCH\_MODE\_LONG\_CUTOFFを選んだ場合のみである。
%ヘッダクラスは入出力ファイルのヘッダ情報を持つ。
Classes \texttt{EssentialParticleI}, \texttt{EssentialParticleJ}, \texttt{Moment},
\texttt{SuperParticleJ} and \texttt{Force} are support classes for the
interactions between two particles. Classes \texttt{EssentialParticleI} and
\texttt{EssentialParticleJ} contain the quantities of $i-$ and $j-$ particles used
for the calculation of interactions. A \texttt{Force} class contains the quantities
of an $i-$ particle used to store the results of the calculations of interactions.
Since these classes contain part of information of \texttt{FullParticle}, it is
possible to use \texttt{FullParticle} in place of these classes. However,
\texttt{FullParticle} may contain other values which are not used to evaluate
the calculations of interactions. It is recommended to use these classes when
high performance is desirable. Classes \texttt{Moment} and \texttt{SuperParticleJ}
contain the information of moment and super particle, respectively. These
classes are necessary if users require the super particle, \textit{i.e.}, when
\texttt{PS::SEARCH\_MODE\_LONG} or \texttt{PS::SEARCH\_MODE\_LONG\_CUTOFF}
are used. A class \texttt{Header} contains the header informations of
input/output file.

%この節で記述するのは、これらのクラスや関数オブジェクトを定義する際の規
%定である。ユーザーはこれらの間でのデータのやりとりや、関数オブジェクト
%内でのデータの加工についてコードに書く必要がある。これらは上に挙げたク
%ラスのメンバ関数と関数オブジェクト内で行われる。以下、必要なメンバ関数
%とその規定について記述する。
The rest of this chapter describes how these classes and functors should
be implemented. Users must write data transfer functions between these
classes and interaction calculation in functor, which are used in the above member
functions and functors.

\subsection{FullParticle}
\label{sec:fullparticle}

\subsubsection{概要}

FullParticleクラスは粒子情報すべてを持つクラスであり、
節\ref{sec:overview_action}の手順0で、粒子群クラスに渡されるユーザー定
義クラスの1つである。ユーザーはこのクラスに対して、どのようなメンバ変
数、メンバ関数を定義してもかまわない。ただし、FDPSからFullParticleクラ
スの情報にアクセスする ために、ユーザーはいくつかの決まった名前のメン
バ関数を定義する必要がある。以下、この節の前提、常に必要なメンバ関数と、
場合によっては必要なメンバ関数について記述する。

\subsubsection{前提}

この節の中では、以下のように、FullParticleクラスとしてFPというクラスを
一例とする。FPという名前は自由に変えることができる。
\begin{screen}
\begin{verbatim}
class FP;
\end{verbatim}
\end{screen}

\subsubsection{必要なメンバ関数}

\subsubsubsection{概要}

常に必要なメンバ関数はFP::getPosとFP::copyFromForceである。FP::getPos
はFullParticleの位置情報をFDPSに読み込ませるための関数で、
FP::copyFromForceは計算された相互作用の結果をFullParticleに書き戻す関
数である。これらのメンバ関数の記述例と解説を以下に示す。

\subsubsubsection{FP::getPos}

\begin{screen}
\begin{verbatim}
class FP {
public:
    PS::F64vec getPos() const;
};
\end{verbatim}
\end{screen}

\begin{itemize}

\item {\bf 引数}

  なし
  
\item {\bf 返値}

  PS::F32vec型またはPS::F64vec型。FPクラスのオブジェクトの位置情報を保
  持したメンバ変数。
  
\item {\bf 機能}

  FPクラスのオブジェクトの位置情報を保持したメンバ変数を返す。
  
\end{itemize}

\subsubsubsection{FP::copyFromForce}

%%%%%%%%%%%%%%%%%%%%%%%%%%%%%%%%%%%%%%%%%%%%%%%%%%%%%%
%%%%%%%%%%%%%%%%%%%%%%%%%%%%%%%%%%%%%%%%%%%%%%%%%%%%%%
\if 0
\begin{screen}
\begin{verbatim}
class Force {
public:
    PS::F64vec acc;
    PS::F64    pot;
};
class FP {
public:
    PS::F64vec acceleration;
    PS::F64    potential;
    void copyFromForce(const Force & force) {
        this->acceleration = force.acc;
        this->potential    = force.pot;
    }
};
\end{verbatim}
\end{screen}

\begin{itemize}

\item {\bf 前提}

  Forceクラスは粒子の相互作用の計算結果を保持するクラス。

\item {\bf 引数}

  force: 入力。const Force \&型。粒子の相互作用の計算結果を保持。
  
\item {\bf 返値}

  なし。
  
\item {\bf 機能}

  粒子の相互作用の計算結果をFPクラスへ書き戻す。Forceクラスのメンバ変
  数acc, potがそれぞれFPクラスのメンバ変数acceleration, potentialに対
  応。
  
\item {\bf 備考}

  Forceクラスというクラス名とそのメンバ変数名は変更可能。FPのメンバ変
  数名は変更可能。メンバ関数FP::copyFromForceの引数名は変更可能。

\end{itemize}
\fi
%%%%%%%%%%%%%%%%%%%%%%%%%%%%%%%%%%%%%%%%%%%%%%%%%%%%%%
%%%%%%%%%%%%%%%%%%%%%%%%%%%%%%%%%%%%%%%%%%%%%%%%%%%%%%

\begin{screen}
\begin{verbatim}
class Force;

class FP {
public:
    void copyFromForce(const Force & force);
};
\end{verbatim}
\end{screen}

\begin{itemize}

\item {\bf 引数}

  force: 入力。const Force \&型。粒子の相互作用の計算結果を保持。
  
\item {\bf 返値}

  なし。
  
\item {\bf 機能}

  粒子の相互作用の計算結果をFPクラスへ書き戻す。
  
\end{itemize}


\subsubsection{場合によっては必要なメンバ関数}

\subsubsubsection{概要}

本節では、以下に示す場合に必要となるメンバ関数について記述する:
\begin{enumerate}%[leftmargin=*,itemsep=-1ex,label={[\arabic*]}]
\item 相互作用ツリークラスのPS::SEARCH\_MODE型にPS::SEARCH\_MODE\_LONG以外を用いる場合
\item 粒子群クラスのファイル入出力APIを用いる場合
\item 粒子群クラスのAPIであるParticleSystem::adjustPositionIntoRootDomainを用いる場合
\item 拡張機能のParticle Meshクラスを用いる場合
\end{enumerate}

\subsubsubsection{相互作用ツリークラスのPS::SEARCH\_MODE型に\\PS::SEARCH\_MODE\_LONG以外を用いる場合}

\subsubsubsubsection{FP::getRSearch}

%%%%%%%%%%%%%%%%%%%%%%%%%%%%%%%%%%%%%%%%%%%%%%%%%%%%%%
%%%%%%%%%%%%%%%%%%%%%%%%%%%%%%%%%%%%%%%%%%%%%%%%%%%%%%
\if 0
\begin{screen}
\begin{verbatim}
class FP {
public:
    PS::F64 search_radius;
    PS::F64 getRSearch() const {
        return this->search_radius;
    }
};
\end{verbatim}
\end{screen}

\begin{itemize}

\item {\bf 前提}

  FPクラスのメンバ変数search\_radiusはある1つの粒子の近傍粒子を探す半
  径の大きさ。このsearch\_radiusのデータ型はPS::F32型またはPS::F64型。
  
\item {\bf 引数}

  なし
  
\item {\bf 返値}

  PS::F32型またはPS::F64型。 FPクラスのオブジェクトの近傍粒子を探す半
  径の大きさを保持したメンバ変数。
  
\item {\bf 機能}

  FPクラスのオブジェクトの近傍粒子を探す半径の大きさを保持したメンバ変
  数を返す。

\item {\bf 備考}

  FPクラスのメンバ変数search\_radiusの変数名は変更可能。
  
\end{itemize}
\fi
%%%%%%%%%%%%%%%%%%%%%%%%%%%%%%%%%%%%%%%%%%%%%%%%%%%%%%
%%%%%%%%%%%%%%%%%%%%%%%%%%%%%%%%%%%%%%%%%%%%%%%%%%%%%%

\begin{screen}
\begin{verbatim}
class FP {
public:
    PS::F64 getRSearch() const;
};
\end{verbatim}
\end{screen}

\begin{itemize}

\item {\bf 引数}

  なし
  
\item {\bf 返値}

  PS::F32型またはPS::F64型。 FPクラスのオブジェクトの近傍粒子を探す半
  径の大きさを保持したメンバ変数。
  
\item {\bf 機能}

  FPクラスのオブジェクトの近傍粒子を探す半径の大きさを保持したメンバ変
  数を返す。

\end{itemize}

%%%%%%%%%%%%%%%%%%%%%%%%%%%%%%%%%%%%%%%%%%%%%%%%%%%%%%
\subsubsubsection{粒子群クラスのファイル入出力APIを用いる場合}
\label{sec:userdefined_fullparticle_io}

粒子群クラスのファイル入出力APIであるParticleSystem::readParticleAscii, ParticleSystem::writeParticleAscii, ParticleSystem::readParticleBinary, ParticleSystem::writeParticleBinaryを使用するときにそれぞれreadAscii, writeAscii, readBinary, writeBinaryというメンバ関数が必要となる(readAscii, writeAscii, readBinary, writeAscii以外の名前を使うことも可能。詳しくは節\ref{sec:ParticleSystem:IO}を参照)。以下、readAsciiとwriteAscii, readBinary, writeBinaryの規定について記述する。

%%%%%%%%%%%%%%%%%%%%%%%%%%%%%%%%%%%%%%%%%%%%%%%%%%%%%%
\subsubsubsubsection{FP::readAscii}
\label{sec:FP_readAscii}

\begin{screen}
\begin{verbatim}
class FP {
public:
    void readAscii(FILE *fp);
};
\end{verbatim}
\end{screen}

\begin{itemize}

\item {\bf 引数}

  fp: FILE *型。粒子データの入力ファイルを指すファイルポインタ。
  
\item {\bf 返値}

  なし。
  
\item {\bf 機能}

  アスキー形式の粒子データの入力ファイル1行からFPクラス1個の情報を読み取り、メンバ変数に値を入れる。ここで1行とは、FPクラス1個のデータの末尾に\textbf{必ず唯一つの}改行コード(\texttt{\textbackslash n})がある状態の意味である。
\end{itemize}

%%%%%%%%%%%%%%%%%%%%%%%%%%%%%%%%%%%%%%%%%%%%%%%%%%%%%%
\subsubsubsubsection{FP::writeAscii}
\label{sec:FP_writeAscii}

\begin{screen}
\begin{verbatim}
class FP {
public:
    void writeAscii(FILE *fp);
};
\end{verbatim}
\end{screen}

\begin{itemize}

\item {\bf 引数}

  fp: FILE *型。粒子データの出力ファイルを指すファイルポインタ。
  
\item {\bf 返値}

  なし。
  
\item {\bf 機能}

  粒子データの出力ファイルへFPクラス1個の情報を1行としてアスキー形式で書き出す。ここで1行とは、FPクラス1個のデータの末尾に\textbf{必ず唯一つの}改行コード(\texttt{\textbackslash n})がある状態の意味である。
  
\end{itemize}


%%%%%%%%%%%%%%%%%%%%%%%%%%%%%%%%%%%%%%%%%%%%%%%%%%%%%%
\subsubsubsubsection{FP::readBinary}
\label{sec:FP_readBinary}

\begin{screen}
\begin{verbatim}
class FP {
public:
    void readBinary(FILE *fp);
};
\end{verbatim}
\end{screen}

\begin{itemize}

\item {\bf 引数}

  fp: FILE *型。粒子データの入力ファイルを指すファイルポインタ。
  
\item {\bf 返値}

  なし。
  
\item {\bf 機能}

  バイナリ形式の粒子データの入力ファイルからFPクラス1個の情報を読み取り、メンバ変数に値を入れる。
  
\end{itemize}

%%%%%%%%%%%%%%%%%%%%%%%%%%%%%%%%%%%%%%%%%%%%%%%%%%%%%%
\subsubsubsubsection{FP::writeBinary}
\label{sec:FP_writeBinary}

\begin{screen}
\begin{verbatim}
class FP {
public:
    void writeBinary(FILE *fp);
};
\end{verbatim}
\end{screen}

\begin{itemize}

\item {\bf 引数}

  fp: FILE *型。粒子データの出力ファイルを指すファイルポインタ。
  
\item {\bf 返値}

  なし。
  
\item {\bf 機能}

  粒子データの出力ファイルへFPクラス1個の情報をバイナリ形式で書き出す。
  
\end{itemize}


\subsubsubsection{ParticleSystem::adjustPositionIntoRootDomainを用いる場合}

\subsubsubsubsection{FP::setPos}

%%%%%%%%%%%%%%%%%%%%%%%%%%%%%%%%%%%%%%%%%%%%%%%%%%%%%%
%%%%%%%%%%%%%%%%%%%%%%%%%%%%%%%%%%%%%%%%%%%%%%%%%%%%%%
\if 0
\begin{screen}
\begin{verbatim}
class FP {
public:
    PS::F64vec pos;
    void setPos(const PS::F64vec pos_new) {
        this->pos = pos_new;
    }
};
\end{verbatim}
\end{screen}

\begin{itemize}

\item {\bf 前提}

  FPクラスのメンバ変数posは1つの粒子の位置情報。このposのデータ型は
  PS::F32vecまたはPS::F64vec。

\item {\bf 引数}

  pos\_new: 入力。const PS::F32vecまたはconst PS::F64vec型。FDPS側で修
  正した粒子の位置情報。

\item {\bf 返値}

  なし。
  
\item {\bf 機能}

  FDPSが修正した粒子の位置情報をFPクラスのオブジェクトの位置情報に書き
  込む。

\item {\bf 備考}

  FPクラスのメンバ変数posの変数名は変更可能。メンバ関数FP::setPosの引
  数名pos\_newは変更可能。posとpos\_newのデータ型が異なる場合の動作は
  保証しない。

\end{itemize}
\fi
%%%%%%%%%%%%%%%%%%%%%%%%%%%%%%%%%%%%%%%%%%%%%%%%%%%%%%
%%%%%%%%%%%%%%%%%%%%%%%%%%%%%%%%%%%%%%%%%%%%%%%%%%%%%%

\begin{screen}
\begin{verbatim}
class FP {
public:
    void setPos(const PS::F64vec pos_new);
};
\end{verbatim}
\end{screen}

\begin{itemize}

\item {\bf 引数}

  pos\_new: 入力。const PS::F32vecまたはconst PS::F64vec型。FDPS側で修
  正した粒子の位置情報。

\item {\bf 返値}

  なし。
  
\item {\bf 機能}

  FDPSが修正した粒子の位置情報をFPクラスのオブジェクトの位置情報に書き
  込む。

\end{itemize}

\subsubsubsection{Particle Meshクラスを用いる場合}

Particle Meshクラスを用いる場合には、メンバ関数
FP::getChargeParticleMeshと\\FP::copyFromForceParticleMeshを用意する必要
がある。以下にそれぞれの規定を記述する。

\subsubsubsubsection{FP::getChargeParticleMesh}

%%%%%%%%%%%%%%%%%%%%%%%%%%%%%%%%%%%%%%%%%%%%%%%%%%%%%%
%%%%%%%%%%%%%%%%%%%%%%%%%%%%%%%%%%%%%%%%%%%%%%%%%%%%%%
\if 0
\begin{screen}
\begin{verbatim}
class FP {
public:
    PS::F64 mass;
    PS::F64 getChargeParticleMesh() const {
        return this->mass;
    }
};
\end{verbatim}
\end{screen}

\begin{itemize}

\item {\bf 前提}

  FPクラスのメンバ変数massは1つの粒子の質量または電荷の情報を持つ変数。
  データ型はPS::F32またはPS::F64型。

\item {\bf 引数}

  なし。

\item {\bf 返値}

  PS::F32型またはPS::F64型。1つの粒子の質量または電荷の変数を返す。
  
\item {\bf 機能}

  1つの粒子の質量または電荷を表すメンバ変数を返す。

\item {\bf 備考}

  FPクラスのメンバ変数massの変数名は変更可能。

\end{itemize}
\fi
%%%%%%%%%%%%%%%%%%%%%%%%%%%%%%%%%%%%%%%%%%%%%%%%%%%%%%
%%%%%%%%%%%%%%%%%%%%%%%%%%%%%%%%%%%%%%%%%%%%%%%%%%%%%%

\begin{screen}
\begin{verbatim}
class FP {
public:
    PS::F64 getChargeParticleMesh() const;
};
\end{verbatim}
\end{screen}

\begin{itemize}

\item {\bf 引数}

  なし。

\item {\bf 返値}

  PS::F32型またはPS::F64型。1つの粒子の質量または電荷の変数を返す。
  
\item {\bf 機能}

  1つの粒子の質量または電荷を表すメンバ変数を返す。

\end{itemize}


\subsubsubsubsection{FP::copyFromForceParticleMesh}

%%%%%%%%%%%%%%%%%%%%%%%%%%%%%%%%%%%%%%%%%%%%%%%%%%%%%%
%%%%%%%%%%%%%%%%%%%%%%%%%%%%%%%%%%%%%%%%%%%%%%%%%%%%%%
\if 0
\begin{screen}
\begin{verbatim}
class FP {
public:
    PS::F64vec accelerationFromPM;
    void copyFromForceParticleMesh(const PS::F32vec & acc_pm) {
        this->accelerationFromPM = acc_pm;
    }
};
\end{verbatim}
\end{screen}

\begin{itemize}

\item {\bf 前提}

  FPクラスのメンバ変数accelerationFromPM\_pmは1つの粒子のParticle
  Meshによる力の情報を保持する変数。このaccelerationFromPM\_pmのデータ
  型はPS::F32vecまたはPS::F64vec。

\item {\bf 引数}

  acc\_pm: const PS::F32vec型またはconst PS::F64vec型。1つの粒子の
  Particle Meshによる力の計算結果。

\item {\bf 返値}

  なし。
  
\item {\bf 機能}

  1つの粒子のParticle Meshによる力の計算結果をこの粒子のメンバ変数に
  書き込む。
  
\item {\bf 備考}

  FPクラスのメンバ変数acc\_pmの変数名は変更可能。メンバ関数
  FP::copyFromForceParticleMeshの引数acc\_pmの引数名は変更可能。

\end{itemize}
\fi
%%%%%%%%%%%%%%%%%%%%%%%%%%%%%%%%%%%%%%%%%%%%%%%%%%%%%%
%%%%%%%%%%%%%%%%%%%%%%%%%%%%%%%%%%%%%%%%%%%%%%%%%%%%%%

\begin{screen}
\begin{verbatim}
class FP {
public:
    void copyFromForceParticleMesh(const PS::F32vec & acc_pm);
};
\end{verbatim}
\end{screen}

\begin{itemize}

\item {\bf 引数}

  acc\_pm: const PS::F32vec型またはconst PS::F64vec型。1つの粒子の
  Particle Meshによる力の計算結果。

\item {\bf 返値}

  なし。
  
\item {\bf 機能}

  1つの粒子のParticle Meshによる力の計算結果をこの粒子のメンバ変数に
  書き込む。
  
\end{itemize}


%%%%%%%%%%%%%%%%%%%%%%%%%%%%%%%%%%%%%%%%%%%%%%%%%%%%%%
%%%%%%%%%%%%%%%%%%%%%%%%%%%%%%%%%%%%%%%%%%%%%%%%%%%%%%
%%%%%%%%%%%%%%%%%%%%%%%%%%%%%%%%%%%%%%%%%%%%%%%%%%%%%%

\subsubsubsection{粒子交換時に粒子データをシリアライズして送る場合}
\label{sec:FP:serialize}

粒子交換時に粒子データをシリアライズして送る場合には、メンバ関数に
FP::packとFP::unpackを用意する必要がある。以下にそれぞれの規定を記述す
る。

%%%%%%%%%%%%%%%%%%%%%%%%%%%%%%%%%%%%%%%%%%%%%%%%%%%%%%

\subsubsubsubsection{FP::pack}

\begin{screen}
\begin{verbatim}
class FP {
public:
    static PS::S32 pack(const PS::S32 n_ptcl, const FP *ptcl[], char *buf, 
                        size_t & packed_size, const size_t max_buf_size);
};
\end{verbatim}
\end{screen}

\begin{itemize}

\item {\bf 引数}

  n\_ptcl: 粒子交換時に送る粒子の数。\\
  ptcl: 送る粒子へのポインタの配列。\\
  buf: 送信バッファーの先頭アドレス。\\
  packed\_size: ユーザーがバッファーへ書き込むサイズ。単位はバイト。\\
  max\_buf\_size: 送信バッファーの書き込み可能な領域のサイズ。単位はバイト。

\item {\bf 返値}

  PS::S32型。packed\_sizeがmax\_buf\_sizeを超えた場合は-1を返す。それ
  以外の場合は0を返す。
  
\item {\bf 機能}

 粒子交換時に送信する粒子をシリアライズし、送信バッファーに書き込む。

\end{itemize}

%%%%%%%%%%%%%%%%%%%%%%%%%%%%%%%%%%%%%%%%%%%%%%%%%%%%%%

\subsubsubsubsection{FP::unPack}

\begin{screen}
\begin{verbatim}
class FP {
public:
    static PS::S32 unPack(const PS::S32 n_ptcl, FP ptcl[], const char *buf);
};
\end{verbatim}
\end{screen}

\begin{itemize}

\item {\bf 引数}

  n\_ptcl: 粒子交換時に受け取る粒子の数。\\
  ptcl: 受け取る粒子の配列。\\
  buf: 受信バッファーの先頭アドレス。

\item {\bf 返値}

  PS::S32型。デシリアライズに失敗した場合は-1を返す。それ以外の場合は0
  を返す。

\item {\bf 機能}

 粒子交換時に受信する粒子をデシリアライズし、粒子配列に書き込む。詳し
 くはセクション\ref{sec:particleSystem:exchangeParticle}を参照。デシリ
 アライズに失敗した場合はPS::Abort()が呼ばれプログラムは終了する。

\end{itemize}


\subsection{EssentialParticleI}
\label{sec:essentialparticlei}

\subsubsection{概要}

EssentialParticleIクラスは相互作用の計算に必要なi粒子の情報を持つクラ
スであり、相互作用の定義(節\ref{sec:overview_action}の手順0)に必要とな
る。EssentialParticleIクラスはFullParticleクラス
(節\ref{sec:fullparticle})のサブセットである。FDPSは、このクラスのデー
タにアクセスする必要がある。そのため、EssentialParticleIクラスはいくつ
かのメンバ関数を持つ必要がある。以下、この節の前提、常に必要なメンバ関
数と、場合によっては必要なメンバ関数について記述する。

\subsubsection{前提}

この節の中では、EssentialParticleIクラスとしてEPIというクラスを一例と
して使う。また、FullParticleクラスの一例としてFPというクラスを使う。
EPI, FPというクラス名は変更可能である。

EPIとFPの宣言は以下の通りである。
\begin{screen}
\begin{verbatim}
class FP;
class EPI;
\end{verbatim}
\end{screen}

\subsubsection{必要なメンバ関数}

\subsubsubsection{概要}

常に必要なメンバ関数はEPI::getPosとEPI::copyfromFPである。EPI::getPos
はEPIクラスの位置情報をFDPSに読み込ませるための関数で、EPI::copyFromFP
はFPクラスの情報をEPIクラスに書きこむ関数である。これらのメンバ関数の
記述例と解説を以下に示す。

\subsubsubsection{EPI::getPos}

%%%%%%%%%%%%%%%%%%%%%%%%%%%%%%%%%%%%%%%%%%%%%%%%%%%%%%
%%%%%%%%%%%%%%%%%%%%%%%%%%%%%%%%%%%%%%%%%%%%%%%%%%%%%%
\if 0
\begin{screen}
\begin{verbatim}
class EPI {
public:
    PS::F64vec pos;
    PS::F64vec getPos() const {
        return this->pos;
    }
};
\end{verbatim}
\end{screen}

\begin{itemize}

\item {\bf 前提}
  
  EPIのメンバ変数posはある1つの粒子の位置情報。このposのデー
  タ型はPS::F64vec型。
  
\item {\bf 引数}

  なし
  
\item {\bf 返値}

  PS::F64vec型。EPIクラスの位置情報を保持したメンバ変数。
  
\item {\bf 機能}

  EPIクラスのオブジェクトの位置情報を保持したメンバ変数を返す。
  
\item {\bf 備考}

  EPIクラスのメンバ変数posの変数名は変更可能。

\end{itemize}
\fi
%%%%%%%%%%%%%%%%%%%%%%%%%%%%%%%%%%%%%%%%%%%%%%%%%%%%%%
%%%%%%%%%%%%%%%%%%%%%%%%%%%%%%%%%%%%%%%%%%%%%%%%%%%%%%

\begin{screen}
\begin{verbatim}
class EPI {
public:
    PS::F64vec getPos() const;
};
\end{verbatim}
\end{screen}

\begin{itemize}

\item {\bf 引数}

  なし
  
\item {\bf 返値}

  PS::F64vec型。EPIクラスの位置情報を保持したメンバ変数。
  
\item {\bf 機能}

  EPIクラスのオブジェクトの位置情報を保持したメンバ変数を返す。
  
\end{itemize}


\subsubsubsection{EPI::copyFromFP}

%%%%%%%%%%%%%%%%%%%%%%%%%%%%%%%%%%%%%%%%%%%%%%%%%%%%%%
%%%%%%%%%%%%%%%%%%%%%%%%%%%%%%%%%%%%%%%%%%%%%%%%%%%%%%
\if 0
\begin{screen}
\begin{verbatim}
class FP {
public:
    PS::S64    identity;
    PS::F64    mass;
    PS::F64vec position;
    PS::F64vec velocity;
    PS::F64vec acceleration;
    PS::F64    potential;
};
class EPI {
public:
    PS::S64    id;
    PS::F64vec pos;
    void copyFromFP(const FP & fp) {
        this->id  = fp.identity;
        this->pos = fp.position;
    }
};
\end{verbatim}
\end{screen}

\begin{itemize}

\item {\bf 前提}

  FPクラスのメンバ変数identity, positionとEPI
  クラスのメンバ変数id, posはそれぞれ対応する情報を持つ。

\item {\bf 引数}

  fp: 入力。const FP \&型。FPクラスの情報を持つ。
  
\item {\bf 返値}

  なし。
  
\item {\bf 機能}

  FPクラスの持つ1粒子の情報の一部をEssnetialParticleIクラス
  に書き込む。
  
\item {\bf 備考}

  FPクラスのメンバ変数の変数名、EPIクラスのメ
  ンバ変数の変数名は変更可能。メンバ関数EPI::copyFromFP
  の引数名は変更可能。EPIクラスの粒子情報はFP
  クラスの粒子情報のサブセット。対応する情報を持つメンバ変数同士のデー
  タ型が一致している必要はないが、実数型とベクトル型(または整数型とベ
  クトル型)という違いがある場合に正しく動作する保証はない。

\end{itemize}
\fi
%%%%%%%%%%%%%%%%%%%%%%%%%%%%%%%%%%%%%%%%%%%%%%%%%%%%%%
%%%%%%%%%%%%%%%%%%%%%%%%%%%%%%%%%%%%%%%%%%%%%%%%%%%%%%

\begin{screen}
\begin{verbatim}
class FP;
class EPI {
public:
    void copyFromFP(const FP & fp);
};
\end{verbatim}
\end{screen}

\begin{itemize}

\item {\bf 引数}

  fp: 入力。const FP \&型。FPクラスの情報を持つ。
  
\item {\bf 返値}

  なし。
  
\item {\bf 機能}

  FPクラスの持つ1粒子の情報の一部をEssnetialParticleIクラス
  に書き込む。
  
\end{itemize}

\subsubsection{場合によっては必要なメンバ関数}

\subsubsubsection{概要}

本節では、場合によっては必要なメンバ関数について記述する。相互作用ツリー
クラスのPS::SEARCH\_MODE型にPS::SEARCH\_MODE\_GATHERまたは
PS::SEARCH\_MODE\_SYMMETRYを用いる場合に必要となるメンバ関数ついて記述
する。

\subsubsubsection{相互作用ツリークラスのPS::SEARCH\_MODE型に\\PS::SEARCH\_MODE\_GATHERまたはPS::SEARCH\_MODE\_SYMMETRYを用いる場合}

\subsubsubsubsection{EPI::getRSearch}

%%%%%%%%%%%%%%%%%%%%%%%%%%%%%%%%%%%%%%%%%%%%%%%%%%%%%%
%%%%%%%%%%%%%%%%%%%%%%%%%%%%%%%%%%%%%%%%%%%%%%%%%%%%%%
\if 0
\begin{screen}
\begin{verbatim}
class EPI {
public:
    PS::F64 search_radius;
    PS::F64 getRSearch() const {
        return this->search_radius;
    }
};
\end{verbatim}
\end{screen}

\begin{itemize}

\item {\bf 前提}

  EPIクラスのメンバ変数search\_radiusはある1つの粒子の
  近傍粒子を探す半径の大きさ。このsearch\_radiusのデータ型はPS::F32型
  またはPS::F64型。
  
\item {\bf 引数}

  なし
  
\item {\bf 返値}

  PS::F32型またはPS::F64型。 EPIクラスの近傍粒子を探す
  半径の大きさを保持したメンバ変数。
  
\item {\bf 機能}

  EPIクラスの近傍粒子を探す半径の大きさを保持したメンバ
  変数を返す。

\item {\bf 備考}

  EPIクラスのメンバ変数search\_radiusの変数名は変更可能。
  
\end{itemize}
\fi
%%%%%%%%%%%%%%%%%%%%%%%%%%%%%%%%%%%%%%%%%%%%%%%%%%%%%%
%%%%%%%%%%%%%%%%%%%%%%%%%%%%%%%%%%%%%%%%%%%%%%%%%%%%%%

\begin{screen}
\begin{verbatim}
class EPI {
public:
    PS::F64 getRSearch() const;
};
\end{verbatim}
\end{screen}

\begin{itemize}

\item {\bf 引数}

  なし
  
\item {\bf 返値}

  PS::F32型またはPS::F64型。 EPIクラスの近傍粒子を探す
  半径の大きさを保持したメンバ変数。
  
\item {\bf 機能}

  EPIクラスの近傍粒子を探す半径の大きさを保持したメンバ
  変数を返す。

\end{itemize}

\subsection{EssentialParticleJ}
\label{sec:essentialparticlej}

\subsubsection{概要}

EssentialParticleJクラスは相互作用の計算に必要なj粒子の情報を持つクラ
スであり、相互作用の定義(節\ref{sec:overview_action}の手順0)に必要とな
る。EssentialParticleJクラスはFullParticleクラス
(節\ref{sec:fullparticle})のサブセットである。FDPSは、このクラスのデー
タにアクセスする必要がある。このために、EssentialParticleJクラスはいく
つかのメンバ関数を持つ必要がある。以下、この節の前提、常に必要なメンバ
関数と、場合によっては必要なメンバ関数について記述する。

\subsubsection{前提}

この節の中では、EssentialParticleJクラスとしてEPJというクラスを一例と
して使う。また、FullParticleクラスの一例としてFPというクラスを使う。
EPJ, FPというクラス名は変更可能である。

EPJとFPの宣言は以下の通りである。
\begin{screen}
\begin{verbatim}
class FP;
class EPJ;
\end{verbatim}
\end{screen}

\subsubsection{必要なメンバ関数}

\subsubsubsection{概要}

常に必要なメンバ関数はEPJ::getPosとEPJ::copyfromFPである。EPJ::getPos
はEPJクラスの位置情報をFDPSに読み込ませるための関数で、EPJ::copyFromFP
はFPクラスの情報をEPJクラスに書きこむ関数である。これら
のメンバ関数の記述例と解説を以下に示す。

\subsubsubsection{EPJ::getPos}

%%%%%%%%%%%%%%%%%%%%%%%%%%%%%%%%%%%%%%%%%%%%%%%%%%%%%%
%%%%%%%%%%%%%%%%%%%%%%%%%%%%%%%%%%%%%%%%%%%%%%%%%%%%%%
\if 0
\begin{screen}
\begin{verbatim}
class EPJ {
public:
    PS::F64vec pos;
    PS::F64vec getPos() const {
        return this->pos;
    }
};
\end{verbatim}
\end{screen}

\begin{itemize}

\item {\bf 前提}
  
  EPJのメンバ変数posはある1つの粒子の位置情報。このposのデー
  タ型はPS::F64vec型。
  
\item {\bf 引数}

  なし
  
\item {\bf 返値}

  PS::F64vec型。EPJクラスの位置情報を保持したメンバ変数。
  
\item {\bf 機能}

  EPJクラスの位置情報を保持したメンバ変数を返す。
  
\item {\bf 備考}

  EPJクラスのメンバ変数posの変数名は変更可能。

\end{itemize}
\fi
%%%%%%%%%%%%%%%%%%%%%%%%%%%%%%%%%%%%%%%%%%%%%%%%%%%%%%
%%%%%%%%%%%%%%%%%%%%%%%%%%%%%%%%%%%%%%%%%%%%%%%%%%%%%%

\begin{screen}
\begin{verbatim}
class EPJ {
public:
    PS::F64vec getPos() const;
};
\end{verbatim}
\end{screen}

\begin{itemize}

\item {\bf 引数}

  なし
  
\item {\bf 返値}

  PS::F64vec型。EPJクラスの位置情報を保持したメンバ変数。
  
\item {\bf 機能}

  EPJクラスの位置情報を保持したメンバ変数を返す。
  
\end{itemize}


\subsubsubsection{EPJ::copyFromFP}

%%%%%%%%%%%%%%%%%%%%%%%%%%%%%%%%%%%%%%%%%%%%%%%%%%%%%%
%%%%%%%%%%%%%%%%%%%%%%%%%%%%%%%%%%%%%%%%%%%%%%%%%%%%%%
\if 0
\begin{screen}
\begin{verbatim}
class FP {
public:
    PS::S64    identity;
    PS::F64    mass;
    PS::F64vec position;
    PS::F64vec velocity;
    PS::F64vec acceleration;
    PS::F64    potential;
};
class EPJ {
public:
    PS::S64    id;
    PS::F64    m;
    PS::F64vec pos;
    void copyFromFP(const FP & fp) {
        this->id  = fp.identity;
        this->m   = fp.mass;
        this->pos = fp.position;
    }
};
\end{verbatim}
\end{screen}

\begin{itemize}

\item {\bf 前提}

  FPクラスのメンバ変数identity, mass, positionと
  EPJクラスのメンバ変数id, m, posはそれぞれ対応する情報
  を持つ。

\item {\bf 引数}

  fp: 入力。const FP \&型。FPクラスの情報を持つ。
  
\item {\bf 返値}

  なし。
  
\item {\bf 機能}

  FPクラスの持つ1粒子の情報の一部をEPJクラスに書き込む。
  
\item {\bf 備考}

  FPクラスのメンバ変数の変数名、EPJクラスのメンバ変数の変数名は変更可
  能。メンバ関数EPJ::copyFromFPの引数名は変更可能。対応する情報を持つ
  メンバ変数同士のデータ型が一致している必要はないが、実数型とベクトル
  型(または整数型とベクトル型)という違いがある場合に正しく動作する保証
  はない。

\end{itemize}
\fi
%%%%%%%%%%%%%%%%%%%%%%%%%%%%%%%%%%%%%%%%%%%%%%%%%%%%%%
%%%%%%%%%%%%%%%%%%%%%%%%%%%%%%%%%%%%%%%%%%%%%%%%%%%%%%

\begin{screen}
\begin{verbatim}
class FP;
class EPJ {
public:
    void copyFromFP(const FP & fp);
};
\end{verbatim}
\end{screen}

\begin{itemize}

\item {\bf 引数}

  fp: 入力。const FP \&型。FPクラスの情報を持つ。
  
\item {\bf 返値}

  なし。
  
\item {\bf 機能}

  FPクラスの持つ1粒子の情報の一部をEPJクラスに書き込む。
  
\end{itemize}

\subsubsection{場合によっては必要なメンバ関数}

\subsubsubsection{概要}

本節では、場合によっては必要なメンバ関数について記述する。相互作用ツリー
クラスのPS::SEARCH\_MODE型にPS::SEARCH\_MODE\_LONG以外を用いる場合に必
要なメンバ関数、列挙型のBOUNDARY\_CONDITION型に
PS::BOUNDARY\_CONDITION\_OPEN以外を選んだ場合に必要となるメンバ関数に
ついて記述する。なお、既存のMomentクラスやSuperParticleJクラスを用いる
際に必要となるメンバ変数はこれら既存のクラスの節を参照のこと。

\subsubsubsection{相互作用ツリークラスのPS::SEARCH\_MODE型に\\PS::SEARCH\_MODE\_LONG以外を用いる場合}

\subsubsubsubsection{EPJ::getRSearch}

\begin{screen}
\begin{verbatim}
class EPJ {
public:
    PS::F64 getRSearch() const;
};
\end{verbatim}
\end{screen}

\begin{itemize}

\item {\bf 引数}

  なし
  
\item {\bf 返値}

  PS::F32型またはPS::F64型。 EPJクラスの近傍粒子を探す
  半径の大きさを保持したメンバ変数。
  
\item {\bf 機能}

  EPJクラスの近傍粒子を探す半径の大きさを保持したメンバ
  変数を返す。
  
  なお、PS::SEARCH\_MODE型にPS::SEARCH\_MODE\_LONG\_CUTOFFを用いる場合、すべてのEPJが同じ値を返すように定義しなければならない。


\end{itemize}

\subsubsubsection{BOUNDARY\_CONDITION型にPS::BOUNDARY\_CONDITION\_OPEN以外を用いる場合}

\subsubsubsubsection{EPJ::setPos}

\begin{screen}
\begin{verbatim}
class EPJ {
public:
    void setPos(const PS::F64vec pos_new);
};
\end{verbatim}
\end{screen}

\begin{itemize}

\item {\bf 引数}

  pos\_new: 入力。const PS::F32vecまたはconst PS::F64vec型。FDPS側で修
  正した粒子の位置情報。

\item {\bf 返値}

  なし。
  
\item {\bf 機能}

  FDPSが修正した粒子の位置情報をEPJクラスの位置情報に書き込む。

\end{itemize}


%%%%%%%%%%%%%%%%%%%%%%%%%%%%%%%%%%%%%%%%%%%%%%%%%%%%%%
%%%%%%%%%%%%%%%%%%%%%%%%%%%%%%%%%%%%%%%%%%%%%%%%%%%%%%
\subsubsubsection{粒子のid番号から対応するEPJを取得したい場合}
\subsubsubsubsection{EPJ::getId}
\label{sec:EPJ:getId}

\begin{screen}
\begin{verbatim}
class EPJ {
public:
    PS::S64 getId();
};
\end{verbatim}
\end{screen}

\begin{itemize}

\item {\bf 引数}

なし。

\item {\bf 返値}

PS::S64型。
  
\item {\bf 機能}

PS::TreeForForce::getEpjFromId()を使用する場合に必要。詳しく
は\ref{sec:getEpjFromId}を参照。


\end{itemize}

%%%%%%%%%%%%%%%%%%%%%%%%%%%%%%%%%%%%%%%%%%%%%%%%%%%%%%
%%%%%%%%%%%%%%%%%%%%%%%%%%%%%%%%%%%%%%%%%%%%%%%%%%%%%%
%%%%%%%%%%%%%%%%%%%%%%%%%%%%%%%%%%%%%%%%%%%%%%%%%%%%%%

\subsubsubsection{LET交換時に粒子データをシリアライズして送る場合}
\label{sec:EPJ:serialize}

LET交換時に粒子データをシリアライズして送る場合には、メンバ関数に
EPJ::packとEPJ::unpackを用意する必要がある。以下にそれぞれの規定を記述
する。

%%%%%%%%%%%%%%%%%%%%%%%%%%%%%%%%%%%%%%%%%%%%%%%%%%%%%%

\subsubsubsubsection{EPJ::pack}

\begin{screen}
\begin{verbatim}
class EPJ {
public:
    static PS::S32 pack(const PS::S32 n_ptcl, const EPJ *ptcl[], char *buf, 
                        size_t & packed_size, const size_t max_buf_size);
};
\end{verbatim}
\end{screen}

\begin{itemize}

\item {\bf 引数}

  n\_ptcl: LET交換時に送る粒子の数。\\
  ptcl: 送る粒子へのポインタの配列。\\
  buf: 送信バッファーの先頭アドレス。\\
  packed\_size: ユーザーがバッファーへ書き込むサイズ。単位はバイト。\\
  max\_buf\_size: 送信バッファーの書き込み可能な領域のサイズ。単位はバイト。

\item {\bf 返値}

  PS::S32型。packed\_sizeがmax\_buf\_sizeを超えた場合は-1を返す。それ
  以外の場合は0を返す。
  
\item {\bf 機能}

  LET交換時に送信する粒子をシリアライズし、送信バッファーに書き込む。

\end{itemize}

%%%%%%%%%%%%%%%%%%%%%%%%%%%%%%%%%%%%%%%%%%%%%%%%%%%%%%

\subsubsubsubsection{EPJ::unPack}

\begin{screen}
\begin{verbatim}
class EPJ {
public:
    static void unPack(const PS::S32 n_ptcl, EPJ ptcl[], const char *buf);
};
\end{verbatim}
\end{screen}

\begin{itemize}

\item {\bf 引数}

  n\_ptcl: LET交換時に受け取る粒子の数。\\
  ptcl: 受け取る粒子の配列。\\
  buf: 受信バッファーの先頭アドレス。\\

\item {\bf 返値}

  なし。

\item {\bf 機能}

 LET交換時に受信する粒子をデシリアライズし、粒子配列に書き込む。詳しく
 はセクション\ref{sec:treeForForceHighLevelAPI}を参照。デシリアライズ
 に失敗した場合はPS::Abort()が呼ばれプログラムは終了する。

\end{itemize}

\subsection{Moment}
\label{sec:moment}

\subsubsection{Summary}

The class \texttt{Moment} encapsulates the values of moment and related
quantities of a set of grouped particles
(see, step 0 in Sec. \ref{sec:overview_action}).
The examples of moment are monopole, dipole and radius of the largest
particle and so on. This class is used for intermediate variables to
make \texttt{SuperParticleJ} from \texttt{EssentialParticleJ}.
Thus, this class has a member function that calculates the moment from
\texttt{EssentialParticleJ} or \texttt{SuperParticleJ}.

Since these processes have prescribed form, FDPS has pre-defined classes.
In this section we first describe the pre-defined class and then describe
the rules in the case users define it.

\subsubsection{Pre-defined class}

\subsubsubsection{Summary}

FDPS has several pre-defined \texttt{Moment} classes, which are used if
specific case are chosen as \texttt{PS::SEARCH\_MODE} in \texttt{PS::TreeForForce}.
Below, we describe about the \texttt{Moment} class which can be chosen in each \texttt{PS::SEARCH\_MODE}.

\subsubsubsection{PS::SEARCH\_MODE\_LONG}

\subsubsubsubsection{PS::MomentMonopole}
\label{sec:MomentMonopole}

This class encapsulates the monopole moment.
The reference point for the calculation of monopole is set to center of mass or center of charge.

\begin{screen}
\begin{verbatim}
namespace ParticleSimulator {
    class MomentMonopole {
    public:
        F64    mass;
        F64vec pos;
    };
}
\end{verbatim}
\end{screen}

\begin{itemize}

\item Name of the class

  \texttt{PS::MomentMonopole}

\item Members and their information

  \texttt{mass}: accumulated mass or electron charge.

  \texttt{pos}: center of mass or center of charge.

\item Terms of use

  The class \texttt{EssentialParticleJ} (see, Sec. \ref{sec:essentialparticlej}) has \texttt{EssentialParticleJ::getCharge} and \texttt{EssentialParticleJ::getPos}
  and returns mass/electron charge and position.
  Users can use an arbitrary name instead of \texttt{EssentialParticleJ}.

\end{itemize}

\subsubsubsubsection{PS::MomentQuadrupole}
\label{sec:MomentQuadrupole}

This class encapsulates the monopole and quadrupole moment.
The reference point for the calculation of moments is set to the center of mass.

\begin{screen}
\begin{verbatim}
namespace ParticleSimulator {
    class MomentQuadrupole {
    public:
        F64    mass;    
        F64vec pos;
        F64mat quad;
    };
}
\end{verbatim}
\end{screen}

\begin{itemize}

\item Name of the class

  \texttt{PS::MomentQuadrupole}

\item Members and their information

  \texttt{mass}: accumulated mass.

  \texttt{pos}: center of mass.

  \texttt{quad}: accumulated quadrupole.

\item Terms of use

  The class \texttt{EssentialParticleJ} (see, Sec. \ref{sec:essentialparticlej}) has \texttt{EssentialParticleJ::getCharge} and \texttt{EssentialParticleJ::getPos}
  and returns mass/electron charge and position.
  Users can use an arbitrary name instead of \texttt{EssentialParticleJ}.

\end{itemize}

\subsubsubsubsection{PS::MomentMonopoleGeometricCenter}
\label{sec:MomentMonopoleGeometricCenter}


This class encapsulates the monopole moment.
The reference point for the calculation of monopole is set to the geometric center.

\begin{screen}
\begin{verbatim}
namespace ParticleSimulator {
    class MomentMonopoleGeometricCenter {
    public:
        F64    charge;    
        F64vec pos;
    };
}
\end{verbatim}
\end{screen}

\begin{itemize}

\item Name of the class

  \texttt{PS::MomentMonopoleGeometricCenter}

\item Members and their information

  \texttt{charge}: accumulated mass/charge.

  \texttt{pos}: geometric center.

\item Terms of use

  The class \texttt{EssentialParticleJ} (see, Sec. \ref{sec:essentialparticlej}) has \texttt{EssentialParticleJ::getCharge} and \texttt{EssentialParticleJ::getPos}
  and returns mass/electron charge and position.
  Users can use an arbitrary name instead of \texttt{EssentialParticleJ}.

\end{itemize}

\subsubsubsubsection{PS::MomentDipoleGeometricCenter}
\label{sec:MomentDipoleGeometricCenter}

This class encapsulates the moments up to dipole.
The reference point for the calculation of moments is set to geometric center.

\begin{screen}
\begin{verbatim}
namespace ParticleSimulator {
    class MomentDipoleGeometricCenter {
    public:
        F64    charge;    
        F64vec pos;
        F64vec dipole;
    };
}
\end{verbatim}
\end{screen}

\begin{itemize}

\item Name of the class

  \texttt{PS::MomentDipoleGeometricCenter}

\item Members and their information

  \texttt{charge}: accumulated mass/charge.

  \texttt{pos}: geometric center.

  \texttt{dipole}: dipole of the particle mass or electric charge.

\item Terms of use

  The class \texttt{EssentialParticleJ} (see, Sec. \ref{sec:essentialparticlej})
  has \texttt{EssentialParticleJ::getCharge} and \texttt{EssentialParticleJ::getPos}
  and returns mass/electron charge and position.
  Users can use an arbitrary name instead of \texttt{EssentialParticleJ}.

\end{itemize}

\subsubsubsubsection{PS::MomentQuadrupoleGeometricCenter}
\label{sec:MomentQuadrupoleGeometricCenter}

This class encapsulates the moments up to quadrupole.
The reference point for the calculation of moments is set to geometric center.

\begin{screen}
\begin{verbatim}
namespace ParticleSimulator {
    class MomentQuadrupoleGeometricCenter {
    public:
        F64    charge;    
        F64vec pos;
        F64vec dipole;
        F64mat quadrupole;
    };
}
\end{verbatim}
\end{screen}

\begin{itemize}

\item Name of the class

  \texttt{PS::MomentQuadrupoleGeometricCenter}

\item Members and their information

  \texttt{charge}: accumulated mass/charge.

  \texttt{pos}: geometric center.

  \texttt{dipole}: dipole of the particle mass or electron charge.

  \texttt{quadrupole}: quadrupole of the particle mass or electron charge.

\item Terms of use

  The class \texttt{EssentialParticleJ} (see, Sec. \ref{sec:essentialparticlej})
  has \texttt{EssentialParticleJ::getCharge} and \texttt{EssentialParticleJ::getPos}
  and returns mass/electron charge and position.
  Users can use an arbitrary name instead of \texttt{EssentialParticleJ}.

\end{itemize}
  
\subsubsubsection{PS::SEARCH\_MODE\_LONG\_SCATTER}

\subsubsubsubsection{PS::MomentMonopoleScatter}
\label{sec:MomentMonopoleScatter}

This class encapsulates the monopole moment. The reference point for the calculation of monopole is set to center of mass or center of charge.

\begin{screen}
\begin{verbatim}
namespace ParticleSimulator {
    class MomentMonopoleScatter {
    public:
        F64    mass;
        F64vec pos;
        F64ort vertex_out_;
        F64ort vertex_in_;
    };
}
\end{verbatim}
\end{screen}

\begin{itemize}
\item Name of the class
  \texttt{PS::MomentMonopoleScatter}

\item Members and their information

  \texttt{mass}: accumulated mass or electron charge.

  \texttt{pos}: center of mass or center of charge.
        
  \texttt{vertex\_out\_}: positions of vertices of the smallest rectangular parallelepiped including all grouped particles when each particle is regarded as a sphere whose radius is the returned value of EssentialParticleJ::getRSearch.
  
  \texttt{vertex\_in\_}: positions of vertices of the smallest rectangular parallelepiped including all grouped particles.

\item Terms of use

  The class \texttt{EssentialParticleJ} (see, Sec. \ref{sec:essentialparticlej}) has \texttt{EssentialParticleJ::getCharge}, \texttt{EssentialParticleJ::getPos} and \texttt{EssentialParticleJ::getRSearch} and returns mass/electron charge, position and cutoff length. Users can use an arbitrary name instead of the name of member functions of \texttt{EssentialParticleJ}.

\end{itemize}

\subsubsubsubsection{PS::MomentQuadrupoleScatter}
\label{sec:MomentQuadrupoleScatter}

This class encapsulates the monopole and quadrupole moments. The reference point for the calculation of monopole is set to center of mass or center of charge.

\begin{screen}
\begin{verbatim}
namespace ParticleSimulator {
    class MomentQuadrupoleScatter {
    public:
        F64    mass;
        F64vec pos;
        F64mat quad;
        F64ort vertex_out_;
        F64ort vertex_in_;
    };
}
\end{verbatim}
\end{screen}

\begin{itemize}
\item Name of the class
        
  \texttt{PS::MomentQuadrupoleScatter}

\item Members and their information

  \texttt{mass}: accumulated mass or electron charge.

  \texttt{pos}: center of mass or center of charge.
        
  \texttt{quad}: accumulated quadrupole.
        
  \texttt{vertex\_out\_}: positions of vertices of the smallest rectangular parallelepiped including all grouped particles when each particle is regarded as a sphere whose radius is the returned value of EssentialParticleJ::getRSearch.
  
  \texttt{vertex\_in\_}: positions of vertices of the smallest rectangular parallelepiped including all grouped particles.

\item Terms of use

  The class \texttt{EssentialParticleJ} (see, Sec. \ref{sec:essentialparticlej}) has \texttt{EssentialParticleJ::getCharge}, \texttt{EssentialParticleJ::getPos} and \texttt{EssentialParticleJ::getRSearch} and returns mass/electron charge, position and cutoff length. Users can use an arbitrary name instead of the name of member functions of \texttt{EssentialParticleJ}.

\end{itemize}

\subsubsubsection{PS::SEARCH\_MODE\_LONG\_SYMMETRY}

\subsubsubsubsection{PS::MomentMonopoleSymmetry}
\label{sec:MomentMonopoleSymmetry}

This class encapsulates the monopole moment. The reference point for the calculation of monopole is set to center of mass or center of charge.

\begin{screen}
\begin{verbatim}
namespace ParticleSimulator {
    class MomentMonopoleSymmetry {
    public:
        F64    mass;
        F64vec pos;
        F64ort vertex_out_;
        F64ort vertex_in_;
    };
}
\end{verbatim}
\end{screen}

\begin{itemize}
\item Name of the class
        
  \texttt{PS::MomentMonopoleSymmetry}

\item Members and their information

  \texttt{mass}: accumulated mass or electron charge.

  \texttt{pos}: center of mass or center of charge.
        
  \texttt{vertex\_out\_}: positions of vertices of the smallest rectangular parallelepiped including all grouped particles when each particle is regarded as a sphere whose radius is the returned value of EssentialParticleJ::getRSearch.
  
  \texttt{vertex\_in\_}: positions of vertices of the smallest rectangular parallelepiped including all grouped particles.

\item Terms of use

  The class \texttt{EssentialParticleJ} (see, Sec. \ref{sec:essentialparticlej}) has \texttt{EssentialParticleJ::getCharge}, \texttt{EssentialParticleJ::getPos} and \texttt{EssentialParticleJ::getRSearch} and returns mass/electron charge, position and cutoff length. Users can use an arbitrary name instead of the name of member functions of \texttt{EssentialParticleJ}.

\end{itemize}

\subsubsubsubsection{PS::MomentQuadrupoleSymmetry}
\label{sec:MomentQuadrupoleSymmetry}

This class encapsulates the monopole and quadrupole moments. The reference point for the calculation of monopole is set to center of mass or center of charge.

\begin{screen}
\begin{verbatim}
namespace ParticleSimulator {
    class MomentQuadrupoleSymmetry {
    public:
        F64    mass;
        F64vec pos;
        F64mat quad;
        F64ort vertex_out_;
        F64ort vertex_in_;
    };
}
\end{verbatim}
\end{screen}

\begin{itemize}

\item Name of the class

  \texttt{PS::MomentQuadrupoleSymmetry}

\item Members and their information

  \texttt{mass}: accumulated mass or electron charge.

  \texttt{pos}: center of mass or center of charge.
        
  \texttt{quad}: accumulated quadrupole.
        
  \texttt{vertex\_out\_}: positions of vertices of the smallest rectangular parallelepiped including all grouped particles when each particle is regarded as a sphere whose radius is the returned value of EssentialParticleJ::getRSearch.
  
  \texttt{vertex\_in\_}: positions of vertices of the smallest rectangular parallelepiped including all grouped particles.

\item Terms of use

  The class \texttt{EssentialParticleJ} (see, Sec. \ref{sec:essentialparticlej}) has \texttt{EssentialParticleJ::getCharge}, \texttt{EssentialParticleJ::getPos} and \texttt{EssentialParticleJ::getRSearch} and returns mass/electron charge, position and cutoff length. Users can use an arbitrary name instead of the name of member functions of \texttt{EssentialParticleJ}.

\end{itemize}

\subsubsubsection{PS::SEARCH\_MODE\_LONG\_CUTOFF}

\subsubsubsubsection{PS::MomentMonopoleCutoff}
\label{sec:MomentMonopoleCutoff}

This class encapsulates the monopole moment.
The reference point for the calculation of monopole is set to center of mass or center of charge.

\begin{screen}
\begin{verbatim}
namespace ParticleSimulator {
    class MomentMonopoleCutoff {
    public:
        F64    mass;
        F64vec pos;
    };
}
\end{verbatim}
\end{screen}

\begin{itemize}

\item Name of the class

  \texttt{PS::MomentMonopoleCutoff}

\item Members and their information

  \texttt{mass}: accumulated mass or electron charge.

  \texttt{pos}: center of mass or center of charge.

\item Terms of use

  The class \texttt{EssentialParticleJ} (see, Sec. \ref{sec:essentialparticlej})
  has \texttt{EssentialParticleJ::getCharge}, \texttt{EssentialParticleJ::getPos}
  and \texttt{EssentialParticleJ::getRSearch} and returns mass/electron charge,
  position and cutoff length. Users can use an arbitrary name instead of the name
  of member functions of \texttt{EssentialParticleJ}.

\end{itemize}

\subsubsection{Required member functions}

\subsubsubsection{Summary}

%以下ではMomentクラスを定義する際に、必要なメンバ関数を記述する。このと
%きMomentクラスのクラス名をMomとする。これは変更自由である。
Below, the required member functions of \texttt{Moment} class are described.
In this section we use the name \texttt{Mom} as a \texttt{Moment} class.

\subsubsubsection{Constructor}

\begin{screen}
\begin{verbatim}
class Mom {
public:
    Mom ();
};
\end{verbatim}
\end{screen}

\begin{itemize}
  
\item {\bf Arguments}

  None.
  
\item {\bf Returns}

  None.

\item {\bf Behaviour}

  %Momクラスのオブジェクトの初期化をする。
  Initialize the instance of \texttt{Mom} class.
  
\end{itemize}

\subsubsubsection{Mom::init}

\begin{screen}
\begin{verbatim}
class Mom {
public:
    void init();
};
\end{verbatim}
\end{screen}

\begin{itemize}
  
\item {\bf Arguments}

  None.
  
\item {\bf Returns}

  None.

\item {\bf Behaviour}

  %Momクラスのオブジェクトの初期化をする。
  Initialize the instance of \texttt{Mom} class.

\end{itemize}

\subsubsubsection{Mom::getPos}

\begin{screen}
\begin{verbatim}
class Mom {
public:
    PS::F32vec getPos();
};
\end{verbatim}
\end{screen}

\begin{itemize}

\item {\bf Arguments}

  None.
  
\item {\bf Returns}

  %PS::F32vecまたはPS::F64vec型。Momクラスのメンバ変数pos。
  \texttt{PS::F32vec} or \texttt{PS::F64vec}.
  Returns the position of a variable of class.

\end{itemize}

\subsubsubsection{Mom::getCharge}

\begin{screen}
\begin{verbatim}
class Mom {
public:
    PS::F32 getCharge() const;
};
\end{verbatim}
\end{screen}

\begin{itemize}

\item {\bf Arguments}

  None.
  
\item {\bf Returns}

  %PS::F32またはPS::F64型。Momクラスのメンバ変数mass。
  \texttt{PS::F32} or \texttt{PS::F64}.
  Returns the mass or charge of a variable of class \texttt{Mom}.

\end{itemize}

\subsubsubsection{Mom::accumulateAtLeaf}

\begin{screen}
\begin{verbatim}
class Mom {
public:
    template <class Tepj>
    void accumulateAtLeaf(const Tepj & epj);
};
\end{verbatim}
\end{screen}

\begin{itemize}

\item {\bf Arguments}

  %epj: 入力。const Tepj \&型。Tepjのオブジェクト。
  \texttt{epj}: Input. \texttt{const Tepj \&} type. The object of \texttt{Tepj}.

\item {\bf Returns}

  None.

\item {\bf Behaviour}

  %EssentialParticleJクラスのオブジェクトからモーメントを計算する。
  Accumulate the multipole moment from \texttt{EssentialParticleJ}.

\end{itemize}

\subsubsubsection{Mom::accumulate}

\begin{screen}
\begin{verbatim}
class Mom {
public:
    void accumulate(const Mom & mom);
};
\end{verbatim}
\end{screen}

\begin{itemize}

\item {\bf Arguments}

  %mom: 入力。const Mom \&型。Momクラスのオブジェクト。
  \texttt{mom}: Input. \texttt{const Mom \&} type. The instance of \texttt{Mom} class.

\item {\bf Returns}

  None.

\item {\bf Behaviour}

  %MomクラスのオブジェクトからさらにMomクラスの情報を計算する。
  Accumulate the multipole moments in \texttt{Mom} class from \texttt{Mom} class objects.

\end{itemize}

\subsubsubsection{Mom::set}

\begin{screen}
\begin{verbatim}
class Mom {
public:
    void set();
};
\end{verbatim}
\end{screen}

\begin{itemize}

\item {\bf Arguments}

  None.
  
\item {\bf Returns}

  None.

\item {\bf Behaviour}

  %上記のメンバ関数Mom::accumulateAtLeaf, Mom::accumulateではモー
  %メントの位置情報の規格化ができていない場合ので、ここで規格化する。
  Normalize the multipole moments, since the member functions
  \texttt{Mom::accumulateAtLeaf} and \texttt{Mom::accumulate} do not
  normalize the multipole moment.

\end{itemize}

\subsubsubsection{Mom::accumulateAtLeaf2}

\begin{screen}
\begin{verbatim}
class Mom {
public:
    template <class Tepj>
    void accumulateAtLeaf2(const Tepj & epj);
};
\end{verbatim}
\end{screen}

\begin{itemize}

\item {\bf Arguments}

  %epj: 入力。const Tepj \&型。Tepjのオブジェクト。
  \texttt{epj}: Input. \texttt{const Tepj \&} type. The instance of \texttt{Teps}.

\item {\bf Returns}

  None.

\item {\bf Behaviour}

  %EssentialParticleJクラスのオブジェクトからモーメントを計算する。
  Accumulate the multipole moments in \texttt{Mom} class from \texttt{EssentialParticleJ} class objects.

\end{itemize}

\subsubsubsection{Mom::accumulate2}

\begin{screen}
\begin{verbatim}
class Mom {
public:
    void accumulate(const Mom & mom);
};
\end{verbatim}
\end{screen}

\begin{itemize}

\item {\bf Arguments}

  %mom: 入力。const Mom \&型。Momクラスのオブジェクト。
  \texttt{mom}: Input. \texttt{const Mom \&} type. The instance of \texttt{Mom} class.

\item {\bf Returns}

  None.

\item {\bf Behaviour}

  %MomクラスのオブジェクトからさらにMomクラスの情報を計算する。
  Accumulate the multipole moments in \texttt{Mom} class from \texttt{Mom} class objects.

\end{itemize}

%%%%%%%%%%%%%%%%%%%%%%%%%%%%%%%%%%%%%%%%%%%%%%%%%%%%%%
\subsubsection{Required member functions for specific cases}

\subsubsubsection{Summary}

Below, the required member functions of \texttt{Moment} class for specific cases are described. In this section we use the name \texttt{Mom} as a \texttt{Moment} class.

\subsubsubsection{One of \texttt{PS::SEARCH\_MODE\_LONG\_CUTOFF}, \newline \texttt{PS::SEARCH\_MODE\_LONG\_SCATTER}, \newline \texttt{PS::SEARCH\_MODE\_LONG\_SYMMETRY} \newline is used as \texttt{PS::SEARCH\_MODE}}
%--------------
\subsubsubsection{Mom::getVertexIn}

\begin{screen}
\begin{verbatim}
class Mom {
public:
    F64ort getVertexIn();
};
\end{verbatim}
\end{screen}

\begin{itemize}

\item {\bf Arguments}

None.
  
\item {\bf Returns}

\texttt{PS::F32ort} or \texttt{PS::F64ort} types.

\item {\bf Behaviour}

  Return the positions of two vertices describing the smallest cuboid (rectangle if the space dimension is two) that contains all the particles corresponding to this \texttt{Mom} class.

\item {\bf Notes}

The positional information above must be calculated correctly in each tree cell. For this end, users must implement coordinate calculations into the member functions \texttt{accumulateAtLeaf} and \texttt{accumulate}, which are used, respectively, to calculate the moment information at the leaf cells (i.e. the tree cells at the deepest levels of a tree structure) from particles and to calculate the moment information of a parent tree cell from its child tree cells. Below, we show an example of the implementation of the member function \texttt{getVertexIn}.
  
\begin{screen}
\begin{verbatim}
class Mom {
public:
    // vertex_in_: Member variable storing the positional
    //             information of a cuboid (or rectangle)
    F64ort vertex_in_; 
    F64ort getVertexIn() const { return vertex_in_; }
    template<class Tepj>
    void accumulateAtLeaf(const Tepj & epj){ 
         // Other calculations
        (this->vertex_in_).merge(epj.getPos());
    }
    void accumulate(const Mom & mom){
        // Other calculations
        (this->vertex_in_).merge(mom.vertex_in_);
    }
};
\end{verbatim}
\end{screen}
where \texttt{merge} is one of the member functions of Orthotope type.
Users can freely change the name of this member variable.
  
\end{itemize}

%--------------
\subsubsubsection{Mom::getVertexOut}

\begin{screen}
\begin{verbatim}
class Mom {
public:
    F64ort getVertexOut();
};
\end{verbatim}
\end{screen}

\begin{itemize}

\item {\bf Arguments}

None.
  
\item {\bf Returns}

\texttt{PS::F32ort} or \texttt{PS::F64ort} type.

\item {\bf Behaviour}

Return the positions of two vertices describing the smallest cuboid (rectangle if the space dimension is two) that contains all of spheres (circles if the space dimension is two) whose centers and radii are, respectively, the positions and the returned values of \texttt{getRSearch} of particles corresponding to this \texttt{Mom} class. 

\item {\bf Notes}

As in the case of member function \texttt{getVertexIn}, the positional information above must be calculated correctly in each tree cell. For this end, users must implement coordinate calculations in the member functions \texttt{accumulateAtLeaf} and \texttt{accumulate}.Below, we show an example of the implementation of the member function \texttt{getVertexOut}.
  
\begin{screen}
\begin{verbatim}
class Mom {
public:
    // vertex_out_: Member variable storing the positional
    //              information of a cuboid (or rectangle)
    F64ort vertex_out_; 
    F64ort getVertexOut() const { return vertex_out_; }
    template<class Tepj>
    void accumulateAtLeaf(const Tepj & epj){ 
         // Other calculations
         (this->vertex_out_).merge(epj.getPos(), epj.getRSearch());
    }
    void accumulate(const Mom & mom){
        // Other calculations
        (this->vertex_out_).merge(mom.vertex_out_);
    }
};
\end{verbatim}
\end{screen}
where \texttt{merge} is a member function of Orthotope type.
Users can freely change the name of the member variable.
  
\end{itemize}

\subsection{SuperParticleJ}
\label{sec:superparticlej}

\subsubsection{概要}

SuperParticleJクラスは近い粒子同士でまとまった複数の粒子を代表してまと
めた超粒子の情報を持つクラスであり、相互作用の定義
(節\ref{sec:overview_action}の手順0)に必要となる。このクラスが必要とな
るのはPS::SEARCH\_MODEに以下のいずれかを選択した場合だけである。
\begin{itemize}[itemsep=-1ex]
\item PS::SEARCH\_MODE\_LONG
\item PS::SEARCH\_MODE\_LONG\_SCATTER
\item PS::SEARCH\_MODE\_LONG\_SYMMETRY
\item PS::SEARCH\_MODE\_LONG\_CUTOFF
\end{itemize}
このクラスのメンバ関数には、超粒子の位置情報をFDPS側とやりとりする
メンバ関数がある。また、超粒子の情報とMomentクラスの情報は対になる
ものである。従って、このクラスのメンバ関数には、Momentクラスから
このクラスへ情報を変換(またはその逆変換)するメンバ関数がある。

SuperParticleJクラスもMomentクラス同様、ある程度決っているものが多いの
で、それらについてはFDPS側で用意した。以下、既存のクラス、
SuperParticleJクラスを作るときに必要なメンバ関数、場合によっては必要な
メンバ関数について記述する。

\subsubsection{既存のクラス}

FDPSはいくつかのSuperParticleJクラスを用意している。以下、各
PS::SEARCH\_MODEに対し選ぶことのできるクラスについて記述する。
PS::SEARCH\_MODE\_LONG、PS::SEARCH\_MODE\_LONG\_SCATTER、
PS::SEARCH\_MODE\_LONG\_SYMMETRY、PS::SEARCH\_MODE\_LONG\_CUTOFF
の場合をこの順序で記述する。その他の\\PS::SEARCH\_MODEでは超粒子を必要としない。

\subsubsubsection{PS::SEARCH\_MODE\_LONG}

\subsubsubsubsection{PS::SPJMonopole}
\label{sec:SPJMonopole}

単極子までの情報を持つMomentクラスPS::MomentMonopoleと対になる
SuperParticleJクラス。以下、このクラスの概要を記述する。
\begin{screen}
\begin{verbatim}
namespace ParticleSimulator {
    class SPJMonopole {
    public:
        F64    mass;
        F64vec pos;
    };
}
\end{verbatim}
\end{screen}

\begin{itemize}
\item クラス名
  PS::SPJMonopole

\item メンバ変数とその情報

  mass: 近傍でまとめた粒子の全質量、または全電荷

  pos: 近傍でまとめた粒子の重心、または粒子電荷の重心

\item 使用条件

  MomentクラスであるPS::MomentMonopoleクラスの使用条件に準ずる。

\end{itemize}

\subsubsubsubsection{PS::SPJQuadrupole}
\label{sec:SPJQuadrupole}

単極子と四重極子を情報を持つMomentクラスPS::MomentQuadrupoleと対になる
SuperParticleJクラス。以下、このクラスの概要を記述する。
\begin{screen}
\begin{verbatim}
namespace ParticleSimulator {
    class SPJQuadrupole {
    public:
        F64    mass;
        F64vec pos;
        F64mat quad;
    };
}
\end{verbatim}
\end{screen}

\begin{itemize}
\item クラス名
  PS::SPJQuadrupole

\item メンバ変数とその情報

  mass: 近傍でまとめた粒子の全質量、または全電荷

  pos: 近傍でまとめた粒子の重心、または粒子電荷の重心

  quad: 近傍でまとめた粒子の四重極子

\item 使用条件

  MomentクラスであるPS::MomentQuadrupoleクラスの使用条件に準ずる。

\end{itemize}

\subsubsubsubsection{PS::SPJMonopoleGeometricCenter}
\label{sec:SPJMonopoleGeometricCenter}

単極子までを情報として持つ(ただしモーメント計算の際の座標系の中心は粒
子の幾何中心)MomentクラスPS::MomentMonopoleGeometricCenterと対となる
SuperParticleJクラス。以下、このクラスの概要を記述する。
\begin{screen}
\begin{verbatim}
namespace ParticleSimulator {
    class SPJMonopoleGeometricCenter {
    public:
        F64    charge;    
        F64vec pos;
    };
}
\end{verbatim}
\end{screen}

\begin{itemize}
\item クラス名
  PS::SPJMonopoleGeometricCenter

\item メンバ変数とその情報

  charge: 近傍でまとめた粒子の全質量、または全電荷

  pos: 近傍でまとめた粒子の幾何中心

\item 使用条件

  PS::MomentMonopoleGeometricCenterの使用条件に準ずる。

\end{itemize}

\subsubsubsubsection{PS::SPJDipoleGeometricCenter}
\label{sec:SPJDipoleGeometricCenter}

双極子までを情報として持つ(ただしモーメント計算の際の座標系の中心は粒
子の幾何中心)MomentクラスPS::MomentDipoleGeometricCenterと対となる
SuperParticleJクラス。以下、このクラスの概要を記述する。
\begin{screen}
\begin{verbatim}
namespace ParticleSimulator {
    class SPJDipoleGeometricCenter {
    public:
        F64    charge;    
        F64vec pos;
        F64vec dipole;
    };
}
\end{verbatim}
\end{screen}

\begin{itemize}
\item クラス名
  PS::SPJDipoleGeometricCenter

\item メンバ変数とその情報

  charge: 近傍でまとめた粒子の全質量、または全電荷

  pos: 近傍でまとめた粒子の幾何中心

  dipole: 粒子の質量または電荷の双極子

\item 使用条件

  PS::MomentDipoleGeometricCenterの使用条件に準ずる。

\end{itemize}

\subsubsubsubsection{PS::SPJQuadrupoleGeometricCenter}
\label{sec:SPJQuadrupoleGeometricCenter}

四重極子までを情報として持つ(ただしモーメント計算の際の座標系の中心は
粒子の幾何中心)MomentクラスPS::MomentQuadrupoleGeometricCenterと対とな
るSuperParticleJクラス。以下、このクラスの概要を記述する。
\begin{screen}
\begin{verbatim}
namespace ParticleSimulator {
    class SPJQuadrupoleGeometricCenter {
    public:
        F64    charge;    
        F64vec pos;
        F64vec dipole;
        F64mat quadrupole;
    };
}
\end{verbatim}
\end{screen}

\begin{itemize}
\item クラス名
  PS::SPJQuadrupoleGeometricCenter

\item メンバ変数とその情報

  charge: 近傍でまとめた粒子の全質量、または全電荷

  pos: 近傍でまとめた粒子の幾何中心

  dipole: 粒子の質量または電荷の双極子

  quadrupole: 粒子の質量または電荷の四重極子

\item 使用条件

  PS::MomentQuadrupoleGeometricCenterの使用条件に準ずる。

\end{itemize}
  
\subsubsubsection{PS::SEARCH\_MODE\_LONG\_SCATTER}

\subsubsubsubsection{PS::SPJMonopoleScatter}
\label{sec:SPJMonopoleScatter}

単極子までを情報として持つMomentクラスPS::MomentMonopoleScatterと
対となるSuperParticleJクラス。以下、このクラスの概要を記述する。
\begin{screen}
\begin{verbatim}
namespace ParticleSimulator {
    class SPJMonopoleScatter {
    public:
        F64    mass;
        F64vec pos;
    };
}
\end{verbatim}
\end{screen}

\begin{itemize}
\item クラス名
  PS::SPJMonopoleScatter

\item メンバ変数とその情報

  mass: 近傍でまとめた粒子の全質量、または全電荷

  pos: 近傍でまとめた粒子の重心、または粒子電荷の重心

\item 使用条件

  PS::MomentMonopoleScatterの使用条件に準ずる。

\end{itemize}

\subsubsubsubsection{PS::SPJQuadrupoleScatter}
\label{sec:SPJQuadrupoleScatter}

単極子と四重極子を情報として持つMomentクラスPS::MomentQuadrupoleScatterと
対となるSuperParticleJクラス。以下、このクラスの概要を記述する。
\begin{screen}
\begin{verbatim}
namespace ParticleSimulator {
    class SPJQuadrupolepoleScatter {
    public:
        F64    mass;
        F64vec pos;
        F64mat quad;
    };
}
\end{verbatim}
\end{screen}

\begin{itemize}
\item クラス名
  PS::SPJQuadrupoleScatter

\item メンバ変数とその情報

  mass: 近傍でまとめた粒子の全質量、または全電荷

  pos: 近傍でまとめた粒子の重心、または粒子電荷の重心
        
  quad: 近傍でまとめた粒子の四重極子

\item 使用条件

  PS::MomentQuadrupoleScatterの使用条件に準ずる。

\end{itemize}

\subsubsubsection{PS::SEARCH\_MODE\_LONG\_SYMMETRY}

\subsubsubsubsection{PS::SPJMonopoleSymmetry}
\label{sec:SPJMonopoleSymmetry}

単極子までを情報として持つMomentクラスPS::MomentMonopoleSymmetryと
対となるSuperParticleJクラス。以下、このクラスの概要を記述する。
\begin{screen}
\begin{verbatim}
namespace ParticleSimulator {
    class SPJMonopoleSymmetry {
    public:
        F64    mass;
        F64vec pos;
    };
}
\end{verbatim}
\end{screen}

\begin{itemize}
\item クラス名
  PS::SPJMonopoleSymmetry

\item メンバ変数とその情報

  mass: 近傍でまとめた粒子の全質量、または全電荷

  pos: 近傍でまとめた粒子の重心、または粒子電荷の重心

\item 使用条件

  PS::MomentMonopoleSymmetryの使用条件に準ずる。

\end{itemize}

\subsubsubsubsection{PS::SPJQuadrupoleSymmetry}
\label{sec:SPJQuadrupoleSymmetry}

単極子と四重極子を情報として持つMomentクラスPS::MomentQuadrupoleSymmetryと
対となるSuperParticleJクラス。以下、このクラスの概要を記述する。
\begin{screen}
\begin{verbatim}
namespace ParticleSimulator {
    class SPJQuadrupolepoleSymmetry {
    public:
        F64    mass;
        F64vec pos;
        F64mat quad;
    };
}
\end{verbatim}
\end{screen}

\begin{itemize}
\item クラス名
  PS::SPJQuadrupoleSymmetry

\item メンバ変数とその情報

  mass: 近傍でまとめた粒子の全質量、または全電荷

  pos: 近傍でまとめた粒子の重心、または粒子電荷の重心
        
  quad: 近傍でまとめた粒子の四重極子

\item 使用条件

  PS::MomentQuadrupoleSymmetryの使用条件に準ずる。

\end{itemize}

\subsubsubsection{PS::SEARCH\_MODE\_LONG\_CUTOFF}

\subsubsubsubsection{PS::SPJMonopoleCutoff}
\label{sec:SPJMonopoleCutoff}

単極子までを情報として持つMomentクラスPS::MomentMonopoleCutoffと
対となるSuperParticleJクラス。以下、このクラスの概要を記述する。
\begin{screen}
\begin{verbatim}
namespace ParticleSimulator {
    class SPJMonopoleCutoff {
    public:
        F64    mass;
        F64vec pos;
    };
}
\end{verbatim}
\end{screen}

\begin{itemize}
\item クラス名
  PS::SPJMonopoleCutoff

\item メンバ変数とその情報

  mass: 近傍でまとめた粒子の全質量、または全電荷

  pos: 近傍でまとめた粒子の重心、または粒子電荷の重心

\item 使用条件

  PS::MomentMonopoleCutoffの使用条件に準ずる。

\end{itemize}

\subsubsection{必要なメンバ関数}

\subsubsubsection{概要}

以下ではSuperParticleJクラスを作る際に必要なメンバ関数を記述する。この
ときSuperParticleJクラスのクラス名をSPJとする。これは変更自由である。

\subsubsubsection{SPJ::getPos}

\if 0
\begin{screen}
\begin{verbatim}
class SPJ {
public:
    PS::F64vec pos;
    PS::F64vec getPos() const {
        return this->pos;
    }
};
\end{verbatim}
\end{screen}

\begin{itemize}

\item {\bf 前提}
  
  SPJのメンバ変数posはある1つの超粒子の位置情報。このposのデータ型は
  PS::F32vecまたはPS::F64vec型。
  
\item {\bf 引数}

  なし
  
\item {\bf 返値}

  PS::F32vec型またはPS::F64vec型。SPJクラスの位置情報を保持したメンバ
  変数。
  
\item {\bf 機能}

  SPJクラスの位置情報を保持したメンバ変数を返す。
  
\item {\bf 備考}

  SPJクラスのメンバ変数posの変数名は変更可能。

\end{itemize}
\fi

\begin{screen}
\begin{verbatim}
class SPJ {
public:
    PS::F64vec getPos() const;
};
\end{verbatim}
\end{screen}

\begin{itemize}

\item {\bf 引数}

  なし
  
\item {\bf 返値}

  PS::F32vec型またはPS::F64vec型。SPJクラスの位置情報を保持したメンバ
  変数。
  
\item {\bf 機能}

  SPJクラスの位置情報を保持したメンバ変数を返す。
  
\end{itemize}

\subsubsubsection{SPJ::setPos}

\if 0
\begin{screen}
\begin{verbatim}
class SPJ {
public:
    PS::F64vec pos;
    void setPos(const PS::F64vec pos_new) {
        this->pos = pos_new;
    }
};
\end{verbatim}
\end{screen}

\begin{itemize}

\item {\bf 前提}
  
  SPJクラスのメンバ変数posは1つの粒子の位置情報。このposのデータ型は
  PS::F32vecまたはPS::F64vec。

\item {\bf 引数}

  pos\_new: 入力。const PS::F32vecまたはconst PS::F64vec型。FDPS側で修
  正した粒子の位置情報。

\item {\bf 返値}

  なし。
  
\item {\bf 機能}

  FDPSが修正した粒子の位置情報をSPJクラスの位置情報に書き込む。

\item {\bf 備考}

  SPJクラスのメンバ変数posの変数名は変更可能。メンバ関数SPJ::setPosの
  引数名pos\_newは変更可能。posとpos\_newのデータ型が異なる場合の動作
  は保証しない。

\end{itemize}
\fi

\begin{screen}
\begin{verbatim}
class SPJ {
public:
    void setPos(const PS::F64vec pos_new);
};
\end{verbatim}
\end{screen}

\begin{itemize}

\item {\bf 引数}

  pos\_new: 入力。const PS::F32vecまたはconst PS::F64vec型。FDPS側で修
  正した粒子の位置情報。

\item {\bf 返値}

  なし。
  
\item {\bf 機能}

  FDPSが修正した粒子の位置情報をSPJクラスの位置情報に書き込む。

\end{itemize}


\subsubsubsection{SPJ::copyFromMoment}

\if 0
\begin{screen}
\begin{verbatim}
class Mom {
public:
    PS::F32    mass;
    PS::F32vec pos;
}
class SPJ {
public:
    PS::F32    mass;
    PS::F32vec pos;
    void copyFromMoment(const Mom & mom) {
        mass = mom.mass;
        pos  = mom.pos;
    }
};
\end{verbatim}
\end{screen}

\begin{itemize}

\item {\bf 前提}

  なし
  
\item {\bf 引数}

  mom: 入力。const Mom \&型。Momにはユーザー定義またはFDPS側で用意した
  Momentクラスが入る。

\item {\bf 返値}

  なし。
  
\item {\bf 機能}

  Momクラスの情報をSPJクラスにコピーする。

\item {\bf 備考}

  Momクラスのクラス名は変更可能。MomクラスとSPJクラスのメンバ変数名は
  変更可能。メンバ関数SPJ::copyFromMomentの引数名は変更可能。

\end{itemize}
\fi

\begin{screen}
\begin{verbatim}
class Mom;
class SPJ {
public:
    void copyFromMoment(const Mom & mom);
};
\end{verbatim}
\end{screen}

\begin{itemize}

\item {\bf 引数}

  mom: 入力。const Mom \&型。Momにはユーザー定義またはFDPS側で用意した
  Momentクラスが入る。

\item {\bf 返値}

  なし。
  
\item {\bf 機能}

  Momクラスの情報をSPJクラスにコピーする。

\end{itemize}

\subsubsubsection{SPJ::convertToMoment}

\if 0
\begin{screen}
\begin{verbatim}
class Mom {
public:
    PS::F32    mass;
    PS::F32vec pos;
    Mom(const PS::F32 m,
        const PS::F32vec & p) {
        mass = m;
        pos  = p;
    }
}
class SPJ {
public:
    PS::F32    mass;
    PS::F32vec pos;
    Mom convertToMoment() const {
        return Mom(mass, pos);
    }
};
\end{verbatim}
\end{screen}

\begin{itemize}

\item {\bf 前提}

  なし
  
\item {\bf 引数}

  なし

\item {\bf 返値}

  Mom型。Momクラスのコンストラクタ。
  
\item {\bf 機能}

  Momクラスのコンストラクタを返す。

\item {\bf 備考}

  Momクラスのクラス名は変更可能。MomクラスとSPJクラスのメンバ変数名は
  変更可能。メンバ関数SPJ::copyFromMomentの引数名は変更可能。メンバ関
  数SPJ::convertToMomentで使用されるMomクラスのコンストラクタが定義さ
  れている必要がある。

\end{itemize}
\fi

\begin{screen}
\begin{verbatim}
class Mom;
class SPJ {
public:
    Mom convertToMoment() const;
};
\end{verbatim}
\end{screen}

\begin{itemize}

\item {\bf 引数}

  なし

\item {\bf 返値}

  Mom型。Momクラスのコンストラクタ。
  
\item {\bf 機能}
  
  超粒子をモーメントに変換し、その変換したものをMomクラスのコンストラク
  タを返す。

\end{itemize}

\subsubsubsection{SPJ::clear}

\if 0
\begin{screen}
\begin{verbatim}
class SPJ {
public:
    PS::F32    mass;
    PS::F32vec pos;
    void clear() {
        mass = 0.0;
        pos  = 0.0;
    }
};
\end{verbatim}
\end{screen}

\begin{itemize}

\item {\bf 前提}

  なし
  
\item {\bf 引数}

  なし

\item {\bf 返値}

  なし
  
\item {\bf 機能}

  SPJクラスのオブジェクトの情報をクリアする。

\item {\bf 備考}

  メンバ変数名は変更可能。

\end{itemize}
\fi

\begin{screen}
\begin{verbatim}
class SPJ {
public:
    void clear();
};
\end{verbatim}
\end{screen}

\begin{itemize}

\item {\bf 引数}

  なし

\item {\bf 返値}

  なし
  
\item {\bf 機能}

  SPJクラスのオブジェクトの情報をクリアする。

\end{itemize}


\subsubsection{場合によっては必要なメンバ関数}

\subsubsubsection{概要}

本節では、場合によっては必要なメンバ関数について記述する。

\subsubsubsection{LET交換時に粒子データをシリアライズして送る場合}
\label{sec:SPJ:serialize}

LET交換時に粒子データをシリアライズして送る場合には、メンバ関数に
SPJ::packとSPJ::unpackを用意する必要がある。以下にそれぞれの規定を記述
する。

%%%%%%%%%%%%%%%%%%%%%%%%%%%%%%%%%%%%%%%%%%%%%%%%%%%%%%

\subsubsubsubsection{SPJ::pack}

\begin{screen}
\begin{verbatim}
class SPJ {
public:
    static PS::S32 pack(const PS::S32 n_ptcl, const SPJ *ptcl[], char *buf, 
                        size_t & packed_size, const size_t max_buf_size);
};
\end{verbatim}
\end{screen}

\begin{itemize}

\item {\bf 引数}

  n\_ptcl: LET交換時に送る粒子の数。
  

  ptcl: 送る粒子へのポインタの配列。

  buf: 送信バッファーの先頭アドレス。
  
  packed\_size: ユーザーがバッファーへ書き込むサイズ。単位はバイト。
  
  max\_buf\_size: 送信バッファーの書き込み可能な領域のサイズ。単位はバイト。

\item {\bf 返値}

  PS::S32型。packed\_sizeがmax\_buf\_sizeを超えた場合は-1を返す。それ
  以外の場合は0を返す。
  
\item {\bf 機能}

 LET交換時に送信する粒子をシリアライズし、送信バッファーに書き込む。

\end{itemize}

%%%%%%%%%%%%%%%%%%%%%%%%%%%%%%%%%%%%%%%%%%%%%%%%%%%%%%

\subsubsubsubsection{SPJ::unPack}

\begin{screen}
\begin{verbatim}
class SPJ {
public:
    static void unPack(const PS::S32 n_ptcl, SPJ[], const char *buf);
};
\end{verbatim}
\end{screen}

\begin{itemize}

\item {\bf 引数}

  n\_ptcl: LET交換時に受け取る粒子の数。
  
  ptcl: 受け取る粒子の配列。
  
  buf: 受信バッファーの先頭アドレス。

\item {\bf 返値}

  なし。

\item {\bf 機能}

 LET交換時に受信する粒子をデシリアライズし、粒子配列に書き込む。

\end{itemize}

\subsection{Force}
\label{sec:force}

\subsubsection{Summary}

%Forceクラスは相互作用の結果を保持するクラスであり、相互作用の定義
%(節\ref{sec:overview_action}の手順0)に必要となる。以下、この節の前提、
%常に必要なメンバ関数について記述する。
The \texttt{Force} class contains the results of the calculation of interactions (see, step 0 in Sec. \ref{sec:overview_action}).
In this section we describe the member functions in any case required.

\subsubsection{Premise}

%この節で用いる例としてForceクラスのクラス名をResultとする。このクラス名
%は変更自由である。
Let us take \texttt{Result} class as an example of \texttt{Force} as below.
Users can use an arbitrary name instead of \texttt{Result}.

\subsubsection{Required member functions}

%常に必要なメンバ関数はResult::clearである。この関数は相互作用の計算結果を初期
%化する。以下、Result::clearについて記述する。
The member function \texttt{Result::clear} is in any case needed.
This function initialize the result on interaction calculations.

\subsubsubsection{Result::clear}

\begin{screen}
\begin{verbatim}
class Result {
public:
    void clear();
};
\end{verbatim}
\end{screen}

\begin{itemize}

  
\item {\bf Arguments}

  None.
  
\item {\bf Returns}

  None.
  
\item {\bf Behaviour}

  %Resultクラスのメンバ変数を初期化する。
  Initializes the result of interaction calculations.
  
\end{itemize}


\subsection{Header}
\label{sec:userdefined_header}

\subsubsection{概要}

ヘッダクラスは入出力ファイルのヘッダの形式を決めるクラスである。ヘッダクラスはFDPSが提供する粒子群クラスのファイル入出力APIを使用し、かつ入出力ファイルにヘッダを含ませたい場合に必要となるクラスである。粒子群クラスのファイル入出力APIとは、ParticleSystem::readParticleAscii, ParticleSystem::writeParticleAscii, ParticleSystem::readParticleBinary, ParticleSystem::writeParticleBinaryである。以下、この節における前提と、これらのAPIを使用する際に必要となるメンバ関数とその記述の規定を述べる。この節において、常に必要なメンバ関数というものは存在しない。

\subsubsection{前提}

この節では、ヘッダクラスのクラス名をHdrとする。このクラス名は変更可能である。

\subsubsection{場合によっては必要なメンバ関数}

\subsubsubsection{Hdr::readAscii}
\label{sec:Hdr_readAscii}

\begin{screen}
\begin{verbatim}
class Hdr {
public:
    PS::S32 readAscii(FILE *fp);
};
\end{verbatim}
\end{screen}

\begin{itemize}

\item {\bf 引数}

  fp: 入力。FILE *型。粒子データの入力ファイルを指すファイルポインタ。
  
\item {\bf 返値}

  PS::S32型。粒子数の情報を返す。ヘッダに粒子数の情報がない場合は-1を
  返す。
  
\item {\bf 機能}

  粒子データの入力ファイルからヘッダ情報を読みこむ。
  
\end{itemize}


\subsubsubsection{Hdr::writeAscii}
\label{sec:Hdr_writeAscii}  


\begin{screen}
\begin{verbatim}
class Hdr {
public:
    void writeAscii(FILE *fp);
};
\end{verbatim}
\end{screen}

\begin{itemize}

\item {\bf 引数}

  fp: 入力。FILE *型。粒子データの出力ファイルを指すファイルポインタ。
  
\item {\bf 返値}

  なし。
  
\item {\bf 機能}

  粒子データの出力ファイルへヘッダ情報を書き込む。
  
\end{itemize}
  
\subsubsubsection{Hdr::readBinary}
\label{sec:Hdr_readBinary}

\begin{screen}
\begin{verbatim}
class Hdr {
public:
    PS::S32 readBinary(FILE *fp);
};
\end{verbatim}
\end{screen}

\begin{itemize}

\item {\bf 引数}

  fp: 入力。FILE *型。粒子データの入力ファイルを指すファイルポインタ。
  
\item {\bf 返値}

  PS::S32型。粒子数の情報を返す。ヘッダに粒子数の情報がない場合は-1を
  返す。
  
\item {\bf 機能}

  粒子データの入力ファイルからヘッダ情報を読みこむ。
  
\end{itemize}


\subsubsubsection{Hdr::writeBinary}
\label{sec:Hdr_writeBinary}

\begin{screen}
\begin{verbatim}
class Hdr {
public:
    void writeBinary(FILE *fp);
};
\end{verbatim}
\end{screen}

\begin{itemize}

\item {\bf 引数}

  fp: 入力。FILE *型。粒子データの出力ファイルを指すファイルポインタ。
  
\item {\bf 返値}

  なし。
  
\item {\bf 機能}

  粒子データの出力ファイルへヘッダ情報を書き込む。
  
\end{itemize}


\subsection{Functor calcForceEpEp}
\label{sec:userdefined_calcForceEpEp}

\subsubsection{Summary}

%関数オブジェクトcalcForceEpEpは粒子同士の相互作用を記述するものであり、
%相互作用の定義(節\ref{sec:overview_action}の手順0)に必要となる。以下、
%これの書き方の規定を記述する。
Functor \texttt{calcForceEpEp} defines the interaction between two particles.
This functor is required for the calculation of interactions (see, step 0 in Sec. \ref{sec:overview_action}).

\subsubsection{Premise}

%ここで示すのは重力N体シミュレーションの粒子間相互作用の記述の仕方であ
%る。関数オブジェクトcalcForceEpEpの名前はgravityEpEpとする。これは変更
%自由である。また、EssentialParitlceIクラスのクラス名をEPI,
%EssentialParitlceJクラスのクラス名をEPJ, Forceクラスのクラス名をResult
%とする。
Here one example of gravitational $N$-body problems are shown.
The name of functor \texttt{calcForceEpEp} is \texttt{gravityEpEp}, which is an arbitrary.
The class name of \texttt{EssentialParitlceI}, \texttt{EssentialParitlceJ} and \texttt{Force} are \texttt{EPI}, \texttt{EPJ} and \texttt{Result}.

\subsubsection{gravityEpEp::operator ()}

\begin{lstlisting}[caption=calcForceEpEp]
class Result;
class EPI;
class EPJ;
struct gravityEpEp {
    static PS::F32 eps2;
    void operator () (const EPI *epi,
                      const PS::S32 ni,
                      const EPJ *epj,
                      const PS::S32 nj,
                      Result *result)
    }
};
PS::F32 gravityEpEp::eps2 = 9.765625e-4;
\end{lstlisting}

\begin{itemize}

\item {\bf Arguments}

  %epi: 入力。const EPI *型またはEPI *型。i粒子情報を持つ配列。
  \texttt{epi}: Input. \texttt{const EPI *} type or \texttt{EPI *} type. Array of $i$ particles.

  %ni: 入力。const PS::S32型またはPS::S32型。i粒子数。
  \texttt{ni}: Input. \texttt{const PS::S32} type or \texttt{PS::S32}. The number of $i$ particles.

  %epj: 入力。const EPJ *型またはEPJ *型。j粒子情報を持つ配列。
  \texttt{epj}: Input. \texttt{const EPJ *} type or \texttt{EPJ *} type. Array of $j$ particles.

  %nj: 入力。const PS::S32型またはPS::S32型。j粒子数。
  \texttt{nj}: Input. \texttt{const PS::S32} type or \texttt{PS::S32} type. Number of $j$ particles.

  %result: 出力。Result *型。i粒子の相互作用結果を返す配列。
  \texttt{result}: Output. \texttt{Result *} type. Array of the results of interaction.

\item {\bf Returns}

  None.
  
\item {\bf Behaviour}

  %j粒子からi粒子への作用を計算する。
  Calculates the interaction to $i-$ particle from $j-$ particle.
  
\end{itemize}

\subsection{Functor calcForceSpEp}
\label{sec:userdefined_calcForceSpEp}

\subsubsection{概要}

関数オブジェクトcalcForceSpEpは超粒子から粒子への作用を記述するもので
あり、相互作用の定義(節\ref{sec:overview_action}の手順0)に必要となる。
以下、これの書き方の規定を記述する。

\subsubsection{前提}

ここで示すのは重力N体シミュレーションにおける超粒子から粒子への作用の
記述の仕方である。超粒子は単極子までの情報で作られているものとする。関
数オブジェクトcalcForceSpEpの名前はgravitySpEpとする。これは変更自由で
ある。また、EssentialParitlceIクラスのクラス名をEPI, SuperParitlceJク
ラスのクラス名をSPJ, Forceクラスのクラス名をResultとする。

\subsubsection{gravitySpEp::operator ()}

\if 0
\begin{lstlisting}[caption=calcForceSpEp]
class Result {
public:
    PS::F32vec accfromspj;
};
class EPI {
public:
    PS::S32    id;
    PS::F32vec pos;
};
class SPJ {
public:
    PS::F32    mass;
    PS::F32vec pos;
};
struct gravitySpEp {
    static PS::F32 eps2;
    void operator () (const EPI *epi,
                      const PS::S32 ni,
                      const SPJ *spj,
                      const PS::S32 nj,
                      Result *result) {
                      
        for(PS::S32 i = 0; i < ni; i++) {
            PS::F32vec xi = epi[i].pos;
            PS::F32vec ai = 0.0;
            for(PS::S32 j = 0; j < nj; j++) {
                PS::F32    mj = spj[j].mass;
                PS::F32vec xj = spj[j].pos;

                PS::F32vec dx   = xi - xj;
                PS::F32    r2   = dx * dx + eps2;
                PS::F32    rinv = 1. / sqrt(r2);

                ai += mj * rinv * rinv * rinv * dx;
            }
            result.accfromspj = ai;
        }
    }
};
PS::F32 gravitySpEp::eps2 = 9.765625e-4;
\end{lstlisting}

\begin{itemize}

\item {\bf 前提}

  クラスResult, EPI, SPJに必要なメンバ関数は省略した。クラスResultのメ
  ンバ変数accfromspjはi粒子が超粒子から受ける重力加速度である。クラス
  EPIとSPJのメンバ変数posはそれぞれの粒子位置である。クラスSPJのメンバ
  変数massは超粒子の質量である。ファンクタgravitySpEpのメンバ変数eps2
  は重力ソフトニングの2乗である。ここの外側でスレッド並列になっている
  ため、ここでOpenMPを記述する必要はない。

\item {\bf 引数}

  epi: 入力。const EPI *型またはEPI *型。i粒子情報を持つ配列。

  ni: 入力。const PS::S32型またはPS::S32型。i粒子数。

  spj: 入力。const SPJ *型またはSPJ *型。超粒子情報を持つ配列。
  
  nj: 入力。const PS::S32型またはPS::S32型。超粒子数。

  result: 出力。Result *型。i粒子の相互作用結果を返す配列。

\item {\bf 返値}

  なし。
  
\item {\bf 機能}

  超粒子からi粒子への作用を計算する。
  
\item {\bf 備考}

  引数名すべて変更可能。関数オブジェクトの内容などはすべて変更可能。
  
\end{itemize}
\fi

\begin{lstlisting}[caption=calcForceSpEp]
class Result;
class EPI;
class SPJ;
struct gravitySpEp {
    void operator () (const EPI *epi,
                      const PS::S32 ni,
                      const SPJ *spj,
                      const PS::S32 nj,
                      Result *result);
};
\end{lstlisting}

\begin{itemize}

\item {\bf 引数}

  epi: 入力。const EPI *型またはEPI *型。i粒子情報を持つ配列。

  ni: 入力。const PS::S32型またはPS::S32型。i粒子数。

  spj: 入力。const SPJ *型またはSPJ *型。超粒子情報を持つ配列。
  
  nj: 入力。const PS::S32型またはPS::S32型。超粒子数。

  result: 出力。Result *型。i粒子の相互作用結果を返す配列。

\item {\bf 返値}

  なし。
  
\item {\bf 機能}

  超粒子からi粒子への作用を計算する。
  
\end{itemize}

\subsection{Functor calcForceDispatch}
\label{sec:userdefined_calcForceDispatch}


\subsubsection{概要}

関数calcForceDispatchは関数calcForceRetrieveと合わせて粒子同士の相互作
用を記述するものであり、calcForceSpEp やcalcForceEpEp の代わりに相互作
用の定義(節\ref{sec:overview_action}の手順0)に使うことができる。
calcForceSpEp やcalcForceEpEp との違いは、calcForceDispatch は複数の相
互作用リストと i粒子リストを受け取ることである。これにより、GPGPU 等の
アクセラレータを起動する回数を削減し、実行効率を向上させる。以下、これ
の書き方の規定を記述する。関数calcForceDispatchの名前はGravityDispatch
とする。これは変更自由である。また、EssentialParitlceIクラスのクラス名
をEPI, EssentialParitlceJクラスのクラス名をEPJ, SuperParitlceJクラスの
クラス名をSPJとする。

\subsubsection{短距離力の場合}

\begin{lstlisting}[caption=calcForceDispatch]
class EPI;
class EPJ;
PS::S32 HydroforceDispatch(const PS::S32  tag,
                           const PS::S32  nwalk,
                           const EPI**      epi,
                           const PS::S32*  ni,
                           const EPJ**      epj,
                           const PS::S32*  nj_ep;
};
\end{lstlisting}

\begin{itemize}

\item {\bf 引数}

  tag: 入力。const PS::S32 型。tagの番号。発行されるtagの番号は\\
  0から関数PS::TreeForForce::calcForceAllandWriteBackMultiWalk()の\\
  第三引数として設定された値から1引いた数までである。\\
  tagの番号はcalcForceRetrieve()で設定するtagの番号と対応させる必要がある。
    

  nwalk: 入力。const PS::S32 型。walkの数。walkの数の最大値は\\
  PS::TreeForForce::calcForceAllandWriteBackMultiWalk()の第六引数の値である。

  epi: 入力。const EPI** 型。i粒子情報を持つポインタのポインタ。

  ni: 入力。const PS::S32*型。i粒子数のポインタ。

  epj: 入力。const EPJ** 型。j粒子情報を持つポインタのポインタ。
  
  nj\_ep: 入力。const PS::S32* 型。j粒子数のポインタ。

\item {\bf 返値}

  PS::S32型。ユーザーは正常に実行された場合は0を、エラーが起こった場合
  は0以外の値を返すようにする。
  
\item {\bf 機能}

epi,epjの情報をアクセラレータに送り、相互作用カーネルを発行する。
  
\end{itemize}

\subsubsection{長距離力の場合}

\begin{lstlisting}[caption=calcForceDispatch]
class EPI;
class EPJ;
class SPJ;
PS::S32 GravityDispatch(const PS::S32   tag,
                        const PS::S32   nwalk,
                        const EPI**     epi,
                        const PS::S32*  ni,
                        const EPJ**     epj,
                        const PS::S32*  nj_ep,
                        const SPJ**     spj,
                        const PS::S32*  nj_sp);
};
\end{lstlisting}

\begin{itemize}

\item {\bf 引数}


  tag: 入力。const PS::S32 型。tagの番号。発行されるtagの番号は0から関
  数PS::TreeForForce::calcForceAllandWriteBackMultiWalk()の第三引数と
  して設定された値から1引いた数までである。tagの番号は
  calcForceRetrieve()で設定するtagの番号と対応させる必要がある。

  nwalk: 入力。const PS::S32 型。walkの数。walkの数の最大値は
  PS::TreeForForce::calcForceAllandWriteBackMultiWalk()の第六引数の値
  である。

  epi: 入力。const EPI** 型。i粒子情報を持つ配列の配列。

  ni: 入力。const PS::S32*型。i粒子数の配列。

  epj: 入力。const EPJ** 型。j粒子情報を持つ配列の配列。
  
  nj\_ep: 入力。const PS::S32* 型。j粒子数の配列。

  spj: 入力。const SPJ** 型。j粒子情報を持つ配列の配列。
  
  nj\_sp: 入力。const PS::S32* 型。j粒子数の配列。

\item {\bf 返値}

  PS::S32型。ユーザーは正常に実行された場合は0を、エラーが起こった場合
  は0以外の値を返すようにする。
  
\item {\bf 機能}

epi,epj,spjの情報をアクセラレータに送り、相互作用カーネルを発行する。
  
\end{itemize}


\subsection{Functor calcForceRetrieve}
\label{sec:userdefined_calcForceRetrieve}

%\subsubsection{概要}
\subsubsection{Summary}

%% 関数calcForceRetrieveは関数calcForceDispatchで行った相互作用の結果を回
%% 収する関数である。以下、これの書き方の規定を記述する。関数
%% calcForceRetrieveの名前はGravityRetrieveとする。これは変更自由である。
%% また、Forceクラスのクラス名をResultとする。

Functor \texttt{calcForceRetrieve} retrieves the result of interaction
calculation dispatched by functor \texttt{calcForceDispatch}.

The name of functor \texttt{calcForceRetrieve}
is \texttt{GravityRetrieve} and the name of \texttt{Force} class
is \texttt{Result}


\begin{lstlisting}[caption=calcForceDispatch]
class EPI;
class EPJ;
class Result;
PS::S32 GravityRetrieve(const PS::S32  tag,
                        const PS::S32  nwalk,
                        const PS::S32  ni     [],
                        Result         result [][]);
};
\end{lstlisting}

\begin{itemize}

%% \item {\bf 引数}

  %% tag: 入力。const PS::S32 型。tagの番号。対応する calcForceDispatch の
  %% tag 番号と一致させる必要がある。

  %% nwalk: 入力。const PS::S32 型。walkの数。対応する calcForceDispatch
  %% に与えた nwalk の値と一致させる必要がある。

  %% ni: 入力。const PS::S32*型。i粒子数の配列。
  
  %% result: 出力。Result *型。i粒子の相互作用結果を返す配列の配列。

%% \item {\bf 返値}

%%   PS::S32型。ユーザーは正常に実行された場合は0を、エラーが起こった場合
%%   は0以外の値を返すようにする。
  

%% \item {\bf 機能}

%% 同じtag番号を持つ関数calcForceDispatchで行った相互作用の結果を回収する。


\item {\bf Arguments}

  {\tt tag}: input. Type {\tt const PS::S32}. Should be the same as
    that for the corresponding call to {\tt  CalcForceDispatch()}.
  
  {\tt nwalk}: input. Type {\tt const PS::S32}. The number of
  interaction lists. Should be the same as
    that for the corresponding call to \texttt{CalcForceDispatch()}.
  

  {\tt ni:} input. Type {\tt const PS::S32*} or {\tt PS::S32*}. Array
  of the numbers of i-particles.

  {\tt result:} output. Type {\tt Result**}. 
  
\item {\bf return value}
   Returns 0 upon normal completion. Otherwize non-zero values are returned.
  
\item {\bf Function}

  Store the results calculated by the call
    to \texttt{CalcForceDispatch()} with the same tag value to the
    array {\tt force}.


\end{itemize}


