\subsubsection{Summary}

In this section, we describe data type \texttt{PS::SEARCH\_MODE}.
It is used only as template arguments of class \texttt{PS::TreeForForce}.
This data type determines the interaction mode of the class.
\texttt{PS::SEARCH\_MODE} can take the following values:
\begin{itemize}[leftmargin=*,itemsep=-1ex]
\item PS::SEARCH\_MODE\_LONG
\item PS::SEARCH\_MODE\_LONG\_CUTOFF
\item PS::SEARCH\_MODE\_GATHER
\item PS::SEARCH\_MODE\_SCATTER
\item PS::SEARCH\_MODE\_SYMMETRY
\item PS::SEARCH\_MODE\_LONG\_SCATTER
\item PS::SEARCH\_MODE\_LONG\_SYMMETRY
\item PS::SEARCH\_MODE\_LONG\_CUTOFF\_SCATTER
\end{itemize}

Each of them corresponds to a mode for interaction calculation.
In the following, we describe them.

\subsubsection{PS::SEARCH\_MODE\_LONG}

This type is used when a group of distant particles is regarded as a
superparticle as in the standard Barnes-Hut tree code.
This type is for gravitational force and Coulomb's force under open boundary condition (Not usable in periodic boundary condition).

\subsubsection{PS::SEARCH\_MODE\_LONG\_CUTOFF}

This type is used when a group of distant particles is regarded as a
superparticle, and when its force does not reach to infinity.  This
type is for gravitational force and Coulomb's force under
periodic boundary condition.

\subsubsection{PS::SEARCH\_MODE\_GATHER}

This type is used when its force decays to zero at a finite distance,
and when the distance is determined by the size of $i$-particle.

\subsubsection{PS::SEARCH\_MODE\_SCATTER}

This type is used when its force decays to zero at a finite distance,
and when the distance is determined by the size of $j$-particle.

\subsubsection{PS::SEARCH\_MODE\_SYMMETRY}

This type is used when its force decays to zero at a finite distance,
and when the distance is determined by the larger of the sizes of $i$-
and $j$-particles.

\subsubsection{PS::SEARCH\_MODE\_LONG\_SCATTER}

Almost the same as {\tt SEARCH\_MODE\_LONG}, but if the distance
between $i$- and $j$-particles is smaller than the search radius of
$j$-particle, the $j$-particle is not included in superparticle.


\subsubsection{PS::SEARCH\_MODE\_LONG\_SYMMETRY}

Almost the same as {\tt SEARCH\_MODE\_LONG}, but if the distance
between $i$- and $j$-particles is smaller than the larger of the search radii of
$i$- and $j$-particle, the $j$-particle is not included in superparticle.


\subsubsection{PS::SEARCH\_MODE\_LONG\_CUTOFF\_SCATTER}

Not implemented yet.

%未実装。
