%\subsubsection{概要}
\subsubsection{Summary}

%% 関数calcForceRetrieveは関数calcForceDispatchで行った相互作用の結果を回
%% 収する関数である。以下、これの書き方の規定を記述する。関数
%% calcForceRetrieveの名前はGravityRetrieveとする。これは変更自由である。
%% また、Forceクラスのクラス名をResultとする。

Functor \texttt{calcForceRetrieve} retrieves the result of interaction
calculation dispatched by functor \texttt{calcForceDispatch}.

The name of functor \texttt{calcForceRetrieve}
is \texttt{GravityRetrieve} and the name of \texttt{Force} class
is \texttt{Result}


\begin{lstlisting}[caption=calcForceDispatch]
class EPI;
class EPJ;
class Result;
PS::S32 GravityRetrieve(const PS::S32  tag,
                        const PS::S32  nwalk,
                        const PS::S32  ni     [],
                        Result         result [][]);
};
\end{lstlisting}

\begin{itemize}

%% \item {\bf 引数}

  %% tag: 入力。const PS::S32 型。tagの番号。対応する calcForceDispatch の
  %% tag 番号と一致させる必要がある。

  %% nwalk: 入力。const PS::S32 型。walkの数。対応する calcForceDispatch
  %% に与えた nwalk の値と一致させる必要がある。

  %% ni: 入力。const PS::S32*型。i粒子数の配列。
  
  %% result: 出力。Result *型。i粒子の相互作用結果を返す配列の配列。

%% \item {\bf 返値}

%%   PS::S32型。ユーザーは正常に実行された場合は0を、エラーが起こった場合
%%   は0以外の値を返すようにする。
  

%% \item {\bf 機能}

%% 同じtag番号を持つ関数calcForceDispatchで行った相互作用の結果を回収する。


\item {\bf Arguments}

  {\tt tag}: input. Type {\tt const PS::S32}. Should be the same as
    that for the corresponding call to {\tt  CalcForceDispatch()}.
  
  {\tt nwalk}: input. Type {\tt const PS::S32}. The number of
  interaction lists. Should be the same as
    that for the corresponding call to \texttt{CalcForceDispatch()}.
  

  {\tt ni:} input. Type {\tt const PS::S32*} or {\tt PS::S32*}. Array
  of the numbers of i-particles.

  {\tt result:} output. Type {\tt Result**}. 
  
\item {\bf return value}
   Returns 0 upon normal completion. Otherwize non-zero values are returned.
  
\item {\bf Function}

  Store the results calculated by the call
    to \texttt{CalcForceDispatch()} with the same tag value to the
    array {\tt force}.


\end{itemize}

