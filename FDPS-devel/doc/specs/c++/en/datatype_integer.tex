%\subsubsection{概要}
\subsubsection{Overview}

The integer types are PS::S32, PS::S64, PS::U32, and PS::U64. We
describe these data types in this section.

%整数型にはPS::S32, PS::S64, PS::U32, PS::U64がある。以下、順にこれらを
%記述する。

\subsubsection{PS::S32}

PS::S32, which is 32-bit signed integer, is defined as follows.
%PS::S32は以下のように定義されている。すなわち32bitの符号付き整数である。
\begin{lstlisting}[caption=S32]
namespace ParticleSimulator {
    typedef int S32;
}
\end{lstlisting}

%Safe performance is ensured, only when users adopt GCC and K compilers.
%ただし、GCCコンパイラとKコンパイラでのみ32bitであることが保証されている。

\subsubsection{PS::S64}

PS::S64, which is 64-bit signed integer, is defined as follows.
%PS::S64は以下のように定義されている。すなわち64bitの符号付き整数である。
\begin{lstlisting}[caption=S64]
namespace ParticleSimulator {
    typedef long long int S64;
}
\end{lstlisting}

%Safe performance is ensured, only when users adopt GCC and K compilers.
%ただし、GCCコンパイラとKコンパイラでのみ64bitであることが保証されている。

\subsubsection{PS::U32}

PS::U32, which is 32-bit unsigned integer, is defined as follows.
%PS::U32は以下のように定義されている。すなわち32bitの符号なし整数である。
\begin{lstlisting}[caption=U32]
namespace ParticleSimulator {
    typedef unsigned int U32;
}
\end{lstlisting}

%Safe performance is ensured, only when users adopt GCC and K compilers.
%ただし、GCCコンパイラとKコンパイラでのみ32bitであることが保証されている。

\subsubsection{PS::U64}

PS::U64, which is 64-bit unsigned integer, is defined as follows.
%PS::U64は以下のように定義されている。すなわち64bitの符号なし整数である。
\begin{lstlisting}[caption=U64]
namespace ParticleSimulator {
    typedef unsigned long long int U64;
}
\end{lstlisting}

%Safe performance is ensured, only when users adopt GCC and K compilers.
%ただし、GCCコンパイラとKコンパイラでのみ64bitであることが保証されている。

