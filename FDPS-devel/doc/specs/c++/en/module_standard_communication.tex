This section describes {\tt PS::Comm} class which is a module used to
communicate data among processes and keep the data for
communication. Since the singleton pattern is applied in this class,
users do not need to create an object.

%本節では、通信用データクラスについて記述する。このクラスはノード間通信
%のための情報の保持や実際の通信を行うモジュールである。このクラスはシン
%グルトンパターンとして管理されており、オブジェクトの生成は必要としない。
%ここではこのモジュールのAPIを記述する。

%%%%%%%%%%%%%%%%%%%%%%%%%%%%%%%%%%%%%%%%%%%%%%%%%%%%%%%%%%%%%%%%%
\subsubsubsection{API}

%このモジュールのAPIの宣言は以下のようになっている。このあと各APIについ
%て記述する。

The APIs are declared below.

\begin{lstlisting}[caption=Communication]
namespace ParticleSimulator {
    class Comm{
    public:
        static S32 getRank();
        static S32 getNumberOfProc();
        static S32 getRankMultiDim(const S32 id);
        static S32 getNumberOfProcMultiDim(const S32 id);
        static bool synchronizeConditionalBranchAND
                    (const bool local);
        static bool synchronizeConditionalBranchOR
                    (const bool local);
        template<class T>
        static T getMinValue(const T val);
        template<class Tfloat, class Tint>
        static void getMinValue(const Tfloat f_in,
                                const Tint i_in,
                                Tfloat & f_out,
                                Tint & i_out);
        template<class T>
        static T getMaxValue(const T val);
        template<class Tfloat, class Tint>
        static void getMaxValue(const Tfloat f_in,
                                const Tint i_in,
                                Tfloat & f_out,
                                Tint & i_out );
        template<class T>
        static T getSum(const T val);
        template<class T>
        static void broadcast(T * val,
                              const S32 n,
                              const S32 src=0);
    };
}
\end{lstlisting}

\subsubsubsubsection{PS::Comm::getRank}

\begin{screen}
\begin{verbatim}
static PS::S32 PS::Comm::getRank();
\end{verbatim}
\end{screen}

\begin{itemize}

\item{\bf arguments}

void.

\item{\bf returned value}

Type {\tt PS::S32}.

\item{\bf function}

Return the rank of the calling process.

\end{itemize}

\subsubsubsubsection{PS::Comm::getNumberOfProc}

\begin{screen}
\begin{verbatim}
static PS::S32 PS::Comm::getNumberOfProc();
\end{verbatim}
\end{screen}

\begin{itemize}

\item{\bf arguments}

void.

\item{\bf returned value}

Type {\tt PS::S32}. The total number of processes.

\item{\bf function}

Return the total number of processes.

\end{itemize}

\subsubsubsubsection{PS::Comm::getRankMultiDim}

\begin{screen}
\begin{verbatim}
static PS::S32 PS::Comm::getRankMultiDim(const PS::S32 id);
\end{verbatim}
\end{screen}

\begin{itemize}

\item{\bf arguments}

{\tt id}: input. Type {\tt const PS::S32}. Id of axes. x-axis:0,
y-axis:1, z-axis:2.

\item{\bf returned value}

Type {\tt PS::S32}. The rank of the calling process along {\tt id}-th
axis. In the case of two dimensional simulations, FDPS returns 0 for
{\tt id}=2.

%PS::S32型。id番目の軸でのランクを返す。2次元の場合、id=2は1を返す。

\end{itemize}

\subsubsubsubsection{PS::Comm::getNumberOfProcMultiDim}

\begin{screen}
\begin{verbatim}
static PS::S32 PS::Comm::getNumberOfProcMultiDim(const PS::S32 id);
\end{verbatim}
\end{screen}

\begin{itemize}

\item{\bf }

{\tt id}: input. Type const {\tt PS::S32}. id of axes. x-axis:0,
y-axis:1, z-axis:2.

%id: 入力。const PS::S32型。軸の番号。x軸:0, y軸:1, z軸:2。

\item{\bf returned value}

Type {\tt PS::S32}. The number of processes along {\tt id}-th axis.
In the case of two dimensional simulations, FDPS returns 1 for {\tt
id}=2.

%PS::S32型。id番目の軸のプロセス数を返す。2次元の場合、id=2は1を返す。

\end{itemize}

\subsubsubsubsection{PS::Comm::synchronizeConditionalBranchAND}

\begin{screen}
\begin{verbatim}
static bool PS::Comm::synchronizeConditionalBranchAND(const bool local)
\end{verbatim}
\end{screen}

\begin{itemize}

\item{\bf arguments}

{\tt local}: input. Type {\tt const bool}.

%local: 入力。const bool型。

\item{\bf returned value}

{\tt bool} type. Logical product of {\tt local} over all processes.
%bool型。全プロセスでlocalの論理積を取り、結果を返す。

\item{\bf function}

Return logical product of {\tt local} over all processes.

\end{itemize}

\subsubsubsubsection{PS::Comm::synchronizeConditionalBranchOR}

\begin{screen}
\begin{verbatim}
static bool PS::Comm::synchronizeConditionalBranchOR(const bool local);
\end{verbatim}
\end{screen}

\begin{itemize}

\item{\bf arguments}

{\tt local}: input. Type {\tt const bool}.
%local: 入力。const bool型。

\item{\bf returned value}

Type {\tt bool}.

\item{\bf function}

Return logical sum of {\tt local} over all processes.
%bool型。全プロセスでlocalの論理和を取り、結果を返す。

\end{itemize}

\subsubsubsubsection{PS::Comm::getMinValue}

\begin{screen}
\begin{verbatim}
template <class T>
static T PS::Comm::getMinValue(const T val);
\end{verbatim}
\end{screen}

\begin{itemize}

\item{\bf arguments}

{\tt val}: input. Type {\tt const T}. Type {\tt T} is allowed to be
{\tt PS::F32}, {\tt PS::F64}, {\tt PS::S32}, {\tt PS::S64}, {\tt
PS::U32} and {\tt PS::U64}

%val: 入力。const T型。

\item{\bf returned value}

Type {\tt T}. The minimum value of val of all processes.

%T型。全プロセスでvalの最小値を取り、結果を返す。

\end{itemize}

\begin{screen}
\begin{verbatim}
template <class Tfloat, class Tint>
static void PS::Comm::getMinValue(const Tfloat f_in,
                                  const Tint i_in,
                                  Tfloat & f_out,
                                  Tint & i_out);
\end{verbatim}
\end{screen}

\begin{itemize}

\item{\bf arguments}

{\tt f\_in}: input. Type {\tt const Tfloat}.

{\tt i\_in}: input. Type {\tt const Tint}.

{\tt f\_out}: output. Type {\tt Tfloat \&}. {\tt Tfloat} must be {\tt
PS::F64} or {\tt PS::F32}. The minimum value of {\tt f\_in} among all
processes.

{\tt i\_out}: output. Type {\tt Tint \&}. {\tt Tint} must be {\tt
PS::S64}, {\tt PS::S32}, {\tt PS::U64} or {\tt PS::U32}. The rank of
the process of which {\tt f\_in} is the minimum.


%f\_in: 入力。const Tfloat型。

%i\_in: 入力。const Tint型。

%f\_out: 出力。Tfloat型。全プロセスでf\_inの最小値を取
%り、結果を返す。

%i\_out: 出力。Tint型。f\_outに伴うIDを返す。

\item{\bf returned value}

void.

\end{itemize}

\subsubsubsubsection{PS::Comm::getMaxValue}

\begin{screen}
\begin{verbatim}
template <class T>
static T PS::Comm::getMaxValue(const T val);
\end{verbatim}
\end{screen}

\begin{itemize}

\item{\bf arguments}

{\tt val}: input. Type {\tt const T}.
%val: 入力。const T型。

\item{\bf returned value}

Type {\tt T}. The maximum value of {\tt val} of all processes.
%T型。全プロセスでvalの最大値を取り、結果を返す。

\end{itemize}

\begin{screen}
\begin{verbatim}
template <class Tfloat, class Tint>
static void PS::Comm::getMaxValue(const Tfloat f_in,
                                  const Tint i_in,
                                  Tfloat & f_out,
                                  Tint & i_out);
\end{verbatim}
\end{screen}

\begin{itemize}

\item{\bf arguments}

{\tt f\_in}: input. Type {\tt const Tfloat}. Type {\tt Tfloat} is {\tt
PS::F32} or {\tt PS::F64}.

{\tt i\_in}: input. Type {\tt const Tint}. Type {\tt Tint} is {\tt
PS::S32}.

{\tt f\_out}: output. Type {\tt Tfloat \&}. The maximum value of {\tt
f\_in} among all processes.

{\tt i\_out}: output. Type {\tt Tint \&}. The rank of the process of
which {\tt f\_in} is the maximum.

%{i\_out}: 出力。{Tint}型。{f\_out}に伴うIDを返す。

%f\_in: 入力。const Tfloat型。

%i\_in: 入力。{const Tint}型。

%{f\_out}: 出力。{Tfloat}型。全プロセスで{f\_in}の最大値を取
%り、結果を返す。

%{i\_out}: 出力。{Tint}型。{f\_out}に伴うIDを返す。

\item{\bf returned value}

void.

\end{itemize}

\subsubsubsubsection{PS::Comm::getSum}

\begin{screen}
\begin{verbatim}
template <class T>
static T PS::Comm::getSum(const T val);
\end{verbatim}
\end{screen}

\begin{itemize}

\item{\bf arguments}

{\tt val}: input. Type {\tt const T}. Type T is allowed to be {\tt
     PS::S32}, {\tt PS::S64}, {\tt PS::F32}, {\tt PS::F64}, {\tt
     PS::U32}, {\tt PS::U64}, {\tt PS::F32vec} and {\tt F64vec}.
     
%{val}: 入力。{const T}型。

\item{\bf returned value}

Type {\tt T}. Return the global sum of {\tt val}.

%{T}型。全プロセスで{val}の総和を取り、結果を返す。

\end{itemize}

\subsubsubsubsection{PS::Comm::broadcast}

\begin{screen}
\begin{verbatim}
template <class T>
static void PS::Comm::broadcast(T * val,
                                const PS::S32 n,
                                const PS::S32 src=0);
\end{verbatim}
\end{screen}

\begin{itemize}

\item{\bf arguments}

{\tt val}: input. Type {\tt T *}. Type {\tt T} is allowed to be any
types.

{\tt n}: input. Type {\tt const PS::S32}. The number of variables.

{\tt src}: input. Type {\tt const PS::S32}. The rank of source
process.

%val: 入力。T *型。

%n: 入力。const PS::F32型。T型変数の数。

%src: 入力。const PS::F32型。放送するプロセスランク。デフォルトのランク
%は0。

\item{\bf returned value}

void.

\item{\bf function}

Broadcast {\tt val} for the {\tt src}-th process.

%プロセスランクsrcのプロセスがn個のT型変数を全プロセスに放送する。

\end{itemize}
