\subsubsection{Summary}

%関数オブジェクトcalcForceSpEpは超粒子から粒子への作用を記述するもので
%あり、相互作用の定義(節\ref{sec:overview_action}の手順0)に必要となる。
%以下、これの書き方の規定を記述する。
Functor \texttt{calcForceSpEp} defines the interaction to a particle from super particle.
This functor is required for the calculation of interactions (see, step 0 in Sec. \ref{sec:overview_action}).

\subsubsection{Premise}

%ここで示すのは重力N体シミュレーションにおける超粒子から粒子への作用の
%記述の仕方である。超粒子は単極子までの情報で作られているものとする。関
%数オブジェクトcalcForceSpEpの名前はgravitySpEpとする。これは変更自由で
%ある。また、EssentialParitlceIクラスのクラス名をEPI, SuperParitlceJク
%ラスのクラス名をSPJ, Forceクラスのクラス名をResultとする。

Here one example of gravitational $N$-body problems are shown.
The superparticles are regarded to be constructed from up to monopole.
The name of functor \texttt{calcForceSpEp} is \texttt{gravitySpEp}, which is an arbitrary.
The class name of \texttt{EssentialParitlceI}, \texttt{SuperParitlceJ} and \texttt{Force} are \texttt{EPI}, \texttt{SPJ} and \texttt{Result}.

\subsubsection{gravitySpEp::operator ()}

\begin{lstlisting}[caption=calcForceSpEp]
class Result;
class EPI;
class SPJ;
struct gravitySpEp {
    static PS::F32 eps2;
    void operator () (const EPI *epi,
                      const PS::S32 ni,
                      const SPJ *spj,
                      const PS::S32 nj,
                      Result *result);
};
\end{lstlisting}

\begin{itemize}

\item {\bf Arguments}

  %epi: 入力。const EPI *型またはEPI *型。i粒子情報を持つ配列。
  \texttt{epi}: Input. \texttt{const EPI *} type or \texttt{EPI *} type. Array of $i$ particles.

  %ni: 入力。const PS::S32型またはPS::S32型。i粒子数。
  \texttt{ni}: Input. \texttt{const PS::S32} type or \texttt{PS::S32}. The number of $i$ particles.

  %spj: 入力。const SPJ *型またはSPJ *型。超粒子情報を持つ配列。
  \texttt{spj}: Input. \texttt{const SPJ *} type or \texttt{SPJ *} type. Array of super particle.

  %nj: 入力。const PS::S32型またはPS::S32型。超粒子数。
  \texttt{nj}: Input. \texttt{const PS::S32} type or \texttt{PS::S32} type. Number of super particles.

  %result: 出力。Result *型。i粒子の相互作用結果を返す配列。
  \texttt{result}: Output. \texttt{Result *} type. Array of the results of interaction.

\item {\bf Returns}

  None.
  
\item {\bf Behaviour}

  %超粒子からi粒子への作用を計算する。
  Calculate interactions from super particle to $i$-th particle.

\end{itemize}
