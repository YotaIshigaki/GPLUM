\subsubsection{Abstract}

The floating point types are PS::F32 and PS::F64. We described
these data types in this section.
%実数型にはPS::F32, PS::F64がある。以下、順にこれらを記述する。

\subsubsection{PS::F32}

PS::F32, which is 32-bit floating point number, is defined as follows.
%PS::F32は以下のように定義されている。すなわち32bitの浮動小数点数である。
\begin{lstlisting}[caption=F32]
namespace ParticleSimulator {
    typedef float F32;
}
\end{lstlisting}

\subsubsection{PS::F64}

PS::F64, which is 64-bit floating point number, is defined as follows.
%PS::F64は以下のように定義されている。すなわち64bitの浮動小数点数である。
\begin{lstlisting}[caption=F64]
namespace ParticleSimulator {
    typedef double F64;
}
\end{lstlisting}
