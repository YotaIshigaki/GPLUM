\subsection{Summary}

We describe the directory strutcures and files of FDPS file
distribution.
%file configuration of FDPS: documents, source files, test
%codes, and sample codes.
%%ここではFDPSのファイル構成について記述する。ドキュメント、ソースファ
%%イル、テストコード、サンプルコードの順に記述する。

\subsection{Documents}

Document files are in directory \texttt{doc}. Files
\texttt{doc\_tutorial\_e.pdf} and \texttt{doc\_specs\_en.pdf} are
tutorial and specification, respectively.
%ドキュメント関係のファイルはディレクトリdocの下にある。チュートリアル
%がdoc\_tutorial.pdfであり、仕様書がdoc\_specs.pdfである。

\subsection{Source files}

Source files are in directory \texttt{src}. Files related to the
standard features are directly under directory \texttt{src}. When
users include header file \texttt{particle\_simulator.hpp} in their
source files, the standard features become available.
%%ソースファイルはディレクトリsrcの下にある。標準機能関係のソースファイ
%%ルはsrcの直下にある。ディレクトリsrcの直下にあるヘッダファイル
%%particle\_simulator.hppをソースファイルにインクルードすれば、FDPSの標
%%準機能を使用できるようになる。

\subsubsection{Extended features}

Source files related to extended features are in directories in
directory \texttt{src}. The extended features include Particle Mesh
and Phantom-GRAPE version x86.
%%拡張機能関係のソースファイルはディレクトリsrcの直下のディレクトリにそ
%%れぞれ入っている。拡張機能にはParticle Mesh、x86版Phantom-GRAPEがある。

\subsubsubsection{Particle Mesh scheme}

Source files related to the Particle Mesh scheme are in directory
\texttt{particle\_mesh}.  When users edit \texttt{Makefile}, and
run \texttt{make}, they get header file
\texttt{particle\_mesh\_class.hpp} and library \texttt{libpm.a}. By
including this header file, and linking this library, they can use the
features of the Particle Mesh scheme.
%%Particle Meshのソースファイルはディレクトリsrc/particle\_meshの下にあ
%%る。ここでMakefileを編集して、makeを実行すると、ヘッダファイル
%%particle\_mesh\_class.hppとライブラリlibpm.aができる。このヘッダファ
%%イルをインクルードし、このライブラリをリンクづけすれば、Particle
%%Meshの機能を使用できるようになる。

\subsubsubsection{x86 version Phantom-GRAPE}

Source files related to x86 version Phantom-GRAPE are in directory
\texttt{src/phantom\_GRAPE\_x86}. This directory has directories containing
Phantom-GRAPE libraries for low precision, for low precision with cutoff, and
for high precision.
%%x86版Phantom-GRAPEのソースファイルはディレクトリ
%%src/phantom\_GRAPE\_x86の下にある。この下には低精度N体シミュレーショ
%%ン用、低精度カットオフ付き相互作用計算用、高精度N体シミュレーション用
%%(ディレクトリG6/libavx)がある。それぞれについて述べる。

\subsubsubsubsection{Low precision}

Files are in directory \texttt{src/phantom\_GRAPE\_x86/G5/newton/libpg5}.
Users should edit \texttt{Makefile} in this directory, and run \texttt{make},
to get library \texttt{libpg5.a}. Users include header file \texttt{gp5util.h}
in this directory, and to link library \texttt{libpg5.a}, to use this Phantom-GRAPE.
%%これはディレクトリsrc/phantom\_GRAPE\_x86/G5/newton/libpg5にある。こ
%%のディレクトリ内のMakefileを編集して、makeを実行すると、ライブラリ
%%libpg5.aができる。このディレクトリ内のヘッダファイルgp5util.hをインク
%%ルードし、ライブラリlibpg5.aをリンクすると、このPhantom-GRAPEが使用可
%%能になる。

\subsubsubsubsection{Low precision with cutoff}

Files are in directory \texttt{src/phantom\_GRAPE\_x86/G5/table}.
Users should edit \texttt{Makefile} in this directory, and run \texttt{make},
to get library \texttt{libpg5.a}. Users include header file \texttt{gp5util.h}
in this directory, and to link library \texttt{libpg5.a}, to use this Phantom-GRAPE.
%%これはディレクトリsrc/phantom\_GRAPE\_x86/G5/table/にある。このディレ
%%クトリ内のMakefileを編集して、makeを実行すると、ライブラリlibpg5.aが
%%できる。このディレクトリ内のヘッダファイルgp5util.hをインクルードし、
%%ライブラリlibpg5.aをリンクすると、このPhantom-GRAPEが使用可能になる。

\subsubsubsubsection{High precision}

Files are in directory \texttt{src/phantom\_GRAPE\_x86/G6/libavx}.
Users should edit \texttt{Makefile} in this directory, and run \texttt{make},
to get library \texttt{libg6avx.a}. Users include header file \texttt{gp6util.h}
in this directory, and to link library \texttt{libg6avx.a}, to use this Phantom-GRAPE.
%%これはディレクトリsrc/phantom\_GRAPE\_x86/G6/libavx/にある。このディ
%%レクトリ内のMakefileを編集して、makeを実行すると、ライブラリ
%%libg6avx.aができる。このディレクトリ内のヘッダファイルgp6util.hをイン
%%クルードし、ライブラリlibg6avx.aをリンクすると、このPhantom-GRAPEが使
%%用可能になる。

\subsection{Test codes}

Test codes are in directory \texttt{tests}. When users run \texttt{make check}
in this directory, the test suite are executed.
%%テストコードはディレクトリtestsの下にある。ディレクトリtestsにカレン
%%トディレクトリを移し、make checkを実行するとテストスィートが動作する。

\subsection{Sample codes}

Sample codes are in directory \texttt{sample}. There are two sample codes:
gravitational $N$-body simulation, and SPH simulation.
%%サンプルコードはディレクトリsampleの下にある。サンプルコードは2つ用
%%意されており、重力N体シミュレーションとSPHシミュレーションである。

\subsubsection{Gravitational $N$-body simulation}

Source files are in directory \texttt{sample/nbody}. In file \texttt{doc\_tutorial.pdf},
we describe how to use this code.
%%ディレクトリsample/nbodyの下にソースファイルがある。サンプルコードの
%%実行方法はチュートリアルを参照のこと。

\subsubsection{SPH Simulation}

Source files are in directory \texttt{sample/sph}. In file \texttt{doc\_tutorial.pdf},
we describe how to use this code.
%%ディレクトリsample/sphの下にソースファイルがある。サンプルコードの実
%%行方法はチュートリアルを参照のこと。
