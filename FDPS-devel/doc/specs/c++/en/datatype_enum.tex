\subsubsection{Summary}

In this section, we describe enumerated types defined in FDPS.
Currently, there is just one datatype. We describe it below.
%%本節ではFDPSで定義されている列挙型について記述する。列挙型には
%%BOUNDARY\_CONDITION型が存在する。以下、各列挙型について記述する。

\subsubsection{PS::BOUNDARY\_CONDITION type}
\label{sec:datatype_enum_boundarycondition}

\subsubsubsection{Summary}

Type BOUNDARY\_CONDITION specifies boundary conditions. The definition
is as follows.
%%BOUNDARY\_CONDITION型は境界条件を指定するためのデータ型である。これは
%%以下のように定義されている。
\begin{lstlisting}[caption=boundarycondition]
namespace ParticleSimulator{
    enum BOUNDARY_CONDITION{
        BOUNDARY_CONDITION_OPEN,
        BOUNDARY_CONDITION_PERIODIC_X,
        BOUNDARY_CONDITION_PERIODIC_Y,
        BOUNDARY_CONDITION_PERIODIC_Z,
        BOUNDARY_CONDITION_PERIODIC_XY,
        BOUNDARY_CONDITION_PERIODIC_XZ,
        BOUNDARY_CONDITION_PERIODIC_YZ,
        BOUNDARY_CONDITION_PERIODIC_XYZ,
        BOUNDARY_CONDITION_SHEARING_BOX,
        BOUNDARY_CONDITION_USER_DEFINED,
    };
}
\end{lstlisting}

We explain each value below.
%以下にどの変数がどの境界条件に対応するかを記述する。

\subsubsubsection{PS::BOUNDARY\_CONDITION\_OPEN}

This specifies the open boundary condition.
%開放境界となる。

\subsubsubsection{PS::BOUNDARY\_CONDITION\_PERIODIC\_X}

This specifies the periodic boundary condition in the direction of x-axis,
and open boundary condition in other directions. The interval is
left-bounded and right-unbounded. This is true for all periodic
boundary conditions.
%%x軸方向のみ周期境界、その他の軸方向は開放境界となる。周期の境界の下限
%%は閉境界、上限は開境界となっている。この境界の規定はすべての軸方向に
%%あてはまる。

\subsubsubsection{PS::BOUNDARY\_CONDITION\_PERIODIC\_Y}

This specifies the periodic boundary condition in the direction of y-axis,
and open boundary condition in other directions.
%y軸方向のみ周期境界、その他の軸方向は開放境界となる。

\subsubsubsection{PS::BOUNDARY\_CONDITION\_PERIODIC\_Z}

This specifies the periodic boundary condition in the direction of z-axis,
and open boundary condition in other directions.
%z軸方向のみ周期境界、その他の軸方向は開放境界となる。

\subsubsubsection{PS::BOUNDARY\_CONDITION\_PERIODIC\_XY}

This specifies the periodic boundary condition in the directions of x- and
y-axes, and open boundary condition in the direction of z-axis.
%x, y軸方向のみ周期境界、その他の軸方向は開放境界となる。

\subsubsubsection{PS::BOUNDARY\_CONDITION\_PERIODIC\_XZ}

This specifies the periodic boundary condition in the directions of x- and
z-axes, and open boundary condition in the direction of y-axis.
%x, z軸方向のみ周期境界、その他の軸方向は開放境界となる。

\subsubsubsection{PS::BOUNDARY\_CONDITION\_PERIODIC\_YZ}

This specifies the periodic boundary condition in the directions of y-
and z-axes, and open boundary condition in the direction of x-axis.
%y, z軸方向のみ周期境界、その他の軸方向は開放境界となる。

\subsubsubsection{PS::BOUNDARY\_CONDITION\_PERIODIC\_XYZ}

This specifies the periodic boundary condition in all three directions.
%x, y, z軸方向すべてが周期境界となる。

\subsubsubsection{PS::BOUNDARY\_CONDITION\_SHEARING\_BOX}

Not implemented yet.
%未実装。

\subsubsubsection{PS::BOUNDARY\_CONDITION\_USER\_DEFINED}

Not implemented yet.
%未実装。

%%%%%%%%%%%%%%%%%%%%%%%%%%%%%%%%%%%
\subsubsection{PS::INTERACTION\_LIST\_MODE type}
\label{sec:datatype_enum_interaction_list_mode}

%\subsubsubsection{概要}
\subsubsubsection{Summary}


Type INTERACTION\_LIST\_MODE is used to determine if user program
reuse the interaction list or not. This type is defined as follows.


%INTERACTION\_LIST\_MODE型は相互作用リストを再利用するかどうかを決定す
%るためのデータ型である。これは以下のように定義されている。

\begin{lstlisting}[caption=boundarycondition]
namespace ParticleSimulator{
    enum INTERACTION_LIST_MODE{
        MAKE_LIST,
        MAKE_LIST_FOR_REUSE,
        REUSE_LIST,
    };
}
\end{lstlisting}

This data type is used as the last argument of the function
calcForceAllAndWriteBack(). For more detail, plese see
section \ref{sec:treeForForceHighLevelAPI}.

%このデータ型はcalcForceAllAndWriteBack()等の関数の引数として使われる
%(詳しくはセクション\ref{sec:treeForForceHighLevelAPI}を参照)。

\subsubsubsection{PS::MAKE\_LIST}

FDPS (re)makes interaction lists for each interaction calculation
(each call of the APIs described above). In this case, we cannot reuse
interaction list in the next interaction calculation because FDPS does
not store the information of interaction list. \textbf{This is the
default operation mode in FDPS}.


%相互作用リストを毎回作り相互作用計算を行う場合に用いる。相互作用リスト
%の再利用はできない。

\subsubsubsection{PS::MAKE\_LIST\_FOR\_REUSE}

FDPS (re)makes interaction lists and stores them internally. Then, it
performs interaction calculation. In this case, we can reuse these
interaction lists in the next interaction calculation if we call the
APIs with the flag PS::REUSE\_LIST. The interaction lists memorized in
FDPS are destroyed if we perform the interaction calculation with the
flags PS::MAKE\_LIST\_FOR\_REUSE or PS::MAKE\_LIST.\\

%相互作用リストを再利用し相互作用計算を行いたい場合に用いる。このオプショ
%ンを選択する事でFDPSは相互作用リストを作りそれを保持する。作成した相互
%作用リストはPS::MAKE\_LIST\_FOR\_REUSEもしくはPS::MAKE\_LISTを用いて相
%互作用計算を行った際に破棄される。

\subsubsubsection{PS::REUSE\_INTERACTION\_LIST}

FDPS performs interaction calculation using the previously-created
interaction lists, which are the lists that are created at the
previous call of the APIs with the flag PS::MAKE\_LIST\_FOR\_REUSE. In
this case, moment information in superparticles are automatically
updated using the latest particle information.\\

%相互作用リストを再利用し相互作用計算を行う。再利用される相互作用リスト
%はPS::MAKE\_LIST\_FOR\_REUSEを選択時に作成した相互作用リストである。
