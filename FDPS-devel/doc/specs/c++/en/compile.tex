\subsection{Summary}

Some of the features of FDPS should be specified at the compile time.
They are (a)coordinate system, (b)method of parallelization, and (c) accuracy of floating point types.
%%FDPSでは、座標系や並列処理の有無を選択できる。この選択はコンパイル時
%%のマクロの定義によってなされる。以下、選択の方法について座標系、並列
%%処理の有無の順に記述する。

\subsection{Coordinate system}
\label{sec:compile_coordinate}

\subsubsection{Summary}

Users have alternatives of 2D and 3D Cartesian coordinate systems.
%%座標系は直角座標系3次元と直角座標系2次元の選択ができる。以下、それ
%%らの選択方法について述べる。

\subsubsection{3D Cartesian coordinate system}

3D Cartesian coordinate system is used by default.
%デフォルトは直角座標系3次元である。なにも行わなくても直角座標系3次元
%となる。

\subsubsection{2D Cartesian coordinate system}

2D Cartesian coordinate system can be used by defining
\texttt{PARTICLE\_SIMULATOR\_TWO\_DIMENSION} as macro.
%コンパイル時にPARTICLE\_SIMULATOR\_TWO\_DIMENSIONをマクロ定義すると直
%交座標系2次元となる。

\subsection{Parallel processing}

\subsubsection{Summary}

Users choose whether OpenMP is used or not, and whether MPI is used or not.
%並列処理に関しては、OpenMPの使用/不使用、MPIの使用/不使用を選択でき
%る。以下、選択の仕方について記述する。

\subsubsection{OpenMP}

OpenMP is disabled by default. If macro
\texttt{PARTICLE\_SIMULATOR\_THREAD\_PARALLEL} is defined, OpenMP becomes
enabled. Compiler option ``-fopenmp'' is required for GCC compiler.
%デフォルトはOpenMP不使用である。使用する場合は、
%\\ PARTICLE\_SIMULATOR\_THREAD\_PARALLELをマクロ定義すればよい。GCCコ
%ンパイラの場合はコンパイラオプションに-fopenmpをつける必要がある。

\subsubsection{MPI}

MPI is disabled by default. If macro
\texttt{PARTICLE\_SIMULATOR\_MPI\_PARALLEL} is defined.
%%デフォルトはMPI不使用である。使用する場合は、
%%PARTICLE\_SIMULATOR\_THREAD\_PARALLELをマクロ定義すればよい。

\subsection{Accuracy of data types}

\subsubsection{Summary}
Users can specify the accuracy of data types of Moment classes described in \S~7.5.2) and SuperParticleJ classes described in \S~7.6.2.

\subsubsection{Accuracy of SuperParticleJ and Moment classes prepared in FDPS}
All the member variables in SuperParticleJ classes and Moment classes are 64 bit accuracy by default. They becomes 32 bit accuracy if macro \texttt{PARTICLE\_SIMULATOR\_SPMOM\_F32BIT} is defined at the compile time.
