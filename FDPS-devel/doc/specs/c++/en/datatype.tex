\subsection{Summary}
In this chapter we define data types used in FDPS.The data types include
integer types, floating point types, vector types, symmetric matrix types,
PS::SEARCH\_MODE type, and enumerated types. We recommend users to
use these data types. The integer and floating point types can be
replaced with data types available in C/C++
languages. PS::SEARCH\_MODE and enumerated types must be used. In the
following, we describe these data types.

%%FDPSでは独自の整数型、実数型、ベクトル型、対称行列型、
%%PS::SEARCH\_MODE型、列挙型が定義されている。整数型、実数型、ベクトル
%%型、対称行列型に関しては必ずしもここに挙げるものを用いる必要はないが、
%%これらを用いることを推奨する。PS::SEARCH\_MODE型、列挙型は必ず用いる
%%必要がある。以下、整数型、実数型、ベクトル型、対称行列型、
%%PS::SEARCH\_MODE型、列挙型の順に記述する。

\subsection{Integer type}

%\subsubsection{概要}
\subsubsection{Overview}

The integer types are PS::S32, PS::S64, PS::U32, and PS::U64. We
describe these data types in this section.

%整数型にはPS::S32, PS::S64, PS::U32, PS::U64がある。以下、順にこれらを
%記述する。

\subsubsection{PS::S32}

PS::S32, which is 32-bit signed integer, is defined as follows.
%PS::S32は以下のように定義されている。すなわち32bitの符号付き整数である。
\begin{lstlisting}[caption=S32]
namespace ParticleSimulator {
    typedef int S32;
}
\end{lstlisting}

%Safe performance is ensured, only when users adopt GCC and K compilers.
%ただし、GCCコンパイラとKコンパイラでのみ32bitであることが保証されている。

\subsubsection{PS::S64}

PS::S64, which is 64-bit signed integer, is defined as follows.
%PS::S64は以下のように定義されている。すなわち64bitの符号付き整数である。
\begin{lstlisting}[caption=S64]
namespace ParticleSimulator {
    typedef long long int S64;
}
\end{lstlisting}

%Safe performance is ensured, only when users adopt GCC and K compilers.
%ただし、GCCコンパイラとKコンパイラでのみ64bitであることが保証されている。

\subsubsection{PS::U32}

PS::U32, which is 32-bit unsigned integer, is defined as follows.
%PS::U32は以下のように定義されている。すなわち32bitの符号なし整数である。
\begin{lstlisting}[caption=U32]
namespace ParticleSimulator {
    typedef unsigned int U32;
}
\end{lstlisting}

%Safe performance is ensured, only when users adopt GCC and K compilers.
%ただし、GCCコンパイラとKコンパイラでのみ32bitであることが保証されている。

\subsubsection{PS::U64}

PS::U64, which is 64-bit unsigned integer, is defined as follows.
%PS::U64は以下のように定義されている。すなわち64bitの符号なし整数である。
\begin{lstlisting}[caption=U64]
namespace ParticleSimulator {
    typedef unsigned long long int U64;
}
\end{lstlisting}

%Safe performance is ensured, only when users adopt GCC and K compilers.
%ただし、GCCコンパイラとKコンパイラでのみ64bitであることが保証されている。



\subsection{Floating point type}

\subsubsection{Abstract}

The floating point types are PS::F32 and PS::F64. We described
these data types in this section.
%実数型にはPS::F32, PS::F64がある。以下、順にこれらを記述する。

\subsubsection{PS::F32}

PS::F32, which is 32-bit floating point number, is defined as follows.
%PS::F32は以下のように定義されている。すなわち32bitの浮動小数点数である。
\begin{lstlisting}[caption=F32]
namespace ParticleSimulator {
    typedef float F32;
}
\end{lstlisting}

\subsubsection{PS::F64}

PS::F64, which is 64-bit floating point number, is defined as follows.
%PS::F64は以下のように定義されている。すなわち64bitの浮動小数点数である。
\begin{lstlisting}[caption=F64]
namespace ParticleSimulator {
    typedef double F64;
}
\end{lstlisting}


\subsection{Vector type}
\label{sec:datatype_vector}
\subsubsection{Abstract}

The vector types are \texttt{PS::Vector2} (2D vector) and
\texttt{PS::Vector3} (3D vector). We describe these vector types first.
These are template classes, which can take basic datatypes
as \texttt{F32} or \texttt{F64} as template arguments. We then present
wrappers for these vector types.

%%ベクトル型には2次元ベクトル型PS::Vector2と3次元ベクトル型PS::Vector3
%%がある。まずこれら2つを記述する。最後にこれらベクトル型のラッパーに
%%ついて記述する。

\subsubsection{PS::Vector2}

PS::Vector2はx, yの2要素を持つ。これらに対する様々なAPIや演算子を定義
した。それらの宣言を以下に記述する。この節ではこれらについて詳しく記述
する。
\begin{lstlisting}[caption=Vector2]
namespace ParticleSimulator{
    template <typename T>
    class Vector2{
    public:
        //メンバ変数2要素
        T x, y;

        //コンストラクタ
        Vector2();
        Vector2(const T _x, const T _y) : x(_x), y(_y) {}
        Vector2(const T s) : x(s), y(s) {}
        Vector2(const Vector2 & src) : x(src.x), y(src.y) {}

        //代入演算子
        const Vector2 & operator = (const Vector2 & rhs);

        //[]演算子
        const T & opertor[](const int i);
        T & operator[](const int i);

        //加減算
        Vector2 operator + (const Vector2 & rhs) const;
        const Vector2 & operator += (const Vector2 & rhs);
        Vector2 operator - (const Vector2 & rhs) const;
        const Vector2 & operator -= (const Vector2 & rhs);

        //ベクトルスカラ積
        Vector2 operator * (const T s) const;
        const Vector2 & operator *= (const T s);
        friend Vector2 operator * (const T s,
                                   const Vector2 & v);
        Vector2 operator / (const T s) const;
        const Vector2 & operator /= (const T s);

        //内積
        T operator * (const Vector2 & rhs) const;

        //外積(返り値はスカラ!!)
        T operator ^ (const Vector2 & rhs) const;

        //Vector2<U>への型変換
        template <typename U>
        operator Vector2<U> () const;
    };
}
namespace PS = ParticleSimulator;
\end{lstlisting}

\subsubsubsection{コンストラクタ}

\begin{screen}
\begin{verbatim}
template<typename T>
PS::Vector2<T>::Vector2()
\end{verbatim}
\end{screen}

\begin{itemize}

\item{{\bf 引数}}

なし。

\item{{\bf 機能}}

デフォルトコンストラクタ。メンバx,yは0で初期化される。

\end{itemize}

%%%%%%%%%%%%%%%%%%%%%%%%%%%%%
\begin{screen}
\begin{verbatim}
template<typename T>
PS::Vector2<T>::Vector2(const T _x, const T _y)
\end{verbatim}
\end{screen}

\begin{itemize}

\item{{\bf 引数}}

{\_x}: 入力。{const T}型。

{\_y}: 入力。{const T}型。

\item{{\bf 機能}}

メンバ{x}、{y}をそれぞれ{\_x}、{\_y}で初期化する。

\end{itemize}

%%%%%%%%%%%%%%%%%%%%%%%%%%%%%
\begin{screen}
\begin{verbatim}
template<typename T>
PS::Vector2<T>::Vector2(const T s);
\end{verbatim}
\end{screen}

\begin{itemize}

\item{{\bf 引数}}

{s}: 入力。{const T}型。

\item{{\bf 機能}}

メンバ{x}、{y}を両方とも{s}の値で初期化する。

\end{itemize}

%%%%%%%%%%%%%%%%%%%%%%%%%%%%%
\subsubsubsection{コピーコンストラクタ}

\begin{screen}
\begin{verbatim}
template<typename T>
PS::Vector2<T>::Vector2(const PS::Vector2<T> & src)
\end{verbatim}
\end{screen}

\begin{itemize}

\item{{\bf 引数}}

{src}: 入力。{const PS::Vector2$<$T$>$ \&}型。

\item{{\bf 機能}}

コピーコンストラクタ。{src}で初期化する。

\end{itemize}

%%%%%%%%%%%%%%%%%%%%%%%%%%%%%
\subsubsubsection{メンバ変数}

\begin{screen}
\begin{verbatim}
template<typename T>
T PS::Vector2<T>::x;

template<typename T>
T PS::Vector2<T>::y;
\end{verbatim}
\end{screen}

\begin{itemize}
  
\item{{\bf 機能}}
  
  メンバ{x}、{y}を直接操作出来る。
  
\end{itemize}

%%%%%%%%%%%%%%%%%%%%%%%%%%%%%
\subsubsubsection{代入演算子}

\begin{screen}
\begin{verbatim}
template<typename T>
const PS::Vector2<T> & PS::Vector2<T>::operator = 
                       (const PS::Vector2<T> & rhs);
\end{verbatim}
\end{screen}

\begin{itemize}

\item{{\bf 引数}}

{rhs}: 入力。{const PS::Vector2$<$T$>$ \&}型。

\item{{\bf 返り値}}

{const PS::Vector2$<$T$>$ \&}型。{rhs}のx,yの値を自身のメンバx,yに
代入し自身の参照を返す。代入演算子。

\end{itemize}

\subsubsubsection{[]演算子}

\begin{screen}
\begin{verbatim}
template<typename T>
const T & PS::Vector2<T>::operator[]
                       (const int i);
\end{verbatim}
\end{screen}

\begin{itemize}

\item{{\bf 引数}}

  {i}: 入力。{const int}型。

\item{{\bf 返り値}}

  {const $<$T$>$ \&}型。ベクトルのi成分を返す。
  
\item{{\bf 備考}}

  直接メンバ変数を指定する場合に比べ、処理が遅くなることがある。

\end{itemize}

\begin{screen}
\begin{verbatim}
template<typename T>
T & PS::Vector2<T>::operator[]
                       (const int i);
\end{verbatim}
\end{screen}

\begin{itemize}

\item{{\bf 引数}}

{i}: 入力。{const int}型。

\item{{\bf 返り値}}

{$<$T$>$ \&}型。ベクトルのi成分を返す。

\item{{\bf 備考}}

  直接メンバ変数を指定する場合に比べ、処理が遅くなることがある。

\end{itemize}

\subsubsubsection{加減算}

\begin{screen}
\begin{verbatim}
template<typename T>
PS::Vector2<T> PS::Vector2<T>::operator + 
               (const PS::Vector2<T> & rhs) const;
\end{verbatim}
\end{screen}

\begin{itemize}

\item{{\bf 引数}}

{rhs}: 入力。{const PS::Vector2$<$T$>$ \&}型。

\item{{\bf 返り値}}

{PS::Vector2$<$T$>$}型。{rhs}のx,yの値と自身のメンバx,yの値の和を
取った値を返す。

\end{itemize}

\begin{screen}
\begin{verbatim}
template<typename T>
const PS::Vector2<T> & PS::Vector2<T>::operator += 
                       (const PS::Vector2<T> & rhs);
\end{verbatim}
\end{screen}

\begin{itemize}

\item{{\bf 引数}}

{rhs}: 入力。{const PS::Vector2$<$T$>$ \&}型。

\item{{\bf 返り値}}

{const PS::Vector2$<$T$>$ \&}型。{rhs}のx,yの値を自身のメンバx,yに足し、自
身を返す。

\end{itemize}

\begin{screen}
\begin{verbatim}
template<typename T>
PS::Vector2<T> PS::Vector2<T>::operator - 
               (const PS::Vector2<T> & rhs) const;
\end{verbatim}
\end{screen}

\begin{itemize}

\item{{\bf 引数}}

{rhs}: 入力。{const PS::Vector2$<$T$>$ \&}型。

\item{{\bf 返り値}}

{PS::Vector2$<$T$>$}型。{rhs}のx,yの値と自身のメンバx,yの値の差を
取った値を返す。

\end{itemize}

\begin{screen}
\begin{verbatim}
template<typename T>
const PS::Vector2<T> & PS::Vector2<T>::operator -= 
                       (const PS::Vector2<T> & rhs);
\end{verbatim}
\end{screen}

\begin{itemize}

\item{{\bf 引数}}

{rhs}: 入力。{const PS::Vector2$<$T$>$ \&}型。

\item{{\bf 返り値}}

{const PS::Vector2$<$T$>$ \&}型。自身のメンバx,yから{rhs}のx,yを引
き自身を返す。

\end{itemize}

\subsubsubsection{ベクトルスカラ積}

\begin{screen}
\begin{verbatim}
template<typename T>
PS::Vector2<T> PS::Vector2<T>::operator * (const T s) const;
\end{verbatim}
\end{screen}

\begin{itemize}

\item{{\bf 引数}}

{s}: 入力。{const T}型。

\item{{\bf 返り値}}

{PS::Vector2$<$T$>$}型。自身のメンバx,yそれぞれに{s}をかけた値を返
す。

\end{itemize}

\begin{screen}
\begin{verbatim}
template<typename T>
const PS::Vector2<T> & PS::Vector2<T>::operator *= (const T s);
\end{verbatim}
\end{screen}

\begin{itemize}

\item{{\bf 引数}}

{rhs}: 入力。{const T}型。

\item{{\bf 返り値}}

{const PS::Vector2$<$T$>$ \&}型。自身のメンバx,yそれぞれに{s}をかけ
自身を返す。

\end{itemize}

\begin{screen}
\begin{verbatim}
template<typename T>
PS::Vector2<T> PS::Vector2<T>::operator / (const T s) const;
\end{verbatim}
\end{screen}

\begin{itemize}

\item{{\bf 引数}}

{s}: 入力。{const T}型。

\item{{\bf 返り値}}

{PS::Vector2$<$T$>$}型。自身のメンバx,yそれぞれを{s}で割った値を返
す。

\end{itemize}

\begin{screen}
\begin{verbatim}
template<typename T>
const PS::Vector2<T> & PS::Vector2<T>::operator /= (const T s);
\end{verbatim}
\end{screen}

\begin{itemize}

\item{{\bf 引数}}

{rhs}: 入力。{const T}型。

\item{{\bf 返り値}}

{const PS::Vector2$<$T$>$ \&}型。自身のメンバx,yそれぞれを{s}で割り
自身を返す。

\end{itemize}

\subsubsubsection{内積、外積}

\begin{screen}
\begin{verbatim}
template<typename T>
T PS::Vector2<T>::operator * (const PS::Vector2<T> & rhs) const;
\end{verbatim}
\end{screen}

\begin{itemize}

\item{{\bf 引数}}

{rhs}: 入力。{const PS::Vector2$<$T$>$ \&}型。

\item{{\bf 返り値}}

{T}型。自身と{rhs}の内積を取った値を返す。

\end{itemize}

\begin{screen}
\begin{verbatim}
template<typename T>
T PS::Vector2<T>::operator ^ (const PS::Vector2<T> & rhs) const;
\end{verbatim}
\end{screen}

\begin{itemize}

\item{{\bf 引数}}

{rhs}: 入力。{const PS::Vector2$<$T$>$ \&}型。

\item{{\bf 返り値}}

{T}型。自身と{rhs}の外積を取った値を返す。

\end{itemize}

\subsubsubsection{{Vector2$<$U$>$}への型変換}

\begin{screen}
\begin{verbatim}
template<typename T>
template <typename U>
PS::Vector2<T>::operator PS::Vector2<U> () const;
\end{verbatim}
\end{screen}

\begin{itemize}

\item{{\bf 引数}}

  なし。

\item{{\bf 返り値}}

  {const PS::Vector2$<$U$>$}型。

\item{{\bf 機能}}

  {const PS::Vector2$<$T$>$}型を{const PS::Vector2$<$U$>$}型にキャ
  ストする。

\end{itemize}




\subsubsection{PS::Vector3}

PS::Vecotr3はx, y, zの3要素を持つ。これらに対する様々なAPIや演算子を定
義した。それらの宣言を以下に記述する。この節ではこれらについて詳しく記
述する。
\begin{lstlisting}[caption=Vector3]
namespace ParticleSimulator{
    template <typename T>
    class Vector3{
    public:
        //メンバ変数は以下の二つのみ。
        T x, y, z;

        //コンストラクタ
        Vector3() : x(T(0)), y(T(0)), z(T(0)) {}
        Vector3(const T _x, const T _y, const T _z) : x(_x), y(_y), z(_z) {}
        Vector3(const T s) : x(s), y(s), z(s) {}
        Vector3(const Vector3 & src) : x(src.x), y(src.y), z(src.z) {}

        //代入演算子
        const Vector3 & operator = (const Vector3 & rhs);

        //[]演算子
        const T & opertor[](const int i);
        T & operator[](const int i);

        //加減算
        Vector3 operator + (const Vector3 & rhs) const;
        const Vector3 & operator += (const Vector3 & rhs);
        Vector3 operator - (const Vector3 & rhs) const;
        const Vector3 & operator -= (const Vector3 & rhs);

        //ベクトルスカラ積
        Vector3 operator * (const T s) const;
        const Vector3 & operator *= (const T s);
        friend Vector3 operator * (const T s, const Vector3 & v);
        Vector3 operator / (const T s) const;
        const Vector3 & operator /= (const T s);

        //内積
        T operator * (const Vector3 & rhs) const;

        //外積(返り値はスカラ!!)
        T operator ^ (const Vector3 & rhs) const;

        //Vector3<U>への型変換
        template <typename U>
        operator Vector3<U> () const;
    };
}
\end{lstlisting}
%%%%%%%%%%%%%%%%%%%%%%%%%%%%%
\subsubsubsection{コンストラクタ}
\mbox{}
%%%%%%%%%%%%%%%%%%%%%%%%%%%%%
%%%%%%%%%%%%%%%%%%%%%%%%%%%%%
\begin{screen}
\begin{verbatim}
template<typename T>
PS::Vector3<T>::Vector3()
\end{verbatim}
\end{screen}

\begin{itemize}

\item{{\bf 引数}}

なし。

\item{{\bf 機能}}

デフォルトコンストラクタ。メンバx,yは0で初期化される。

\end{itemize}

%%%%%%%%%%%%%%%%%%%%%%%%%%%%%
\begin{screen}
\begin{verbatim}
template<typename T>
PS::Vector3<T>::Vector3(const T _x, const T _y)
\end{verbatim}
\end{screen}

\begin{itemize}

\item{{\bf 引数}}

{\_x}: 入力。{const T}型。

{\_y}: 入力。{const T}型。

\item{{\bf 機能}}

メンバ{x}、{y}をそれぞれ{\_x}、{\_y}で初期化する。

\end{itemize}

%%%%%%%%%%%%%%%%%%%%%%%%%%%%%
\begin{screen}
\begin{verbatim}
template<typename T>
PS::Vector3<T>::Vector3(const T s);
\end{verbatim}
\end{screen}

\begin{itemize}

\item{{\bf 引数}}

{s}: 入力。{const T}型。

\item{{\bf 機能}}

メンバ{x}、{y}を両方とも{s}の値で初期化する。

\end{itemize}

%%%%%%%%%%%%%%%%%%%%%%%%%%%%%
\subsubsubsection{コピーコンストラクタ}
\mbox{}
%%%%%%%%%%%%%%%%%%%%%%%%%%%%%

%%%%%%%%%%%%%%%%%%%%%%%%%%%%%
\begin{screen}
\begin{verbatim}
template<typename T>
PS::Vector3<T>::Vector3(const PS::Vector3<T> & src)
\end{verbatim}
\end{screen}

\begin{itemize}

\item{{\bf 引数}}

{src}: 入力。{const PS::Vector3$<$T$>$ \&}型。

\item{{\bf 機能}}

コピーコンストラクタ。{src}で初期化する。

\end{itemize}

%%%%%%%%%%%%%%%%%%%%%%%%%%%%%
\subsubsubsection{メンバ変数}

\begin{screen}
\begin{verbatim}
template<typename T>
T PS::Vector3<T>::x;

template<typename T>
T PS::Vector3<T>::y;

template<typename T>
T PS::Vector3<T>::z;
\end{verbatim}
\end{screen}

\begin{itemize}
  
\item{{\bf 機能}}
  
  メンバ{x}、{y}、{z}を直接操作出来る。
  
\end{itemize}

%%%%%%%%%%%%%%%%%%%%%%%%%%%%%
\subsubsubsection{代入演算子}
\mbox{}
%%%%%%%%%%%%%%%%%%%%%%%%%%%%%

%%%%%%%%%%%%%%%%%%%%%%%%%%%%%
\begin{screen}
\begin{verbatim}
template<typename T>
const PS::Vector3<T> & PS::Vector3<T>::operator = 
                       (const PS::Vector3<T> & rhs);
\end{verbatim}
\end{screen}

\begin{itemize}

\item{{\bf 引数}}

{rhs}: 入力。{const PS::Vector3$<$T$>$ \&}型。

\item{{\bf 返り値}}

{const PS::Vector3$<$T$>$ \&}型。{rhs}のx,yの値を自身のメンバx,yに
代入し自身の参照を返す。代入演算子。

\end{itemize}

\subsubsubsection{[]演算子}

\begin{screen}
\begin{verbatim}
template<typename T>
const T & PS::Vector3<T>::operator[]
                       (const int i);
\end{verbatim}
\end{screen}

\begin{itemize}

\item{{\bf 引数}}

{i}: 入力。{const int}型。

\item{{\bf 返り値}}

{const $<$T$>$ \&}型。ベクトルのi成分を返す。

\item{{\bf 備考}}

  直接メンバ変数を指定する場合に比べ、処理が遅くなることがある。

\end{itemize}

\begin{screen}
\begin{verbatim}
template<typename T>
T & PS::Vector3<T>::operator[]
                       (const int i);
\end{verbatim}
\end{screen}

\begin{itemize}

\item{{\bf 引数}}

{i}: 入力。{const int}型。

\item{{\bf 返り値}}

{$<$T$>$ \&}型。ベクトルのi成分を返す。

\item{{\bf 備考}}

  直接メンバ変数を指定する場合に比べ、処理が遅くなることがある。

\end{itemize}


%%%%%%%%%%%%%%%%%%%%%%%%%%%%%
\subsubsubsection{加減算}
\mbox{}
%%%%%%%%%%%%%%%%%%%%%%%%%%%%%

%%%%%%%%%%%%%%%%%%%%%%%%%%%%%
\begin{screen}
\begin{verbatim}
template<typename T>
PS::Vector3<T> PS::Vector3<T>::operator + 
               (const PS::Vector3<T> & rhs) const;
\end{verbatim}
\end{screen}

\begin{itemize}

\item{{\bf 引数}}

{rhs}: 入力。{const PS::Vector3$<$T$>$ \&}型。

\item{{\bf 返り値}}

{PS::Vector3$<$T$>$}型。{rhs}のx,yの値と自身のメンバx,yの値の和を
取った値を返す。

\end{itemize}


%%%%%%%%%%%%%%%%%%%%%%%%%%%%%
\begin{screen}
\begin{verbatim}
template<typename T>
const PS::Vector3<T> & PS::Vector3<T>::operator += 
                       (const PS::Vector3<T> & rhs);
\end{verbatim}
\end{screen}

\begin{itemize}

\item{{\bf 引数}}

{rhs}: 入力。{const PS::Vector3$<$T$>$ \&}型。

\item{{\bf 返り値}}

{const PS::Vector3$<$T$>$ \&}型。{rhs}のx,yの値を自身のメンバx,yに足し、自
身を返す。

\end{itemize}


%%%%%%%%%%%%%%%%%%%%%%%%%%%%%
\begin{screen}
\begin{verbatim}
template<typename T>
PS::Vector3<T> PS::Vector3<T>::operator - 
               (const PS::Vector3<T> & rhs) const;
\end{verbatim}
\end{screen}

\begin{itemize}

\item{{\bf 引数}}

{rhs}: 入力。{const PS::Vector3$<$T$>$ \&}型。

\item{{\bf 返り値}}

{PS::Vector3$<$T$>$}型。{rhs}のx,yの値と自身のメンバx,yの値の差を
取った値を返す。

\end{itemize}


%%%%%%%%%%%%%%%%%%%%%%%%%%%%%
\begin{screen}
\begin{verbatim}
template<typename T>
const PS::Vector3<T> & PS::Vector3<T>::operator -= 
                       (const PS::Vector3<T> & rhs);
\end{verbatim}
\end{screen}

\begin{itemize}

\item{{\bf 引数}}

{rhs}: 入力。{const PS::Vector3$<$T$>$ \&}型。

\item{{\bf 返り値}}

{const PS::Vector3$<$T$>$ \&}型。自身のメンバx,yから{rhs}のx,yを引
き自身を返す。

\end{itemize}

%%%%%%%%%%%%%%%%%%%%%%%%%%%%%
\subsubsubsection{ベクトルスカラ積}
\mbox{}
%%%%%%%%%%%%%%%%%%%%%%%%%%%%%

%%%%%%%%%%%%%%%%%%%%%%%%%%%%%
\begin{screen}
\begin{verbatim}
template<typename T>
PS::Vector3<T> PS::Vector3<T>::operator * (const T s) const;
\end{verbatim}
\end{screen}

\begin{itemize}

\item{{\bf 引数}}

{s}: 入力。{const T}型。

\item{{\bf 返り値}}

{PS::Vector3$<$T$>$}型。自身のメンバx,yそれぞれに{s}をかけた値を返
す。

\end{itemize}


%%%%%%%%%%%%%%%%%%%%%%%%%%%%%
\begin{screen}
\begin{verbatim}
template<typename T>
const PS::Vector3<T> & PS::Vector3<T>::operator *= (const T s);
\end{verbatim}
\end{screen}

\begin{itemize}

\item{{\bf 引数}}

{rhs}: 入力。{const T}型。

\item{{\bf 返り値}}

{const PS::Vector3$<$T$>$ \&}型。自身のメンバx,yそれぞれに{s}をかけ
自身を返す。

\end{itemize}


%%%%%%%%%%%%%%%%%%%%%%%%%%%%%
\begin{screen}
\begin{verbatim}
template<typename T>
PS::Vector3<T> PS::Vector3<T>::operator / (const T s) const;
\end{verbatim}
\end{screen}

\begin{itemize}

\item{{\bf 引数}}

{s}: 入力。{const T}型。

\item{{\bf 返り値}}

{PS::Vector3$<$T$>$}型。自身のメンバx,yそれぞれを{s}で割った値を返
す。

\end{itemize}


%%%%%%%%%%%%%%%%%%%%%%%%%%%%%
\begin{screen}
\begin{verbatim}
template<typename T>
const PS::Vector3<T> & PS::Vector3<T>::operator /= (const T s);
\end{verbatim}
\end{screen}

\begin{itemize}

\item{{\bf 引数}}

{rhs}: 入力。{const T}型。

\item{{\bf 返り値}}

{const PS::Vector3$<$T$>$ \&}型。自身のメンバx,yそれぞれを{s}で割り
自身を返す。

\end{itemize}


%%%%%%%%%%%%%%%%%%%%%%%%%%%%%
\subsubsubsection{内積、外積}
\mbox{}
%%%%%%%%%%%%%%%%%%%%%%%%%%%%%

%%%%%%%%%%%%%%%%%%%%%%%%%%%%%
\begin{screen}
\begin{verbatim}
template<typename T>
T PS::Vector3<T>::operator * (const PS::Vector3<T> & rhs) const;
\end{verbatim}
\end{screen}

\begin{itemize}

\item{{\bf 引数}}

{rhs}: 入力。{const PS::Vector3$<$T$>$ \&}型。

\item{{\bf 返り値}}

{T}型。自身と{rhs}の内積を取った値を返す。

\end{itemize}

%%%%%%%%%%%%%%%%%%%%%%%%%%%%%
\begin{screen}
\begin{verbatim}
template<typename T>
T PS::Vector3<T>::operator ^ (const PS::Vector3<T> & rhs) const;
\end{verbatim}
\end{screen}

\begin{itemize}

\item{{\bf 引数}}

{rhs}: 入力。{const PS::Vector3$<$T$>$ \&}型。

\item{{\bf 返り値}}

{T}型。自身と{rhs}の外積を取った値を返す。

\end{itemize}


%%%%%%%%%%%%%%%%%%%%%%%%%%%%%
\subsubsubsection{{Vector3$<$U$>$}への型変換}
\mbox{}
%%%%%%%%%%%%%%%%%%%%%%%%%%%%%

%%%%%%%%%%%%%%%%%%%%%%%%%%%%%
\begin{screen}
\begin{verbatim}
template<typename T>
template <typename U>
PS::Vector3<T>::operator PS::Vector3<U> () const;
\end{verbatim}
\end{screen}

\begin{itemize}

\item{{\bf 引数}}

  なし

\item{{\bf 返り値}}

  {const PS::Vector3$<$U$>$}型。

\item{{\bf 機能}}

  {const PS::Vector3$<$T$>$}型を{const PS::Vector3$<$U$>$}型にキャ
  ストする。

\end{itemize}


\subsubsection{Wrappers}

The wrappers of vector types are defined as follows.
%ベクトル型のラッパーの定義を以下に示す。
\begin{lstlisting}[caption=vectorwrapper]
namespace ParticleSimulator{
    typedef Vector2<F32> F32vec2;
    typedef Vector3<F32> F32vec3;
    typedef Vector2<F64> F64vec2;
    typedef Vector3<F64> F64vec3;
#ifdef PARTICLE_SIMULATOR_TWO_DIMENSION
    typedef F32vec2 F32vec;
    typedef F64vec2 F64vec;
#else
    typedef F32vec3 F32vec;
    typedef F64vec3 F64vec;
#endif
}
\end{lstlisting}

PS::F32vec2, PS::F32vec3, PS::F64vec2, and PS::F64vec3 are,
respectively, 2D vector in single precision, 3D vector in single
presicion, 2D vector in double precision, and 3D vector in double
precision. If users set 2D (3D) coordinate system, PS::F32vec and
PS::F64vec is wrappers of PS::F32vec2 and PS::F64vec2 (PS::F32vec3 and
PS::F64vec3).
%%すなわちPS::F32vec2, PS::F32vec3, PS::F64vec2, PS::F64vec3はそれぞれ
%%単精度2次元ベクトル、倍精度2次元ベクトル、単精度3次元ベクトル、倍精度
%%3次元ベクトルである。FDPSで扱う空間座標系を2次元とした場合、
%%PS::F32vecとPS::F64vecはそれぞれ単精度2次元ベクトル、倍精度2次元ベク
%%トルとなる。一方、FDPSで扱う空間座標系を3次元とした場合、PS::F32vecと
%%PS::F64vecはそれぞれ単精度3次元ベクトル、倍精度3次元ベクトルとなる。





\subsection{Orthotope type}
\label{sec:datatype_orthotope}
\subsubsection{Abstract}

The orthotope types are \texttt{PS::Orthotope2} (rectangle) and
\texttt{PS::Orthotope3} (cuboid). We describe these orthotope types first.
These are template classes, which can take basic datatypes
as \texttt{F32} or \texttt{F64} as template arguments. We then present
wrappers for these orthotope types.


\subsubsection{PS::Orthotope2}

\texttt{PS::Orthotope2} has two components: \texttt{low\_} and \texttt{high\_}, both of which are \texttt{PS::Vector2} class.
We define various APIs and operators for these components.
In the following, we describe them.

\begin{lstlisting}[caption=Orthotope2]
namespace ParticleSimulator{
    template<class T>
    class Orthotope2{
    public:
        Vector2<T> low_;
        Vector2<T> high_;

        Orthotope2(): low_(9999.9), high_(-9999.9){}
        
        Orthotope2(const Vector2<T> & _low, const Vector2<T> & _high)
            : low_(_low), high_(_high){}
        
        Orthotope2(const Orthotope2 & src) : low_(src.low_), high_(src.high_){}

        Orthotope2(const Vector2<T> & center, const T length) :
            low_(center-(Vector2<T>)(length)), high_(center+(Vector2<T>)(length)) {
        }

        void initNegativeVolume(){
            low_ = std::numeric_limits<float>::max() / 128;
            high_ = -low_;
        }

        void init(){
            initNegativeVolume();
        }

        void merge( const Orthotope2 & ort ){
            this->high_.x = ( this->high_.x > ort.high_.x ) ? this->high_.x : ort.high_.x;
            this->high_.y = ( this->high_.y > ort.high_.y ) ? this->high_.y : ort.high_.y;
            this->low_.x = ( this->low_.x <= ort.low_.x ) ? this->low_.x : ort.low_.x;
            this->low_.y = ( this->low_.y <= ort.low_.y ) ? this->low_.y : ort.low_.y;
        }

        void merge( const Vector2<T> & vec ){
            this->high_.x = ( this->high_.x > vec.x ) ? this->high_.x : vec.x;
            this->high_.y = ( this->high_.y > vec.y ) ? this->high_.y : vec.y;
            this->low_.x = ( this->low_.x <= vec.x ) ? this->low_.x : vec.x;
            this->low_.y = ( this->low_.y <= vec.y ) ? this->low_.y : vec.y;
        }

        void merge( const Vector2<T> & vec, const T size){
            this->high_.x = ( this->high_.x > vec.x + size ) ? this->high_.x : vec.x + size;
            this->high_.y = ( this->high_.y > vec.y + size ) ? this->high_.y : vec.y + size;
            this->low_.x = ( this->low_.x <= vec.x - size) ? this->low_.x : vec.x - size;
            this->low_.y = ( this->low_.y <= vec.y - size) ? this->low_.y : vec.y - size;
        }
    };
}
namespace PS = ParticleSimulator;
\end{lstlisting}

%%%%%%%%%%%%%%%%%%%%%%%%%%%%%
\subsubsubsection{Member variables}

\begin{screen}
\begin{verbatim}
template<typename T>
PS::Vector2<T> PS::Orthotope2<T>::low_;

template<typename T>
PS::Vector2<T> PS::Orthotope2<T>::high_;
\end{verbatim}
\end{screen}

\begin{itemize}
  
\item{{\bf Feature}}
  
Member variables, \texttt{low\_} and \texttt{high\_} can be directly handled.
  
\end{itemize}


%%%%%%%%%%%%%%%%%%%%%%%%%%%%%%%%
\subsubsubsection{Constructors}
%---------------
\begin{screen}
\begin{verbatim}
template<typename T>
PS::Orthotope2<T>::Orthotope2();
\end{verbatim}
\end{screen}

\begin{itemize}

\item{{\bf Arguments}}

None. But, the template argument \texttt{T} must be either \texttt{PS::F32} or \texttt{PS::F64}.

\item{{\bf Feature}}

Default constructor.
Member variables \texttt{low\_} and \texttt{high\_} are initialized by (9999.9, 9999.9) and (-9999.9, -9999.9), respectively.

\end{itemize}
%---------------
\begin{screen}
\begin{verbatim}
template<typename T>
PS::Orthotope2<T>::Orthotope2(const Vector2<T> _low, const Vector2<T> _high);
\end{verbatim}
\end{screen}

\begin{itemize}

\item{{\bf Arguments}}

\texttt{\_low}: Input. Type \texttt{const Vector2$<$T$>$}.

\texttt{\_high}: Input. Type \texttt{const Vector2$<$T$>$}.

where \texttt{T} must be either \texttt{PS::F32} or \texttt{PS::F64}.

\item{{\bf Feature}}

Member variables \texttt{low\_} and \texttt{high\_} are initialized by \texttt{\_low} and \texttt{\_high}, respectively.

\end{itemize}
%---------------
\begin{screen}
\begin{verbatim}
template<typename T>
PS::Orthotope2<T>::Orthotope2(const Vector2<T> & center, const T length);
\end{verbatim}
\end{screen}

\begin{itemize}

\item{{\bf Arguments}}

\texttt{center}: Input. Type \texttt{const Vector2$<$T$>$ \&}.
\texttt{length}: Input. Type \texttt{const T}.

\item{{\bf Feature}}

Member variables \texttt{low\_} and \texttt{high\_} are initialized by \texttt{center-(Vector2$<$T$>$)(length)} and \texttt{center+(Vector2$<$T$>$)(length)}, respectively.

\end{itemize}


%%%%%%%%%%%%%%%%%%%%%%%%%%%%%%%%
\subsubsubsection{Copy constructor}
%---------------
\begin{screen}
\begin{verbatim}
template<typename T>
PS::Orthotope2<T>::Orthotope2(const Orthotope2<T> & src);
\end{verbatim}
\end{screen}

\begin{itemize}

\item{{\bf Argument}}

\texttt{src}: Input. Type \texttt{const Orthotope2$<$T$>$ \&}.

\item{{\bf Feature}}

Member variables \texttt{low\_} and \texttt{high\_} are initialized by \texttt{src.low\_} and \texttt{src.high\_}, respectively.

\end{itemize}


%%%%%%%%%%%%%%%%%%%%%%%%%%%%%
\subsubsubsection{Initialize}
%---------------
\begin{screen}
\begin{verbatim}
template<typename T>
PS::Orthotope2<T>::initNegativeVolume();
\end{verbatim}
\end{screen}

\begin{itemize}

\item{{\bf Arguments}}

None.

\item{{\bf Feature}}

Member variables \texttt{low\_} and \texttt{high\_} are initialized by ($a$, $a$) and (-$a$, -$a$), respectively,
where $a$=\texttt{std::numeric\_limits$<$float$>$::max() / 128}.

\end{itemize}

%---------------
\begin{screen}
\begin{verbatim}
template<typename T>
PS::Orthotope2<T>::init();
\end{verbatim}
\end{screen}

\begin{itemize}

\item{{\bf Arguments}}

None.

\item{{\bf Feature}}

This member function is the same as \texttt{initNegativeVolume}.

\end{itemize}

%%%%%%%%%%%%%%%%%%%%%%%%%%%%%
\subsubsubsection{Merge operations}
%---------------
\begin{screen}
\begin{verbatim}
template<typename T>
void PS::Orthotope2<T>::merge( const Orthotope2 & ort )();
\end{verbatim}
\end{screen}

\begin{itemize}

\item{{\bf Arguments}}

\texttt{ort}: Input. Type \texttt{const Orthotope2 \&}.

\item{{\bf Feature}}

Member variables \texttt{low\_} and \texttt{high\_} are updated so that they represents the smallest rectangle that contains both the original rectangle and rectangle \texttt{ort}.

\end{itemize}
%---------------
\begin{screen}
\begin{verbatim}
template<typename T>
void PS::Orthotope2<T>::merge( const Vector2<T> & vec )();
\end{verbatim}
\end{screen}

\begin{itemize}

\item{{\bf Arguments}}

\texttt{vec}: Input. Type \texttt{const Vector2$<$T$>$ \&}.

\item{{\bf Feature}}

Member variables \texttt{low\_} and \texttt{high\_} are updated so that they represents the smallest rectangle that contains the point \texttt{vec}.

\end{itemize}
%---------------
\begin{screen}
\begin{verbatim}
template<typename T>
void PS::Orthotope2<T>::merge( const Vector2<T> & vec, const T size )();
\end{verbatim}
\end{screen}

\begin{itemize}

\item{{\bf Arguments}}

\texttt{vec}: Input. Type \texttt{const Vector2$<$T$>$ \&}.
\texttt{size}: Input. Type \texttt{const T}.

\item{{\bf Feature}}

Member variables \texttt{low\_} and {high\_} are updated so that they represents the smallest rectangle that contains a circle whose center is \texttt{vec} and radius is \texttt{size}.

\end{itemize}



\subsubsection{PS::Orthotope3}

\texttt{PS::Orthotope3}は\texttt{PS::Vector3}型のメンバ変数\texttt{low\_}, \texttt{high\_}を持つ。
これらに対する様々なAPIや演算子を定義した。それらの宣言を以下に記述する。
この節ではこれらについて詳しく記述する。
\begin{lstlisting}[caption=Orthotope3]
namespace ParticleSimulator{
    template<class T>
    class Orthotope3{
    public:
        Vector3<T> low_;
        Vector3<T> high_;

        Orthotope3(): low_(9999.9), high_(-9999.9){}

        Orthotope3(const Vector3<T> & _low, const Vector3<T> & _high)
            : low_(_low), high_(_high) {}

        Orthotope3(const Orthotope3 & src) : low_(src.low_), high_(src.high_){}

        Orthotope3(const Vector3<T> & center, const T length) :
            low_(center-(Vector3<T>)(length)), high_(center+(Vector3<T>)(length)) {
        }

        void initNegativeVolume(){
            low_ = std::numeric_limits<float>::max() / 128;
            high_ = -low_;
        }

        void init(){
            initNegativeVolume();
        }

        void merge( const Orthotope3 & ort ){
            this->high_.x = ( this->high_.x > ort.high_.x ) ? this->high_.x : ort.high_.x;
            this->high_.y = ( this->high_.y > ort.high_.y ) ? this->high_.y : ort.high_.y;
            this->high_.z = ( this->high_.z > ort.high_.z ) ? this->high_.z : ort.high_.z;
            this->low_.x = ( this->low_.x <= ort.low_.x ) ? this->low_.x : ort.low_.x;
            this->low_.y = ( this->low_.y <= ort.low_.y ) ? this->low_.y : ort.low_.y;
            this->low_.z = ( this->low_.z <= ort.low_.z ) ? this->low_.z : ort.low_.z;
        }

        void merge( const Vector3<T> & vec ){
            this->high_.x = ( this->high_.x > vec.x ) ? this->high_.x : vec.x;
            this->high_.y = ( this->high_.y > vec.y ) ? this->high_.y : vec.y;
            this->high_.z = ( this->high_.z > vec.z ) ? this->high_.z : vec.z;
            this->low_.x = ( this->low_.x <= vec.x ) ? this->low_.x : vec.x;
            this->low_.y = ( this->low_.y <= vec.y ) ? this->low_.y : vec.y;
            this->low_.z = ( this->low_.z <= vec.z ) ? this->low_.z : vec.z;
        }

        void merge( const Vector3<T> & vec, const T size){
            this->high_.x = ( this->high_.x > vec.x + size ) ? this->high_.x : vec.x + size;
            this->high_.y = ( this->high_.y > vec.y + size ) ? this->high_.y : vec.y + size;
            this->high_.z = ( this->high_.z > vec.z + size) ? this->high_.z : vec.z + size;
            this->low_.x = ( this->low_.x <= vec.x - size) ? this->low_.x : vec.x - size;
            this->low_.y = ( this->low_.y <= vec.y - size) ? this->low_.y : vec.y - size;
            this->low_.z = ( this->low_.z <= vec.z - size) ? this->low_.z : vec.z - size;
        }
    };
}
namespace PS = ParticleSimulator;
\end{lstlisting}

%%%%%%%%%%%%%%%%%%%%%%%%%%%%%
\subsubsubsection{メンバ変数}

\begin{screen}
\begin{verbatim}
template<typename T>
PS::Vector3<T> PS::Orthotope3<T>::low_;

template<typename T>
PS::Vector3<T> PS::Orthotope3<T>::high_;
\end{verbatim}
\end{screen}

\begin{itemize}
  
\item{{\bf 機能}}
  
メンバ変数\texttt{low\_}、\texttt{high\_}を直接操作出来る。
  
\end{itemize}


%%%%%%%%%%%%%%%%%%%%%%%%%%%%%%%%
\subsubsubsection{コンストラクタ}
%---------------
\begin{screen}
\begin{verbatim}
template<typename T>
PS::Orthotope3<T>::Orthotope3();
\end{verbatim}
\end{screen}

\begin{itemize}

\item{{\bf 引数}}

なし。ただし、テンプレート引数\texttt{T}は \texttt{PS::F32} または \texttt{PS::F64} でなければならない。

\item{{\bf 機能}}

デフォルトコンストラクタ。
メンバ変数\texttt{low\_}, \texttt{high\_}は、それぞれ、(9999.9, 9999.9, 9999.9), (-9999.9, -9999.9, -9999.9)で初期化される。

\end{itemize}
%---------------
\begin{screen}
\begin{verbatim}
template<typename T>
PS::Orthotope3<T>::Orthotope3(const Vector3<T> _low, const Vector3<T> _high);
\end{verbatim}
\end{screen}

\begin{itemize}

\item{{\bf 引数}}

\texttt{\_low}: 入力。\texttt{const Vector3$<$T$>$}型。

\texttt{\_high}: 入力。\texttt{const Vector3$<$T$>$}型。

ここで、\texttt{T} は \texttt{PS::F32} または \texttt{PS::F64} でなければならない。

\item{{\bf 機能}}

メンバ変数\texttt{low\_}、\texttt{high\_}を、それぞれ\texttt{\_low}、\texttt{\_high}で初期化する。

\end{itemize}
%---------------
\begin{screen}
\begin{verbatim}
template<typename T>
PS::Orthotope3<T>::Orthotope3(const Vector3<T> & center, const T length);
\end{verbatim}
\end{screen}

\begin{itemize}

\item{{\bf 引数}}

\texttt{center}: 入力。\texttt{const Vector3$<$T$>$ \&}

\texttt{length}: 入力。\texttt{const T}型。

\item{{\bf 機能}}

メンバ変数\texttt{low\_}、\texttt{high\_}を、それぞれ、\texttt{center-(Vector3$<$T$>$)(length)}、\texttt{center+(Vector3$<$T$>$)(length)}で初期化する。

\end{itemize}


%%%%%%%%%%%%%%%%%%%%%%%%%%%%%%%%
\subsubsubsection{コピーコンストラクタ}
%---------------
\begin{screen}
\begin{verbatim}
template<typename T>
PS::Orthotope3<T>::Orthotope3(const Orthotope3<T> & src);
\end{verbatim}
\end{screen}

\begin{itemize}

\item{{\bf 引数}}

\texttt{src}: 入力。\texttt{const Orthotope3$<$T$>$ \&}型。

\item{{\bf 機能}}

メンバ変数\texttt{low\_}、\texttt{high\_}を、それぞれ、\texttt{src.low\_}、\texttt{src.high\_}で初期化する。

\end{itemize}


%%%%%%%%%%%%%%%%%%%%%%%%%%%%%
\subsubsubsection{初期化}
%---------------
\begin{screen}
\begin{verbatim}
template<typename T>
PS::Orthotope3<T>::initNegativeVolume();
\end{verbatim}
\end{screen}

\begin{itemize}

\item{{\bf 引数}}

なし。

\item{{\bf 機能}}

メンバ変数\texttt{low\_}、\texttt{high\_}を、それぞれ、($a$, $a$, $a$)、(-$a$, -$a$, -$a$)で初期化する。
ここで、$a$=\texttt{std::numeric\_limits$<$float$>$::max() / 128}である。

\end{itemize}

%---------------
\begin{screen}
\begin{verbatim}
template<typename T>
PS::Orthotope3<T>::init();
\end{verbatim}
\end{screen}

\begin{itemize}

\item{{\bf 引数}}

なし。

\item{{\bf 機能}}

メンバ関数 \texttt{initNegativeVolume} と同じ機能を提供する。

\end{itemize}

%%%%%%%%%%%%%%%%%%%%%%%%%%%%%
\subsubsubsection{結合操作}
%---------------
\begin{screen}
\begin{verbatim}
template<typename T>
void PS::Orthotope3<T>::merge( const Orthotope3 & ort )();
\end{verbatim}
\end{screen}

\begin{itemize}

\item{{\bf 引数}}

\texttt{ort}: 入力。\texttt{const Orthotope3 \&}型。

\item{{\bf 機能}}

メンバ変数\texttt{low\_}、\texttt{high\_}を、当該長方形と長方形\texttt{ort}を包含する最小の長方形を記述するように更新する。

\end{itemize}
%---------------
\begin{screen}
\begin{verbatim}
template<typename T>
void PS::Orthotope3<T>::merge( const Vector3<T> & vec )();
\end{verbatim}
\end{screen}

\begin{itemize}

\item{{\bf 引数}}

\texttt{vec}: 入力。\texttt{const Vector3$<$T$>$ \&}型。

\item{{\bf 機能}}

メンバ変数\texttt{low\_}、\texttt{high\_}を、点\texttt{vec}を包含するように更新する。

\end{itemize}
%---------------
\begin{screen}
\begin{verbatim}
template<typename T>
void PS::Orthotope3<T>::merge( const Vector3<T> & vec, const T size )();
\end{verbatim}
\end{screen}

\begin{itemize}

\item{{\bf 引数}}

\texttt{vec}: 入力。\texttt{const Vector3$<$T$>$ \&}型。

\texttt{size}: 入力。\texttt{const T}型。

\item{{\bf 機能}}

メンバ変数\texttt{low\_}、\texttt{high\_}を、中心\texttt{vec}、半径\texttt{size}の球を包含するように更新する。

\end{itemize}



\subsubsection{Wrappers}

The wrappers of orthotope types are defined as follows.

\begin{lstlisting}[caption=Orthotope wrapper]
namespace ParticleSimulator{
    typedef Orthotope2<F32> F32ort2;
    typedef Orthotope3<F32> F32ort3;
    typedef Orthotope2<F64> F64ort2;
    typedef Orthotope3<F64> F64ort3;
#ifdef PARTICLE_SIMULATOR_TWO_DIMENSION
    typedef F32ort2 F32ort;
    typedef F64ort2 F64ort;
#else
    typedef F32ort3 F32ort;
    typedef F64ort3 F64ort;
#endif
}
\end{lstlisting}

PS::F32ort2, PS::F32ort3, PS::F64ort2, and PS::F64ort3 are,
respectively, 2D orthotope in single precision, 3D orthotope in single
presicion, 2D orthotope in double precision, and 3D orthotope in double
precision. If users set 2D (3D) coordinate system, PS::F32ort and
PS::F64ort are wrappers of PS::F32ort2 and PS::F64ort2 (PS::F32ort3 and
PS::F64ort3).






\subsection{Symmetric matrix type}

\subsubsection{概要}

対称行列型には2x2対称行列型PS::MatrixSym2と3x3対称行列型PS::MatrixSym3
がある。まずこれら2つを記述する。最後にこれら対称行列型のラッパーにつ
いて記述する。

\subsubsection{PS::MatrixSym2}

\texttt{PS::MatrixSym2} has three components: \texttt{xx}, \texttt{yy}, and \texttt{xy}.
We define various APIs and operators for these components.
In the following, we list them.
%%PS::MatrixSym2はxx, yy, xyの3要素を持つ。これらに対する様々なAPIや演
%%算子を定義した。それらの宣言を以下に記述する。この節ではこれらについ
%%て詳しく記述する。
\begin{lstlisting}[caption=MatrixSym2]
namespace ParticleSimulator{
    template<class T>
    class MatrixSym2{
    public:
        // Three member variables
        T xx, yy, xy;

        // Constructors
        MatrixSym2() : xx(T(0)), yy(T(0)), xy(T(0)) {}
        MatrixSym2(const T _xx, const T _yy, const T _xy)
            : xx(_xx), yy(_yy), xy(_xy) {}
        MatrixSym2(const T s) : xx(s), yy(s), xy(s){}
        MatrixSym2(const MatrixSym2 & src) : xx(src.xx), yy(src.yy), xy(src.xy) {}

        // Assignment operator
        const MatrixSym2 & operator = (const MatrixSym2 & rhs);

        // Addition and subtraction
        MatrixSym2 operator + (const MatrixSym2 & rhs) const;
        const MatrixSym2 & operator += (const MatrixSym2 & rhs) const;
        MatrixSym2 operator - (const MatrixSym2 & rhs) const;
        const MatrixSym2 & operator -= (const MatrixSym2 & rhs) const;

        // Trace
        T getTrace() const;

        // Typecast to MatrixSym2<U>
        template <typename U>
        operator MatrixSym2<U> () const;
    }
}
namespace PS = ParticleSimulator;
\end{lstlisting}

%%%%%%%%%%%%%%%%%%%%%%%%%%%%%%%%%%%%%%%%%%%%%%%%%%%%%
\subsubsubsection{Constructor}

\begin{screen}
\begin{verbatim}
template<typename T>
PS::MatrixSym2<T>::MatrixSym2();
\end{verbatim}
\end{screen}

\begin{itemize}

\item{{\bf Argument}}

  None.

\item{{\bf Feature}}

Default constructor. The member variables \texttt{xx}, \texttt{yy} and \texttt{xy} are initialized as 0.
%デフォルトコンストラクタ。メンバxx,yy,xyは0で初期化される。

\end{itemize}

\begin{screen}
\begin{verbatim}
template<typename T>
PS::MatrixSym2<T>::MatrixSym2
            (const T _xx,
             const T _yy,
             const T _xy);
\end{verbatim}
\end{screen}

\begin{itemize}

\item{{\bf Argument}}

\texttt{\_xx}: Input. Type \texttt{const T}.
%{\_xx}: 入力。{const T}型。

\texttt{\_yy}: Input. Type \texttt{const T}.
%{\_yy}: 入力。{const T}型。

\texttt{\_xy}: Input. Type \texttt{const T}.
%{\_xy}: 入力。{const T}型。

\item{{\bf Feature}}

Values of members \texttt{xx}, \texttt{yy} and \texttt{xy} are set to values of arguments \texttt{\_xx}, \texttt{\_yy} and \texttt{\_xy}, respectively.
%%メンバ{xx}、{yy}、{xy}をそれぞれ{\_xx}、{\_yy}、{\_xy}で初期化する。

\end{itemize}

\begin{screen}
\begin{verbatim}
template<typename T>
PS::MatrixSym2<T>::MatrixSym2(const T s);
\end{verbatim}
\end{screen}

\begin{itemize}

\item{{\bf Argument}}

\texttt{s}: Input. Type \texttt{const T}.
%{s}: 入力。{const T}型。

\item{{\bf Feature}}

Values of members \texttt{xx}, \texttt{yy} and \texttt{xy} are set to the value of argument \texttt{s}.
%メンバ{xx}、{yy}、{xy}すべてを{s}の値で初期化する。

\end{itemize}

%%%%%%%%%%%%%%%%%%%%%%%%%%%%%%%%%%%%%%%%%%%%%%%%%%%%%
\subsubsubsection{Copy constructor}

\begin{screen}
\begin{verbatim}
template<typename T>
PS::MatrixSym2<T>::MatrixSym2(const PS::MatrixSym2<T> & src)
\end{verbatim}
\end{screen}

\begin{itemize}

\item{{\bf Argument}}

\texttt{src}: Input. Type \texttt{const PS::MatrixSym2$<$T$>$ \&}.
%{src}: 入力。{const PS::MatrixSym2$<$T$>$ \&}型。

\item{{\bf Feature}}

Copy constructor. The new variable will have the same value as \texttt{src}.
%コピーコンストラクタ。{src}で初期化する。

\end{itemize}

%%%%%%%%%%%%%%%%%%%%%%%%%%%%%%%%%%%%%%%%%%%%%%%%%%%%%
\subsubsubsection{Assignment operator}

\begin{screen}
\begin{verbatim}
template<typename T>
const PS::MatrixSym2<T> & PS::MatrixSym2<T>::operator = 
                       (const PS::MatrixSym2<T> & rhs);
\end{verbatim}
\end{screen}

\begin{itemize}

\item{{\bf Argument}}

\texttt{rhs}: Input. Type \texttt{const PS::MatrixSym2$<$T$>$ \&}.
%{rhs}: 入力。{const PS::MatrixSym2$<$T$>$ \&}型。

\item{{\bf Return value}}

Type \texttt{const PS::MatrixSym2$<$T$>$ \&}. Assigns the values of components of \texttt{rhs}
to its components, and returns the vector itself.
%%{const PS::MatrixSym2$<$T$>$ \&}型。{rhs}のxx,yy,xyの値を自身のメンバ
%%xx,yy,xyに代入し自身の参照を返す。代入演算子。

\end{itemize}

%%%%%%%%%%%%%%%%%%%%%%%%%%%%%%%%%%%%%%%%%%%%%%%%%%%%%
\subsubsubsection{Addition and subtraction}

\begin{screen}
\begin{verbatim}
template<typename T>
PS::MatrixSym2<T> PS::MatrixSym2<T>::operator + 
               (const PS::MatrixSym2<T> & rhs) const;
\end{verbatim}
\end{screen}

\begin{itemize}

\item{{\bf Argument}}

\texttt{rhs}: Input. Type \texttt{const PS::MatrixSym2$<$T$>$ \&}.
%{rhs}: 入力。{const PS::MatrixSym2$<$T$>$ \&}型。

\item{{\bf Return value}}

Type \texttt{PS::MatrixSym2$<$T$>$}. Add the components of \texttt{rhs} and its
components, and return the results.

%%{PS::MatrixSym2$<$T$>$}型。{rhs}のxx,yy,xyの値と自身のメンバxx,yy,xy
%%の値の和を取った値を返す。

\end{itemize}

\begin{screen}
\begin{verbatim}
template<typename T>
const PS::MatrixSym2<T> & PS::MatrixSym2<T>::operator += 
                       (const PS::MatrixSym2<T> & rhs);
\end{verbatim}
\end{screen}

\begin{itemize}

\item{{\bf Argument}}

\texttt{rhs}: Input. Type \texttt{const PS::MatrixSym2$<$T$>$ \&}.
%{rhs}: 入力。{const PS::MatrixSym2$<$T$>$ \&}型。

\item{{\bf Return value}}

Type \texttt{const PS::MatrixSym2$<$T$>$ \&}. Add the components of \texttt{rhs} to its
components, and return itself (lhs is changed).
%%{const PS::MatrixSym2$<$T$>$ \&}型。{rhs}のxx,yy,xyの値を自身のメンバ
%%xx,yy,xyに足し、自身を返す。

\end{itemize}

\begin{screen}
\begin{verbatim}
template<typename T>
PS::MatrixSym2<T> PS::MatrixSym2<T>::operator - 
               (const PS::MatrixSym2<T> & rhs) const;
\end{verbatim}
\end{screen}

\begin{itemize}

\item{{\bf Argument}}

\texttt{rhs}: Input. Type \texttt{const PS::MatrixSym2$<$T$>$ \&}.
%{rhs}: 入力。{const PS::MatrixSym2$<$T$>$ \&}型。

\item{{\bf Return value}}

Type \texttt{PS::MatrixSym2$<$T$>$}. Subtract the components of \texttt{rhs} from its
components, and return the results.

%%{PS::MatrixSym2$<$T$>$}型。{rhs}のxx,yy,xyの値と自身のメンバxx,yy,xy
%%の値の差を取った値を返す。

\end{itemize}

\begin{screen}
\begin{verbatim}
template<typename T>
const PS::MatrixSym2<T> & PS::MatrixSym2<T>::operator -= 
                       (const PS::MatrixSym2<T> & rhs);
\end{verbatim}
\end{screen}

\begin{itemize}

\item{{\bf Argument}}

\texttt{rhs}: Input. Type \texttt{const PS::MatrixSym2$<$T$>$ \&}.
%{rhs}: 入力。{const PS::MatrixSym2$<$T$>$ \&}型。

\item{{\bf Return value}}

Type \texttt{const PS::MatrixSym2$<$T$>$ \&}. Subtract the components of \texttt{rhs} from
its components, and return itself (lhs is changed).
%%{const PS::MatrixSym2$<$T$>$ \&}型。自身のメンバxx,yy,xyから{rhs}の
%%xx,yy,xyを引き自身を返す。

\end{itemize}

%%%%%%%%%%%%%%%%%%%%%%%%%%%%%%%%%%%%%%%%%%%%%%%%%%%%%
\subsubsubsection{Trace}

\begin{screen}
\begin{verbatim}
template<typename T>
T PS::MatrixSym2<T>::getTrace() const;
\end{verbatim}
\end{screen}

\begin{itemize}

\item{{\bf Argument}}

  None.

\item{{\bf Return value}}

  Type \texttt{T}.

\item{{\bf Feature}}

  Calculate the trace, and return the result.

\end{itemize}

%%%%%%%%%%%%%%%%%%%%%%%%%%%%%%%%%%%%%%%%%%%%%%%%%%%%%
\subsubsubsection{Typecast to MatrixSym2$<$U$>$}

\begin{screen}
\begin{verbatim}
template<typename T>
template<typename U>
PS::MatrixSym2<T>::operator PS::MatrixSym2<U> () const;
\end{verbatim}
\end{screen}

\begin{itemize}

\item{{\bf Argument}}

  None.

\item{{\bf Return value}}

  Type \texttt{const PS::MatrixSym2$<$U$>$}.

\item{{\bf Feature}}

  Typecast from type \texttt{const PS::MatrixSym2$<$T$>$} to \texttt{const
  PS::MatrixSym2$<$U$>$}.
%%  {const PS::MatrixSym2$<$T$>$}型を{const PS::MatrixSym2$<$U$>$}型に
%%  キャ ストする

\end{itemize}



\subsubsection{PS::MatrixSym3}

\texttt{PS::MatrixSym3} has six components: \texttt{xx}, \texttt{yy}, \texttt{zz}, \texttt{xy}, \texttt{xz}, and \texttt{yz}.
We define various APIs and operators for these components.
In the following, we list them.
%%PS::MatrixSym3はxx, yy, zz, xy, xz, yzの6要素を持つ。これらに対する様々
%%なAPIや演算子を定義した。それらの宣言を以下に記述する。この節ではこれ
%%らについて詳しく記述する。
\begin{lstlisting}[caption=MatrixSym3]
namespace ParticleSimulator{
    template<class T>
    class MatrixSym3{
    public:
        // Six member variables
        T xx, yy, zz, xy, xz, yz;

        // Constructors
        MatrixSym3() : xx(T(0)), yy(T(0)), zz(T(0)),
                       xy(T(0)), xz(T(0)), yz(T(0)) {}
        MatrixSym3(const T _xx, const T _yy, const T _zz,
                   const T _xy, const T _xz, const T _yz )
                       : xx(_xx), yy(_yy), zz(_zz),
                       xy(_xy), xz(_xz), yz(_yz) {}
        MatrixSym3(const T s) : xx(s), yy(s), zz(s),
                                xy(s), xz(s), yz(s) {}
        MatrixSym3(const MatrixSym3 & src) :
            xx(src.xx), yy(src.yy), zz(src.zz),
            xy(src.xy), xz(src.xz), yz(src.yz) {}

        // Assignment operator
        const MatrixSym3 & operator = (const MatrixSym3 & rhs);

        // Addition and subtraction
        MatrixSym3 operator + (const MatrixSym3 & rhs) const;
        const MatrixSym3 & operator += (const MatrixSym3 & rhs) const;
        MatrixSym3 operator - (const MatrixSym3 & rhs) const;
        const MatrixSym3 & operator -= (const MatrixSym3 & rhs) const;

        // Trace
        T getTrace() const;

        // Typecast to MatrixSym3<U>
        template <typename U>
        operator MatrixSym3<U> () const;
    }
}
namespace PS = ParticleSimulator;
\end{lstlisting}

%%%%%%%%%%%%%%%%%%%%%%%%%%%%%%%%%%%%%%%%%%%%%%%%%%%%%
\subsubsubsection{Constructor}

\begin{screen}
\begin{verbatim}
template<typename T>
PS::MatrixSym3<T>::MatrixSym3();
\end{verbatim}
\end{screen}

\begin{itemize}

\item{{\bf Argument}}

  None.

\item{{\bf Feature}}

Default constructor. All the member variables are initialized as 0.
%デフォルトコンストラクタ。6要素は0で初期化される。

\end{itemize}

\begin{screen}
\begin{verbatim}
template<typename T>
PS::MatrixSym3<T>::MatrixSym3(const T _xx,
                              const T _yy,
                              const T _zz,
                              const T _xy,
                              const T _xz,
                              const T _yz);
\end{verbatim}
\end{screen}

\begin{itemize}

\item{{\bf Argument}}

\texttt{\_xx}: Input. Type \texttt{const T}.
%%{\_xx}: 入力。{const T}型。

\texttt{\_yy}: Input. Type \texttt{const T}.
%{\_yy}: 入力。{const T}型。

\texttt{\_zz}: Input. Type \texttt{const T}.
%{\_zz}: 入力。{const T}型。

\texttt{\_xy}: Input. Type \texttt{const T}.
%{\_xy}: 入力。{const T}型。

\texttt{\_xz}: Input. Type \texttt{const T}.
%{\_xz}: 入力。{const T}型。

\texttt{\_yz}: Input. Type \texttt{const T}.
%{\_yz}: 入力。{const T}型。

\item{{\bf Feature}}

Values of members \texttt{xx}, \texttt{yy}, \texttt{zz}, \texttt{xy},
\texttt{xz} and \texttt{yz} are set to values of arguments \texttt{\_xx},
\texttt{\_yy}, \texttt{\_zz}, \texttt{\_xy}, \texttt{xz} and \texttt{\_yz},
respectively.
%%メンバ{xx}、{yy}、{zz}、{xy}、{xz}、{yz}をそれぞれ{\_xx}、{\_yy}、
%%{\_zz}、{\_xy}、{\_xz}、{\_yz}で初期化する。

\end{itemize}

\begin{screen}
\begin{verbatim}
template<typename T>
PS::MatrixSym3<T>::MatrixSym3(const T s);
\end{verbatim}
\end{screen}

\begin{itemize}

\item{{\bf Argument}}

\texttt{s}: Input. Type \texttt{const T}.
%{s}: 入力。{const T}型。

\item{{\bf Feature}}

Values of members are set to the value of argument \texttt{s}.
%6要素すべてを{s}の値で初期化する。

\end{itemize}

%%%%%%%%%%%%%%%%%%%%%%%%%%%%%%%%%%%%%%%%%%%%%%%%%%%%%
\subsubsubsection{Copy constructor}

\begin{screen}
\begin{verbatim}
template<typename T>
PS::MatrixSym3<T>::MatrixSym3(const PS::MatrixSym3<T> & src)
\end{verbatim}
\end{screen}

\begin{itemize}

\item{{\bf Argument}}

\texttt{src}: Input. Type const \texttt{PS::MatrixSym3$<$T$>$ \&}.
%{src}: 入力。{const PS::MatrixSym3$<$T$>$ \&}型。

\item{{\bf Feature}}

Copy constructor. The new variable will have the same value as \texttt{src}.
%コピーコンストラクタ。{src}で初期化する。

\end{itemize}

%%%%%%%%%%%%%%%%%%%%%%%%%%%%%%%%%%%%%%%%%%%%%%%%%%%%%
\subsubsubsection{Assignment operator}

\begin{screen}
\begin{verbatim}
template<typename T>
const PS::MatrixSym3<T> & PS::MatrixSym3<T>::operator = 
                       (const PS::MatrixSym3<T> & rhs);
\end{verbatim}
\end{screen}

\begin{itemize}

\item{{\bf Argument}}

\texttt{rhs}: Input. Type \texttt{const PS::MatrixSym3$<$T$>$ \&}.
%{rhs}: 入力。{const PS::MatrixSym3$<$T$>$ \&}型。

\item{{\bf Return value}}

Type \texttt{const PS::MatrixSym3$<$T$>$ \&}. Assigns the values of components of \texttt{rhs}
to its components, and returns the vector itself.
%%{const PS::MatrixSym3$<$T$>$ \&}型。{rhs}の6要素それぞれの値を自身の
%%6要素それぞれに代入し自身の参照を返す。代入演算子。

\end{itemize}

%%%%%%%%%%%%%%%%%%%%%%%%%%%%%%%%%%%%%%%%%%%%%%%%%%%%%
\subsubsubsection{Addition and subtraction}

\begin{screen}
\begin{verbatim}
template<typename T>
PS::MatrixSym3<T> PS::MatrixSym3<T>::operator + 
               (const PS::MatrixSym3<T> & rhs) const;
\end{verbatim}
\end{screen}

\begin{itemize}

\item{{\bf Argument}}

\texttt{rhs}: Input. Type \texttt{const PS::MatrixSym3$<$T$>$ \&}.
%{rhs}: 入力。{const PS::MatrixSym3$<$T$>$ \&}型。

\item{{\bf Return value}}

Type \texttt{PS::MatrixSym3$<$T$>$}. Add the components of \texttt{rhs} and its
components, and return the results.
%%{PS::MatrixSym3$<$T$>$ }型。{rhs}の6要素それぞれの値と自身の6要素の
%%値の和を取った値を返す。

\end{itemize}

\begin{screen}
\begin{verbatim}
template<typename T>
const PS::MatrixSym3<T> & PS::MatrixSym3<T>::operator += 
                       (const PS::MatrixSym3<T> & rhs);
\end{verbatim}
\end{screen}

\begin{itemize}

\item{{\bf Argument}}

\texttt{rhs}: Input. Type \texttt{const PS::MatrixSym3$<$T$>$ \&}.
%{rhs}: 入力。{const PS::MatrixSym3$<$T$>$ \&}型。

\item{{\bf Return value}}

Type \texttt{const PS::MatrixSym3$<$T$>$ \&}. Add the components of \texttt{rhs} to its
components, and return itself (lhs is changed).
%%{const PS::MatrixSym3$<$T$>$ \&}型。{rhs}の6要素それぞれの値を自身の
%%6要素それぞれに足し、自身を返す。

\end{itemize}

\begin{screen}
\begin{verbatim}
template<typename T>
PS::MatrixSym3<T> PS::MatrixSym3<T>::operator - 
               (const PS::MatrixSym3<T> & rhs) const;
\end{verbatim}
\end{screen}

\begin{itemize}

\item{{\bf Argument}}

\texttt{rhs}: Input. Type \texttt{const PS::MatrixSym3$<$T$>$ \&}.
%{rhs}: 入力。{const PS::MatrixSym3$<$T$>$ \&}型。

\item{{\bf Return value}}

Type \texttt{PS::MatrixSym3$<$T$>$}. Subtract the components of \texttt{rhs} from its
components, and return the results.
%%{PS::MatrixSym3$<$T$>$}型。{rhs}の6要素それぞれの値と自身の6要素そ
%%れぞれの値の差を取った値を返す。

\end{itemize}

\begin{screen}
\begin{verbatim}
template<typename T>
const PS::MatrixSym3<T> & PS::MatrixSym3<T>::operator -= 
                       (const PS::MatrixSym3<T> & rhs);
\end{verbatim}
\end{screen}

\begin{itemize}

\item{{\bf Argument}}

\texttt{rhs}: Input. Type \texttt{const PS::MatrixSym3$<$T$>$ \&}.
%{rhs}: 入力。{const PS::MatrixSym3$<$T$>$ \&}型。

\item{{\bf Return value}}

Type \texttt{const PS::MatrixSym3$<$T$>$ \&}. Subtract the components of \texttt{rhs} from
its components, and return itself (lhs is changed).
%%{const PS::MatrixSym3$<$T$>$ \&}型。自身の6要素それぞれから{rhs}の6
%%要素それぞれを引き自身を返す。

\end{itemize}

%%%%%%%%%%%%%%%%%%%%%%%%%%%%%%%%%%%%%%%%%%%%%%%%%%%%%
\subsubsubsection{Trace}

\begin{screen}
\begin{verbatim}
template<typename T>
T PS::MatrixSym3<T>::getTrace() const;
\end{verbatim}
\end{screen}

\begin{itemize}

\item{{\bf Argument}}

  None.

\item{{\bf Return value}}

  Type \texttt{T}.

\item{{\bf Feature}}

  Calculate the trace, and return the result.
%  トレースを計算し、その結果を返す。

\end{itemize}

%%%%%%%%%%%%%%%%%%%%%%%%%%%%%%%%%%%%%%%%%%%%%%%%%%%%%
\subsubsubsection{Typecast to MatrixSym3$<$U$>$}

\begin{screen}
\begin{verbatim}
template<typename T>
template<typename U>
PS::MatrixSym3<T>::operator PS::MatrixSym3<U> () const;
\end{verbatim}
\end{screen}

\begin{itemize}

\item{{\bf Argument}}

  None.

\item{{\bf Return value}}

  Type \texttt{const PS::MatrixSym3$<$U$>$}.

%{const PS::MatrixSym3$<$U$>$}型。

\item{{\bf Feature}}

  Typecast from type \texttt{const PS::MatrixSym3$<$T$>$} to \texttt{const
  PS::MatrixSym3$<$U$>$}.
%%  {const PS::MatrixSym3$<$T$>$}型を{const PS::MatrixSym3$<$U$>$}型に
%%  キャ ストする

\end{itemize}



\subsubsection{対称行列型のラッパー}

The wrappers of symmetric matrix types are defined as follows.
%対称行列型のラッパーの定義を以下に示す。
\begin{lstlisting}[caption=matrixsymwrapper]
namespace ParticleSimulator{
    typedef MatrixSym2<F32> F32mat2;
    typedef MatrixSym3<F32> F32mat3;
    typedef MatrixSym2<F64> F64mat2;
    typedef MatrixSym3<F64> F64mat3;
#ifdef PARTICLE_SIMULATOR_TWO_DIMENSION
    typedef F32mat2 F32mat;
    typedef F64mat2 F64mat;
#else
    typedef F32mat3 F32mat;
    typedef F64mat3 F64mat;
#endif
}
namespace PS = ParticleSimulator;
\end{lstlisting}

\texttt{PS::F32mat2}, \texttt{PS::F32mat3}, \texttt{PS::F64mat2}, and \texttt{PS::F64mat3} are,
respectively, 2x2 symmetric matrix in single precision, 3x3 symmetric
matrix in single presicion, 2x2 symmetric matrix in double precision,
and 3x3 symmetric matrix in double precision. If users set 2D (3D)
coordinate system, \texttt{PS::F32mat} and \texttt{PS::F64mat} is wrappers of
\texttt{PS::F32mat2} and \texttt{PS::F64mat2} (\texttt{PS::F32mat3} and \texttt{PS::F64mat3}).
%%すなわちPS::F32mat2, PS::F32mat3, PS::F64mat2, PS::F64mat3はそれぞれ
%%単精度2x2対称行列、倍精度2x2対称行列、単精度3x3対称行列、倍精度3x3対
%%称行列である。FDPSで扱う空間座標系を2次元とした場合、PS::F32matと
%%PS::F64matはそれぞれ単精度2x2対称行列、倍精度2x2対称行列となる。一方、
%%FDPSで扱う空間座標系を3次元とした場合、PS::F32matとPS::F64matはそれぞ
%%れ単精度3x3対称行列、倍精度3x3対称行列となる。





\subsection{PS::SEARCH\_MODE type}

\subsubsection{概要}

本節では、PS::SEARCH\_MODE型について記述する。PS::SEARCH\_MODE型は相互
作用ツリークラスのテンプレート引数としてのみ使用されるものである。この
型によって、相互作用ツリークラスで計算する相互作用のモードを決定する。
PS::SEARCH\_MODE型には以下がある:
\begin{itemize}[leftmargin=*,itemsep=-1ex]
\item PS::SEARCH\_MODE\_LONG
\item PS::SEARCH\_MODE\_LONG\_CUTOFF
\item PS::SEARCH\_MODE\_GATHER
\item PS::SEARCH\_MODE\_SCATTER
\item PS::SEARCH\_MODE\_SYMMETRY
\item PS::SEARCH\_MODE\_LONG\_SCATTER
\item PS::SEARCH\_MODE\_LONG\_SYMMETRY
\item PS::SEARCH\_MODE\_LONG\_CUTOFF\_SCATTER
\end{itemize}
以下で、それぞれが対応する相互作用のモードについて記述する。

\subsubsection{PS::SEARCH\_MODE\_LONG}

この型を使用するのは、遠くの粒子からの寄与を複数の粒子にまとめた超粒子
からの寄与として計算する場合である。開放境界条件における重力やクーロン
力に適用できる(周期境界条件では使用不可能)。

\subsubsection{PS::SEARCH\_MODE\_LONG\_CUTOFF}

この型を使用するのは、遠くの粒子からの寄与を複数の粒子にまとめた超粒子
からの寄与として計算し、かつ有限の距離までの寄与しか計算しない場合であ
る。周期境界条件における重力やクーロン力(Particle Mesh法の並用が必要)
などに適用できる。

\subsubsection{PS::SEARCH\_MODE\_GATHER}

この型を使用するのは、相互作用の到達距離が有限でかつ、その到達距離がi
粒子の大きさで決まる場合である。

\subsubsection{PS::SEARCH\_MODE\_SCATTER}

この型を使用するのは、相互作用の到達距離が有限でかつ、その到達距離がj
粒子の大きさで決まる場合である。

\subsubsection{PS::SEARCH\_MODE\_SYMMETRY}

この型を使用するのは、相互作用の到達距離が有限でかつ、その到達距離がi,
j粒子のうち大きいほうのサイズで決まる場合である。

\subsubsection{PS::SEARCH\_MODE\_LONG\_SCATTER}

基本的にはSEARCH\_MODE\_LONGと同じであるが、i粒子とj粒子の距離がj粒子の
探査半径よりも短い場合は、そのj粒子は超粒子に含めない(超粒子としてではなく粒子として扱う)。

\subsubsection{PS::SEARCH\_MODE\_LONG\_SYMMETRY}

基本的にはSEARCH\_MODE\_LONGと同じであるが、i粒子とj粒子の距離が、i粒子とj粒子の
探査半径のどちらか大きい方よりも短い場合は、そのj粒子は超粒子に含めない(超粒子としてではなく粒子として扱う)。


\subsubsection{PS::SEARCH\_MODE\_LONG\_CUTOFF\_SCATTER}

未実装。


\subsection{Enumerated type}
\label{sec:datatype_enum}

\subsubsection{Summary}

In this section, we describe enumerated types defined in FDPS.
Currently, there is just one datatype. We describe it below.
%%本節ではFDPSで定義されている列挙型について記述する。列挙型には
%%BOUNDARY\_CONDITION型が存在する。以下、各列挙型について記述する。

\subsubsection{PS::BOUNDARY\_CONDITION type}
\label{sec:datatype_enum_boundarycondition}

\subsubsubsection{Summary}

Type BOUNDARY\_CONDITION specifies boundary conditions. The definition
is as follows.
%%BOUNDARY\_CONDITION型は境界条件を指定するためのデータ型である。これは
%%以下のように定義されている。
\begin{lstlisting}[caption=boundarycondition]
namespace ParticleSimulator{
    enum BOUNDARY_CONDITION{
        BOUNDARY_CONDITION_OPEN,
        BOUNDARY_CONDITION_PERIODIC_X,
        BOUNDARY_CONDITION_PERIODIC_Y,
        BOUNDARY_CONDITION_PERIODIC_Z,
        BOUNDARY_CONDITION_PERIODIC_XY,
        BOUNDARY_CONDITION_PERIODIC_XZ,
        BOUNDARY_CONDITION_PERIODIC_YZ,
        BOUNDARY_CONDITION_PERIODIC_XYZ,
        BOUNDARY_CONDITION_SHEARING_BOX,
        BOUNDARY_CONDITION_USER_DEFINED,
    };
}
\end{lstlisting}

We explain each value below.
%以下にどの変数がどの境界条件に対応するかを記述する。

\subsubsubsection{PS::BOUNDARY\_CONDITION\_OPEN}

This specifies the open boundary condition.
%開放境界となる。

\subsubsubsection{PS::BOUNDARY\_CONDITION\_PERIODIC\_X}

This specifies the periodic boundary condition in the direction of x-axis,
and open boundary condition in other directions. The interval is
left-bounded and right-unbounded. This is true for all periodic
boundary conditions.
%%x軸方向のみ周期境界、その他の軸方向は開放境界となる。周期の境界の下限
%%は閉境界、上限は開境界となっている。この境界の規定はすべての軸方向に
%%あてはまる。

\subsubsubsection{PS::BOUNDARY\_CONDITION\_PERIODIC\_Y}

This specifies the periodic boundary condition in the direction of y-axis,
and open boundary condition in other directions.
%y軸方向のみ周期境界、その他の軸方向は開放境界となる。

\subsubsubsection{PS::BOUNDARY\_CONDITION\_PERIODIC\_Z}

This specifies the periodic boundary condition in the direction of z-axis,
and open boundary condition in other directions.
%z軸方向のみ周期境界、その他の軸方向は開放境界となる。

\subsubsubsection{PS::BOUNDARY\_CONDITION\_PERIODIC\_XY}

This specifies the periodic boundary condition in the directions of x- and
y-axes, and open boundary condition in the direction of z-axis.
%x, y軸方向のみ周期境界、その他の軸方向は開放境界となる。

\subsubsubsection{PS::BOUNDARY\_CONDITION\_PERIODIC\_XZ}

This specifies the periodic boundary condition in the directions of x- and
z-axes, and open boundary condition in the direction of y-axis.
%x, z軸方向のみ周期境界、その他の軸方向は開放境界となる。

\subsubsubsection{PS::BOUNDARY\_CONDITION\_PERIODIC\_YZ}

This specifies the periodic boundary condition in the directions of y-
and z-axes, and open boundary condition in the direction of x-axis.
%y, z軸方向のみ周期境界、その他の軸方向は開放境界となる。

\subsubsubsection{PS::BOUNDARY\_CONDITION\_PERIODIC\_XYZ}

This specifies the periodic boundary condition in all three directions.
%x, y, z軸方向すべてが周期境界となる。

\subsubsubsection{PS::BOUNDARY\_CONDITION\_SHEARING\_BOX}

Not implemented yet.
%未実装。

\subsubsubsection{PS::BOUNDARY\_CONDITION\_USER\_DEFINED}

Not implemented yet.
%未実装。

%%%%%%%%%%%%%%%%%%%%%%%%%%%%%%%%%%%
\subsubsection{PS::INTERACTION\_LIST\_MODE type}
\label{sec:datatype_enum_interaction_list_mode}

%\subsubsubsection{概要}
\subsubsubsection{Summary}


Type INTERACTION\_LIST\_MODE is used to determine if user program
reuse the interaction list or not. This type is defined as follows.


%INTERACTION\_LIST\_MODE型は相互作用リストを再利用するかどうかを決定す
%るためのデータ型である。これは以下のように定義されている。

\begin{lstlisting}[caption=boundarycondition]
namespace ParticleSimulator{
    enum INTERACTION_LIST_MODE{
        MAKE_LIST,
        MAKE_LIST_FOR_REUSE,
        REUSE_LIST,
    };
}
\end{lstlisting}

This data type is used as the last argument of the function
calcForceAllAndWriteBack(). For more detail, plese see
section \ref{sec:treeForForceHighLevelAPI}.

%このデータ型はcalcForceAllAndWriteBack()等の関数の引数として使われる
%(詳しくはセクション\ref{sec:treeForForceHighLevelAPI}を参照)。

\subsubsubsection{PS::MAKE\_LIST}

FDPS (re)makes interaction lists for each interaction calculation
(each call of the APIs described above). In this case, we cannot reuse
interaction list in the next interaction calculation because FDPS does
not store the information of interaction list. \textbf{This is the
default operation mode in FDPS}.


%相互作用リストを毎回作り相互作用計算を行う場合に用いる。相互作用リスト
%の再利用はできない。

\subsubsubsection{PS::MAKE\_LIST\_FOR\_REUSE}

FDPS (re)makes interaction lists and stores them internally. Then, it
performs interaction calculation. In this case, we can reuse these
interaction lists in the next interaction calculation if we call the
APIs with the flag PS::REUSE\_LIST. The interaction lists memorized in
FDPS are destroyed if we perform the interaction calculation with the
flags PS::MAKE\_LIST\_FOR\_REUSE or PS::MAKE\_LIST.\\

%相互作用リストを再利用し相互作用計算を行いたい場合に用いる。このオプショ
%ンを選択する事でFDPSは相互作用リストを作りそれを保持する。作成した相互
%作用リストはPS::MAKE\_LIST\_FOR\_REUSEもしくはPS::MAKE\_LISTを用いて相
%互作用計算を行った際に破棄される。

\subsubsubsection{PS::REUSE\_INTERACTION\_LIST}

FDPS performs interaction calculation using the previously-created
interaction lists, which are the lists that are created at the
previous call of the APIs with the flag PS::MAKE\_LIST\_FOR\_REUSE. In
this case, moment information in superparticles are automatically
updated using the latest particle information.\\

%相互作用リストを再利用し相互作用計算を行う。再利用される相互作用リスト
%はPS::MAKE\_LIST\_FOR\_REUSEを選択時に作成した相互作用リストである。


\subsection{PS::TimeProfile type}
\label{sec:datatype_timeprofile}

\subsubsection{Abstract}

In this section, we describe data type PS::TimeProfile. This data type
is class to store calculation time for each function, and is used for
three classes: DomainInfo, ParticleSystem, and TreeForForce. These
three classes have ``PS::TimeProfile getTimeProfile()''. Users get the
calculation time of each function by using the function
``getTimeProfile()''.
%%本節では、PS::TimeProfile型について記述する。PS::TimeProfile型はFDPS
%%で使われる3つのクラス、領域分割クラス、粒子群クラス、相互作用ツリー
%%クラス、各メソッドの計算時間を格納するクラスである。これら3つのクラ
%%スには{\tt PS::TimeProfile getTimeProfile()}というメソッドが存在し、
%%このメソッドをつかって、ユーザーは各メソッドの計算時間を取得出来る。

This class is described as follows.
%このクラスは以下のように記述されている。

\begin{lstlisting}[caption=TimeProfile]
namespace ParticleSimulator{
    class TimeProfile{
    publid:
        F64 collect_sample_particle;
        F64 decompose_domain;
        F64 exchange_particle;
        F64 make_local_tree;
        F64 make_global_tree;
        F64 calc_force;
        F64 calc_moment_local_tree;
        F64 calc_moment_global_tree;
        F64 make_LET_1st;
        F64 make_LET_2nd;        
        F64 exchange_LET_1st;
        F64 exchange_LET_2nd;
    };
}    
\end{lstlisting}

%%%%%%%%%%%%%%%%%%%%%%%%%%%%%
\subsubsubsection{Addition}
\mbox{}
%%%%%%%%%%%%%%%%%%%%%%%%%%%%%

%%%%%%%%%%%%%%%%%%%%%%%%%%%%%
\begin{screen}
\begin{verbatim}
PS::TimeProfile PS::TimeProfile::operator + 
               (const PS::TimeProfile & rhs) const;
\end{verbatim}
\end{screen}

\begin{itemize}

\item{{\bf Argument}}

{rhs}: Input. Type const TimeProfile \&.
%{rhs}: 入力。{const TimeProfile \&}型。

\item{{\bf Return value}}

Type PS::TimeProfile. Add the components of rhs to its own components,
and return the results.
%%{PS::TimeProfile}型。{rhs}のすべてのメンバ変数の値と自身のメンバ変数
%%の値の和を取った値を返す。

\end{itemize}

%%%%%%%%%%%%%%%%%%%%%%%%%%%%%
\subsubsubsection{Reduction}
\mbox{}
%%%%%%%%%%%%%%%%%%%%%%%%%%%%%

%%%%%%%%%%%%%%%%%%%%%%%%%%%%%
\begin{screen}
\begin{verbatim}
PS::F64 PS::TimeProfile::getTotalTime() const;
\end{verbatim}
\end{screen}

\begin{itemize}

\item{{\bf Argument}}

  None.

\item{{\bf Return value}}

  Type PS::F64. Return values of all the member variables.
%{PS::F64}型。すべてのメンバ変数の値の和を返す。

\end{itemize}

%%%%%%%%%%%%%%%%%%%%%%%%%%%%%
\subsubsubsection{Initialize}
\mbox{}
%%%%%%%%%%%%%%%%%%%%%%%%%%%%%

%%%%%%%%%%%%%%%%%%%%%%%%%%%%%
\begin{screen}
\begin{verbatim}
void PS::TimeProfile::clear();
\end{verbatim}
\end{screen}

\begin{itemize}

\item{{\bf Argument}}

 None.

\item{{\bf Return value}}

 None.

\item{{\bf Feature}}

 Assign 0 to all the member variables.
%すべてのメンバ変数に0を代入する。

\end{itemize}


%%\subsection{MPIデータ型}

%%\subsubsection{概要}

本節ではFDPSで定義されているMPIデータ型について記述する。

\subsubsection{PS::GetDataType$<$S32$>$()}

PS::S32に対応するMPIデータ型である。

\subsubsection{PS::GetDataType$<$S64$>$()}

PS::S64に対応するMPIデータ型である。

\subsubsection{PS::GetDataType$<$U32$>$()}

PS::U32に対応するMPIデータ型である。

\subsubsection{PS::GetDataType$<$U64$>$()}

PS::U64に対応するMPIデータ型である。

\subsubsection{PS::GetDataType$<$F32$>$()}

PS::F32に対応するMPIデータ型である。

\subsubsection{PS::GetDataType$<$F64$>$()}

PS::F64に対応するMPIデータ型である。

\subsubsection{PS::MPI\_F32VEC}

PS::F32vecに対応するMPIデータ型である。PS::F32vecは、2次元直交座標系を
扱っている場合には2次元ベクトル、3次元直交座標系を扱っている場合には3
次元ベクトルである。

\subsubsection{PS::MPI\_F64VEC}

PS::F64vecに対応するMPIデータ型である。PS::F64vecは、2次元直交座標系を
扱っている場合には2次元ベクトル、3次元直交座標系を扱っている場合には3
次元ベクトルである。


