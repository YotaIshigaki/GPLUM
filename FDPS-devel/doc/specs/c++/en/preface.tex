This document is the specification of FDPS (Framework for Developing
Particle Simulator), which supports the development of massively
parallel particle simulation codes. This document is written by
Ataru Tanikawa, Masaki Iwasawa, Natsuki Hosono, Keigo Nitadori,
Takayuki Muranushi, and Junichiro Makino at RIKEN Advanced Institute
for Computational Science.
%%この文書は大規模並列粒子シミュレーションの開発を支援するFramework
%%forDeveloping Particle Simulator (FDPS)の仕様書である。この文書は理化
%%学研究所計算科学研究機構粒子系シミュレータ研究チームの谷川衝、岩澤全
%%規、細野七月、似鳥啓吾、村主崇行、牧野淳一郎によって記述された。

This document is structured as follows.
%%この文書は以下のような構成となっている。

In sections~\ref{sec:overview}, \ref{sec:configuration},
and \ref{sec:compile}, we present prerequisites for programing with
FDPS. In section~\ref{sec:overview}, we show the concept of FDPS. In
section~\ref{sec:configuration}, we present the file configuration of
FDPS. In section~\ref{sec:compile}, we describe how to compile
simulation codes with FDPS.
%%節\ref{sec:overview}、\ref{sec:configuration}、\ref{sec:compile}には、
%%FDPSを使ってプログラムを書く際に前提となる情報が記述されている。
%%節\ref{sec:overview}には、FDPSの概要として、FDPSの基本的な考えかたや
%%動作が記述されている。節\ref{sec:configuration}には、FDPSのファイル構
%%成が記述されている。節\ref{sec:compile}には、FDPSのAPIを使用したコー
%%ドをコンパイルする時にどのようなマクロを用いればよいかが記述されてい
%%る。

In sections~\ref{sec:namespace}, \ref{sec:datatype}, \ref{sec:userdefined},
\ref{sec:initfin}, and \ref{sec:module}, we present necessary information
to develop simulation codes with FDPS. In section~\ref{sec:namespace},
we describe the structure of namespaces unused in FDPS. In
section~\ref{sec:datatype}, we described data types defined in
FDPS. In section~\ref{sec:userdefined}, we introduce user-defined
classes and function objects necessary for developing codes with
FDPS. In section~\ref{sec:initfin}, we describe APIs used to
initialize and finalize FDPS. In section~\ref{sec:module}, we present
modules in FDPS and their APIs.
%%節\ref{sec:namespace}、\ref{sec:datatype}、\ref{sec:userdefined}、
%%\ref{sec:initfin}、\ref{sec:module}には、FDPSを使ってプログラムを書く
%%際に必要となる情報が提供されている。節\ref{sec:namespace}には、FDPS内
%%での名前空間の構造についてが記述されている。節\ref{sec:datatype}には、
%%FDPSで独自に定義されているデータ型が記述されている。
%%節\ref{sec:userdefined}には、FDPSのAPIを使用する際にユーザーが定義す
%%る必要があるクラスや関数オブジェクトについて記述されている。
%%節\ref{sec:initfin}には、FDPSを開始するときと終了するときに呼ぶ必要の
%%あるAPIについて記述されている。節\ref{sec:module}には、FDPSにあるモ
%%ジュールとそのAPIについて記述されている。

In
sections~\ref{sec:errormessage}, \ref{sec:knownbug}, \ref{sec:limitation},
and \ref{sec:usersupport}, we present information useful for
troubleshooting.  In section~\ref{sec:errormessage}, we describe error
messages. In section~\ref{sec:knownbug}, we present known bugs. In
section~\ref{sec:limitation}, we describe the limitation of FDPS. In
section~\ref{sec:usersupport}, we present our current system for user
supports.
%%節\ref{sec:errormessage}、\ref{sec:knownbug}、\ref{sec:limitation}、
%%\ref{sec:usersupport}には、FDPSのAPIを使用したコードを記述したがコー
%%ドが思ったように動作しない場合に有用な情報が記載されている。
%%節\ref{sec:errormessage}にはエラーメッセージについてが記述されている。
%%節\ref{sec:knownbug}には、よく知られているバグについて記述されている。
%%節\ref{sec:limitation}には、FDPSの限界について記述されている。
%%節\ref{sec:usersupport}には、ユーザーサポートに関する情報が記述されて
%%いる。

Finally, we describe the license of FDPS in section~\ref{sec:license},
and the change log of this document in section~\ref{sec:changelog}.
%%最後に節\ref{sec:license}にはFDPSのライセンスに関する情報が、
%%節\ref{sec:changelog}にはこの文書の変更履歴が記述されている。
