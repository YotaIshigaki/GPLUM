\subsubsection{Summary}

%Forceクラスは相互作用の結果を保持するクラスであり、相互作用の定義
%(節\ref{sec:overview_action}の手順0)に必要となる。以下、この節の前提、
%常に必要なメンバ関数について記述する。
The \texttt{Force} class contains the results of the calculation of interactions (see, step 0 in Sec. \ref{sec:overview_action}).
In this section we describe the member functions in any case required.

\subsubsection{Premise}

%この節で用いる例としてForceクラスのクラス名をResultとする。このクラス名
%は変更自由である。
Let us take \texttt{Result} class as an example of \texttt{Force} as below.
Users can use an arbitrary name instead of \texttt{Result}.

\subsubsection{Required member functions}

%常に必要なメンバ関数はResult::clearである。この関数は相互作用の計算結果を初期
%化する。以下、Result::clearについて記述する。
The member function \texttt{Result::clear} is in any case needed.
This function initialize the result on interaction calculations.

\subsubsubsection{Result::clear}

\begin{screen}
\begin{verbatim}
class Result {
public:
    void clear();
};
\end{verbatim}
\end{screen}

\begin{itemize}

  
\item {\bf Arguments}

  None.
  
\item {\bf Returns}

  None.
  
\item {\bf Behaviour}

  %Resultクラスのメンバ変数を初期化する。
  Initializes the result of interaction calculations.
  
\end{itemize}

