\subsubsection{Abstract}

In this section, we describe data type PS::TimeProfile. This data type
is class to store calculation time for each function, and is used for
three classes: DomainInfo, ParticleSystem, and TreeForForce. These
three classes have ``PS::TimeProfile getTimeProfile()''. Users get the
calculation time of each function by using the function
``getTimeProfile()''.
%%本節では、PS::TimeProfile型について記述する。PS::TimeProfile型はFDPS
%%で使われる3つのクラス、領域分割クラス、粒子群クラス、相互作用ツリー
%%クラス、各メソッドの計算時間を格納するクラスである。これら3つのクラ
%%スには{\tt PS::TimeProfile getTimeProfile()}というメソッドが存在し、
%%このメソッドをつかって、ユーザーは各メソッドの計算時間を取得出来る。

This class is described as follows.
%このクラスは以下のように記述されている。

\begin{lstlisting}[caption=TimeProfile]
namespace ParticleSimulator{
    class TimeProfile{
    publid:
        F64 collect_sample_particle;
        F64 decompose_domain;
        F64 exchange_particle;
        F64 make_local_tree;
        F64 make_global_tree;
        F64 calc_force;
        F64 calc_moment_local_tree;
        F64 calc_moment_global_tree;
        F64 make_LET_1st;
        F64 make_LET_2nd;        
        F64 exchange_LET_1st;
        F64 exchange_LET_2nd;
    };
}    
\end{lstlisting}

%%%%%%%%%%%%%%%%%%%%%%%%%%%%%
\subsubsubsection{Addition}
\mbox{}
%%%%%%%%%%%%%%%%%%%%%%%%%%%%%

%%%%%%%%%%%%%%%%%%%%%%%%%%%%%
\begin{screen}
\begin{verbatim}
PS::TimeProfile PS::TimeProfile::operator + 
               (const PS::TimeProfile & rhs) const;
\end{verbatim}
\end{screen}

\begin{itemize}

\item{{\bf Argument}}

{rhs}: Input. Type const TimeProfile \&.
%{rhs}: 入力。{const TimeProfile \&}型。

\item{{\bf Return value}}

Type PS::TimeProfile. Add the components of rhs to its own components,
and return the results.
%%{PS::TimeProfile}型。{rhs}のすべてのメンバ変数の値と自身のメンバ変数
%%の値の和を取った値を返す。

\end{itemize}

%%%%%%%%%%%%%%%%%%%%%%%%%%%%%
\subsubsubsection{Reduction}
\mbox{}
%%%%%%%%%%%%%%%%%%%%%%%%%%%%%

%%%%%%%%%%%%%%%%%%%%%%%%%%%%%
\begin{screen}
\begin{verbatim}
PS::F64 PS::TimeProfile::getTotalTime() const;
\end{verbatim}
\end{screen}

\begin{itemize}

\item{{\bf Argument}}

  None.

\item{{\bf Return value}}

  Type PS::F64. Return values of all the member variables.
%{PS::F64}型。すべてのメンバ変数の値の和を返す。

\end{itemize}

%%%%%%%%%%%%%%%%%%%%%%%%%%%%%
\subsubsubsection{Initialize}
\mbox{}
%%%%%%%%%%%%%%%%%%%%%%%%%%%%%

%%%%%%%%%%%%%%%%%%%%%%%%%%%%%
\begin{screen}
\begin{verbatim}
void PS::TimeProfile::clear();
\end{verbatim}
\end{screen}

\begin{itemize}

\item{{\bf Argument}}

 None.

\item{{\bf Return value}}

 None.

\item{{\bf Feature}}

 Assign 0 to all the member variables.
%すべてのメンバ変数に0を代入する。

\end{itemize}
