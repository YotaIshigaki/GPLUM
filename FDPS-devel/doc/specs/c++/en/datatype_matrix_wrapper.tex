The wrappers of symmetric matrix types are defined as follows.
%対称行列型のラッパーの定義を以下に示す。
\begin{lstlisting}[caption=matrixsymwrapper]
namespace ParticleSimulator{
    typedef MatrixSym2<F32> F32mat2;
    typedef MatrixSym3<F32> F32mat3;
    typedef MatrixSym2<F64> F64mat2;
    typedef MatrixSym3<F64> F64mat3;
#ifdef PARTICLE_SIMULATOR_TWO_DIMENSION
    typedef F32mat2 F32mat;
    typedef F64mat2 F64mat;
#else
    typedef F32mat3 F32mat;
    typedef F64mat3 F64mat;
#endif
}
namespace PS = ParticleSimulator;
\end{lstlisting}

\texttt{PS::F32mat2}, \texttt{PS::F32mat3}, \texttt{PS::F64mat2}, and \texttt{PS::F64mat3} are,
respectively, 2x2 symmetric matrix in single precision, 3x3 symmetric
matrix in single presicion, 2x2 symmetric matrix in double precision,
and 3x3 symmetric matrix in double precision. If users set 2D (3D)
coordinate system, \texttt{PS::F32mat} and \texttt{PS::F64mat} is wrappers of
\texttt{PS::F32mat2} and \texttt{PS::F64mat2} (\texttt{PS::F32mat3} and \texttt{PS::F64mat3}).
%%すなわちPS::F32mat2, PS::F32mat3, PS::F64mat2, PS::F64mat3はそれぞれ
%%単精度2x2対称行列、倍精度2x2対称行列、単精度3x3対称行列、倍精度3x3対
%%称行列である。FDPSで扱う空間座標系を2次元とした場合、PS::F32matと
%%PS::F64matはそれぞれ単精度2x2対称行列、倍精度2x2対称行列となる。一方、
%%FDPSで扱う空間座標系を3次元とした場合、PS::F32matとPS::F64matはそれぞ
%%れ単精度3x3対称行列、倍精度3x3対称行列となる。

