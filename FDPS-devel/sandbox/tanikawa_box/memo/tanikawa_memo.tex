\documentclass[12pt,a4paper]{jarticle}
%
\topmargin=-5mm
\oddsidemargin=-5mm
\evensidemargin=-5mm
\textheight=235mm
\textwidth=165mm
%
\title{FDPS議題}
\author{谷川}
\date{}
%\pagestyle{empty}
\usepackage{graphicx}
\usepackage{wrapfig}
\usepackage{lscape}
\usepackage{amssymb}
\usepackage{amsmath}
\usepackage{bm}
\usepackage{setspace}
\usepackage{listings,jlisting}
\usepackage{color}
\usepackage{ascmac}
\usepackage{here}
\usepackage[dvipdfmx]{hyperref}
\usepackage{pxjahyper}

\newcommand{\underbold}[1]{\underline{\bf #1}}
\newcommand{\redtext}[1]{\textcolor{red}{#1}}


%\setcounter{secnumdepth}{4}
%%%%%%%%%%%%%%%%%%%%%%%%%%%%%%%%%%
\setcounter{secnumdepth}{6}
\makeatletter
\newcounter{subsubparagraph}[subparagraph]
\renewcommand\thesubsubparagraph{\thesubparagraph.\@arabic\c@subsubparagraph}
\newcommand\subsubparagraph{\@startsection{subsubparagraph}{6}{\parindent}%
                                       {3.25ex \@plus1ex \@minus .2ex}%
                                       {-1em}%
                              {\normalfont\normalsize\bfseries}}
\newcommand*\l@subsubparagraph{\@dottedtocline{6}{10em}{5em}}
\newcommand{\subsubsubsection}{\@startsection{paragraph}{4}{\z@}%
{1.5\baselineskip \@plus.5\dp0 \@minus.2\dp0}%
{.5\baselineskip \@plus2.3\dp0}%
{\reset@font\normalsize\bfseries}
}
\newcommand{\subsubsubsubsection}{\@startsection{subparagraph}{5}{\z@}%
{1.5\baselineskip \@plus.5\dp0 \@minus.2\dp0}%
{.5\baselineskip \@plus2.3\dp0}%
{\reset@font\normalsize\itshape}
}
\newcommand{\subsubsubsubsubsection}{\@startsection{subsubparagraph}{6}{\z@}%
{1.5\baselineskip \@plus.5\dp0 \@minus.2\dp0}%
{.5\baselineskip \@plus2.3\dp0}%
{\reset@font\normalsize\itshape}
}
\setcounter{tocdepth}{6}
%%%%%%%%%%%%%%%%%%%%%%%%%%%%%%%%%%

%\twocolumn
%\setstretch{1.5}

\lstset{language = C,
numbers = left,
numbersep = 8pt,
breaklines = true,
breakindent = 40pt,
frame = lines,
basicstyle = \ttfamily,
}

\begin{document}
\maketitle
\tableofcontents

\newpage

%%%%%%%%%%%%%%%%%%%%%%%%%%%%%%%%%%%%%%%%%%%%%%%%%%%%%
\section{15/02/10}

\begin{itemize}
\item ユーザーチュートリアル構成
  \begin{itemize}
  \item 目次作った
  \end{itemize}
\item 外部仕様書構成
  \begin{itemize}
  \item 目次作った
  \end{itemize}
\item テストスィート構成
  \begin{itemize}
  \item 項目作った
  \end{itemize}
\item エラー処理素案
  \begin{itemize}
  \item エラー処理の仕方 \redtext{(わかりやすいメッセージを吐いて
    MPI\_Abort)}
    \begin{itemize}
    \item 例外処理(メッセージを吐いて、throwしっぱなし/こっちが
      MPI\_abort)
    \item 例外処理をやらないというオプション
    \end{itemize}    
  \item エラーコードいる?(やりなおせないのでは?)
  \item エラーメッセージの様式 \redtext{(PSERROR: わかりやすいメッセー
    ジ、関数名、こまかいメッセージ)}
  \item エラー処理必要なところ
    \begin{itemize}
    \item 入力ファイルない
    \item メモリ確保失敗
    \item 規定(2G)より大きな配列
    \item ツリーのルートセル外に粒子
    \item ユーザーの不適切な初期設定(コンパイルエラーにできないものとかの
    場合)
    \item \redtext{浮動小数点例外?}
    \item \redtext{セグメンテーションフォールト?}
    \end{itemize}
  \item \redtext{signal handler要検討}
  \end{itemize}
\item API
  \begin{itemize}
  \item 相互作用リスト \redtext{(v1.0ではスレッドセーフではないというこ
    とを書く)}
  \item 通信データクラスにブロードキャストがほしい
  \end{itemize}
\item 実装
  \begin{itemize}
  \item Paritlce Meshどういうテストすべきか? \redtext{(検討)}
  \item 領域分割をユーザーが手で変更できるようにするか? \redtext{(裏API)}
  \item ツリー内の位置情報を64bitにするか \redtext{(する)}
  \item 相互作用ツリークラスの高レベル関数について
  \item メルセンヌツイスターの名前空間
  \end{itemize}
\item \redtext{テストスィートAPIごと}
  \redtext{
  \item {\bf 最優先} 3/1までにアルファ版をつくる
    \begin{itemize}
    \item インストールガイド
      \begin{itemize}
      \item ソースの場所
      \item 取得の仕方
      \item 展開の仕方
      \item サンプルコードの場所
      \item サンプルコードのコンパイルの仕方
      \item サンプルコードの動かし方
      \end{itemize}
    \end{itemize}
  }
\end{itemize}

\newpage

%%%%%%%%%%%%%%%%%%%%%%%%%%%%%%%%%%%%%%%%%%%%%%%%%%%%%
\section{15/02/18}

\begin{itemize}
\item S64のtypedefがlongだけど大丈夫?
\item clangを除くではなく、openmp対応のコンパイラ
\item C++03以降対応のコンパイラ
\item MPI対応のコンパイラ
\item configureについてはできてないならチュートリアルにはかかない(今の
  ところconfigureは必要でない)
\item サンプルコードの結果の見方
\item MPIデータ型の修正(これはなし)
\item 来週までにgithubに上げる
\end{itemize}

\newpage

%%%%%%%%%%%%%%%%%%%%%%%%%%%%%%%%%%%%%%%%%%%%%%%%%%%%%
\section{15/02/24}

\begin{itemize}
\item FDPS内のsetPosの手前でshiftを足すの危くない?
  \begin{itemize}
  \item そもそも内部で桁落ちする
  \item ユーザーがF32vecにしたときにも桁落ちする   
  \end{itemize}
\item 詳細記述書(fdps/document/document\_description/doc.pdf)の動作概
  略的なのがチュートリアルか仕様書のどこかに必要な気がする。
  \redtext{縮小版をチュートリアルに、フルを仕様書に}
\end{itemize}

\newpage

%%%%%%%%%%%%%%%%%%%%%%%%%%%%%%%%%%%%%%%%%%%%%%%%%%%%%
\section{15/03/03}

\begin{itemize}
\item ファイル
  \begin{itemize}
  \item このファイル: \$(FDPS)/sandbox/tanikawa\_box/memo/memo\_tanikawa.pdf
  \item 仕様書: \$(FDPS)/doc/doc\_specs.pdf
  \item チュートリアル: \$(FDPS)/doc/doc\_tutorial.pdf
  \item 論文下書き: \$(FDPS)/sc15/fdps\_sc15.pdf
  \end{itemize}
\item ユーザーの問い合せ先のメールアドレス(今は仮に
  ataru.tanikawa@riken.jpになっているが、専用のアドレスを作った方がよ
  いのではないか) \redtext{fdps-support@mail.jmlab.jp}
\item アルファ版について
\item SC15 registration
  \begin{itemize}
  \item Rules
    \begin{itemize}
    \item Up to a maximum of 10 pages, not including references
    \end{itemize}
  \item Title
  \item Author
  \item Requested Areas (第2希望まで)
    \begin{itemize}
    \item Algorithms
    \item Applications \redtext{第2希望}
    \item Architecture and Networks
    \item Clouds \& Distributed Computing
    \item Data Analytics, Visualization \& Storage
    \item Performance
    \item Programming Systems \redtext{第1希望}
    \item State of the Practice
    \item System Software
    \end{itemize}
  \item Keywords (2 -- 5 in total, 1 -- 3 per area from the above)
    \begin{itemize}
    \item Applications: Computational earth and atmospheric sciences
      \redtext{これ}
    \item Applications: Computational astrophysics/astronomy,
      chemistry, fluid dynamics, mechanics and physics \redtext{これ}
    \item Performance: Analysis, modeling or simulation for
      performance, power and/or resilience
    \item Programming Systems: Parallel application frameworks
      \redtext{これ}
    \item Programming Systems: Parallel programming languages,
      libraries, models and notations
    \item Programming Systems: Tools for parallel program development
      (e.g., debuggers and integrated development environments)
    \end{itemize}
  \item Abstract
    \begin{itemize}
    \item No more than 150 words
    \item Paragraph breaks by a blank line in the text field
    \end{itemize}
  \item Conference Presentations
  \end{itemize}
\item SC15 paper 色々
  \begin{itemize}
  \item Categories and Subject Descriptors
    (http://www.acm.org/about/class/ccs98-html)
%%    (I.6.7 [Simulation and Modeling]: Simulation Support Systems)
%%    (I.6.8 [Simulation and Modeling]: Types of Simulation -- Parallel)    
  \item General Terms (Algorithms, Design, Documentation, Economics,
    Experimentation, Human Factors, Languages, Legal Aspects,
    Management, Measurement, Performance, Reliability, Security,
    Standardization, Theory, Verification) いくつでも選べる(0でもよい)
  \item Keywords (選択肢なし?)
  \end{itemize}
\item SC15 paper 構成
  \begin{itemize}
  \item Introduction (なんでFDPSみたいなものが必要か?)
    \begin{itemize}
    \item 粒子シミュレーションは有用
      \begin{itemize}
      \item 構造が複雑な場合は格子法に比べて有利
      \item 様々な分野で利用
      \end{itemize}
    \item 粒子シミュレーションの大規模化に対する要請
      \begin{itemize}
      \item より計算コストの大きい、複雑な現象を解きたい
      \item 大規模並列計算機の存在
      \end{itemize}
    \item 粒子シミュレーションを大規模化する時の問題点
      \begin{itemize}
      \item プログラムが困難:ロードバランスのための動的領域分割、 領
        域分割に合わせた粒子交換 ノード間通信の削減と最適化、 キャッシュ
        利用効率の向上、SIMDユニット利用効率の向上、アクセラレータ
      \item やることは同じなのに各グループが個別に開発:
      \end{itemize}
    \item 共通のフレームワークを作ることで問題を解決
      \begin{itemize}
      \item 利点:研究者がよりクリエイティブなことに専念できる
      \item 過去のアプローチとその問題点:\redtext{アプリケーションご
        とにはある(Gadget, PAM-CRASH etc.), 汎用?(portable
        parallel particle program)}
      \item 我々のアプローチとその利点:わかりやすいインターフェース、
        高い性能
      \end{itemize}
    \item 論文の構成:実装、サンプルコード、性能、デモ、結論
    \end{itemize}
  \item Implementation
    \begin{itemize}
    \item 粒子シミュレーションのループのoverview (我々による抽象化?)
      \begin{itemize}
      \item 概念式
        \begin{align}
          \frac{d\bm{u}_i}{dt} = \sum_j f(\bm{u}_i,\bm{u}_j) + \sum_j
          g(\bm{u}_i,\bm{v}_i)
        \end{align}
        $f$: 粒子--粒子相互作用, $g$: 粒子--超粒子(相互?)作用,
        $\bm{u}_i$: 粒子の物理量ベクトル, $\bm{v}_j$: 超粒子の物理量ベ
        クトル
      \item ロードバランスのための領域分割 (微分方程式の分担決め1)
      \item 領域分割に合わせた粒子交換 (微分方程式の分担決め2)
      \item 相互作用リストの作成:ローカルツリー、LET交換、グローバル
        ツリー、$i$グループ作成、$i$グループに対する相互作用リスト作成
        (シグマ内の$j$決め)
      \item 相互作用計算 (シグマ内)
      \item 時間積分, 組成進化, etc.
      \end{itemize}
    \item 考え方、FDPSとユーザーの役割分担
      \begin{itemize}
      \item FDPS: 並列化がからんだ複雑な処理(領域分割、粒子交換、相互
        作用リストの作成)、これらをモジュール化する
      \item ユーザー: 特殊化(粒子の定義、相互作用の定義など)、APIの呼
        出、並列化のからまない単純な処理(相互作用計算、時間積分、組成
        進化、外場など)
      \item スキーマティックな図(領域クラス、粒子群クラス、相互作用ツ
        リークラス、ユーザー)
      \end{itemize}
    \item ユーザーの仕事
      \begin{itemize}
      \item 特殊化: 粒子の定義 (粒子群クラスのテンプレート引数)、 相互
        作用の定義 (相互作用クラスのテンプレート引数(既存のものから選
        択も可); 関数オブジェクト; 境界条件)
      \item APIの呼出(C++)
      \item 時間積分とか
      \end{itemize}
    \item 使用言語: C++ \redtext{なくてもいい}
    \item モジュール構成 \redtext{なくてもいい}
      \begin{itemize}
      \item 領域クラス (データ: 領域情報; API: decomposeDomainAll)
      \item 粒子群クラス (データ: 粒子情報; API: exchangeParticle; テ
        ンプレート引数: FP)
      \item 相互作用ツリークラス (データ: ツリー構造; API:
        calcForceAllAndWriteBack; テンプレート引数: SEARCH\_MODE,
        Force, EPI, EPJ, MomentLocal, MomentGlobal, SPJ; 関数オブジェ
        クト: calcForceEpEp, calcForceSpEp; マクロ: 座標系指定,
        MPI/OpenMPのオンオフ; 動的指定: 境界条件)
      \item 通信データクラス
      \item Particle Meshクラス
      \end{itemize}
    \end{itemize}
  \item Sample codes
    \begin{itemize}
    \item N-body \redtext{こちらだけ; 記述は、粒子定義、相互作用定義、
      (main関数)}
    \item SPH
    \end{itemize}
  \item Performance
    \begin{itemize}
    \item N-body \redtext{(weak scaling, strong scaling)}
    \item \redtext{SPH+N-body, giant impact (weak scaling, strong
      scaling)}
    \item MD \redtext{無しで}
    \item \redtext{長時間やっても動くと一言}
    \item \redtext{なんかきれいな絵(スパイラル?)}
    \end{itemize}
  \item Discussion and Conclusion
    \begin{itemize}
    \item 補足
      \begin{itemize}
      \item 独立時間刻みへの対応
      \item SIMD, アクセラレータへの対応
      \item etc.
      \end{itemize}      
    \end{itemize}      
  \end{itemize}
\item SC15 paper 予定
  \begin{itemize}
  \item -- 4/3 (UTC-12) (アブストラクト締切)
    \begin{itemize}
    \item Introduction, Implementation, Sample code完成
    \item Performanceの数字
    \end{itemize}
  \item 4/3 (UTC-12) -- 4/17 (UTC-12)
    \begin{itemize}
    \item Performance, Demonstration, Conclusion完成
    \end{itemize}
  \end{itemize}
\item 役割責任者
  \begin{itemize}
  \item Performance(N-body):岩澤
  \item Performance(SPH)、きれいな図作成:細野
  \item 本文、他細々したこと:谷川
  \item \redtext{SIMD: 似鳥}
  \item \redtext{英語チェック: 村主}
  \end{itemize}
\end{itemize}

\newpage

%%%%%%%%%%%%%%%%%%%%%%%%%%%%%%%%%%%%%%%%%%%%%%%%%%%%%
\section{15/03/10}

\begin{itemize}
\item 今日のファイル
  \begin{itemize}
  \item このファイル: \$(FDPS)/sandbox/tanikawa\_box/mem/memo\_tanikawa.pdf
  \item このファイル: \$(FDPS)/doc/doc\_specs.pdf
  \end{itemize}
\item Phantom-GRAPEとかをライブラリとして加える?
\end{itemize}

%%%%%%%%%%%%%%%%%%%%%%%%%%%%%%%%%%%%%%%%%%%%%%%%%%%%%
\section{15/03/17}

\begin{itemize}
\item 今日のファイル
  \begin{itemize}
  \item このファイル: \$(FDPS)/sandbox/tanikawa\_box/memo/memo\_tanikawa.pdf
  \item SC15メモ: \$(FDPS)/sc15/memo/memo\_sc15.pdf
  \end{itemize}
\item 性能がでない場合の相談をユーザーサポートに加えるか?
\item 相互作用計算用の関数オブジェクトの高速化についての相談をユーザー
  サポートに加えるか?
\end{itemize}


%%%%%%%%%%%%%%%%%%%%%%%%%%%%%%%%%%%%%%%%%%%%%%%%%%%%%
\section{15/03/23}

\begin{itemize}
\item 今日のファイル
  \begin{itemize}
  \item このファイル: \$(FDPS)/sandbox/tanikawa\_box/memo/tanikawa\_memo.pdf
  \item SC15メモ: \$(FDPS)/sandbox/tanikawa\_box/sc15\_memo/sc15\_memo.pdf
  \end{itemize}
\item SC15メモ見る
  \begin{itemize}
  \item 概念式確認(sec.2)
  \item サンプルコード相談(sec.3)
  \item 性能評価の役割確認とチケット発行(sec.4)
  \end{itemize}
\end{itemize}


%%%%%%%%%%%%%%%%%%%%%%%%%%%%%%%%%%%%%%%%%%%%%%%%%%%%%
\section{15/04/01}

\begin{itemize}

\item 今日のファイル
  \begin{itemize}
  \item このファイル: \$(FDPS)/sandbox/tanikawa\_box/memo/tanikawa\_memo.pdf
  \item SC15メモ: \$(FDPS)/sandbox/tanikawa\_box/sc15\_memo/sc15\_memo.pdf
  \end{itemize}

\item fdps-supportにメールが来たときの対応 \redtext{(済)}
  \begin{itemize}
  \item 担当決め: 谷川
  \item 自動返信はあり。
  \item 解決法: 担当者はチケットのところに対応を書く。対応がOKかどうか
    は牧野さん。
  \item 返信先: メール作成は担当者。宛先は質問者への返信。Ccに
    fdps-support。差出人はfdps-support。
  \end{itemize}

\item 変更履歴について \redtext{(済)}
  \begin{itemize}
  \item ソースコードの変更はsrcのしたのCHANGELOGへ
  \end{itemize}

\item リリースのバージョンについて
  \begin{itemize}
  \item 微修正としてver. 1.0.1
  \item 微修正しましたとREADME.mdに書く。文書の変更履歴はここ、ソースの
    変更履歴はここ。
  \item 来週くらいにリリース
  \end{itemize}

\item SC15 Technical paper
  \begin{itemize}
  \item Title: FDPS: A Novel Framework for Developing High-Performance
    Particle Simulation Codes for Distributed-Memory Systems
  \item Author Information: SC15メモへ
  \item Requested Areas: 第1希望 Programming Systems, 第2希望
    Applications
  \item Keywords (2 -- 5): 
    \begin{itemize}
    \item Algorithm: Numerical methods, linear and nonlinear systems
    \item Applications: Computational earth and atmospheric sciences
    \item Applicatoins: Computational astrophysics/astronomy,
      chemistry, fluid dynamics, mechanics and physics
    \item Programming Systems: Parallel application frameworks
    \item Programming System: Tools for parallel program development
      (e.g., debuggers and integrated development environments)
    \end{itemize}
  \item Abstract: SC15メモへ
  \item Conference Presentations: Yes
  \end{itemize}

\item 論文のサンプルコードまでを来週の水曜までに

\end{itemize}


%%%%%%%%%%%%%%%%%%%%%%%%%%%%%%%%%%%%%%%%%%%%%%%%%%%%%
\section{15/06/17}

\begin{itemize}

\item 仕様書の節7について

  \begin{itemize}
  \item getPosは位置を返すことが仕様
  \item 前提、備考は実装例の説明(実装例は付録もしくは別の節)
  \end{itemize}

\end{itemize}


%%%%%%%%%%%%%%%%%%%%%%%%%%%%%%%%%%%%%%%%%%%%%%%%%%%%%
\section{15/06/24}

\begin{itemize}

\item Appendixに実装例のサンプルを加える

\item 実習の手引(レジュメ的なもの) (-- 7/7)

\item FDPS概要のスライド (-- 7/7)

\item 実際のテスト (-- 7/7)

\item PS::SEARCH\_MODEに\\
  PS::SEARCH\_MODE\_LONG\_SCATTER, \\
  PS::SEARCH\_MODE\_LONG\_CUTOFF\_SCATTER, \\
  PS::SEARCH\_MODE\_LONG\_SYMMETRY, \\
  PS::SEARCH\_MODE\_LONG\_CUTOFF\_SYMMETRYを追加

\end{itemize}


\end{document}
