%%%%%%%%%%%%%%%%%%%%%%%%%%%%%%%%%%%%%%%%%%%%%%%%%%%%%%%%%%%%%%%%%%%%%%
%
% 分子シミュレーション研究会会誌「アンサンブル」用テンプレートファイル
%
% Ver. 3.0  2016/07/08
%
%%%%%%%%%%%%%%%%%%%%%%%%%%%%%%%%%%%%%%%%%%%%%%%%%%%%%%%%%%%%%%%%%%%%%%

\documentclass[twocolumn,10pt]{jarticle}
\usepackage[reviewb]{ensemble}
%
% === 使用方法 ===
%
% \usepackage[option]{ensemble}
%
% option のところを,記事の種類に合わせて変更してください.
% 
% review : 論文・研究紹介(著者一人)
% reviewb : 論文・研究紹介(著者二人)
% intro : 特集の前書き,研究室紹介,海外紹介(著者一人)
% introb : 特集の前書き,研究室紹介(著者二人)
% report : 夏の学校報告等(著者有り、著者紹介無し)
% noauther : 報告(著者無し)
%
% amsmath, amssymb, cite, color, graphicx, here, 
% は,ensemble.sty の中で読み込まれています.
%
\setcounter{page}{101}  % 開始ページ (編集委員用)

\title{% タイトル
[1]分子シミュレーション研究会会誌の書き方
}

\author{% 著者名
分子太郎
}

\email{% e-mail address
bunshi@aaa.bbb.ccc
}

\affiliation{% 所属
分子大学\quad シミュレーション学部
}

\career{% 著者紹介
分子太郎(博士(理学)):〔経歴〕1980年分子科学大学理工学研究科博士課程修了,同年分子科学研究所に入所.1990年から現所属.〔専門〕統計力学,液体論.〔趣味〕演劇鑑賞.\\
写真サイズは縦35mm横25mm程度.
}

% 第2著者
\authorb{原子次郎}
\emailb{genshi@aaa.bbb.ccc}
\affiliationb{% 所属
原子大学\quad シミュレーション学部
}
\careerb{% 著者紹介
原子次郎(博士(工学)):〔経歴〕1985年原子科学大学理工学研究科博士課程修了,同年原子科学研究所に入所.1995年から現所属.〔専門〕原子力工学〔趣味〕演劇鑑賞.\\
写真サイズは縦35mm横25mm程度.
}

\abst{% 概要
 数行程度(概要の部分は、1段組)\\
\textcolor{red}{(1行あける)}}

\keyword{% キーワード
5〜10個程度 (キーワードの部分は、1段組)\\
\textcolor{red}{(1行あける)}}

% 顔写真のファイル名.デフォルトはphoto.eps, photob.eps
% 第1著者
\photofile{photo.eps} 
% 第2著者
\photofileb{photob.eps} 

%------------------------------------------------------------------------
% ユーザー定義のマクロはここに書く.
%------------------------------------------------------------------------
%\renewcommand{\v}[1]{{\bf #1}}
%\newcommand{\ave}[1]{\left< #1 \right>}
%------------------------------------------------------------------------
\begin{document}
\maketitle
%------------------------------------------------------------------------
% 原稿ここから
%------------------------------------------------------------------------

\section{はじめに}
以下の注意事項に留意して,原稿を作成すること.\\
\textcolor{red}{(1行あける)}
%
\section{「アンサンブル」用原稿作成上の注意}

\subsection{標準形式}
原稿はMicrosoft Word,TeX等を用いて作成し,図や写真等は原稿に張り込み一つのファイルとして完結させる.原稿の標準形式を表\ref{format-table}に示す.

\begin{table}[H]
\caption{原稿の標準形式}
\begin{tabular}{|c|l|}
\hline
用紙サイズ  &  A4縦長(210mm×297mm),横書き  \\
\hline
余白サイズ  &
\begin{minipage}{17zw}
上余白28mm,下余白20mm  \par
左余白20mm,右余白20mm
\end{minipage}  \\
\hline
タイトル  &
\begin{minipage}{17zw}
1段組,所属,著者氏名,emailを明記
\end{minipage}  \\
\hline
本文  &
\begin{minipage}{17zw}
2段組,1段80mm,段間隔余白10mm
\end{minipage}  \\
\hline
活字  &
\begin{minipage}{17zw}
10ポイント(10×0.3514mm) \par
タイトル \par
\qquad   \textsf{MSゴシック体} \par
所属,著者氏名 \par
\qquad   \textsf{MSゴシック体} \par
著者email \par
\qquad   {\fontfamily{phv}\fontseries{m}\fontshape{n}\selectfont Arial} \par
本文  \par
\qquad   {\mc MS明朝体} \par
見出し \par
\qquad   \textsf{MSゴシック体} \par
英文字・数字 \par
\qquad   \textrm{Times New Roman}\par
\qquad   または\textrm{Symbol}
\end{minipage}  \\
\hline
1行の字数  &  1段あたり23文字程度  \\
\hline
行送り  &
\begin{minipage}{17zw}
15ポイント(15×0.3514=5.271mm)\par
1ページあたり45行
\end{minipage}  \\
\hline
\end{tabular}
\label{format-table}
\end{table}


\subsection{見出しなど}
見出しは\textgt{ゴシック体}を用い,大見出しは左寄せして前に1行空ける.中見出しは2.2などのように番号をつけ左寄せする.見出しの数字は半角とする.行の始めに,括弧
やハイフン等がこないように禁則処理を行うこと.
\\

\subsection{句読点}
句読点は ,および .を用い, 、や 。は避けること.
\\

\subsection{図について}
図中のフォントは本文中のフォントと同じものを用いること.図や表はなるべく上側か下側の隅に固めること.
\\

\subsection{参考文献について}
\subsubsection{番号の付け方}
参考文献は本文中の該当する個所に\cite{ueda},\cite{ueda,metropolis},
\cite{ueda,alder,metropolis,allen}のように番号を入れて示す.
\\
%それぞれ [1],[1,3],[1-4]のように表示される

\subsubsection{参考文献の引き方}
著者名,誌名,巻,頁,年の順とする.毎号頁の改まる雑誌は巻-号数のようにして号数
も入れる.著者名は,名前のイニシャル.名字,のように記述する.雑誌名の省略法は科学技術文献速報(JICST)に準拠する.文献の表題は省略する.日本語の雑誌・書籍の場合
は著者名・書名とも省略しない.
\\

\section{TeX で執筆する場合}
TeX で原稿を執筆する場合はアンサンブル用のスタイルファイル ensemble.sty を使用すること.  
\begin{verbatim}
\documentclass[twocolumn,10pt]{jarticle}
\usepackage[option]{ensemble}	
\end{verbatim}
\verb|option| に指定できるコマンド一覧を表~\ref{option}に示す.
執筆する記事の内容(著者表示の有無, 著者紹介の有無, 概要・キーワードの有無)にあわせて使い分けること.
\verb|option| を省略すると \verb|noauther| と同じになる.
よく使用される \verb|amsmath|, \verb|amssymb|, \verb|cite|, \verb|color|, \verb|graphicx|, \verb|here|  は \verb|ensemble.sty| の中で読み込まれるため, \verb|\usepackage{...}| を使用して定義する必要はない.



\begin{table}[H]
\caption{オプション一覧.}
\begin{tabular}{|c|l|}
\hline
 \verb|option| & 用途\\
\hline\hline
 \verb|review| & 研究紹介(単著用) \\
\hline
 \verb|reviewb| & 研究紹介(著者2人用) \\
\hline
 \verb|intro| & 研究以外の記事 (概要・キーワード無し)\\
\hline
 \verb|introb| & 研究以外の記事(同上, 著者2人用) \\
\hline
 \verb|report| & 夏の学校等(著者有, 著者紹介無し)\\
\hline
 \verb|noauther| & 幹事会報告, 事務局連絡(タイトル\&本文)\\
\hline
\end{tabular}  
\label{option}
\end{table}





% 謝辞
\acknowledgement{○○氏に感謝します.}

% 参考文献
\begin{thebibliography}{9}
\bibitem{ueda}
上田顕,分子シミュレーション—古典系から量子系手法まで−,裳華房 (2003).
\bibitem{alder} B. J. Alder and T. E. Wainwright, \textit{J. Chem. Phys.}, {\bf 27}, 1208 (1957).
\bibitem{metropolis} N. Metropolis, A. W. Rosenbluth, M. N. Rosenbluth, A. H. Teller and E. Teller, \textit{J. Chem. Phys.}, {\bf 21}, 1087 (1953).
\bibitem{allen} M. P. Allen and D. J. Tildesley, \textit{Computer Simulation of Liquids}, Oxford University Press Inc., New York (1987).
\end{thebibliography}

%------------------------------------------------------------------------
% 原稿ここまで
%------------------------------------------------------------------------

% 著者紹介出力
\profile




\end{document}

