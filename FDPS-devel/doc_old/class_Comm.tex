\subsubsection{概要}

並列化手法(MPI, OpenMP, SIMD)には様々な最適化パラメータが存在する。MPI
関連のパラメータやコミュニケータはシングルトンパターンを使って管理する。

以下のように定義する。

\begin{lstlisting}[caption={\tt Comm}の定義]
namespace ParticleSimulator {
    class Comm{
    public:
        static S32 getRank();
        static S32 getNumberOfProc();
        static S32 getRankMultiDim(const S32 id);
        static S32 getNumberOfProcMultiDim(const S32 id);
        static bool synchronizeConditionalBranchAND(const bool local);        
        static bool synchronizeConditionalBranchOR(const bool local);        
        template<class T>
        static T getMinValue(const T val);
        template<class T>
        static T getMaxValue(const T val);
        template<class Tfloat, class Tint>
        static void getMinValue(const Tfloat f_in, const Tint i_in, Tfloat & f_out, Tint & i_out);
        template<class Tfloat, class Tint>
        static void getMaxValue(const Tfloat f_in, const Tint i_in, Tfloat & f_out, Tint & i_out);
        template<class T>
        static T getSum(const T val);
}
\end{lstlisting}

\subsubsection{API}

%%%%%%%%%%%%%%%%%%%%%%%%%%%
\begin{screen}
\begin{verbatim}
static PS::S32 getRank();
\end{verbatim}
\end{screen}

\begin{itemize}

\item{{\bf 引数}}

なし。

\item{{\bf 返り値}}

{\tt PS::S32}型。全プロセス中でのランクを返す。

\end{itemize}

%%%%%%%%%%%%%%%%%%%%%%%%%%%
\begin{screen}
\begin{verbatim}
static PS::S32 PS::Comm::getNumberOfProc();
\end{verbatim}
\end{screen}

\begin{itemize}

\item{{\bf 引数}}

なし。

\item{{\bf 返り値}}

{\tt PS::S32}型。全プロセス数を返す。

\end{itemize}

%%%%%%%%%%%%%%%%%%%%%%%%%%%
\begin{screen}
\begin{verbatim}
static PS::S32 PS::Comm::getRankMultiDim(const PS::S32 id);
\end{verbatim}
\end{screen}

\begin{itemize}

\item{{\bf 引数}}

{\tt id}: 入力。{\tt const PS::S32}型。軸の番号。x軸:0, y軸:1, z軸:2。

\item{{\bf 返り値}}

{\tt PS::S32}型。id番目の軸でのランクを返す。2次元の場合、id=2は1を返す。

\end{itemize}

%%%%%%%%%%%%%%%%%%%%%%%%%%%
\begin{screen}
\begin{verbatim}
static PS::S32 PS::Comm::getNumberOfProcMultiDim(const PS::S32 id);
\end{verbatim}
\end{screen}

\begin{itemize}

\item{{\bf 引数}}

{\tt id}: 入力。{\tt const PS::S32}型。軸の番号。x軸:0, y軸:1, z軸:2。

\item{{\bf 返り値}}

{\tt PS::S32}型。id番目の軸のプロセス数を返す。2次元の場合、id=2は1を返す。

\end{itemize}

%%%%%%%%%%%%%%%%%%%%%%%%%%%
\begin{screen}
\begin{verbatim}
static bool PS::Comm::synchronizeConditionalBranchAND(const bool local)
\end{verbatim}
\end{screen}

\begin{itemize}

\item{{\bf 引数}}

{\tt local}: 入力。{\tt const bool}型。

\item{{\bf 返り値}}

{\tt bool}型。全プロセスで{\tt local}の{\tt AND}を取り、結果を返す。

\end{itemize}

%%%%%%%%%%%%%%%%%%%%%%%%%%%
\begin{screen}
\begin{verbatim}
static bool PS::Comm::synchronizeConditionalBranchOR(const bool local);
\end{verbatim}
\end{screen}

\begin{itemize}

\item{{\bf 引数}}

{\tt local}: 入力。{\tt const bool}型。

\item{{\bf 返り値}}

{\tt bool}型。全プロセスで{\tt local}の{\tt OR}を取り、結果を返す。

\end{itemize}

%%%%%%%%%%%%%%%%%%%%%%%%%%%
\begin{screen}
\begin{verbatim}
template <class T>
static T PS::Comm::getMinValue(const T val);
\end{verbatim}
\end{screen}

\begin{itemize}

\item{{\bf 引数}}

{\tt val}: 入力。{\tt const T}型。

\item{{\bf 返り値}}

{\tt T}型。全プロセスで{\tt val}の最小値を取り、結果を返す。

\end{itemize}

%%%%%%%%%%%%%%%%%%%%%%%%%%%
\begin{screen}
\begin{verbatim}
template <class T>
static T PS::Comm::getMaxValue(const T val);
\end{verbatim}
\end{screen}

\begin{itemize}

\item{{\bf 引数}}

{\tt val}: 入力。{\tt const T}型。

\item{{\bf 返り値}}

{\tt T}型。全プロセスで{\tt val}の最大値を取り、結果を返す。

\end{itemize}

%%%%%%%%%%%%%%%%%%%%%%%%%%%
\begin{screen}
\begin{verbatim}
template <class Tfloat, class Tint>
static void PS::Comm::getMinValue(const Tfloat f_in, const Tint i_in,
                                  Tfloat & f_out, Tint & i_out);
\end{verbatim}
\end{screen}

\begin{itemize}

\item{{\bf 引数}}

{\tt f\_in}: 入力。{\tt const Tfloat}型。

{\tt i\_in}: 入力。{\tt const Tint}型。

{\tt f\_out}: 出力。{\tt Tfloat}型。全プロセスで{\tt f\_in}の最小値を取
り、結果を返す。

{\tt i\_out}: 出力。{\tt Tint}型。{\tt f\_out}に伴うIDを返す。

\item{{\bf 返り値}}

なし。

\end{itemize}

%%%%%%%%%%%%%%%%%%%%%%%%%%%
\begin{screen}
\begin{verbatim}
template <class Tfloat, class Tint>
static void PS::Comm::getMaxValue(const Tfloat f_in, const Tint i_in,
                                  Tfloat & f_out, Tint & i_out);
\end{verbatim}
\end{screen}

\begin{itemize}

\item{{\bf 引数}}

{\tt f\_in}: 入力。{\tt const Tfloat}型。

{\tt i\_in}: 入力。{\tt const Tint}型。

{\tt f\_out}: 出力。{\tt Tfloat}型。全プロセスで{\tt f\_in}の最大値を取
り、結果を返す。

{\tt i\_out}: 出力。{\tt Tint}型。{\tt f\_out}に伴うIDを返す。

\item{{\bf 返り値}}

なし。

\end{itemize}

%%%%%%%%%%%%%%%%%%%%%%%%%%%
\begin{screen}
\begin{verbatim}
template <class T>
static T PS::Comm::getSum(const T val);
\end{verbatim}
\end{screen}

\begin{itemize}

\item{{\bf 引数}}

{\tt val}: 入力。{\tt const T}型。

\item{{\bf 返り値}}

{\tt T}型。全プロセスで{\tt val}の総和を取り、結果を返す。

\end{itemize}


%%%%%%%%%%%%%%%%%%%%%%%%%%%
%\begin{screen}
%\begin{verbatim}
%static void PS::Comm::setNProcXD(const PS::S32 nx, const PS::S32 ny, const PS::S32 nz);
%\end{verbatim}
%\end{screen}

%\begin{itemize}

%\item{{\bf 引数}}

%{\tt nx}: 入力。{\tt const PS::S32}型。x軸方向のプロセスの分割数。

%{\tt ny}: 入力。{\tt const PS::S32}型。y軸方向のプロセスの分割数。

%{\tt nz}: 入力。{\tt const PS::S32}型。z軸方向のプロセスの分割数。

%\item{{\bf 返り値}}

%なし。

%\item{{\bf 機能}}

%プロセッサの分割数を設定する。nx,ny,nzの積が全プロセス数にならない場合
%は例外を送出する。
%\end{itemize}
