本節では、PSLが実行すること、またそのアルゴリズムについて記述する。一般
に、分散メモリ環境での粒子シミュレーションの手順は以下の3つに分けられ
る。
\begin{enumerate}
\item[0.] 各プロセスの役割分担を決定 \label{proc:decompexchange}
\item[1.] $i$粒子に対する$j$粒子リストを作成 \label{proc:jlist}
\item[2.] $i$粒子に対する$j$粒子の作用を計算し、$i$粒子の物理量とその時間変
  化量を導出 \label{proc:interact}
\item[3.] $i$粒子の物理量を時間積分 \label{proc:integrate}
\end{enumerate}
ここで、作用される粒子を$i$粒子、作用する粒子を$j$粒子と呼び、ある$i$粒
子に対する全$j$粒子を$j$粒子リストと呼んだ。PSLは手順0から2を行う関数群
を提供する。

手順0は以下2つの手順に分割される。
\begin{enumerate}
\renewcommand{\labelenumi}{0.\arabic{enumi}.}
\item 各プロセスの担当粒子を決定 \label{proc:decomp}
\item 各プロセスにそれぞれの担当粒子を分配 \label{proc:exchange}
\end{enumerate}
PSLでは、それぞれの手順はGreeMとほぼ同様のアルゴリズムを用いる。

手順1は、さらに以下のような手順に分割される。
\begin{enumerate}
\renewcommand{\labelenumi}{1.\arabic{enumi}.}
\item 各プロセスが自分の担当粒子の探査を高速にするデータ構造を構
  築 \label{proc:lstruct}
\item 各プロセスが自分の担当粒子のうち別プロセスの担当粒子の$j$粒子リス
  トに入る可能性のある粒子を探査し、そうであればその粒子データを別プロ
  セスへ送信 \label{proc:communicate}
\item 各プロセスが自分の担当粒子と別プロセスから送信されてきた粒子の粒
  子探査を高速にするデータ構造を構築
  \label{proc:gstruct}
\item 各プロセスが担当粒子の$j$粒子リストを作成 \label{proc:jlistd}
\end{enumerate}

PSLでは、手順1.1と手順1.3でツリー構造が構築される。前者のツリーをローカ
ルツリー、後者のツリーをグローバルツリーと呼ぶ。

手順1.2と手順1.4で行われる粒子探査は、扱う相互作用が短距離力か長距離力
かで大きく異なる。ここで短距離力とは力の到達距離が有限である場合であり、
長距離力とは力の到達距離が無限である場合である。短距離力において探査す
るのは粒子そのものである。一方長距離力においては、粒子そのものだけでな
く、遠くの複数の粒子をまとめた超粒子も探査する。ただし、粒子に超粒子が
作用しつつ力の到達距離が有限となる場合がある。例としてはTreePM法を用い
た$N$体シミュレーションである。PSLでは、これは長距離力に分類される。

長距離力の場合の手順1.2のアルゴリズムは、GreeMに準じる。短距離力の場合
の手順1.2のアルゴリズムは、相互作用の性質に応じて場合分けされる。相互作
用の性質は以下の4種類が存在する。
\begin{itemize}
\item 力の到達距離が全粒子で等しい相互作用 (固定長モード)
\item 力の到達距離が$i$粒子のサイズで決まる相互作用(収集モード)
\item 力の到達距離が$j$粒子のサイズで決まる相互作用(散乱モード)
\item 力の到達距離が$ij$粒子の両方のサイズで決まる相互作用 (対称モード)
\end{itemize}
固定長モードのアルゴリズムはまだ決めていない。他のモードはASURAに準じる。

PSLでは手順2においてSIMD演算が用いられる。SIMD演算を用いて計算性能を得
るために、複数の$i$粒子が共通の$j$粒子リストを持つようにする。本文書で
は、PSLにおけるSIMD演算の実装については言及しない。