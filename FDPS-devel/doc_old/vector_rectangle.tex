\subsubsection{{\tt PS::Vector2}型} 

x,yの2要素を持つ。それらの要素の型は
{PS::S32},{PS::S64},{PS::U32},{PS::U64},{PS::F32},{PS::F64}に限られる。
外積演算も定義しており、結果はスカラーとして返す。このクラスは以下の様
に記述される。

\begin{lstlisting}[caption=Vector2]
namespace ParticleSimulator{
    template <typename T>
    class Vector2{
    public:
        //メンバ変数は以下の二つのみ。
        T x, y;

        //コンストラクタ
        Vector2() : x(T(0)), y(T(0)) {}
        Vector2(const T _x, const T _y) : x(_x), y(_y) {}
        Vector2(const T s) : x(s), y(s) {}
        Vector2(const Vector2 & src) : x(src.x), y(src.y) {}

        //代入演算子
        const Vector2 & operator = (const Vector2 & rhs){
            x = rhs.x;
            y = rhs.y;
            return (*this);
        }

        //加減算
        Vector2 operator + (const Vector2 & rhs) const{
            return Vector2(x + rhs.x, y + rhs.y);
        }
        const Vector2 & operator += (const Vector2 & rhs){
            (*this) = (*this) + rhs;
            return (*this);
        }
        Vector2 operator - (const Vector2 & rhs) const{
            return Vector2(x - rhs.x, y - rhs.y);
        }
        const Vector2 & operator -= (const Vector2 & rhs){
            (*this) = (*this) - rhs;
            return (*this);
        }

        //ベクトルスカラ積
        Vector2 operator * (const T s) const{
            return Vector2(x * s, y * s);
        }
        const Vector2 & operator *= (const T s){
            (*this) = (*this) * s;
            return (*this);
        }
        friend Vector2 operator * (const T s, const Vector2 & v){
            return (v * s);
        }
        Vector2 operator / (const T s) const{
            return Vector2(x / s, y / s);
        }
        const Vector2 & operator /= (const T s){
            (*this) = (*this) / s;
            return (*this);
        }

        //内積
        T operator * (const Vector2 & rhs) const{
            return (x * rhs.x) + (y * rhs.y);
        }

        //外積(返り値はスカラ!!)
        T operator ^ (const Vector2 & rhs) const{
            const T z = (x * rhs.y) - (y * rhs.x);
            return z;
        }

        //Vector2<U>への型変換
        template <typename U>
        operator Vector2<U> () {
            return Vector2<U> (static_cast<U>(x),
                               static_cast<U>(y));
        }
    };


    template <>
    inline Vector2<float> Vector2<float>::operator / (const float s) const {
        const float inv_s = 1.0f/s;
        return Vector2(x * inv_s, y * inv_s);
    }
    template <>
    inline Vector2<double> Vector2<double>::operator / (const double s) const {
        const double inv_s = 1.0/s;
        return Vector2(x * inv_s, y * inv_s);
    }

}
\end{lstlisting}

\subsubsection{{\tt PS::Vector3}型}

x,y,zの3要素を持つ。それらの要素の型は
{PS::S32},{PS::S64},{PS::U32},{PS::U64},{PS::F32},{PS::F64}に限られる。
定義されているメソッド等は{\tt Vector2}型と同じであるが外積は{\tt
Vector2}型と違い{\tt Vector3}型で返す。このクラスは以下の様に記述される。

\begin{lstlisting}[caption=Vector3]
namespace ParticleSimulator{
    template<class T>
    class Vector3{
    public:
        //メンバ変数は以下の三つのみ。
        T x, y, z;

        //コンストラクタ
        Vector3() : x(T(0)), y(T(0)) {}
        Vector3(const T _x, const T _y) : x(_x), y(_y), z(_z) {}
        Vector3(const T s) : x(s), y(s), z(s) {}
        Vector3(const Vector3 & src) : x(src.x), y(src.y), z(src.z) {}

        //代入演算子
        const Vector3 & operator = (const Vector3 & rhs){
            x = rhs.x;
            y = rhs.y;
            z = rhs.z;
            return (*this);
        }

        //加減算
        Vector3 operator + (const Vector3 & rhs) const{
            return Vector3(x + rhs.x, y + rhs.y, z + rhs.z);
        }
        const Vector3 & operator += (const Vector3 & rhs){
            (*this) = (*this) + rhs;
            return (*this);
        }
        Vector3 operator - (const Vector3 & rhs) const{
            return Vector2(x - rhs.x, y - rhs.y, z - rhs.z);
        }
        const Vector3 & operator -= (const Vector3 & rhs){
            (*this) = (*this) - rhs;
            return (*this);
        }

        //ベクトルスカラ積
        Vector3 operator * (const T s) const{
            return Vector3(x * s, y * s, z * s);
        }
        const Vector3 & operator *= (const T s){
            (*this) = (*this) * s;
            return (*this);
        }
        friend Vector3 operator * (const T s, const Vector3 & v){
            return (v * s);
        }
        Vector3 operator / (const T s) const{
            return Vector3(x / s, y / s, z / s);
        }
        const Vector3 & operator /= (const T s){
            (*this) = (*this) / s;
            return (*this);
        }

        //内積
        T operator * (const Vector3 & rhs) const{
            return (x * rhs.x) + (y * rhs.y) + (z * rhs.z);
        }

        //外積
        Vector3 operator ^ (const Vector3 & rhs) const{
            return Vector3( (y * rhs.z - z * rhs.y),
                            (z * rhs.x - x * rhs.z),
                            (x * rhs.y - y * rhs.x) );
        }

        //Vector2<U>への型変換
        template <typename U>
        operator Vector3<U> () {
            return Vector3<U> (static_cast<U>(x),
                               static_cast<U>(y),
                               static_cast<U>(z));
        }
    };
    template <>
    inline Vector3<float> Vector3<float>::operator / (const float s) const {
        const float inv_s = 1.0f/s;
        return Vector3(x * inv_s, y * inv_s, z * inv_s);
    }
    template <>
    inline Vector3<double> Vector3<double>::operator / (const double s) const {
        const double inv_s = 1.0/s;
        return Vector3(x * inv_s, y * inv_s, z * inv_s);
    }
}
\end{lstlisting}

