\subsection{$N$体シミュレーション}
\label{sec:Nbody}

本節では、PSLの使用例を記述する。サンプルコードには省略したものだけを示
した。

$N$体シミュレーションの時間積分のループは以下の手順で実行される。ここで
時間積分法はリープフロッグ法を使用する。
\begin{enumerate}
\item $m_i$, $\bm{x}_i^{(0)}$, $\bm{v}_i^{(0)}$, $\bm{a}_i^{(0)}$は既知。
  上付き添字``$(0)$''は時刻を表す。
\item $\bm{x}_i^{(1)}$を以下の式で導出し、
  \begin{align}
    \bm{x}_i^{(1)} &= \bm{x}_i^{(0)} + \Delta t \bm{v}_i^{(0)} + 0.5
    \Delta t^2 \bm{a}_i^{(0)} \label{eq:ipos}
  \end{align}
  $\bm{v}_i^{(1/2)}$を以下の式で導出($X=\bm{v}$)。
  \begin{align}
    X_i^{(1/2)} &= X_i^{(0)} + 0.5 \Delta t
    \dot{X}_i^{(0)} \label{eq:half}
  \end{align}
\item $m_i$と$\bm{x}_i^{(1)}$を用いて$\bm{a}_i^{(1)}$を導出。
\item 以下の式を用いて$\bm{v}_i^{(1)}$を導出($X=\bm{v}$)。
  \begin{align}
    X_i^{(1)} = X_i^{(1/2)} + 0.5 \Delta t \dot{X}_i^{(1)} \label{eq:corr}
  \end{align}
\item 計算領域の分割と粒子の分配を実行。
\item ステップ1に戻る。
\end{enumerate}

以上の手順をPSLを用いて実装すると以下のようなコードとなる。

\begin{lstlisting}[caption=$N$体シミュレーションのサンプルコード]
#include<particle_simulator.h> // 必須

class ForceGrav{
public:
    PS::F64vec acc;
    PS::F64 pot;
    void clear(){
        acc = 0.0;
        pot = 0.0;
    }
};

class FPGrav{
public:
    PS::S64 id;
    PS::F64 mass;
    PS::F64vec pos;
    PS::F64vec vel;
    PS::F64vec acc;
    PS::F64 pot;
    PS::F64vec getPos() const { return pos; }
    void copyFromForce(const ForceGrav & force){
        acc = force.acc;
        pot = force.pot;
    }
};

class EPIGrav{
public:
    PS::S64 id;
    PS::F64vec pos;
    static PS::F64 eps;
    PS::F64vec getPos() const {pos;}
    void copyFromFP(const FPGrav & fp){ 
        pos = fp.pos;
        id = fp.id;
    }
};

PS::F64 EPIGrav::eps = 1.0/32.0;

class EPJGrav{
public:
    PS::S64 id;
    PS::F64 mass;
    PS::F64vec pos;
    void copyFromFP(const FPGrav & fp){ 
        mass = fp.mass;
        pos = fp.pos;
        id = fp.id;
    }
    PS::F64vec getPos() const { return pos; }
    PS::F64 getCharge() const { return mass; }
};

struct CalcForceEpEp{
    void operator () (const EPIGrav * ep_i,
                      const PS::S32 n_ip,
                      const EPJGrav * ep_j,
                      const PS::S32 n_jp,
                      ForceGrav * force){
        PS::F64 eps2 = EPIGrav::eps * EPIGrav::eps;
        for(PS::S32 i=0; i<n_ip; i++){
            PS::F64vec xi = ep_i[i].pos;
            PS::F64vec ai = 0.0;
            PS::F64 poti = 0.0;
            PS::S64 idi = ep_i[i].id;
            for(PS::S32 j=0; j<n_jp; j++){
                if( idi == ep_j[j].id ) continue;
                PS::F64vec rij = xi - ep_j[j].pos;
                PS::F64 r3_inv = rij * rij + eps2;
                PS::F64 r_inv = 1.0/sqrt(r3_inv);
                r3_inv = r_inv * r_inv;
                r_inv *= ep_j[j].mass;
                r3_inv *= r_inv;
                ai -= r3_inv * rij;
                poti -= r_inv;
            }
            force[i].acc += ai;
            force[i].pot += poti;
        }
    }
};

struct CalcForceSpEp{
    void operator () (const EPIGrav * ep_i,
                      const PS::S32 n_ip,
                      const PS::SPJMonoPole * sp_j,
                      const PS::S32 n_jp,
                      ForceGrav * force){
        PS::F64 eps2 = EPIGrav::eps * EPIGrav::eps;
        for(PS::S32 i=0; i<n_ip; i++){
            PS::F64vec xi = ep_i[i].pos;
            PS::F64vec ai = 0.0;
            PS::F64 poti = 0.0;
            for(PS::S32 j=0; j<n_jp; j++){
                PS::F64vec rij = xi - sp_j[j].pos;
                PS::F64 r3_inv = rij * rij + eps2;
                PS::F64 r_inv = 1.0/sqrt(r3_inv);
                r3_inv = r_inv * r_inv;
                r_inv *= sp_j[j].mass;
                r3_inv *= r_inv;
                ai -= r3_inv * rij;
                poti -= r_inv;
            }
            force[i].acc += ai;
            force[i].pot += poti;
        }
    }
};

template<class Tpsys>
void Kick(Tpsys & system,
          const PS::F64 dt){
    PS::S32 n = system.getNumberOfParticleLocal();
    for(int i=0; i<n; i++){
        system[i].vel  += system[i].acc * dt;
    }
}

template<class Tpsys>
void Drift(Tpsys & system,
           const PS::F64 dt){
    PS::S32 n = system.getNumberOfParticleLocal();
    for(int i=0; i<n; i++){
        system[i].pos  += system[i].vel * dt;
    }
}

int main(int argc, char *argv[]){
    std::cout<<std::setprecision(15);
    std::cerr<<std::setprecision(15);
    PS::Initialize(argc, argv);

    char sinput[1024];
    int c;
    while((c=getopt(argc,argv,"i:h")) != -1){
        switch(c){
        case 'i':
            sprintf(sinput,optarg);
            break;
        case 'h':
            std::cerr<<"i: input file name (nemo ascii)"<<std::endl;
            return 0;
        }
    }

    const PS::F32 dt = 1.0/128.0;
    const PS::F32 time_end = 10.0;
    PS::ParticleSystem<FPGrav> system_grav;
    system_grav.initialize();
    PS::S32 n_grav_glb, n_grav_loc;
    PS::F32 time_sys;
    ReadNemoAscii(system_grav, n_grav_glb, n_grav_loc, time_sys, sinput);

    PS::F32 coef_ema = 0.7;
    PS::DomainInfo dinfo;
    dinfo.initialize(coef_ema);
    dinfo.decomposeDomainAll();

    system_grav.exchangeParticle(dinfo);
    n_grav_loc = system_grav.getNumberOfParticleLocal();

    PS::TreeType<ForceGrav, EPIGrav, EPJGrav>::TreeForMonoPole tree_grav;

    PS::F32 theta = 0.5;
    tree_grav.initialize(n_grav_glb, theta);

    tree_grav.calcForceAllAndWriteBack(CalcForceEpEp(), CalcForceSpEp(), system_grav, dinfo);

    Kick(system_grav, dt*0.5);

    while(time_sys < time_end){
        time_sys += dt;
        Drift(system_grav, dt);
        if( fmod(time_sys, 1.0/32.0) == 0.0){
            dinfo.decomposeDomainAll();
        }
        system_grav.exchangeParticle(dinfo);
        tree_grav.calcForceAllAndWriteBack(CalcForceEpEp(), CalcForceSpEp(), system_grav, dinfo);
        Kick(system_grav, dt);
    }

    PS::Finalize();
    return 0;
}

\end{lstlisting}


\subsection{stdSPHシミュレーション}
\label{sec:stdsph}

stdSPHシミュレーションは以下の手順で実行される。時間積分法は先の$N$体シ
ミュレーションと同様にリープフロッグ法である。
\begin{enumerate}
\item $m_i$, $\bm{x}_i^{(0)}$, $\bm{v}_i^{(0)}$, $\bm{a}_i^{(0)}$,
  $u_i^{(0)}$, $\dot{u}_i^{(0)}$, $\alpha_i^{(0)}$,
  $\dot{\alpha}_i^{(0)}$, $h_i^{(0)}$は既知。$u$は内部エネルギー、
  $\alpha$は人工粘性に用いる変数である。
\item $\bm{x}_i^{(1)}$を式(\ref{eq:ipos})で導出、$\bm{v}_i^{(1/2)}$,
  $u_i^{(1/2)}$ $\alpha_i^{(1/2)}$を式(\ref{eq:half})で導出、
  $h_i^{({\rm p})}=h_i^{(0)}$とし、$\bm{v}_i^{({\rm p})}$, $u_i^{({\rm
    p})}$, $\alpha_i^{({\rm p})}$は以下のように導出($X=\bm{v}$, $u$,
  $\alpha$)。ここで上付き添字$({\rm p})$は予測値であることを示す。
  \begin{align}
    X_i^{({\rm p})} &= X_i^{(0)} + \Delta t
    \dot{X}_i^{(0)} \label{eq:pred}
  \end{align}
\item 以下のループにより$\rho_i^{(1)}$, $h_i^{(1)}$, $(\partial \rho_i
  / \partial h_i)^{(1)}$, $(\nabla \cdot \bm{v}_i)^{(1)}$, $(\nabla
  \times \bm{v}_i)^{(1)}$を導出。
  \label{proc:lpsph}
  \begin{enumerate}
    \renewcommand{\labelenumii}{\ref{proc:lpdisph}.\arabic{enumii}.}
  \item 以下の式を用いて$\rho_i^{(1)}$, $(\partial \rho_i / \partial
    h_i)^{(1)}$, $(\nabla \cdot \bm{v}_i)^{(1)}$, $(\nabla \times
    \bm{v}_i)^{(1)}$を導出(上付き添字``$(1)$''は省略。以下、式中は同様)。
    このとき近傍粒子数$N_{{\rm n},i}$($j$粒子数と同義)もカウント。
    \begin{align}
      \rho_i &= \sum_j m_j W(\bm{x}_{ij},h_i^{({\rm
          p})}) \label{eq:rhoi} \\ \left( \frac{\partial
        \rho_i}{\partial h_i} \right) &= \sum_j m_j \frac{\partial
        W}{\partial h_i}(\bm{x}_{ij}, h_i^{({\rm p})}) \\ \rho_i
      \left( \nabla \cdot \bm{v}_i \right) &= - \sum_j m_j
      \bm{v}_{ij}^{({\rm p})} \cdot \nabla W(\bm{x}_{ij}, h_i^{({\rm
          p})}) \\ \rho_i \left( \nabla \times \bm{v}_i \right) &= -
      \sum_j m_j \bm{v}_{ij}^{({\rm p})} \times \nabla W(\bm{x}_{ij},
      h_i^{({\rm p})})
    \end{align}
  \item $\rho_i^{(1)}$, $m_i$, $h_i^{({\rm p})}$, $N_{{\rm n},i}$が拘束
    式を満たしたら、$h_i^{(1)}=h_i^{({\rm p})}$として次ステップへ。そう
    でなければ、$h_i^{({\rm p})}$を$\rho_i^{(1)}$と$m_i$から導出してス
    テップ3.1へ戻る。
  \end{enumerate}
\item 圧力$p_i^{(1)}$と音速$c_{{\rm s},i}^{(1)}$を$\rho_i^{(1)}$と
  $u_i^{({\rm p})}$から導出。
\item $\bm{a}_i^{(1)}$と$\dot{u}_i^{(1)}$を以下の式を用いて導出。
  \begin{align}
    \bm{a}_i &= - \sum_j m_j \left\{ f_i^{\rm grad}
    \frac{p_i}{\rho_i^2} \nabla W(\bm{x}_{ij},h_i) + f_j^{\rm grad}
    \frac{p_j}{\rho_j^2} \nabla W(\bm{x}_{ij},h_j) + \Pi_{ij} \nabla
    W(\bm{x}_{ij},h_{ij}) \right\} \\ \dot{u}_i &= f_i^{\rm grad}
    \frac{p_i}{ \rho_i^2} \sum_j m_j \bm{v}_{ij}^{({\rm p})} \cdot
    \nabla W(\bm{x}_{ij},h_i) + \frac{1}{2} \sum_j m_j \Pi_{ij}
    \bm{v}_{ij}^{({\rm p})} \cdot \nabla W(\bm{x}_{ij},h_{ij})
  \end{align}
  ここで、
  \begin{align}
    f_i^{\rm grad} &= \left( 1 + \frac{1}{3} \frac{h_i}{\rho_i}
    \frac{\partial \rho_i}{\partial h_i} \right)^{-1} \\
    \Pi_{ij} &= \left\{
    \begin{array}{cc}
      \displaystyle F_i \frac{-\alpha_i c_{{\rm s},ij} \mu_{ij} +
        \beta_i \mu_{ij}^2}{\rho_{ij}} & (\bm{x}_{ij} \cdot
      \bm{v}_{ij} < 0) \\ 0 & (\mbox{otherwise})
    \end{array}
    \right. \\
  \end{align}
  さらに
  \begin{align}
    \mu_{ij} &= \frac{h_{ij} \bm{x}_{ij} \cdot
      \bm{v}_{ij}}{|\bm{r}_{ij}|^2+\varepsilon h_{ij}^2} \\ F_i &=
    \frac{\nabla \cdot \bm{v}_i}{|\nabla \cdot \bm{v}_i| + |\nabla
      \times \bm{v}_i| + \varepsilon_{\rm b} c_{{\rm s}, i}/h_i} \\
    \beta_i &\propto \alpha_i
  \end{align}
\item $\dot{\alpha}_i^{(1)}$を以下の式から導出。
  \begin{align}
    \dot{\alpha}_i = - \frac{\alpha_i - \alpha_{\rm min}}{\tau_i} +
    \max \left[ -(\nabla \cdot \bm{v}_i) (\alpha_{\rm max} -
      \alpha_i), 0 \right] \label{eq:alphai}
  \end{align}
  ここで
  \begin{align}
    \tau_i = \frac{h_i}{\xi c_{{\rm s},i}}
  \end{align}
\item $\bm{v}_i^{(1)}$, $u_i^{(1)}$, $\alpha_i^{(1)}$を式
  (\ref{eq:corr})から導出。
\item 計算領域の分割と粒子の分配を実行。
\item ステップ1に戻る。
\end{enumerate}

以下のサンプルはステップ3の密度とカーネル半径を求めるコードである。
拘束条件はカーネル半径内の粒子数を一定とした。

\begin{lstlisting}[caption=stdSPHシミュレーションのサンプルコード]
#include<particle_simulator.h> // 必須

class ResultDens{
public:
    void clear(){ //絶対必要。名前固定。
        density = 0.0;
        r_search_next = 0.0;
        n_neighbour = 0;
        kernel_length = 0.0;
    }
    PS::F64 density;
    PS::F64 r_search_next;
    PS::S32 n_neighbour;
    PS::F64 kernel_length;
};

class FullPtcl{
public:
    PS::F64 getCharge() const { //絶対必要。名前固定。
        return this->mass;
    }
    PS::F64vec getPos() const { //絶対必要。名前固定。
        return this->pos;
    }
    PS::F64 getRSearch() const { //絶対必要。名前固定。
        return this->r_search;
    }
    void copyFromForce(const ResultDens & _dens){//絶対必要。名前固定。
        this->density = _dens.density;
        this->kernel_length = _dens.kernel_length;
    }

    //以下名前含めユーザー適宜。
    PS::S64 id;
    PS::F64 mass;
    PS::F64 kernel_length;
    PS::F64 r_search;
    PS::F64 density;
    PS::F64vec pos;
    PS::F64vec vel;
    PS::S32 loadOneParticle(FILE * fp) {
        PS::S32 ret = 0;
        ret = fscanf(fp, "%ld%lf%lf%lf%lf%lf%lf%lf%lf",
                     &this->id, &this->mass, &this->kernel_length,
                     &this->pos.x, &this->pos.y, &this->pos.z,
                     &this->vel.x, &this->vel.y, &this->vel.z);
        return ret;
    }
    void dumpOneParticle(FILE * fp){
        fprintf(fp, "%lf   %lf   %lf   %lf   %lf   %lf   %lf   \n", 
                this->mass,
                this->pos.x, this->pos.y, this->pos.z,
                this->vel.x, this->vel.y, this->vel.z);
    }
};

class EPIDens{
public:
    enum{
        n_neighbour_crit = 50,
    };
    PS::F64vec pos;
    PS::F64 r_search;

    void copyFromFP(const FullPtcl & rp){
        this->r_search = rp.getRSearch();
        this->pos = rp.getPos();
    }
};

class EPJDens{
public:
    PS::F64 mass;
    PS::F64vec pos;
    void copyFromFP(const FullPtcl & rp){
        this->mass = rp.getCharge();
        this->pos = rp.getPos();
    }
};

PS::F64 CubicSpline(const PS::F64 r_sq, 
                    const PS::F64 h_inv){
    PS::F64 xi = sqrt(r_sq)*h_inv;
    PS::F64 xi10 = (1.0-xi > 0.0) ? 1.0-xi : 0.0;
    PS::F64 xi05 = (0.5-xi > 0.0) ? 0.5-xi : 0.0;
    return xi10*xi10*xi10 - 4.0*xi05*xi05*xi05;
}

void CalcDensityEpEp(const EPIDens * ep_i,
                     const PS::S32   n_ip,
                     const EPJDens * ep_j,
                     const PS::S32   n_jp,
                     ResultDens * dens)
{
    std::vector < std::pair < PS::F64, PS::F64 > > r_neighbour_sq_with_mass;
    r_neighbour_sq_with_mass.reserve(EPIDens::n_neighbour_crit*3+100);
    for(PS::S32 i=0; i<n_ip; i++){
        if (ep_i[i].r_search == 0.0) continue; 
        r_neighbour_sq_with_mass.clear();
        const PS::F64 rsearch2 = ep_i[i].r_search*ep_i[i].r_search;
        for(PS::S32 j=0; j<n_ip; j++){
            const PS::F64vec dr = ep_i[i].pos - ep_j[j].pos;
            const PS::F64 r_sq = dr * dr;
            const PS::F64 mj = ep_j[j].mass;
            if( r_sq < rsearch2 ){
                r_neighbour_sq_with_mass.push_back( std::make_pair(r_sq, mj) );
            }
        }
        const size_t ncrit = EPIDens::n_neighbour_crit;
        if( r_neighbour_sq_with_mass.size() <=  ncrit ){
            // search radius was too small !!
            dens[i].r_search_next = ep_i[i].r_search * cbrt(2.0);
            continue;
        }
        else{
            dens[i].r_search_next = 0.0; // DON'T need to search next

            std::sort(r_neighbour_sq_with_mass.begin(), r_neighbour_sq_with_mass.end());
            dens[i].kernel_length = sqrt( r_neighbour_sq_with_mass[ncrit-1].first );
            PS::F64 tmp_dens = 0.0;
            PS::F64 h_inv = 1.0/dens[i].kernel_length;
            PS::F64 Cnorm = (16.0/M_PI) * (h_inv * h_inv * h_inv);
            for(PS::S32 k=0; k<PS::S32(ncrit); k++){
                tmp_dens += r_neighbour_sq_with_mass[k].second * CubicSpline(r_neighbour_sq_with_mass[k].first, h_inv);
            }
            dens[i].density = Cnorm * tmp_dens;
        }
    }
}

int main(int argc, char *argv[]){

    PS::Initialize(argc, argv); //初期化

    PS::TreeType<ResultDens, EPIDens, EPJDens>::TreeForGatherSearch tree_dens; // for density

    PS::ParticleSystem<FullPtcl> sph_system; //粒子種の数だけ生成

    sph_system.loadParticleSingle(argv[1], "r", &FullPtcl::loadOneParticle); //単一ファイル用

    PS::DomainInfo dinfo;
    dinfo.initialize("domain_info.para");
    tree_dens.initialize(sph_system.getNumberOfParticleTotal()); //引数は配列の領域確保に使う。

    dinfo.decomposeDomainAll(sph_system);

    PS::S32 nloc = sph_system.getNumberOfParticleLocal();
    //探査する半径を膨らます。
    for(PS::S32 i=0; i<nloc; i++){
        sph_system[i].r_search = sph_system[i].kernel_length;
    }
    for(bool repeat = true; repeat; ){
        tree_dens.calcForceAll(CalcDensityEpEp, dinfo, sph_system);
        PS::S32 nploc = tree_dens.getNumberOfParticleLocal();
        repeat = false;

        for(PS::S32 i=0; i<nploc; i++){
            ResultDens dens_tmp = tree_dens.getForce(i);
            sph_system[i].r_search    = dens_tmp.r_search_next;
            sph_system[i].density     = dens_tmp.density;
            sph_system[i].kernel_length = dens_tmp.kernel_length;
            if(dens_tmp.r_search_next > 0.0){
                repeat = true;
            }
        }
        repeat = PS::Comm::synchronizeConditionalBranch(repeat);
    }
    PS::Finalize();
    return 0;
}
\end{lstlisting}

相互作用クラスは、{\tt tree\_dens}であり、ツリークラスのタイプは{\tt
TreeForSearch}を使っている。for文の内側はカーネル半径(\tt
kernel\_length)を求めるためのループである。LET交換時に、推定されるカー
ネル半径よりやや大きめの値(\tt r\_search)を使ってLETをとってくることで、
通信が頻繁に起こるのを防いている。もし、(\tt r\_search)より内側にいる粒
子数が拘束条件を下回ったら、再び通信からやり直す。この際拘束条件を満た
した粒子は{\tt r\_search}に0を代入することで、無駄な通信を減らす。

\subsection{複数種類のシミュレーション}

$N$体+SPHシミュレーションを行う場合(sphは密度のみ求めてる)、以下のよう
に実装される。

\begin{lstlisting}[caption=$N$体+SPHシミュレーションのサンプルコード]
#include<particle_simulator.h> // 必須

// Gravity //
class ResultForce{ //名前自由。
public:
    void clear(){ //絶対必要。名前固定。
        acc = 0.0;
        pot = 0.0;
    }

    //以下名前含めユーザー定義。
    PS::F64vec acc;
    PS::F64 pot;
};

class GravFP{ //絶対必要。名前自由。
public:
    PS::F64vec getPos() const { //絶対必要。名前固定。
        return this->pos;
    }
    PS::F64 getCharge() const { //絶対必要。名前固定。
        return this->mass;
    }
    void copyFromForce(const ResultForce & _force){//絶対必要。名前固定。
        this->acc = _force.acc;
        this->pot = _force.pot;
    }

    //以下名前含めユーザー定義。
    PS::S64 id;
    PS::F64 mass;
    PS::F64vec pos;
    PS::F64vec vel;
    PS::F64vec acc;
    PS::F64 pot;
    PS::S32 loadOneParticle(FILE * fp) {
        PS::S32 ret = 0;
        ret = fscanf(fp, "%lf%lf%lf%lf%lf%lf%lf",
                     &this->mass,
                     &this->pos[0], &this->pos[1], &this->pos[2],
                     &this->vel[0], &this->vel[1], &this->vel[2]);
        std::cout<<"this->mass"<<this->mass<<std::endl;
        return ret;
    }
    void dumpOneParticle(FILE * fp){
        fprintf(fp, "%lf   %lf   %lf   %lf   %lf   %lf   %lf   \n", 
                this->mass,
                this->pos[0], this->pos[1], this->pos[2],
                this->vel[0], this->vel[1], this->vel[2]);
    }
};

class GravEPI{
public:
    void copyFromFP(const GravFP & rp){ //絶対必要。名前固定。
        pos = rp.pos;
        id = rp.id;
    }

    //以下名前含めユーザー定義。
    PS::F64vec pos;
    PS::S64 id;
};

class GravEPJ{
public:
    void copyFromFP(const GravFP & rp){ //絶対必要。名前固定。
        mass = rp.mass;
        pos = rp.pos;
        id = rp.id;
    }

    //以下名前含めユーザー定義。
    PS::S64 id;
    PS::F64 mass;
    PS::F64vec pos;
};

//相互作用関数
void CalcGravEpEp(const GravEPI * ep_i,
                   const PS::S32 n_ip,
                   const GravEPJ * ep_j,
                   const PS::S32 n_jp,
                   ResultForce * force){
    for(PS::S32 i=0; i<n_ip; i++){
        PS::F64vec xi = ep_i[i].pos;
        PS::F64vec ai = 0.0;
        PS::F64 poti = 0.0;
        PS::S64 idi = ep_i[i].id;
        for(PS::S32 j=0; j<n_jp; j++){
            if( idi == ep_j[j].id ) continue;
            PS::F64vec rij = xi - ep_j[j].pos;
            PS::F64 r3_inv = rij * rij;
            PS::F64 r_inv = 1.0/sqrt(r3_inv);
            r3_inv = r_inv * r_inv;
            r_inv *= ep_j[j].mass;
            r3_inv *= r_inv;
            ai -= r3_inv * rij;
            poti -= r_inv;
        }
        force[i].acc = ai;
        force[i].pot = poti;
    }
}

// SPH //
class ResultDens{
public:
    void clear(){ //絶対必要。名前固定。
        density = 0.0;
        r_search_next = 0.0;
        n_neighbour = 0;
        kernel_length = 0.0;
    }
    PS::F64 density;
    PS::F64 r_search_next;
    PS::S32 n_neighbour;
    PS::F64 kernel_length;
};

class DensFP{
public:
    PS::F64 getCharge() const { //絶対必要。名前固定。
        return this->mass;
    }
    PS::F64vec getPos() const { //絶対必要。名前固定。
        return this->pos;
    }
    PS::F64 getRSearch() const { //絶対必要。名前固定。
        return this->r_search;
    }
    void copyFromForce(const ResultDens & _dens){//絶対必要。名前固定。
        this->density = _dens.density;
        this->kernel_length = _dens.kernel_length;
    }

    //以下名前含めユーザー定義。
    PS::S64 id;
    PS::F64 mass;
    PS::F64 kernel_length;
    PS::F64 r_search;
    PS::F64 density;
    PS::F64vec pos;
    PS::F64vec vel;
    PS::S32 loadOneParticle(FILE * fp) {
        PS::S32 ret = 0;
        ret = fscanf(fp, "%ld%lf%lf%lf%lf%lf%lf%lf%lf",
                     &this->id, &this->mass, &this->kernel_length,
                     &this->pos.x, &this->pos.y, &this->pos.z,
                     &this->vel.x, &this->vel.y, &this->vel.z);
        return ret;
    }
    void dumpOneParticle(FILE * fp){
        fprintf(fp, "%lf   %lf   %lf   %lf   %lf   %lf   %lf   \n", 
                this->mass,
                this->pos.x, this->pos.y, this->pos.z,
                this->vel.x, this->vel.y, this->vel.z);
    }
};

class EPIDens{
public:
    void copyFromFP(const DensFP & rp){//絶対必要。名前固定。
        this->r_search = rp.getRSearch();
        this->pos = rp.getPos();
    }
    //以下名前含めユーザー定義。
    enum{
        n_neighbour_crit = 50,
    };
    PS::F64vec pos;
    PS::F64 r_search;

};

class EPJDens{
public:
    void copyFromFP(const GravFP & rp){//絶対必要。名前固定。
        this->mass = rp.getCharge();
        this->pos = rp.getPos();
    }
    //以下名前含めユーザー定義。
    PS::F64 mass;
    PS::F64vec pos;

};

// 相互作用関数(密度)
PS::F64 CubicSpline(const PS::F64 r_sq, 
                    const PS::F64 h_inv){
    PS::F64 xi = sqrt(r_sq)*h_inv;
    PS::F64 xi10 = (1.0-xi > 0.0) ? 1.0-xi : 0.0;
    PS::F64 xi05 = (0.5-xi > 0.0) ? 0.5-xi : 0.0;
    return xi10*xi10*xi10 - 4.0*xi05*xi05*xi05;
}

void CalcDensityEpEp(const EPIDens * ep_i,
                     const PS::S32   n_ip,
                     const EPJDens * ep_j,
                     const PS::S32   n_jp,
                     ResultDens * dens)
{
    std::vector < std::pair < PS::F64, PS::F64 > > r_neighbour_sq_with_mass;
    r_neighbour_sq_with_mass.reserve(EPIDens::n_neighbour_crit*3+100);
    for(PS::S32 i=0; i<n_ip; i++){
        if (ep_i[i].r_search == 0.0) continue; 
        r_neighbour_sq_with_mass.clear();
        const PS::F64 rsearch2 = ep_i[i].r_search*ep_i[i].r_search;
        for(PS::S32 j=0; j<n_ip; j++){
            const PS::F64vec dr = ep_i[i].pos - ep_j[j].pos;
            const PS::F64 r_sq = dr * dr;
            const PS::F64 mj = ep_j[j].mass;
            if( r_sq < rsearch2 ){
                r_neighbour_sq_with_mass.push_back( std::make_pair(r_sq, mj) );
            }
        }
        const size_t ncrit = EPIDens::n_neighbour_crit;
        if( r_neighbour_sq_with_mass.size() <=  ncrit ){
            // search radius was too small !!
            dens[i].r_search_next = ep_i[i].r_search * cbrt(2.0);
            continue;
        }
        else{
            dens[i].r_search_next = 0.0; // DON'T need to search next

            std::sort(r_neighbour_sq_with_mass.begin(), r_neighbour_sq_with_mass.end());
            dens[i].kernel_length = sqrt( r_neighbour_sq_with_mass[ncrit-1].first );
            PS::F64 tmp_dens = 0.0;
            PS::F64 h_inv = 1.0/dens[i].kernel_length;
            PS::F64 Cnorm = (16.0/M_PI) * (h_inv * h_inv * h_inv);
            for(PS::S32 k=0; k<PS::S32(ncrit); k++){
                tmp_dens += r_neighbour_sq_with_mass[k].second * CubicSpline(r_neighbour_sq_with_mass[k].first, h_inv);
            }
            dens[i].density = Cnorm * tmp_dens;
        }
    }
}

int main(int argc, char *argv[]){

    PS::Initialize(argc, argv); //初期化

    //領域分割クラスは1つ。
    PS::DomainInfo dinfo;
    dinfo.initialize("domain_info.para");

    // 粒子群クラス2種類生成
    PS::ParticleSystem<GravFP> nbody_system; 
    PS::ParticleSystem<DensFP> sph_system; 
    //ファイル読み込み。
    nbody_system.loadParticleSingle(argv[1], "r", &GravFP::loadOneParticle); 
    sph_system.loadParticleSingle(argv[2], "r", &DensFP::loadOneParticle); 

    // 相互作用クラス2種類生成
    PS::TreeType<ResultDens, EPIDens, EPJDens>::TreeForGahterSearch tree_dens; // 流体(密度計算)用
    PS::TreeType<ResultForce, GravEPI, GravEPJ>::TreeForMonoBaryCenter grav_tree; // 重力用
    // 領域確保
    tree_dens.initialize(sph_system.getNumberOfParticleTotal()); 
    grav_tree.initialize( nbody_system.getNumberOfParticleTotal() + sph_system.getNumberOfParticleTotal() ); 

    //複数種あるので、decomposeDomainAllは呼べない。
    dinfo.sampleParticle(nbody_system, true);
    dinfo.sampleParticle(sph_system, false);
    dinfo.decomposeDomain();

    //重力計算。
    grav_tree.calcForce(CalcGravEpEp, dinfo, nbody_system);

    //以下、密度計算。
    //探査する半径を膨らます。
    PS::S32 nloc = sph_system.getNumberOfParticleLocal();
    for(PS::S32 i=0; i<nloc; i++){
        sph_system[i].r_search = sph_system[i].kernel_length*1.2;
    }
    for(bool repeat = true; repeat; ){
        tree_dens.calcForceAll(CalcDensityEpEp, dinfo, sph_system);
        PS::S32 nploc = tree_dens.getNumberOfParticleLocal();
        repeat = false;
        for(PS::S32 i=0; i<nploc; i++){
            ResultDens dens_tmp = tree_dens.getForce(i);
            sph_system[i].r_search    = dens_tmp.r_search_next;
            sph_system[i].density     = dens_tmp.density;
            sph_system[i].kernel_length = dens_tmp.kernel_length;
            if(dens_tmp.r_search_next > 0.0){
                repeat = true;
            }
        }
        repeat = PS::Comm::synchronizeConditionalBranch(repeat);
    }

    PS::Finalize();
    return 0;
}

\end{lstlisting}
