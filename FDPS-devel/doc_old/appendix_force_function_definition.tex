ユーザーは任意の相互作用関数を書くことができる。関数名は任意であるが、
引数は第一引数から順に、I粒子へのポインタ、I粒子の個数、J粒子へのポイン
タ、J粒子の個数、forceへのポインタとなる。粒子の型と力の型は任意である
が、相互作用関数内で定義されている型と対応していなければならない。
以下は重力の場合の例である。

\begin{lstlisting}
void CalcForceEpEp(const EssentialPtclI * ep_i,
                   const PS::S32 n_ip,
                   const EssentialPtclJ * ep_j,
                   const PS::S32 n_jp,
                   Force * force){
    for(PS::S32 i=0; i<n_ip; i++){
        PS::F64vec xi = ep_i[i].pos;
        PS::F64vec ai = 0.0;
        PS::F64 poti = 0.0;
        PS::S32 idi = ep_i[i].id;
        for(PS::S32 j=0; j<n_jp; j++){
            if(idi == ep_j[j].id) continue;
            PS::F64vec rij = xi - ep_j[j].pos;
            PS::F64 r3_inv = rij * rij;
            PS::F64 r_inv = 1.0/sqrt(r3_inv);
            r3_inv = r_inv * r_inv;
            r_inv *= ep_j[j].mass;
            r3_inv *= r_inv;
            ai -= r3_inv * rij;
            poti -= r_inv;
        }
        force[i].acc += ai;
        force[i].pot += poti;
    }
}
\end{lstlisting}
