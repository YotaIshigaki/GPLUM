\documentclass[12pt,a4paper]{jarticle}
%
\topmargin=-5mm
\oddsidemargin=-5mm
\evensidemargin=-5mm
\textheight=235mm
\textwidth=165mm
%
\title{FDPS: Framework for Developing Particle Simulator}
\author{谷川衝、岩澤全規}
\date{}
%\pagestyle{empty}
\usepackage{graphicx}
\usepackage{wrapfig}
\usepackage{lscape}
\usepackage{amssymb}
\usepackage{amsmath}
\usepackage{bm}
\usepackage{setspace}
\usepackage{listings,jlisting}
\usepackage{color}
\usepackage{ascmac}
\usepackage{here}

\newcommand{\underbold}[1]{\underline{\bf #1}}
\newcommand{\redtext}[1]{\textcolor{red}{#1}}


%\setcounter{secnumdepth}{4}
%%%%%%%%%%%%%%%%%%%%%%%%%%%%%%%%%%
\setcounter{secnumdepth}{5}
\makeatletter
\newcommand{\subsubsubsection}{\@startsection{paragraph}{4}{\z@}%
{1.5\baselineskip \@plus.5\dp0 \@minus.2\dp0}%
{.5\baselineskip \@plus2.3\dp0}%
{\reset@font\normalsize\bfseries}
}
\newcommand{\subsubsubsubsection}{\@startsection{subparagraph}{5}{\z@}%
{1.5\baselineskip \@plus.5\dp0 \@minus.2\dp0}%
{.5\baselineskip \@plus2.3\dp0}%
{\reset@font\normalsize\itshape}
}
\makeatother
\setcounter{tocdepth}{5}
%%%%%%%%%%%%%%%%%%%%%%%%%%%%%%%%%%

%\twocolumn
%\setstretch{1.5}

\lstset{language = C,
numbers = left,
numbersep = 8pt,
breaklines = true,
breakindent = 40pt,
frame = lines,
basicstyle = \ttfamily,
}

\begin{document}
\maketitle
\tableofcontents

\section{change log}
\begin{itemize}
\item 2014年5月xx日 名前変更
\begin{itemize}
\item GhostParticleからEssentialParticleに。
\item BHParticleからTreeParticleに。
\item BHParticleからTreeCellに。
\item 型名は大文字始まりキャメルスタイルに。
\item メンバ関数名は小文字始まりキャメルスタイルに。
\item メンバ変数名は小文字でアンダースコアでつなぎアンダスコアで終わらす。
\end{itemize}
\item 2014年5月xx日 EP型をEPIとEPJに分離。
\item 2014年5月xx日 ドキュメント、APIと実装を分離。
\item 2014年5月xx日 領域の座標の表現を{\tt vector}クラス二つから{\tt rectangle}クラスに変更。
\item 2014年5月xx日 Nbodyのサンプルコード、STDSPH、Nbody+SPHのサンプルコードを新しく。継続中。
\item 2014年6月10日 {\tt Vector}型、{\tt Rectangle}型の記述を加える。
\item 2014年6月11日 {\tt Rectangle}型から{\tt Orthotope}型へ。
\item 2014年6月11日 マクロ{\tt PS\_DIM}から{\tt PARTICLE\_SIMULATOR\_DIMENSION}へ。
\item 2014年6月11日 {\tt Vector}型の説明更新。
\item 2014年6月11日 \ref{sec:includefile}更新。
\item 2014年6月25日 {\tt TreeForForce}クラスのテンプレートパラメータに{\tt search\_mode}を追加。
\item 2014年6月25日 {\tt TreeParticle}クラス、{\tt TreeCell}クラスを変更。
\item 2014年8月15日 境界条件API追加。
\item 2014年8月28日 境界条件の仕様拡張に伴う仕様変更(領域分割クラス、粒
  子群クラス、相互作用クラスの節のみ変更)
\item 2014年8月28日 周期境界条件などを用いる場合に現れる粒子のコピーを
  「イメージ粒子」と定義\redtext{(定義を文中に)}
\item 2014年8月28日 粒子の実体が存在しうる領域を「ルートドメイン」と定
  義、イメージ粒子が存在する領域を「イメージドメイン」と定義、粒子の実
  体に力を及ぼすイメージ粒子が存在する領域を「ホライゾン」と定義
  \redtext{(定義を文中に)}
\item 2014年8月28日 1プロセスが担当する領域を「領域」から「ドメイン」
  に変更\redtext{(定義を文中に)}
\item 2014年8月29日 境界条件指定方法を\ref{sec:includefile}に追加。
\item 2014年8月29日 相互作用クラスの相互作用関数({\tt TreeForForce::calcForce()}等)をテンプレート関数に変更し、関数オブジェクトも使える様に仕様変更。
\item 2014年10月6日 領域クラスのAPI変更。
\item 2014年12月19日 treeforforceのテンプレート引数を変更。
\item 2015年2月12日 File I/O周りのAPI変更。
\end{itemize}

\section{TO DO}
\begin{itemize}
\item イメージ粒子に対する粒子の実体の名前を決定する。
\item ネイバーリスト検討。
\end{itemize}

\section{FDPSとは}
FDPSは任意の粒子シミュレーションコードの開発を支援するフレームワークで
ある。FDPSユーザ(以下、ユーザ)が1粒子の持つ情報と粒子間の相互作用の形
などをFDPSコードジェネレータ(以下、コードジェネレータ)に入力すると、粒
子シミュレーション用のライブラリ(PSライブラリ、以後PSL)が出力される。ユー
ザはPSLを基に粒子シミュレーションコードを書くことができる。PSLは、任意
の粒子と相互作用する粒子群の探査と、それらの相互作用の計算を、あらゆる
階層の並列化(マルチプロセス、マルチスレッド、SIMD演算)を用いて高速に行
う関数群を提供する。PSL使用の最大の利点は、ユーザが粒子シミュレーション
コードを高速化を意識せずに作成できることである。

PSLはC++で記述されている。ユーザはC++を用いて粒子シミュレーションコード
を記述することが推奨される。

PSLは2粒子間相互作用の計算のみサポートする。3粒子以上の相互作用はサポー
トしない。また、独立時間刻み法もサポートしない。ただし、近傍粒子を返す
APIを用意するので、ユーザがP$^3$T法を用いて独立時間刻み法を実装すること
は可能である。

本文書では、PSLについて記述する。粒子間相互作用の形の定義の仕方などは必
要がない限り記述しない。コードジェネレータについては全く触れない。本文
書の構成は以下の通りである。\ref{sec:algorithm}節では、PSLで実行される
ことについて簡潔に記述する。\ref{sec:api}節では、PSL のAPIを示す。
\ref{sec:sample}節では、PSLを使用した粒子シミュレーションコードの例を示
す。最後に\ref{sec:roadmap}節では、FDPS作成のロードマップを記述する。


\section{PSL}
\label{sec:algorithm}
本節では、PSLが実行すること、またそのアルゴリズムについて記述する。一般
に、分散メモリ環境での粒子シミュレーションの手順は以下の3つに分けられ
る。
\begin{enumerate}
\item[0.] 各プロセスの役割分担を決定 \label{proc:decompexchange}
\item[1.] $i$粒子に対する$j$粒子リストを作成 \label{proc:jlist}
\item[2.] $i$粒子に対する$j$粒子の作用を計算し、$i$粒子の物理量とその時間変
  化量を導出 \label{proc:interact}
\item[3.] $i$粒子の物理量を時間積分 \label{proc:integrate}
\end{enumerate}
ここで、作用される粒子を$i$粒子、作用する粒子を$j$粒子と呼び、ある$i$粒
子に対する全$j$粒子を$j$粒子リストと呼んだ。PSLは手順0から2を行う関数群
を提供する。

手順0は以下2つの手順に分割される。
\begin{enumerate}
\renewcommand{\labelenumi}{0.\arabic{enumi}.}
\item 各プロセスの担当粒子を決定 \label{proc:decomp}
\item 各プロセスにそれぞれの担当粒子を分配 \label{proc:exchange}
\end{enumerate}
PSLでは、それぞれの手順はGreeMとほぼ同様のアルゴリズムを用いる。

手順1は、さらに以下のような手順に分割される。
\begin{enumerate}
\renewcommand{\labelenumi}{1.\arabic{enumi}.}
\item 各プロセスが自分の担当粒子の探査を高速にするデータ構造を構
  築 \label{proc:lstruct}
\item 各プロセスが自分の担当粒子のうち別プロセスの担当粒子の$j$粒子リス
  トに入る可能性のある粒子を探査し、そうであればその粒子データを別プロ
  セスへ送信 \label{proc:communicate}
\item 各プロセスが自分の担当粒子と別プロセスから送信されてきた粒子の粒
  子探査を高速にするデータ構造を構築
  \label{proc:gstruct}
\item 各プロセスが担当粒子の$j$粒子リストを作成 \label{proc:jlistd}
\end{enumerate}

PSLでは、手順1.1と手順1.3でツリー構造が構築される。前者のツリーをローカ
ルツリー、後者のツリーをグローバルツリーと呼ぶ。

手順1.2と手順1.4で行われる粒子探査は、扱う相互作用が短距離力か長距離力
かで大きく異なる。ここで短距離力とは力の到達距離が有限である場合であり、
長距離力とは力の到達距離が無限である場合である。短距離力において探査す
るのは粒子そのものである。一方長距離力においては、粒子そのものだけでな
く、遠くの複数の粒子をまとめた超粒子も探査する。ただし、粒子に超粒子が
作用しつつ力の到達距離が有限となる場合がある。例としてはTreePM法を用い
た$N$体シミュレーションである。PSLでは、これは長距離力に分類される。

長距離力の場合の手順1.2のアルゴリズムは、GreeMに準じる。短距離力の場合
の手順1.2のアルゴリズムは、相互作用の性質に応じて場合分けされる。相互作
用の性質は以下の4種類が存在する。
\begin{itemize}
\item 力の到達距離が全粒子で等しい相互作用 (固定長モード)
\item 力の到達距離が$i$粒子のサイズで決まる相互作用(収集モード)
\item 力の到達距離が$j$粒子のサイズで決まる相互作用(散乱モード)
\item 力の到達距離が$ij$粒子の両方のサイズで決まる相互作用 (対称モード)
\end{itemize}
固定長モードのアルゴリズムはまだ決めていない。他のモードはASURAに準じる。

PSLでは手順2においてSIMD演算が用いられる。SIMD演算を用いて計算性能を得
るために、複数の$i$粒子が共通の$j$粒子リストを持つようにする。本文書で
は、PSLにおけるSIMD演算の実装については言及しない。

\section{API}
\label{sec:api}

PSLは前節で示した手順0、1、2を実行するクラスを提供する。クラスは3つあ
る。1つ目は手順0.1を実行する領域分割クラス{\tt DomainInfo}、2つ目は手
順0.2を実行する粒子群クラス{\tt ParticleSystem}、3つ目は手順1、2を実行
する相互作用クラス{\tt TreeforForce}である。まずこれらのクラスが属する
名前空間について述べる。次に、初期化等の関数について触れる。次にシミュ
レーションを行う上でコアとなる上記3つのクラスそれぞれについて記述する。
これらのクラス内部では、MPI、OpenMP、SIMDなど様々な並列化手法が用いられ
ている。最後にこれら並列化手法を最適化するためのパラメータ等を管理する
コミュニケーションクラス{\tt Comm}についてを述べる。

PSLでは計算や通信の効率化の為、粒子情報やツリーの情報をいくつかの小さい
サブクラスに持たしている。これらの定義は付録
\ref{sec:particle_subclass}節と\ref{sec:tree_subclass}節に記されている。

\subsection{インクルードファイル}
\label{sec:includefile}

ユーザーはPSLを使う為に{\tt particle\_simulator.h}というヘッダーファイ
ルをインクルードしなければならない。このファイルをインクルードする事で、
以下に記述されるAPIを使うことが出来る。記述されているAPIはすべて{\tt
ParticleSimulator}という名前空間に属する。これでは長いので、省略名{\tt
PS}も使うことが出来る。ユーザーが{\tt ParticleSimulator}と{\tt PS}とい
う名前の名前空間を定義する事は出来ない。

ユーザはコンパイル時に{\tt PARTICLE\_SIMULATOR\_TWO\_DIMENSION}を定義す
る事で、2次元用のPSLを使うことが出来る。何も定義しない場合は3次元用の
PSLを使うことが出来る。

{\tt particle\_simulator.h}は以下の様に記述される。

\begin{lstlisting}[caption={\tt particle\_simulator.h}]
namespace ParticleSimulator{
#ifdef PARTICLE_SIMULATOR_TOW_DIMENSION
    static const int DIMENSION = 2;
#else
    static const int DIMENSION = 3;
#endif
}

#include<ps_defs.hpp>
#include<domain_info.hpp>
#include<particle_system.hpp>
#include<tree_for_force.hpp>
namespace PS = ParticleSimulator;
\end{lstlisting}

{\tt domain\_info.hpp}, {\tt particle\_system.hpp}, {\tt
tree\_for\_force.hpp}はそれぞれ、領域クラス{\tt DomainInfo}、粒子群クラ
ス{\tt ParticleSystem}、相互作用クラス{\tt TreeforForce}について記述さ
れているヘッダーファイルである。

すでにユーザー使っているライブラリなどで、これらの名前が使われている場
合は、ユーザーは以下の様に{\tt PS}を包含する名前空間の中で{\tt
particle\_simulator.h}をインクルードする事で{\tt PS}をネストさせ回避す
ることが出来る。

\begin{lstlisting}[caption=名前空間の衝突の回避方法]
namespce hoge{
    #include<particle_simulator.h>
}
\end{lstlisting}

これにより、ユーザーは{\tt PSL}のAPIには{\tt hoge::PS::}という形で各種
APIにアクセスできるようになる。

\subsection{変数型の定義}
\label{sec:def_variable}

PSLでは以下のように変数型を提供する。以下に出てくる{\tt Vector2}、{\tt
Vector3}型については次節で説明する。

\begin{lstlisting}[caption=変数型]
namespace ParticleSimulator{
    enum SEARCH_MODE{
        SEARCH_MODE_LONG,
        SEARCH_MODE_LONG_CUTOFF,
        SEARCH_MODE_GATHER,
        SEARCH_MODE_SCATTER,
        SEARCH_MODE_SYMMETRY,
        SEARCH_MODE_FIXED_LENGTH,
    };

    enum BOUNDARY_CONDITION{
        BOUNDARY_CONDITION_PRRIODIC____,
        BOUNDARY_CONDITION_PRRIODIC_X__,
        BOUNDARY_CONDITION_PRRIODIC__Y_,
        BOUNDARY_CONDITION_PRRIODIC___Z,
        BOUNDARY_CONDITION_PRRIODIC_XY_,
        BOUNDARY_CONDITION_PRRIODIC_X_Z,
        BOUNDARY_CONDITION_PRRIODIC__YZ,
        BOUNDARY_CONDITION_PRRIODIC_XYZ,
        BOUNDARY_CONDITION_SHEARING_BOX,
        BOUNDARY_CONDITION_USER_DEFINED,
    };

    typedef int S32;
    typedef unsigned int U32;
    typedef float F32;
    typedef long S64;
    typedef unsigned long U64;
    typedef double F64;
    typedef Vector2<F32> F32vec2;
    typedef Vector3<F32> F32vec3;
    typedef Vector2<F64> F64vec2;
    typedef Vector3<F64> F64vec3;
#ifdef PARTICLE_SIMULATOR_TOW_DIMENSION
    typedef F32vec2 F32vec;
    typedef F64vec2 F64vec;
    typedef F32ort2 F32ort;
    typedef F64ort2 F64ort;
    static const S32 N_CHILDREN = 4;
#else
    typedef F32vec3 F32vec;
    typedef F64vec3 F64vec;
    typedef F32ort3 F32ort;
    typedef F64ort3 F64ort;
    static const S32 N_CHILDREN = 8;
#endif
    MPI::Datatype MPI_F32VEC;
    MPI::Datatype MPI_F64VEC;
}
\end{lstlisting}

\subsection{{\tt Vector}型}

\subsubsection{{\tt PS::Vector2}型} 

x,yの2要素を持つ。それらの要素の型は
{PS::S32},{PS::S64},{PS::U32},{PS::U64},{PS::F32},{PS::F64}に限られる。
外積演算も定義しており、結果はスカラーとして返す。このクラスは以下の様
に記述される。

\begin{lstlisting}[caption=Vector2]
namespace ParticleSimulator{
    template <typename T>
    class Vector2{
    public:
        //メンバ変数は以下の二つのみ。
        T x, y;

        //コンストラクタ
        Vector2() : x(T(0)), y(T(0)) {}
        Vector2(const T _x, const T _y) : x(_x), y(_y) {}
        Vector2(const T s) : x(s), y(s) {}
        Vector2(const Vector2 & src) : x(src.x), y(src.y) {}

        //代入演算子
        const Vector2 & operator = (const Vector2 & rhs);

        //加減算
        Vector2 operator + (const Vector2 & rhs) const;
        const Vector2 & operator += (const Vector2 & rhs);
        Vector2 operator - (const Vector2 & rhs) const;
        const Vector2 & operator -= (const Vector2 & rhs);

        //ベクトルスカラ積
        Vector2 operator * (const T s) const;
        const Vector2 & operator *= (const T s);
        friend Vector2 operator * (const T s, const Vector2 & v);
        Vector2 operator / (const T s) const;
        const Vector2 & operator /= (const T s);

        //内積
        T operator * (const Vector2 & rhs) const;

        //外積(返り値はスカラ!!)
        T operator ^ (const Vector2 & rhs) const;

        //Vector2<U>への型変換
        template <typename U>
        operator Vector2<U> () const;
    };
}
\end{lstlisting}
%%%%%%%%%%%%%%%%%%%%%%%%%%%%%
\paragraph{コンストラクタ}
\mbox{}
%%%%%%%%%%%%%%%%%%%%%%%%%%%%%
%%%%%%%%%%%%%%%%%%%%%%%%%%%%%
\begin{screen}
\begin{verbatim}
template<typename T>
PS::Vector2<T>()
\end{verbatim}
\end{screen}

\begin{itemize}

\item{{\bf 引数}}

なし。

\item{{\bf 機能}}

デフォルトコンストラクタ。メンバx,yは0で初期化される。

\end{itemize}

%%%%%%%%%%%%%%%%%%%%%%%%%%%%%
\begin{screen}
\begin{verbatim}
template<typename T>
PS::Vector2<T>(const T _x, const T _y)
\end{verbatim}
\end{screen}

\begin{itemize}

\item{{\bf 引数}}

{\tt \_x}: 入力。{\tt const T}型。

{\tt \_y}: 入力。{\tt const T}型。

\item{{\bf 機能}}

メンバ{\tt x}、{\tt y}をそれぞれ{\tt \_x}、{\tt \_y}で初期化する。

\end{itemize}

%%%%%%%%%%%%%%%%%%%%%%%%%%%%%
\begin{screen}
\begin{verbatim}
template<typename T>
PS::Vector2<T>(const T s);
\end{verbatim}
\end{screen}

\begin{itemize}

\item{{\bf 引数}}

{\tt s}: 入力。{\tt const T}型。

\item{{\bf 機能}}

メンバ{\tt x}、{\tt y}を両方とも{\tt s}の値で初期化する。

\end{itemize}

%%%%%%%%%%%%%%%%%%%%%%%%%%%%%
\begin{screen}
\begin{verbatim}
template<typename T>
PS::Vector2<T>(const PS::Vector2<T> & src)
\end{verbatim}
\end{screen}

\begin{itemize}

\item{{\bf 引数}}

{\tt src}: 入力。{\tt const PS::Vector2<T> \&}型。

\item{{\bf 機能}}

コピーコンストラクタ。{\tt src}で初期化する。

\end{itemize}

%%%%%%%%%%%%%%%%%%%%%%%%%%%%%
\paragraph{代入演算子}
\mbox{}
%%%%%%%%%%%%%%%%%%%%%%%%%%%%%

%%%%%%%%%%%%%%%%%%%%%%%%%%%%%
\begin{screen}
\begin{verbatim}
template<typename T>
const PS::Vector2<T> & PS::Vector2<T>::operator = 
                       (const PS::Vector2<T> & rhs);
\end{verbatim}
\end{screen}

\begin{itemize}

\item{{\bf 引数}}

{\tt rhs}: 入力。{\tt const PS::Vector2<T> \&}型。

\item{{\bf 返り値}}

{\tt const PS::Vector2<T> \&}型。{\tt rhs}のx,yの値を自身のメンバx,yに
代入し自身の参照を返す。代入演算子。

\end{itemize}


%%%%%%%%%%%%%%%%%%%%%%%%%%%%%
\paragraph{加減算}
\mbox{}
%%%%%%%%%%%%%%%%%%%%%%%%%%%%%

%%%%%%%%%%%%%%%%%%%%%%%%%%%%%
\begin{screen}
\begin{verbatim}
template<typename T>
PS::Vector2<T> PS::Vector2<T>::operator + 
               (const PS::Vector2<T> & rhs) const;
\end{verbatim}
\end{screen}

\begin{itemize}

\item{{\bf 引数}}

{\tt rhs}: 入力。{\tt const PS::Vector2<T> \&}型。

\item{{\bf 返り値}}

{\tt PS::Vector2<T> }型。{\tt rhs}のx,yの値と自身のメンバx,yの値の和を
取った値を返す。

\end{itemize}


%%%%%%%%%%%%%%%%%%%%%%%%%%%%%
\begin{screen}
\begin{verbatim}
template<typename T>
const PS::Vector2<T> & PS::Vector2<T>::operator += 
                       (const PS::Vector2<T> & rhs);
\end{verbatim}
\end{screen}

\begin{itemize}

\item{{\bf 引数}}

{\tt rhs}: 入力。{\tt const PS::Vector2<T> \&}型。

\item{{\bf 返り値}}

{\tt const PS::Vector2<T> \&}型。{\tt rhs}のx,yの値を自身のメンバx,yに足し、自
身を返す。

\end{itemize}


%%%%%%%%%%%%%%%%%%%%%%%%%%%%%
\begin{screen}
\begin{verbatim}
template<typename T>
PS::Vector2<T> PS::Vector2<T>::operator - 
               (const PS::Vector2<T> & rhs) const;
\end{verbatim}
\end{screen}

\begin{itemize}

\item{{\bf 引数}}

{\tt rhs}: 入力。{\tt const PS::Vector2<T> \&}型。

\item{{\bf 返り値}}

{\tt PS::Vector2<T> }型。{\tt rhs}のx,yの値と自身のメンバx,yの値の差を
取った値を返す。

\end{itemize}


%%%%%%%%%%%%%%%%%%%%%%%%%%%%%
\begin{screen}
\begin{verbatim}
template<typename T>
const PS::Vector2<T> & PS::Vector2<T>::operator += 
                       (const PS::Vector2<T> & rhs);
\end{verbatim}
\end{screen}

\begin{itemize}

\item{{\bf 引数}}

{\tt rhs}: 入力。{\tt const PS::Vector2<T> \&}型。

\item{{\bf 返り値}}

{\tt const PS::Vector2<T> \&}型。自身のメンバx,yから{\tt rhs}のx,yを引
き自身を返す。

\end{itemize}

%%%%%%%%%%%%%%%%%%%%%%%%%%%%%
\paragraph{ベクトルスカラ積}
\mbox{}
%%%%%%%%%%%%%%%%%%%%%%%%%%%%%

%%%%%%%%%%%%%%%%%%%%%%%%%%%%%
\begin{screen}
\begin{verbatim}
template<typename T>
PS::Vector2<T> PS::Vector2<T>::operator * (const T s) const;
\end{verbatim}
\end{screen}

\begin{itemize}

\item{{\bf 引数}}

{\tt s}: 入力。{\tt const T}型。

\item{{\bf 返り値}}

{\tt PS::Vector2<T>}型。自身のメンバx,yそれぞれに{\tt s}をかけた値を返
す。

\end{itemize}


%%%%%%%%%%%%%%%%%%%%%%%%%%%%%
\begin{screen}
\begin{verbatim}
template<typename T>
const PS::Vector2<T> & PS::Vector2<T>::operator *= (const T s);
\end{verbatim}
\end{screen}

\begin{itemize}

\item{{\bf 引数}}

{\tt rhs}: 入力。{\tt const T}型。

\item{{\bf 返り値}}

{\tt const PS::Vector2<T> \&}型。自身のメンバx,yそれぞれに{\tt s}をかけ
自身を返す。

\end{itemize}


%%%%%%%%%%%%%%%%%%%%%%%%%%%%%
\begin{screen}
\begin{verbatim}
template<typename T>
PS::Vector2<T> PS::Vector2<T>::operator / (const T s) const;
\end{verbatim}
\end{screen}

\begin{itemize}

\item{{\bf 引数}}

{\tt s}: 入力。{\tt const T}型。

\item{{\bf 返り値}}

{\tt PS::Vector2<T>}型。自身のメンバx,yそれぞれを{\tt s}で割った値を返
す。

\end{itemize}


%%%%%%%%%%%%%%%%%%%%%%%%%%%%%
\begin{screen}
\begin{verbatim}
template<typename T>
const PS::Vector2<T> & PS::Vector2<T>::operator /= (const T s);
\end{verbatim}
\end{screen}

\begin{itemize}

\item{{\bf 引数}}

{\tt rhs}: 入力。{\tt const T}型。

\item{{\bf 返り値}}

{\tt const PS::Vector2<T> \&}型。自身のメンバx,yそれぞれを{\tt s}で割り
自身を返す。

\end{itemize}


%%%%%%%%%%%%%%%%%%%%%%%%%%%%%
\paragraph{内積、外積}
\mbox{}
%%%%%%%%%%%%%%%%%%%%%%%%%%%%%

%%%%%%%%%%%%%%%%%%%%%%%%%%%%%
\begin{screen}
\begin{verbatim}
template<typename T>
T PS::Vector2<T>::operator * (const PS::Vector2<T> & rhs) const;
\end{verbatim}
\end{screen}

\begin{itemize}

\item{{\bf 引数}}

{\tt rhs}: 入力。{\tt const PS::Vector2<T> \&}型。

\item{{\bf 返り値}}

{\tt T}型。自身と{\tt rhs}の内積を取った値を返す。

\end{itemize}

%%%%%%%%%%%%%%%%%%%%%%%%%%%%%
\begin{screen}
\begin{verbatim}
template<typename T>
T PS::Vector2<T>::operator ^ (const PS::Vector2<T> & rhs) const;
\end{verbatim}
\end{screen}

\begin{itemize}

\item{{\bf 引数}}

{\tt rhs}: 入力。{\tt const PS::Vector2<T> \&}型。

\item{{\bf 返り値}}

{\tt T}型。自身と{\tt rhs}の外積を取った値を返す。

\end{itemize}


%%%%%%%%%%%%%%%%%%%%%%%%%%%%%
\paragraph{{\tt Vector2<U>}への型変換}
\mbox{}
%%%%%%%%%%%%%%%%%%%%%%%%%%%%%

%%%%%%%%%%%%%%%%%%%%%%%%%%%%%
\begin{screen}
\begin{verbatim}
template<typename T>
template <typename U>
PS::Vector2<T>::operator PS::Vector2<U> () const;
\end{verbatim}
\end{screen}

\begin{itemize}

\item{{\bf 引数}}

{\tt rhs}: 入力。{\tt const PS::Vector2<T> \&}型。

\item{{\bf 返り値}}

{\tt const PS::Vector2<U> \&}型。

\item{{\bf 機能}}

{\tt const PS::Vector2<T> \&}型を{\tt const PS::Vector2<U> \&}型にキャ
ストする。

\end{itemize}


%%%%%%%%%%%%%%%%%%%%%%%%%%%%
%%%%%%%%% Vectror3 %%%%%%%%%
%%%%%%%%%%%%%%%%%%%%%%%%%%%%

\subsubsection{{\tt PS::Vector3}型}

x,y,zの3要素を持つ。それらの要素の型は
{PS::S32},{PS::S64},{PS::U32},{PS::U64},{PS::F32},{PS::F64}に限られる。
定義されているメソッド等は{\tt Vector2}型と同じであるが外積は{\tt
Vector2}型と違い{\tt Vector3}型で返す。このクラスは以下の様に記述される。

\begin{lstlisting}[caption=Vector3]
namespace ParticleSimulator{
    template <typename T>
    class Vector3{
    public:
        //メンバ変数は以下の二つのみ。
        T x, y;

        //コンストラクタ
        Vector3() : x(T(0)), y(T(0)) {}
        Vector3(const T _x, const T _y) : x(_x), y(_y) {}
        Vector3(const T s) : x(s), y(s) {}
        Vector3(const Vector3 & src) : x(src.x), y(src.y) {}

        //代入演算子
        const Vector3 & operator = (const Vector3 & rhs);

        //加減算
        Vector3 operator + (const Vector3 & rhs) const;
        const Vector3 & operator += (const Vector3 & rhs);
        Vector3 operator - (const Vector3 & rhs) const;
        const Vector3 & operator -= (const Vector3 & rhs);

        //ベクトルスカラ積
        Vector3 operator * (const T s) const;
        const Vector3 & operator *= (const T s);
        friend Vector3 operator * (const T s, const Vector3 & v);
        Vector3 operator / (const T s) const;
        const Vector3 & operator /= (const T s);

        //内積
        T operator * (const Vector3 & rhs) const;

        //外積(返り値はスカラ!!)
        T operator ^ (const Vector3 & rhs) const;

        //Vector3<U>への型変換
        template <typename U>
        operator Vector3<U> () const;
    };
}
\end{lstlisting}
%%%%%%%%%%%%%%%%%%%%%%%%%%%%%
\paragraph{コンストラクタ}
\mbox{}
%%%%%%%%%%%%%%%%%%%%%%%%%%%%%
%%%%%%%%%%%%%%%%%%%%%%%%%%%%%
\begin{screen}
\begin{verbatim}
template<typename T>
PS::Vector3<T>()
\end{verbatim}
\end{screen}

\begin{itemize}

\item{{\bf 引数}}

なし。

\item{{\bf 機能}}

デフォルトコンストラクタ。メンバx,yは0で初期化される。

\end{itemize}

%%%%%%%%%%%%%%%%%%%%%%%%%%%%%
\begin{screen}
\begin{verbatim}
template<typename T>
PS::Vector3<T>(const T _x, const T _y)
\end{verbatim}
\end{screen}

\begin{itemize}

\item{{\bf 引数}}

{\tt \_x}: 入力。{\tt const T}型。

{\tt \_y}: 入力。{\tt const T}型。

\item{{\bf 機能}}

メンバ{\tt x}、{\tt y}をそれぞれ{\tt \_x}、{\tt \_y}で初期化する。

\end{itemize}

%%%%%%%%%%%%%%%%%%%%%%%%%%%%%
\begin{screen}
\begin{verbatim}
template<typename T>
PS::Vector3<T>(const T s);
\end{verbatim}
\end{screen}

\begin{itemize}

\item{{\bf 引数}}

{\tt s}: 入力。{\tt const T}型。

\item{{\bf 機能}}

メンバ{\tt x}、{\tt y}を両方とも{\tt s}の値で初期化する。

\end{itemize}

%%%%%%%%%%%%%%%%%%%%%%%%%%%%%
\begin{screen}
\begin{verbatim}
template<typename T>
PS::Vector3<T>(const PS::Vector3<T> & src)
\end{verbatim}
\end{screen}

\begin{itemize}

\item{{\bf 引数}}

{\tt src}: 入力。{\tt const PS::Vector3<T> \&}型。

\item{{\bf 機能}}

コピーコンストラクタ。{\tt src}で初期化する。

\end{itemize}

%%%%%%%%%%%%%%%%%%%%%%%%%%%%%
\paragraph{代入演算子}
\mbox{}
%%%%%%%%%%%%%%%%%%%%%%%%%%%%%

%%%%%%%%%%%%%%%%%%%%%%%%%%%%%
\begin{screen}
\begin{verbatim}
template<typename T>
const PS::Vector3<T> & PS::Vector3<T>::operator = 
                       (const PS::Vector3<T> & rhs);
\end{verbatim}
\end{screen}

\begin{itemize}

\item{{\bf 引数}}

{\tt rhs}: 入力。{\tt const PS::Vector3<T> \&}型。

\item{{\bf 返り値}}

{\tt const PS::Vector3<T> \&}型。{\tt rhs}のx,yの値を自身のメンバx,yに
代入し自身の参照を返す。代入演算子。

\end{itemize}


%%%%%%%%%%%%%%%%%%%%%%%%%%%%%
\paragraph{加減算}
\mbox{}
%%%%%%%%%%%%%%%%%%%%%%%%%%%%%

%%%%%%%%%%%%%%%%%%%%%%%%%%%%%
\begin{screen}
\begin{verbatim}
template<typename T>
PS::Vector3<T> PS::Vector3<T>::operator + 
               (const PS::Vector3<T> & rhs) const;
\end{verbatim}
\end{screen}

\begin{itemize}

\item{{\bf 引数}}

{\tt rhs}: 入力。{\tt const PS::Vector3<T> \&}型。

\item{{\bf 返り値}}

{\tt PS::Vector3<T> }型。{\tt rhs}のx,yの値と自身のメンバx,yの値の和を
取った値を返す。

\end{itemize}


%%%%%%%%%%%%%%%%%%%%%%%%%%%%%
\begin{screen}
\begin{verbatim}
template<typename T>
const PS::Vector3<T> & PS::Vector3<T>::operator += 
                       (const PS::Vector3<T> & rhs);
\end{verbatim}
\end{screen}

\begin{itemize}

\item{{\bf 引数}}

{\tt rhs}: 入力。{\tt const PS::Vector3<T> \&}型。

\item{{\bf 返り値}}

{\tt const PS::Vector3<T> \&}型。{\tt rhs}のx,yの値を自身のメンバx,yに足し、自
身を返す。

\end{itemize}


%%%%%%%%%%%%%%%%%%%%%%%%%%%%%
\begin{screen}
\begin{verbatim}
template<typename T>
PS::Vector3<T> PS::Vector3<T>::operator - 
               (const PS::Vector3<T> & rhs) const;
\end{verbatim}
\end{screen}

\begin{itemize}

\item{{\bf 引数}}

{\tt rhs}: 入力。{\tt const PS::Vector3<T> \&}型。

\item{{\bf 返り値}}

{\tt PS::Vector3<T> }型。{\tt rhs}のx,yの値と自身のメンバx,yの値の差を
取った値を返す。

\end{itemize}


%%%%%%%%%%%%%%%%%%%%%%%%%%%%%
\begin{screen}
\begin{verbatim}
template<typename T>
const PS::Vector3<T> & PS::Vector3<T>::operator += 
                       (const PS::Vector3<T> & rhs);
\end{verbatim}
\end{screen}

\begin{itemize}

\item{{\bf 引数}}

{\tt rhs}: 入力。{\tt const PS::Vector3<T> \&}型。

\item{{\bf 返り値}}

{\tt const PS::Vector3<T> \&}型。自身のメンバx,yから{\tt rhs}のx,yを引
き自身を返す。

\end{itemize}

%%%%%%%%%%%%%%%%%%%%%%%%%%%%%
\paragraph{ベクトルスカラ積}
\mbox{}
%%%%%%%%%%%%%%%%%%%%%%%%%%%%%

%%%%%%%%%%%%%%%%%%%%%%%%%%%%%
\begin{screen}
\begin{verbatim}
template<typename T>
PS::Vector3<T> PS::Vector3<T>::operator * (const T s) const;
\end{verbatim}
\end{screen}

\begin{itemize}

\item{{\bf 引数}}

{\tt s}: 入力。{\tt const T}型。

\item{{\bf 返り値}}

{\tt PS::Vector3<T>}型。自身のメンバx,yそれぞれに{\tt s}をかけた値を返
す。

\end{itemize}


%%%%%%%%%%%%%%%%%%%%%%%%%%%%%
\begin{screen}
\begin{verbatim}
template<typename T>
const PS::Vector3<T> & PS::Vector3<T>::operator *= (const T s);
\end{verbatim}
\end{screen}

\begin{itemize}

\item{{\bf 引数}}

{\tt rhs}: 入力。{\tt const T}型。

\item{{\bf 返り値}}

{\tt const PS::Vector3<T> \&}型。自身のメンバx,yそれぞれに{\tt s}をかけ
自身を返す。

\end{itemize}


%%%%%%%%%%%%%%%%%%%%%%%%%%%%%
\begin{screen}
\begin{verbatim}
template<typename T>
PS::Vector3<T> PS::Vector3<T>::operator / (const T s) const;
\end{verbatim}
\end{screen}

\begin{itemize}

\item{{\bf 引数}}

{\tt s}: 入力。{\tt const T}型。

\item{{\bf 返り値}}

{\tt PS::Vector3<T>}型。自身のメンバx,yそれぞれを{\tt s}で割った値を返
す。

\end{itemize}


%%%%%%%%%%%%%%%%%%%%%%%%%%%%%
\begin{screen}
\begin{verbatim}
template<typename T>
const PS::Vector3<T> & PS::Vector3<T>::operator /= (const T s);
\end{verbatim}
\end{screen}

\begin{itemize}

\item{{\bf 引数}}

{\tt rhs}: 入力。{\tt const T}型。

\item{{\bf 返り値}}

{\tt const PS::Vector3<T> \&}型。自身のメンバx,yそれぞれを{\tt s}で割り
自身を返す。

\end{itemize}


%%%%%%%%%%%%%%%%%%%%%%%%%%%%%
\paragraph{内積、外積}
\mbox{}
%%%%%%%%%%%%%%%%%%%%%%%%%%%%%

%%%%%%%%%%%%%%%%%%%%%%%%%%%%%
\begin{screen}
\begin{verbatim}
template<typename T>
T PS::Vector3<T>::operator * (const PS::Vector3<T> & rhs) const;
\end{verbatim}
\end{screen}

\begin{itemize}

\item{{\bf 引数}}

{\tt rhs}: 入力。{\tt const PS::Vector3<T> \&}型。

\item{{\bf 返り値}}

{\tt T}型。自身と{\tt rhs}の内積を取った値を返す。

\end{itemize}

%%%%%%%%%%%%%%%%%%%%%%%%%%%%%
\begin{screen}
\begin{verbatim}
template<typename T>
T PS::Vector3<T>::operator ^ (const PS::Vector3<T> & rhs) const;
\end{verbatim}
\end{screen}

\begin{itemize}

\item{{\bf 引数}}

{\tt rhs}: 入力。{\tt const PS::Vector3<T> \&}型。

\item{{\bf 返り値}}

{\tt T}型。自身と{\tt rhs}の外積を取った値を返す。

\end{itemize}


%%%%%%%%%%%%%%%%%%%%%%%%%%%%%
\paragraph{{\tt Vector3<U>}への型変換}
\mbox{}
%%%%%%%%%%%%%%%%%%%%%%%%%%%%%

%%%%%%%%%%%%%%%%%%%%%%%%%%%%%
\begin{screen}
\begin{verbatim}
template<typename T>
template <typename U>
PS::Vector3<T>::operator PS::Vector3<U> () const;
\end{verbatim}
\end{screen}

\begin{itemize}

\item{{\bf 引数}}

{\tt const PS::Vector3<T>}型。

\item{{\bf 返り値}}

{\tt const PS::Vector3<U>}型。

\item{{\bf 機能}}

{\tt const PS::Vector3<T>}型を{\tt const PS::Vector3<U>}型にキャストす
る。

\end{itemize}


\subsection{PSL初期化、終了}

\subsubsection{概要}

PSLを初期化又は終了するために必要な関数。

\subsubsection{API}

\begin{screen}
\begin{verbatim}
void PS::Initialize(PS::S32 & argc, 
                    char **& argv)
\end{verbatim}
\end{screen}

\begin{itemize}

\item{{\bf 引数}}

{\tt argc}: コマンドライン引数の総数

{\tt argv}: コマンドライン引数の文字列を指すポインタのポインタ

\item{{\bf 返り値}}

なし。

\item{{\bf 機能}}

PSLの初期化を行う。PSLを使う前に読み出す必要がある。内部ではMPI::Initを
呼び出すため、引数{\tt argc}と{\tt argv}が変わっている可能性がある。

\end{itemize}

%%%%%%%%%%%%%%%%%%%%%%%%%%%
\begin{screen}
\begin{verbatim}
void PS::Finalize()
\end{verbatim}
\end{screen}

\begin{itemize}

\item{{\bf 引数}}

なし。

\item{{\bf 返り値}}

なし。

\item{{\bf 機能}}

PSLの終了処理を行う。

\end{itemize}



%%%%%%%%%%%%%%%%%%%%%%%%%%%

\subsection{領域クラス({\tt PS::DomainInfo})}

\subsubsection{概要}

ルートドメインの情報を持ち、その分割を行うクラス。クラス内に全ドメイン
の境界と、ルートドメインの分割の際に使われるサンプル粒子の座標を持つ。
ユーザー定義境界条件の場合、イメージドメインの情報も持つ。

\subsubsubsection{ルートドメインの分割の方法}

ルートドメインの分割を行う際、各ドメインから粒子の位置座標をサンプリン
グし、サンプリングされた粒子が各ドメインで均等になるように、ドメインの
境界を決める。この際、サンプル粒子が少ないとポアソンノイズによる影響を
受ける。この影響を減らすため、境界に指数移動平均を用いる。具体的にドメ
インの境界の決め方は以下の様になる。

\begin{equation}
X_{\rm EMA, t+\Delta t} = \alpha x_{t+\Delta t} + (1-\alpha)X_{\rm
EMA, t} \quad (0 \le \alpha \le 1).
\end{equation}

ここで、$\alpha$は平滑化係数、$X_{\rm EMA, t}$は時刻tでの平均化された境
界、$x_{t+\Delta t}$は時刻$t+\Delta t$での平均化する前の境界である。

領域クラス({\tt DomainInfo})はルートドメインの分割の為に必要な平滑化係
数と、サンプルする最大粒子数を内部のprivateメンバとして持つ。

領域クラス({\tt DomainInfo})は以下のように記述される。
\begin{lstlisting}[caption=領域クラス]
namespace  ParticleSimulator{
    class DomainInfo{
    public:
        void initialize(const S32 nsamplemax, const F32 alpha);
        void initialize(std::string domainparam);
        void setNProcMultiDim(const S32 nx, const S32 ny, const S32 nz);
        void setBoundaryCondition(enum BOUNDARY\_CONDITION);
        void setPosOfRootDomain(const F32vec & low,
                                const F32vec & high);
        void setNumberOfImageDomain(const S32 nimage);
        void collectSample(const PS::ParticleSystem & psys, 
                           const bool clear=true)
        void decomposeDomain();

    };
}
\end{lstlisting}

\subsubsubsection{境界条件}

ユーザーは領域クラスのメソッド{\tt setBoundarConditino(enum)}を使う事で、
以下の境界条件のシミュレーションを行う事が出来る。具体的な引数について
はAPIのセクションを参照。

{\bf 直方体型の周期境界条件}

ユーザーはx,y,z,の中の任意の軸を周期境界、もしくは開放境界に設定する事
が出来る。

{\bf シアリングボックス}

ユーザーはx軸を動径方向、y軸を接戦方向とするシアリングボックスのシミュ
レーションを扱う事が出来る(y軸方向は周期境界、x軸方向のボックスがずれて
いく)。3次元の場合z軸方向は開放境界となる。ボックスはy軸方向にずれてい
くので、ユーザーは{\tt TreeForForce::setShiftY(const PS::F32)}を使って
y軸方向のシフト量を設定する。この境界条件を扱う場合は短距離力でなくては
ならず、{\tt PSL}はカットオフ半径内にある粒子を全て見つける。

{\bf ユーザー定義境界条件}

ユーザーはイメージ粒子を定義する関数を定義し(イメージマップ関数)、
{\tt PS::DomainInfo::SetNumberOfImage(const PS::S32)}を使って全イメージ
数を設定する必要がある。その為、例えば、近距離力のシミュレーションでは、
ユーザーは定義したイメージマップ関数が十分にシミュレーション領域を覆い
尽くす様にしなければならない。もし、カットオフ半径内をユーザー定義イメー
ジが十分に覆い尽くせていない場合もPSはエラーや、例外を送出する事はない。

%下の例は、ケプラーポテンシャル中でのシアリングボックスのシミュレーショ
%ンの場合である。

%\begin{lstlisting}[caption=シアリングボックスでイメージ粒子を作る関数の例]
%void MapImage(const PS::F32vec & pos_in, 
%              const PS::F32 time,
%              const PS::S32 id,
%              PS::F32vec & pos_out){
%    static PS::F32 lx = 0.001;
%    static PS::F32 ly = 0.001;
%    static PS::F32 omega = 1.0;
%    pos_out.z = pos_in.z;
%    if(id == 0){
%        pos_out = pos_in;
%    }
%    else if(id == 1){
%        pos_out.x = pos_in.x - lx;
%        pos_out.y = pos_in.y + 1.5*lx*omega*time - ly;
%    }
%    else if(id == 2){
%        pos_out.x = pos_in.x;
%        pos_out.y = pos_in.y - ly;
%    }
%    else if(id == 3){
%        pos_out.x = pos_in.x + lx;
%        pos_out.y = pos_in.y - 1.5*lx*omega*time - ly;
%    }
%    else if(id == 4){
%        pos_out.x = pos_in.x - lx;
%        pos_out.y = pos_in.y + 1.5*lx*omega*time;
%    }
%    else if(id == 5){
%        pos_out.x = pos_in.x + lx;
%        pos_out.y = pos_in.y - 1.5*lx*omega*time;
%    }
%
%    else if(id == 6){
%        pos_out.x = pos_in.x - lx;
%        pos_out.y = pos_in.y + 1.5*lx*omega*time +ly;
%    }
%    else if(id == 7){
%        pos_out.x = pos_in.x;
%        pos_out.y = pos_in.y + ly;
%    }
%    else if(id == 8){
%        pos_out.x = pos_in.x + lx;
%        pos_out.y = pos_in.y - 1.5*lx*omega*time + ly;
%    }
%}
%\end{lstlisting}

%第一引数{\tt pos\_in}は粒子の実体の座標。第二引数{\tt time}は時刻であり、
%時間に依存するイメージ粒子を作ることが出来る。第三引数{\tt id}は1つの
%粒子の実体から複数のイメージ粒子を作る場合にイメージ粒子を区別するため
%の番号である。{\tt id}は0からイメージ粒子の数-1,までの整数にしなければ
%ならない。また、ユーザーは{\tt
%PS::DomainInfo::setNumberOfImageDomain(const PS::S32)を使って全イメージ
%の数をPS側に設定する必要がある。}第四引数{\tt pos\_out}はイメージ粒子の
%座標。

\subsubsection{API}

%%%%%%%%%%%%%%%%%%%%%%%%%
%%% format
%%%%%%%%%%%%%%%%%%%%%%%%%
%%%  \begin{screen}
%%%  \begin{verbatim}
%%%  function()
%%%  \end{verbatim}
%%%  \end{screen}
%%%
%%%  \begin{itemize}
%%%
%%%  \item{{\bf 引数}}
%%%
%%%  \item{{\bf 返り値}}
%%%
%%%  \item{{\bf 機能}}
%%%
%%%  \end{itemize}
%%%
%%%%%%%%%%%%%%%%%%%%%%%%%

%%%%%%%%%%%%%%%%%%%%%%%%%
\paragraph{初期化}
\mbox{}
%%%%%%%%%%%%%%%%%%%%%%%%%

%%%%%%%%%%%%%%%%%%%%%%%%%
\begin{screen}
\begin{verbatim}
void PS::DomainInfo::initialize(const PS::S32 nsamplemax,
                                const PS::F32 alpha=1.0)
\end{verbatim}
\end{screen}

\begin{itemize}

\item{{\bf 引数}}

{\tt nsamplemax}: 入力。{\tt const PS::S32}型。サンプル粒子数の最大値。

{\tt alpha}: 入力。{\tt const PS::F32}型。指数移動平均の平滑化係数。デフォルト1.0。

\item{{\bf 返り値}}
なし。

\item{{\bf 機能}}

最大のサンプル粒子数と平滑化係数を設定し、サンプリングされる粒子の配列
と全ドメインの境界の配列を確保する。実際にサンプルされる粒子数はここで
設定した数と同じかわずかに小さくなる。最大のサンプル数が全粒子数より多
い場合、もしくは全プロセス数より小さい場合は例外を送出する。

\end{itemize}

%%%%%%%%%%%%%%%%%%%%%%%%%
\begin{screen}
\begin{verbatim}
void PS::DomainInfo::initialize(std::string domainparam);
\end{verbatim}
\end{screen}

\begin{itemize}

\item{{\bf 引数}}

{\tt domainparam}: 入力。{\tt std::string}型。{\tt PS::DomainInfo}に関
するパラメータファイル名。

\item{{\bf 返り値}}
なし。

\item{{\bf 機能}}

最大のサンプル粒子数と平滑化係数を設定し、サンプリングされる粒子の配列
と全ドメインの境界の配列を確保する。実際にサンプルされる粒子数はここで
設定した数と同じがわずかに小さくなる。最大のサンプル数が全粒子数より多
い場合、もしくは全プロセス数より小さい場合は例外を送出する。

パラメータファイルの記述方法は以下のとおりである。

一行に設定する値は一つとし、第一カラムで設定する変数を指定し、第二カラ
ムにその値をいれる。カラムの間は空白で区切る。サンプル粒子の最大値を設
定する場合は第一カラムに{\tt NSAMPLEMAX}、平滑化係数を設定する場合は
{\tt COEFFICIENT\_OF\_EMA}。\#で始まる行はコメントアウトされる。

サンプル粒子の最大値が10000で、平滑化係数が0.5の時のファイルは以下の様に書く。
\begin{lstlisting}
NSAMPLEMAX 10000
COEFFICIENT_OF_EMA  0.5
\end{lstlisting}

\end{itemize}

%%%%%%%%%%%%%%%%%%%%%%%%%
\paragraph{ルートドメイン分割数の設定}
\mbox{}
%%%%%%%%%%%%%%%%%%%%%%%%%

%%%%%%%%%%%%%%%%%%%%%%%%%
\begin{screen}
\begin{verbatim}
void PS::DomainInfo::setNProcMultiDim(const PS::S32 nx, const PS::S32 ny, const PS::S32 nz);
\end{verbatim}
\end{screen}

\begin{itemize}

\item{{\bf 引数}}

{\tt nx}: 入力。{\tt const PS::S32}型。x軸方向のドメインの分割数。

{\tt ny}: 入力。{\tt const PS::S32}型。y軸方向のドメインの分割数。

{\tt nz}: 入力。{\tt const PS::S32}型。z軸方向のドメインの分割数。

\item{{\bf 返り値}}

なし。

\item{{\bf 機能}}

ルートドメインの分割数を設定する。nx,ny,nz(2次元の場合はnx,ny)の積が全
プロセス数にならない場合は例外を送出する。2次元の場合は{\tt nz}に任意の
値を入れてよい。このメソッドを使わない場合は{\tt PSL}が自動的に
nx,ny,nzを決定する。

\end{itemize}


%%%%%%%%%%%%%%%%%%%%%%%%%
\paragraph{境界条件設定}
\mbox{}
%%%%%%%%%%%%%%%%%%%%%%%%%

%%%%%%%%%%%%%%%%%%%%%%%%%
\begin{screen}
\begin{verbatim}
void PS::DomainInfo::setBoundaryCondition(enum BOUNDARY\_CONDITION)
\end{verbatim}
\end{screen}

\begin{itemize}

\item{{\bf 引数}}

{\tt BOUNDARY\_CONDITION}: 入力。{\tt enum}型。境界条件。

\item{{\bf 返り値}}
なし。

\item{{\bf 機能}}

引数に対応する、境界条件を設定する。
このメソッドを呼ばない場合は自動的に開放境界条件となる。

\mbox{}
\begin{table}[h!]
\begin{tabular}{llll}
{\tt BOUNDARY\_CONDITION} &=& {\tt PS::BOUNDARY\_CONDITION\_OPEN} & 開放境界条件。デフォルト。 \\
     	          &=& {\tt PS::BOUNDARY\_CONDITION\_PERIODIC\_X} & x軸方向周期境界。y,z軸開放境界。 \\
     	          &=& {\tt PS::BOUNDARY\_CONDITION\_PERIODIC\_Y} & y軸方向周期境界。x,z軸開放境界。 \\
     	          &=& {\tt PS::BOUNDARY\_CONDITION\_PERIODIC\_Z} & z軸方向周期境界。x,y軸開放境界。 \\
     	          &=& {\tt PS::BOUNDARY\_CONDITION\_PERIODIC\_XY} & x,y軸方向周期境界。z軸開放境界。 \\
     	          &=& {\tt PS::BOUNDARY\_CONDITION\_PERIODIC\_XZ} & x,z軸方向周期境界。y軸開放境界。 \\
     	          &=& {\tt PS::BOUNDARY\_CONDITION\_PERIODIC\_YZ} & y,z軸方向周期境界。x軸開放境界。 \\
     	          &=& {\tt PS::BOUNDARY\_CONDITION\_PERIODIC\_XYZ} & x,y,z軸方向周期境界。 \\
     	          &=& {\tt PS::BOUNDARY\_CONDITION\_SHEARING\_BOX} & シアリングボックス。 \\
     	          &=& {\tt PS::BOUNDARY\_CONDITION\_USER\_DEFINED} & ユーザー指定境界条件。
\end{tabular}
\end{table}
\mbox{}

\end{itemize}

\newpage

%%%%%%%%%%%%%%%%%%%%%%%%%
\paragraph{非開放境界用ルートドメイン設定}
\mbox{}
%%%%%%%%%%%%%%%%%%%%%%%%%

%%%%%%%%%%%%%%%%%%%%%%%%%
\begin{screen}
\begin{verbatim}
void PS::DomainInfo::setPosOfRootDomain(const PS::F32vec & low,
                                        const PS::F32vec & high)
\end{verbatim}
\end{screen}

\begin{itemize}

\item{{\bf 引数}}

{\tt low}: 入力。{\tt const PS::F32vec}型。ルートドメインの座標値が小さ
い側の頂点の座標。

{\tt high}: 入力。{\tt const PS::F32vec}型。ルートドメインの座標値が大
きい側の頂点の座標。

\item{{\bf 返り値}}
なし。

\item{{\bf 機能}}

ルートドメインを設定する。開放境界の場合、このメソッドを使う必要はなく、
{\tt PS}がルートドメインの設定を自動で行う。また、軸方向に0を与えた場合
も{\tt PS}がドメインの設定を行う。

\end{itemize}

%%%%%%%%%%%%%%%%%%%%%%%%%
%\paragraph{任意境界条件用イメージドメイン設定}
%\mbox{}
%%%%%%%%%%%%%%%%%%%%%%%%%

%%%%%%%%%%%%%%%%%%%%%%%%%
%\begin{screen}
%\begin{verbatim}
%void PS::DomainInfo::setImageDomainFunc(void (*pmap)(const PS::F32vec & pos_in,
%                                                     const PS::F32 time,
%                                                     const PS::S32 id,
%                                                     const PS::F32vec & pos_out))
%\end{verbatim}
%\end{screen}

%\begin{itemize}

%\item{{\bf 引数}}

%{\tt *pmap}: 入力。返り値がvoid型のイメージドメイン設定用関数ポインタ。
%関数の引数は第一引数から順に{\tt const PS::F32vec型}、{\tt const
%PS::F32型}、{\tt const PS::S32型}、{\tt const PS::F32vec型}。

%\item{{\bf 返り値}}
%なし。

%\item{{\bf 機能}}

%任意境界条件の場合のイメージドメインを設定する。任意境界条件以外の境界
%条件の場合に呼び出されると、例外を送出する。

%\end{itemize}


%%%%%%%%%%%%%%%%%%%%%%%%%
\paragraph{ユーザー定義境界条件用イメージドメイン数の設定}
\mbox{}
%%%%%%%%%%%%%%%%%%%%%%%%%

%%%%%%%%%%%%%%%%%%%%%%%%%
\begin{screen}
\begin{verbatim}
void PS::DomainInfo::setNumberOfImageDomain(const PS::S32 nimage)
\end{verbatim}
\end{screen}

\begin{itemize}

\item{{\bf 引数}}

{\tt nimage}: 入力。{\tt const PS::S32}型。イメージドメインの数。

\item{{\bf 返り値}}
なし。

\item{{\bf 機能}}

ユーザー指定境界条件の場合のイメージドメインの数を設定する。ユーザー指
定境界条件以外の境界条件の場合に呼び出されると、例外を送出する。

\end{itemize}

%%%%%%%%%%%%%%%%%%%%%%%%%
\paragraph{サンプル粒子回収}
\mbox{}
%%%%%%%%%%%%%%%%%%%%%%%%%

%%%%%%%%%%%%%%%%%%%%%%%%%
\begin{screen}
\begin{verbatim}
void PS::DomainInfo::collectSample(const PS::ParticleSystem & psys,
                                   const PS::F32 wgh = 1.0,
                                   const bool clear=true)
\end{verbatim}
\end{screen}

\begin{itemize}

\item{{\bf 引数}}

{\tt psys}: 入力。{\tt const PS::ParticleSystem \&}型。

{\tt wgh}: 入力。{\tt PS::F32}型。サンプル数調整用重み。

{\tt clear}: 入力。{\tt const bool}型。クリアフラグ。デフォルトtrue。

\item{{\bf 返り値}}

なし。

\item{{\bf 機能}}

各プロセスが自分のドメインからいくつかの粒子の座標をサンプルし、そのサ
ンプルをルートプロセスに送る。\verb|clear|でサンプルする前にサンプル粒
子の位置データを保持したメモリをクリアするかどうか決める。\verb|wgh|で
自分のドメインからサンプルする粒子の数を調整する。

{\bf サンプル数の調整方法}

各ドメインから供出されるサンプル粒子の数が、1/{\tt wgh}の比になるように、
粒子サンプルを行う。例えば、計算時間でサンプル数の調整を行う場合、{\tt
wgh}に各プロセスでかかった計算時間を設定することで、時間のかかったプロ
セス程、担当するi粒子が少なくなり、ロードバランスがとりやすくなる。

%1プロセス当たりの最大サンプル数は決まっており、ユーザーが指定する事も出
%来る。\redtext{どの様に最大粒子数を与えるかはまだ決めていない。領域クラ
%スメソッド?、粒子群クラメソッドス?}

\end{itemize}

%%%%%%%%%%%%%%%%%%%%%%%%%


%%%%%%%%%%%%%%%%%%%%%%%%%
\paragraph{領域分割}
\mbox{}
%%%%%%%%%%%%%%%%%%%%%%%%%

%%%%%%%%%%%%%%%%%%%%%%%%%
\begin{screen}
\begin{verbatim}
void PS::DomainInfo::decomposeDomain()
\end{verbatim}
\end{screen}

\begin{itemize}

\item{{\bf 引数}}

なし。

\item{{\bf 返り値}}

なし。

\item{{\bf 機能}}

全プロセスのドメインの境界を決定し、その座標を領域クラス内に格納する。

\end{itemize}

%%%%%%%%%%%%%%%%%%%%%%%%%
%\begin{screen}
%\begin{verbatim}
%void PS::DomainInfo::decomposeDomainAll(const PS::ParticleSystem & psys, 
%                                        const PS::F32 wgh=1.0)
%\end{verbatim}
%\end{screen}
%
%\begin{itemize}
%
%\item{{\bf 引数}}
%
%{\tt psys}: 入力。{\tt const PS::ParticleSystem \&}型。
%
%{\tt wgh}: 入力。{\tt const PS::F32}型。サンプリング重み。デフォルト1.0。
%
%\item{{\bf 返り値}}
%
%なし。
%
%\item{{\bf 機能}}
%
%関数{\tt PS::DomainInfo::sampleParticle}と{\tt
%PS::DomainInfo::decomposeDomain}を順次実行する。これ以前に保持されてい
%たサンプル粒子はクリアされ、その後にサンプルを行う。
%
%\end{itemize}
%
%%%%%%%%%%%%%%%%%%%%%%%%%


\subsection{粒子群クラス({\tt PS::ParticleSystem})}

\subsubsection{概要}

{\tt FullParticle}の配列を持ち、粒子の交換等を行うクラス。粒子の読み込
みや書き込み等はこのクラスを通して行う。粒子種の数だけインスタンスを作
る必要がある。

粒子群クラス({\tt PS::ParticleSystem})は以下のように記述される。

\begin{lstlisting}[caption=粒子群クラス]
namespace ParticleSimulator{
    template<class Tptcl>
    class ParticleSystem{
    public:

        template <class Theader>
        void readParticleAscii(const char * const filename,
                               const char * const format,
                               const Theader& header);
        template <class Theader>
        void readParticleAscii(const char * const filename,
                               const Theader& header);
        void readParticleAscii(const char * const filename,
                               const char * const format);
        void readParticleAscii(const char * const filename);

        template <class Theader>
        void writeParticleAscii(const char * const filename,
                                const char * const format,
                                Theader& header);
        template <class Theader>
        void writeParticleAscii(const char * const filename,
                                Theader& header);
        void writeParticleAscii(const char * const filename,
                                const char * const format);
        void writeParticleAscii(const char * const filename);


        void createParticleArray(const S32 array_size);

        template<class Tdinfo>
        void exchangeParticle(const Tdinfo & dinfo);
        Tptcl & operator [] (const S32 id);
        const Tptcl & operator [] (const S32 id) const;
        Tptcl * getParticlePointer(const S32 id=0);
        S32 getNumberOfParticleLocal() const;
        S64 getNumberOfParticleTotal() const;
        S32 getSizeOfParticle() const;
    };
}
\end{lstlisting}

\subsubsection{API}

%%%%%%%%%%%%%%%%%%%%%%%%%
%%% format
%%%%%%%%%%%%%%%%%%%%%%%%%
%%%  \begin{screen}
%%%  \begin{verbatim}
%%%  function()
%%%  \end{verbatim}
%%%  \end{screen}
%%%
%%%  \begin{itemize}
%%%
%%%  \item{{\bf 引数}}
%%%
%%%  \item{{\bf 返り値}}
%%%
%%%  \item{{\bf 機能}}
%%%
%%%  \end{itemize}
%%%
%%%%%%%%%%%%%%%%%%%%%%%%%

%%%%%%%%%%%%%%%%%%%%%%%%%
\paragraph{ファイル入出力}
\mbox{}
%%%%%%%%%%%%%%%%%%%%%%%%%

\begin{screen}
\begin{verbatim}
template <class Theader>
void PS::ParticleSystem::readParticleAscii(const char * const filename,
                                           const char * const format,
                                           Theader& header);
\end{verbatim}
\end{screen}

\begin{itemize}

\item{{\bf 引数}}

{\tt filename}: 入力。{\tt const char *}型。入力ファイル名のベースとなる部分。

{\tt format}: 入力。{\tt const char *}型。分散ファイルから粒子データを読み込む際のファイルフォーマット。

{\tt header}: 入力。{\tt Theader\&}型。ファイルのヘッダ情報。

\item{{\bf 返り値}}

なし。

\item{{\bf 機能}}

各プロセスが{\tt filename}と{\tt format}で指定された入力ファイルから粒子データを読み出し、データを{\tt FullParticle}として格納する。

{\tt filename}で、分散しているファイルのベースとなる名前を指定する。
{\tt format}でファイル名のフォーマットを指定する。
フォーマットの指定方法は標準Cライブラリの関数\verb|printf|の第1引数と同じである。
ただし変換指定は必ず3つであり、その指定子は1つめは文字列、残りはどちらも整数である。
2つ目の変換指定にはそのジョブの全プロセス数が、3つ目の変換指定にはプロセス番号が入る。
例えば、{\tt filename}が\verb|nbody|、{\tt format}が\verb|%s_%03d_%03d.init|ならば、全プロセス数$64$のジョブのプロセス番号$12$のプロセスは、\verb|nbody_064_012.init|というファイルを読み込む。

1粒子のデータを読み取る関数は{\tt FullParticle}のメンバ関数でユーザが定義する。
名前は{\tt readAscii}であり、引数は{\tt FILE*}型である。
例えば3次元の重力計算の場合、以下のように定義できる。
読み込むべき1粒子のデータは、質量(\verb|PS::F64 mass|)、位置(\verb|PS::F64vec pos|)、速度(\verb|PS::F64vec vel|)であり、そのフォーマットが10進数アスキーであるとする。

\begin{verbatim}
void FullParticle::readAscii(FILE *fp){
    fscanf(fp, "%lf%lf%lf%lf%lf%lf%lf",
                  &this->mass,
                  &this->pos[0], &this->pos[1], &this->pos[2],
                  &this->vel[0], &this->vel[1], &this->vel[2]);
    return;
}
\end{verbatim}

ユーザがこの関数を定義するに当って、以下の制限がある。
\begin{itemize}
\item 返値の型が\verb|void|。
\item 引数にファイルポインタを取り、このファイルポインタを入力先に指定すること
\end{itemize}

ファイルのヘッダのデータを読み取る関数は{\tt Theader}のメンバ関数でユーザが定義する。
名前は{\tt readAscii}であり、引数は{\tt FILE*}型である。
もし、ヘッダーに粒子数が含まれていれば、この関数は粒子数を返さなければならない。
この時、返ってきた粒子数分、ファイルから粒子データを読み込む。
ヘッダーに粒子数が含まれていない場合は-1以下の整数を返さなければならない。
この時、1回ファイルを読み込んで行数を取得した後、もう一度ファイルを読み込みなおし、粒子データを読み込む。
例えばヘッダーに粒子数(\verb|nbody|)、時刻(\verb|time|)が入っている場合、以下の様になる。

\begin{verbatim}
S32 Theader::readAscii(FILE *fp) {
    fscanf(fp, "%d%lf", &nbody, &time);
    return nbody;
}
\end{verbatim}

ユーザがこの関数を定義するに当って、以下の制限がある。
\begin{itemize}
\item 返値の型が\verb|PS::S32|。
\item 引数にファイルポインタを取り、このファイルポインタを入力先に指定すること
\end{itemize}

\end{itemize}

%%%%%%%%%%%%%%%%%%%%%%%%%

\begin{screen}
\begin{verbatim}
void PS::ParticleSystem::readParticleAscii(const char * const filename,
                                           const char * const format);
\end{verbatim}
\end{screen}

\begin{itemize}

\item{{\bf 引数}}

{\tt filename}: 入力。{\tt const char * const}型。入力ファイル名のベースとなる部分。

{\tt format}: 入力。{\tt const char * const}型。分散ファイルから粒子データを読み込む際のファイルフォーマット。

\item{{\bf 返り値}}

なし。

\item{{\bf 機能}}

各プロセスが{\tt filename}と{\tt format}で指定された入力ファイルから粒子データを読み出し、データを{\tt FullParticle}として格納する。
この時、1回ファイルを読み込んで行数を取得した後、もう一度ファイルを読み込みなおし、粒子データを読み込む。

{\tt filename}で、分散しているファイルのベースとなる名前を指定する。
{\tt format}でファイル名のフォーマットを指定する。
{\tt format}の指定の仕方は、{\tt Theader}が存在する場合の時と同様である。

1粒子のデータを読み取る関数は{\tt FullParticle}のメンバ関数でユーザが定義する。
このメンバ関数の書式と規約は、分散ファイルから読み出す場合と同様である。

\end{itemize}

%%%%%%%%%%%%%%%%%%%%%%%%%

\begin{screen}
\begin{verbatim}
template <class Theader>
void PS::ParticleSystem::readParticleAscii(const char * const filename,
                                           Theader& header);
\end{verbatim}
\end{screen}

\begin{itemize}

\item{{\bf 引数}}

{\tt filename}: 入力。{\tt const char * const}型。入力ファイル名。

{\tt header}: 入力。{\tt Theader\&}型。ファイルのヘッダ情報。

\item{{\bf 返り値}}

なし。

\item{{\bf 機能}}

ルートプロセスが{\tt filename}で指定された入力ファイルから粒子データを読み出し、データを{\tt FullParticle}として格納した後、各プロセスに分配する。

1粒子のデータを読み取る関数は{\tt FullParticle}のメンバ関数でユーザが定義する。
ファイルのヘッダのデータを読み取る関数は{\tt Theader}のメンバ関数でユーザが定義する。
これら2つのメンバ関数の書式と規約は、分散ファイルから読み出す場合と同様である。

\end{itemize}

%%%%%%%%%%%%%%%%%%%%%%%%%

\begin{screen}
\begin{verbatim}
void PS::ParticleSystem::readParticleAscii(const char * const filename);
\end{verbatim}
\end{screen}

\begin{itemize}

\item{{\bf 引数}}

{\tt filename}: 入力。{\tt const char * const}型。入力ファイル名。

\item{{\bf 返り値}}

なし。

\item{{\bf 機能}}

ルートプロセスが{\tt filename}で指定された入力ファイルから粒子データを読み出し、データを{\tt FullParticle}として格納した後、各プロセスに分配する。
この時、1回ファイルを読み込んで行数を取得した後、もう一度ファイルを読み込みなおし、粒子データを読み込む。

1粒子のデータを読み取る関数は{\tt FullParticle}のメンバ関数でユーザが定義する。
このメンバ関数の書式と規約は、分散ファイルから読み出す場合と同様である。

\end{itemize}

%%%%%%%%%%%%%%%%%%%%%%%%%

\begin{screen}
\begin{verbatim}
template <class Theader>
void PS::ParticleSystem::void writeParticleAscii(const char * const filename,
                                                 const char * const format,
                                                 const Theader& header);
\end{verbatim}
\end{screen}

\begin{itemize}

\item{{\bf 引数}}

{\tt filename}: 入力。{\tt const char * const}型。出力ファイル名のベースとなる部分。

{\tt format}: 入力。{\tt const char * const}型。分散ファイルに粒子データを書き込む際のファイルフォーマット。

{\tt header}: 入力。{\tt const Theader\&}型。ファイルのヘッダ情報。

\item{{\bf 返り値}}

なし。

\item{{\bf 機能}}

各プロセスが{\tt filename}と{\tt format}で指定された出力ファイルに{\tt FullParticle}型の粒子データと、{\tt Theader}型のヘッダー情報を出力する。
出力ファイルのフォーマットはメンバ関数\verb|PS::ParticleSystem::readParticleAscii|と同様である。

1粒子のデータを書き込む関数は{\tt FullParticle}のメンバ関数でユーザが定義する。
名前は{\tt writeAscii}であり、引数は{\tt FILE*}型である。
例えば重力計算の場合、以下のように定義できる。
読み込むべき1粒子のデータは、質量(\verb|PS::F64 mass|)、位置(\verb|PS::F64vec pos|)、速度(\verb|PS::F64vec vel|)であり、そのフォーマットが10進数アスキーであるとする。

\begin{verbatim}
void FullParticle::writeAscii(FILE *fp) const{
    fprintf(fp, "%lf%lf%lf%lf%lf%lf%lf",
                this->mass,
                this->pos[0], this->pos[1], this->pos[2],
                this->vel[0], this->vel[1], this->vel[2]);
}
\end{verbatim}

ユーザがこの関数を定義するに当って、以下の制限がある。
\begin{itemize}
\item 返値の型が\verb|void|。
\item 引数にファイルポインタを取り、このファイルポインタを入力先に指定すること。
\item \verb|const|メンバ関数であること。
\end{itemize}

ファイルのヘッダのデータを書き込む関数は{\tt Theader}のメンバ関数でユーザが定義する。
名前は{\tt writeAscii}であり、引数は{\tt FILE*}型である。
例えばヘッダーに粒子数(\verb|nbody|)、時刻(\verb|time|)が入っている場合、以下の様になる。

\begin{verbatim}
void Theader::writeAscii(FILE *fp) const{
     fprintf(fp, "%d%lf", nbody, time);
}
\end{verbatim}

ユーザがこの関数を定義するに当って、以下の制限がある。
\begin{itemize}
\item 返値の型が\verb|void|。
\item 引数にファイルポインタを取り、このファイルポインタを入力先に指定すること。
\item \verb|const|メンバ関数であること。
\end{itemize}

\end{itemize}

%%%%%%%%%%%%%%%%%%%%%%%%%

\begin{screen}
\begin{verbatim}
void PS::ParticleSystem::void writeParticleAscii(const char * const filename,
                                                 const char * const format);
\end{verbatim}
\end{screen}

\begin{itemize}

\item{{\bf 引数}}

{\tt filename}: 入力。{\tt const char * const}型。出力ファイル名のベースとなる部分。

{\tt format}: 入力。{\tt const char * const}型。分散ファイルに粒子データを書き込む際のファイルフォーマット。

\item{{\bf 返り値}}

なし。

\item{{\bf 機能}}

各プロセスが{\tt filename}と{\tt format}で指定された出力ファイルに{\tt FullParticle}型の粒子データを出力する。
出力ファイルのフォーマットはメンバ関数\verb|PS::ParticleSystem::readParticleAscii|と同様である。

1粒子のデータを書き込む関数は{\tt FullParticle}のメンバ関数でユーザが定義する。
このメンバ関数の書式と規約は、分散ファイルに書き込む場合と同様である。

\end{itemize}

%%%%%%%%%%%%%%%%%%%%%%%%%

\begin{screen}
\begin{verbatim}
template <class Theader>
void PS::ParticleSystem::void writeParticleAscii(const char * const filename,
                                                 const Theader& header);
\end{verbatim}
\end{screen}

\begin{itemize}

\item{{\bf 引数}}

{\tt filename}: 入力。{\tt const char * const}型。出力ファイル名。

{\tt header}: 入力。{\tt const Theader\&}型。ファイルのヘッダ情報。

\item{{\bf 返り値}}

なし。

\item{{\bf 機能}}

各プロセスが{\tt filename}で指定された出力ファイルに{\tt FullParticle}型の粒子データと、{\tt Theader}型のヘッダー情報を出力する。

1粒子のデータを書き込む関数は{\tt FullParticle}のメンバ関数でユーザが定義する。
ファイルのヘッダのデータを書き込む関数は{\tt Theader}のメンバ関数でユーザが定義する。
これら2つのメンバ関数の書式と規約は、分散ファイルに書き込む場合と同様である。

\end{itemize}

%%%%%%%%%%%%%%%%%%%%%%%%%

\begin{screen}
\begin{verbatim}
void PS::ParticleSystem::void writeParticleAscii(const char * const filename);
\end{verbatim}
\end{screen}

\begin{itemize}

\item{{\bf 引数}}

{\tt filename}: 入力。{\tt const char * const}型。出力ファイル名。

\item{{\bf 返り値}}

なし。

\item{{\bf 機能}}

各プロセスが{\tt filename}で指定された出力ファイルに{\tt FullParticle}型の粒子データを出力する。

1粒子のデータを書き込む関数は{\tt FullParticle}のメンバ関数でユーザが定義する。
このメンバ関数の書式と規約は、分散ファイルに書き込む場合と同様である。

\end{itemize}

%%%%%%%%%%%%%%%%%%%%%%%%%


%%%%%%%%%%%%%%%%%%%%%%%%%
\paragraph{粒子交換}
\mbox{}
%%%%%%%%%%%%%%%%%%%%%%%%%


%%%%%%%%%%%%%%%%%%%%%%%%%
\begin{screen}
\begin{verbatim}
void PS::ParticleSystem::exchangeParticle(const PS::DomainInfo & dinfo)
\end{verbatim}
\end{screen}

\begin{itemize}

\item{{\bf 引数}}

{\tt dinfo}: 入力。{\tt const PS::DomainInfo \&}型。

\item{{\bf 返り値}}

なし。

\item{{\bf 機能}}

粒子が適切なドメインに配置されるように、粒子の交換を行う。どのドメイン
にも属さない粒子が現れた場合、例外を送出する。

\end{itemize}

%%%%%%%%%%%%%%%%%%%%%%%%%


%%%%%%%%%%%%%%%%%%%%%%%%%
\paragraph{その他}
\mbox{}
%%%%%%%%%%%%%%%%%%%%%%%%%

%%%%%%%%%%%%%%%%%%%%%%%%%
\begin{screen}
\begin{verbatim}
FullPtcl & PS::ParticleSystem::operator [] (const PS::S32 id)
\end{verbatim}
\end{screen}

\begin{itemize}

\item{{\bf 引数}}

{\tt id}: {\tt const PS::S32}型。

\item{{\bf 返り値}}

{\tt FullPtcl \&}型。{\tt id}番目の粒子の参照を返す。

\end{itemize}

%%%%%%%%%%%%%%%%%%%%%%%%%
\begin{screen}
\begin{verbatim}
FullPtcl * PS::ParticleSystem::getParticlePointer(const PS::S32 id)
\end{verbatim}
\end{screen}

\begin{itemize}

\item{{\bf 引数}}

{\tt id}: 入力。{\tt const PS::S32}型。

\item{{\bf 返り値}}

{\tt FullPtcl *}型。id番目の粒子へのポインタを返す。


\end{itemize}


%%%%%%%%%%%%%%%%%%%%%%%%%
\begin{screen}
\begin{verbatim}
PS::S32 PS::ParticleSystem::getNumberOfParticleLocal()
\end{verbatim}
\end{screen}

\begin{itemize}

\item{{\bf 引数}}

なし。

\item{{\bf 返り値}}

{\tt PS::S32}型。自身が担当する粒子数を返す。

\end{itemize}


%%%%%%%%%%%%%%%%%%%%%%%%%
\begin{screen}
\begin{verbatim}
PS::S64 PS::ParticleSystem::getNumberOfParticleGlobal()
\end{verbatim}
\end{screen}

\begin{itemize}

\item{{\bf 引数}}

なし。

\item{{\bf 返り値}}

{\tt PS::S64}型。全プロセスでの粒子の総数を返す。

\end{itemize}


%%%%%%%%%%%%%%%%%%%%%%%%%
\begin{screen}
\begin{verbatim}
PS::S32 PS::ParticleSystem::getSizeOfParticle()
\end{verbatim}
\end{screen}

\begin{itemize}

\item{{\bf 引数}}

なし。

\item{{\bf 返り値}}

{\tt PS::S32}型。自身が持つ粒子配列のサイズを返す。

\end{itemize}




\subsection{相互作用クラス({\tt PS::TreeForForce})}
\subsubsection{概要}

{\tt PS::ParticleSystem}から粒子の情報を読み取り、内部でツリーを構築し
相互作用の計算を行い、結果を格納するクラス。{\tt PS::ParticleSystem}に
結果を書き戻す事も出来る。内部には相互作用演算を効率的に行う為の3種類の
粒子型\verb|EssentialParticleI|型、\verb|EssentialParticleJ|型、
\verb|SuperParticleJ|型、また、相互作用の結果を格納する\verb|Force|型、
がありさらに、ツリー構造を実現するための\verb|TreeParticle|型、
\verb|TreeCell|型という、全6種類のクラスを配列として持っている。これら
の配列はprvate空間に存在し、ユーザーが直接アクセスする事は出来ない。

相互作用クラス({\tt TreeForForce})は以下のように記述される。

\begin{lstlisting}[caption=相互作用クラス]
    template<int SEARCH_MODE, class Tforce, class Tepi,
        class Tepj, class Tmomloc, class Tmomglb, class Tspj>
    class TreeForForce{
    public:
        void initialize(const S64 ntot,
                        const F32 theta,
                        const S32 nleafmax,
                        const S32 ngroupmax);

        %void initializeLocalTree(const F32 len_half);

        template<class Tpsys>
        void setParticleLocalTree(const Tpsys & psys,
                                  const bool clear=true);

        void makeLocalTree(const bool cancel=false);

        void makeGlobalTree(const bool cancel=false);

        void exchangeLocalEssentialTree(const DomainInfo & dinfo);

        void setLocalEssentialTreeToGlobalTree();

        Tforce & getForce(const S32 id);

        template<class Tfunc_ep_ep, class Tpsys>
        void calcForceAll(Tfunc_ep_ep pfunc_ep_ep, 
                          Tpsys & psys,
                          DomainInfo & dinfo,
                          const bool clear_force = true);

        template<class Tfunc_ep_ep, class Tpsys>
        void calcForceAllAndWriteBack(Tfunc_ep_ep pfunc_ep_ep, 
                                      Tpsys & psys,
                                      DomainInfo & dinfo,
                                      const bool clear_force = true);

        template<class Tfunc_ep_ep, class Tfunc_ep_sp, class Tpsys>
        void calcForceAll(Tfunc_ep_ep pfunc_ep_ep, 
                          Tfunc_ep_sp pfunc_ep_sp,  
                          Tpsys & psys,
                          DomainInfo & dinfo,
                          const bool clear_force=true);

        template<class Tfunc_ep_ep, class Tfunc_ep_sp, class Tpsys>
        void calcForceAllAndWriteBack(Tfunc_ep_ep pfunc_ep_ep, 
                                      Tfunc_ep_sp pfunc_ep_sp,  
                                      Tpsys & psys,
                                      DomainInfo & dinfo,
                                      const bool clear_force=true);

        template<class Tfunc_ep_ep, class Tpsys>
        void makeTreeAndCalcForce(Tfunc_ep_ep pfunc_ep_ep, 
                                  DomainInfo & dinfo,
                                  const bool clear_force = true);

        template<class Tfunc_ep_ep, class Tfunc_ep_sp, class Tpsys>
        void makeTreeAndCalcForce(Tfunc_ep_ep pfunc_ep_ep, 
                                  Tfunc_ep_sp pfunc_ep_sp,  
                                  DomainInfo & dinfo,
                                  const bool clear_force=true){



        template<class Tfunc_ep_ep, class Tfunc_ep_sp, class Tpsys>
        void calcForceAll(Tfunc_ep_ep pfunc_ep_ep, 
                          Tfunc_ep_sp pfunc_ep_sp,  
                          DomainInfo & dinfo,
                          const bool clear_force=true);
    };
}
\end{lstlisting}

テンプレートパラメータの第一引数である{\tt SEARCH\_MODE}により、ツリー
ウォークの方法を選択できる。

\mbox{}
\begin{table}[h!]
\begin{tabular}{llll}
{\tt SEARCH\_MODE} &=& {\tt PS::SEARCH\_MODE\_LONG} & cutoffなし長距離力モード。 \\
     	          &=& {\tt PS::SEARCH\_MODE\_LONG\_CUTOFF} & cutoffあり長距離力モード。 \\
     	          &=& {\tt PS::SEARCH\_MODE\_GATHER} & 収集モード。 \\
     	          &=& {\tt PS::SEARCH\_MODE\_SCATTER} & 散乱モード。 \\
     	          &=& {\tt PS::SEARCH\_MODE\_SYMMETRY} & 対称モード。
\end{tabular}
\end{table}
\mbox{}

テンプレートパラメータの第二引数は{\tt FORCE}クラス、第三引数は{\tt
EPI}クラス、第四引数は{\tt EPJ}クラス。第五、六引数はそれぞれローカルツ
リー、グローバルツリーの持つ{\tt MOMENT}クラス。第七引数は{\tt SPJ}クラ
スである。

{\tt TreeForForce}はテンプレートパラメータが多いため、以下の様なラッパー
クラスが用意されており、ユーザーはこちらを使う事が推奨される。長距離力
の場合はテンプレート引数が{\tt FORCE}, {\tt EPI}, {\tt EPJ}, {\tt
MOMENT}, {\tt SPJ}の5つ。短距離量の場合はテンプレート引数が{\tt
FORCE}, {\tt EPI}, {\tt EPJ}の3つである。長距離力の場合はさらに、四重極
モーメント迄の計算については{\tt MOMENT}, {\tt SPJ}クラスが用意されてお
り、それに対応したラッパーも存在する。


\begin{lstlisting}

namespace ParticleSimulator{
    template<class Tforce, class Tepi, class Tepj, class Tmom=void, class Tsp=void>
    class TreeForForceLong{
    public:
        typedef TreeForForce
        <SEARCH_MODE_LONG,
         Tforce, Tepi, Tepj,
         Tmom, Tmom, Tsp> Normal; // cutoffなし長距離力モード

        typedef TreeForForce
        <SEARCH_MODE_LONG_CUTOFF,
         Tforce, Tepi, Tepj,
         Tmom, Tmom, Tsp> WithCutoff; // cutoffあり長距離力モード
    };

    template<class Tforce, class Tepi, class Tepj>
    class TreeForForceLong<Tforce, Tepi, Tepj, void, void>{
    public:
        typedef TreeForForce
        <SEARCH_MODE_LONG,
         Tforce, Tepi, Tepj,
         MomentMonopole,
         MomentMonopole,
         SPJMonoPole> Monopole;

        typedef TreeForForce
        <SEARCH_MODE_LONG,
         Tforce, Tepi, Tepj,
         MomentDipole,
         MomentDipole,
         SPJMonoPole> Dipole;

        typedef TreeForForce
        <SEARCH_MODE_LONG,
         Tforce, Tepi, Tepj,
         MomentQuadrupole,
         MomentQuadrupole,
         SPJMonoPole> Quadrupole;
    };

    template<class Tforce, class Tepi, class Tepj>
    class TreeForForceShort{
    public:
        typedef TreeForForce 
        <SEARCH_MODE_SYMMETRY,
         Tforce, Tepi, Tepj,
         MomentSearchInAndOut,
         MomentSearchInAndOut,
         SuperParticleBase> Symmetry; // 短距離、対称モード

        typedef TreeForForce 
        <SEARCH_MODE_GATHER,
         Tforce, Tepi, Tepj,
         MomentSearchInAndOut,
         MomentSearchInOnly,
         SuperParticleBase> Gather; // 短距離、収集モード

        typedef TreeForForce
        <SEARCH_MODE_SCATTER,
         Tforce, Tepi, Tepj,
         MomentSearchInOnly,
         MomentSearchInAndOut,
         SuperParticleBase> Scatter; // 短距離、散乱モード
    };
}
\end{lstlisting}

\subsubsection{API}

%%%%%%%%%%%%%%%%%%%%%%%%%
%%% format
%%%%%%%%%%%%%%%%%%%%%%%%%
%%%  \begin{screen}
%%%  \begin{verbatim}
%%%  function()
%%%  \end{verbatim}
%%%  \end{screen}
%%%
%%%  \begin{itemize}
%%%
%%%  \item{{\bf 引数}}
%%%
%%%  \item{{\bf 返り値}}
%%%
%%%  \item{{\bf 機能}}
%%%
%%%  \end{itemize}
%%%
%%%%%%%%%%%%%%%%%%%%%%%%%

%%%%%%%%%%%%%%%%%%%%%%%%%
\paragraph{初期化}
\mbox{}
%%%%%%%%%%%%%%%%%%%%%%%%%

\begin{screen}
\begin{verbatim}
void PS::TreeForForce::initialize(const PS::S64 ntot,
                                  const PS::F32 theta=0.5,
                                  const PS::S32 nleafmax=8,
                                  const PS::S32 ngroupmax=64)
\end{verbatim}
\end{screen}

\begin{itemize}

\item{{\bf 引数}}

{\tt ntot}: 入力。{\tt const PS::S64}型。相互作用にかかわる全粒子数。

{\tt theta}: 入力。{\tt const PS::F32}型。ツリーのオープニングクライテ
リオン。長距離相互作用において、あるツリーセルからある粒子(実際にはi粒
子グループ)への力を計算する時に見込角が{\tt theta}以下の時はツリーセル
の多重極モーメントを用いて相互作用を評価する。近距離力では任意の値でよ
い。デフォルト0.5。

{\tt nleafmax}: 入力。{\tt const PS::S32}型。ツリーリーフセル内の最大粒
子数。あるセル内にこの数を超える粒子が入っていた場合、そのセルはさらに
分割される。デフォルト8。

{\tt ngroupmax}: 入力。{\tt const PS::S32}型。i粒子グループ内の最大粒子
数。i粒子グループとは相互作用リスト(\verb|EssentialParticleJ|型、
\verb|SuperParticleJ|型の配列)を共有するi粒子のグループである。ツリー
構造にそってグルーピングする為、グループ内の粒子は固まって存在している。
グループの粒子数は{\tt ngroupmax}を超えないようにグルーピングが行われる。
デフォルト64。

\item{{\bf 返り値}}

なし。

\item{{\bf 機能}}

相互作用ツリーの初期設定を行い、\verb|EssentialParticleI|型、
\verb|EssentialParticleJ|型、\verb|SuperParticleJ|型、\verb|Force|型、
\verb|PS::TreeParticle|型、\verb|PS::TreeCell|型のメンバの配列を確保す
る。配列のサイズを推定するのに{\tt ntot}を使う。自クラス内にi粒子グルー
プ内の最大粒子数とツリーリーフセル内の最大粒子数をセットする。{\tt
theta}は長距離力用のツリーのオープニングクライテリオンであり、短距離力
の場合は値は何でもよい。

\end{itemize}
%%%%%%%%%%%%%%%


%\begin{screen}
%\begin{verbatim}
%void PS::TreeForForce::initializeLocalTree(const PS::F32 len_half);
%\end{verbatim}
%\end{screen}

%\begin{itemize}

%\item{{\bf 引数}}

%{\tt len\_half}: 入力。{\tt const PS::F32}型。ツリーのルートセルの一辺の
半分の長さ。

%\item{{\bf 返り値}}

%なし。

%\item{{\bf 機能}}

%ツリーのルートセルの一辺の半分の長さを与える。

%\end{itemize}
%%%%%%%%%%%%%%%


%%%%%%%%%%%%%%%
\paragraph{粒子読み込み}
\mbox{}
%%%%%%%%%%%%%%%

%%%%%%%%%%%%%%%
\begin{screen}
\begin{verbatim}
void PS::TreeForForce::setParticleLocalTree(
                       const PS::ParticleSystem & psys,
                       const bool clear=true);
\end{verbatim}
\end{screen}

\begin{itemize}

\item{{\bf 引数}}

{\tt psys}: 入力。{\tt const PS::ParticleSystem \&}型。

{\tt clear}: 入力。{\tt const bool clear}型。クリアフラグ。デフォルトtrue。

\item{{\bf 返り値}}

なし。

\item{{\bf 機能}}

{\tt psys}から{\tt FullParticle}を読み込み、メンバの{\tt
PS::TreeParticle}型、{\tt EssentailParticleI}、{\tt
EssentailParticleJ}型の配列にツリー生成や力の計算に必要な情報をコピーす
る。

\end{itemize}



%%%%%%%%%%%%%%%
\begin{screen}
\begin{verbatim}
void PS::TreeForForce::setLocalEssentialTreeToGlobalTree();

\end{verbatim}
\end{screen}

\begin{itemize}

\item{{\bf 引数}}

\item{{\bf 返り値}}

なし。

\item{{\bf 機能}}

{\tt exchangeLocalEssentialTree}で送られて来た粒子を{\tt TreeParticle}
型に変換する。

\end{itemize}

%%%%%%%%%%%%%%%
\paragraph{ツリー生成}
\mbox{}
%%%%%%%%%%%%%%%

%%%%%%%%%%%%%%%
\begin{screen}
\begin{verbatim}
void PS::TreeForForce::makeLocalTree(const bool cancel=false);
\end{verbatim}
\end{screen}

\begin{itemize}

\item{{\bf 引数}}

{\tt cancel}: 入力。{\tt const bool}型。キャンセルフラグ。デフォルトfalse。

\item{{\bf 返り値}}

なし。

\item{{\bf 機能}}

LTを作り、モートンソートを行い、モーメント計算まで行う。LTづくりとモー
トンソートは{\tt cancel}が{\tt true}なら行わない。

\end{itemize}



%%%%%%%%%%%%%%%
\begin{screen}
\begin{verbatim}
void PS::TreeForForce::exchangeLocalEssentialTree()

\end{verbatim}
\end{screen}

\begin{itemize}

\item{{\bf 引数}}

なし。

\item{{\bf 返り値}}

なし。

\item{{\bf 機能}}

LocalEssentialTreeを作り、各プロセスに送る。イメージ粒子が存在する場合
は、それらもLocalEssentialTreeの中に組み込まれる。

\end{itemize}


%%%%%%%%%%%%%%%
\begin{screen}
\begin{verbatim}
void PS::TreeForForce::makeGlobalTree(const bool cancel=false);
\end{verbatim}
\end{screen}
\begin{itemize}

\item{{\bf 引数}}

{\tt cancel}: 入力。{\tt const bool}型。キャンセルフラグ。デフォルトfalse。

\item{{\bf 返り値}}

なし。

\item{{\bf 機能}}

GTを作り、モートンソートを行い、モーメント計算まで行う。
GTづくりとモートンソートは{\tt cancel}が{\tt true}なら行わない。

\end{itemize}


%%%%%%%%%%%%%%%
\paragraph{力の計算}
\mbox{}
%%%%%%%%%%%%%%%

以下に述べる6つの関数はテンプレート関数となっており、関数の引数として
{\tt Tfunc\_ep\_ep}型や{\tt Tfunc\_ep\_sp}型を取る。これらはユーザーが定義
した相互作用関数のポインタもしくは相互作用関数オブジェクトを取る。

関数ポインタを使う場合
\begin{lstlisting}[caption=関数ポインタを使う場合]
void CalcForceEpEp(const EssentialPtclI * ep_i,
                   const PS::S32 n_ip,
                   const EssentialPtclJ * ep_j,
                   const PS::S32 n_jp,
                   ResultForce * force){
......
}
\end{lstlisting}

関数オブジェクトを使う場合
\begin{lstlisting}[caption=関数ポインタを使う場合]
class CalcForceEpEp{
public:
    void operator()(const EssentialPtclI * ep_i,
                    const PS::S32 n_ip,
                    const EssentialPtclJ * ep_j,
                    const PS::S32 n_jp,
                    ResultForce * force){
......
    }
}
\end{lstlisting}

関数ポインタ、関数オブジェクトどちらを使う場合も返り値はvoid型とし、引
数は第一引数から順に{\tt const EssentialParticleI *}型、{\tt PS::S32}型、
{\tt const EssentialParticleJ *}型、{\tt PS::S32}型、{\tt Force *}型。


%%%%%%%%%%%%%%%
\begin{screen}
\begin{verbatim}
template<class Tfunc_ep_ep>
void PS::TreeForForce::calcForce( Tfunc_ep_ep func_ep_ep);

\end{verbatim}
\end{screen}

\begin{itemize}

\item{{\bf 引数}}

{\tt (func\_ep\_ep)}: 入力。返り値がvoid型の{EssentialParticleI}と
{EssentialParticleJ}間の相互作用計算用関数ポインタ、もしくは関数オブジェ
クト。関数の引数は第一引数から順に{\tt const EssentialParticleI *}型、
{\tt PS::S32}型、{\tt const EssentialParticleJ *}型、{\tt PS::S32}型、
{\tt Force *}型。

\item{{\bf 返り値}}

なし。

\item{{\bf 機能}}

全粒子に対して、GT内のツリーウォークと相互作用計算を行い、結果を\\
{\tt PS::TreeForForce::setParticleFromFullParticle()}で読み込んだ時の粒
子の並びと同じ順番で格納する。この関数を呼ぶ前に
\verb|PS::TreeForForce::makeIPGroup()|で、i粒子をグルーピングしておく必
要がある。

\end{itemize}

%%%%%%%%%%%%%%%
\begin{screen}
\begin{verbatim}
template<class Tfunc_ep_ep>
void PS::TreeForForce::calcForceAll( Tfunc_ep_ep func_ep_ep,  PS::ParticleSystem & psys) 

\end{verbatim}
\end{screen}

\begin{itemize}

\item{{\bf 引数}}

{\tt (func\_ep\_ep)}: 入力。返り値がvoid型の{EssentialParticleI}と
{EssentialParticleJ}間の相互作用計算用関数ポインタ、もしくは関数オブジェ
クト。関数の引数は第一引数から順に{\tt const EssentialParticleI *}型、
{\tt PS::S32}型、{\tt const EssentialParticleJ *}型、{\tt PS::S32}型、
{\tt Force *}型。

{\tt psys}: 入力。{\tt PS::ParticleSystem \&}型。

\item{{\bf 返り値}}

なし。

\item{{\bf 機能}}

粒子移動、LT作り、LET交換、GT作り、相互作用計算まで行い、結果を\\ {\tt
PS::TreeForForce::setParticleFromFullParticle()}で読み込んだ時の粒子の
並びと同じ順番で格納する。

\end{itemize}

%%%%%%%%%%%%%%%
\begin{screen}
\begin{verbatim}
template<class Tfunc_ep_ep>
void PS::TreeForForce::calcForceAll( Tfunc_ep_ep func_ep_ep,  PS::ParticleSystem & psys) 

\end{verbatim}
\end{screen}

\begin{itemize}

\item{{\bf 引数}}

{\tt (func\_ep\_ep)}: 入力。返り値がvoid型の{EssentialParticleI}と
{EssentialParticleJ}間の相互作用計算用関数ポインタ、もしくは関数オブジェ
クト。関数の引数は第一引数から順に{\tt const EssentialParticleI *}型、
{\tt PS::S32}型、{\tt const EssentialParticleJ *}型、{\tt PS::S32}型、
{\tt Force *}型。

{\tt psys}: 入力。{\tt PS::ParticleSystem \&}型。

\item{{\bf 返り値}}

なし。

\item{{\bf 機能}}

粒子移動、LT作り、LET交換、GT作り、相互作用計算まで行い、結果を\\
{\tt PS::TreeForForce::setParticleFromFullParticle()}で読み込んだ時の粒
子の並びと同じ順番で格納し、さらに{\tt psys}にも書き戻す。

\end{itemize}









%%%%%%%%%%%%%%%
\begin{screen}
\begin{verbatim}
template<class Tfunc_ep_ep, class Tfunc_ep_sp>
void PS::TreeForForce::calcForce( Tfunc_ep_ep func_ep_ep, Tfunc_ep_sp func_epsp);

\end{verbatim}
\end{screen}

\begin{itemize}

\item{{\bf 引数}}

{\tt (func\_ep\_ep)}: 入力。返り値がvoid型の{EssentialParticleI}と
{EssentialParticleJ}間の相互作用計算用関数ポインタ、もしくは関数オブジェ
クト。関数の引数は第一引数から順に{\tt const EssentialParticleI *}型、
{\tt PS::S32}型、{\tt const EssentialParticleJ *}型、{\tt PS::S32}型、
{\tt Force *}型。

{\tt (func\_ep\_sp)}: 入力。返り値がvoid型の{EssentialParticleI}と
{SuperParticleJ}間の相互作用計算用関数ポインタ、もしくは関数オブジェ
クト。関数の引数は第一引数から順に{\tt const EssentialParticleI *}型、
{\tt PS::S32}型、{\tt const SuperParticleJ *}型、{\tt PS::S32}型、
{\tt Force *}型。

\item{{\bf 返り値}}

なし。

\item{{\bf 機能}}

全粒子に対して、GT内のツリーウォークと相互作用計算を行い、結果を\\
{\tt PS::TreeForForce::setParticleFromFullParticle()}で読み込んだ時の粒
子の並びと同じ順番で格納する。この関数を呼ぶ前に
\verb|PS::TreeForForce::makeIPGroup()|で、i粒子をグルーピングしておく必
要がある。

\end{itemize}

%%%%%%%%%%%%%%%
\begin{screen}
\begin{verbatim}
template<class Tfunc_ep_ep, class Tfunc_ep_sp>
void PS::TreeForForce::calcForceAll( Tfunc_ep_ep func_ep_ep,   Tfunc_ep_sp func_ep_sp,  
                                     PS::ParticleSystem & psys) 

\end{verbatim}
\end{screen}

\begin{itemize}

\item{{\bf 引数}}

{\tt (func\_ep\_ep)}: 入力。返り値がvoid型の{EssentialParticleI}と
{EssentialParticleJ}間の相互作用計算用関数ポインタ、もしくは関数オブジェ
クト。関数の引数は第一引数から順に{\tt const EssentialParticleI *}型、
{\tt PS::S32}型、{\tt const EssentialParticleJ *}型、{\tt PS::S32}型、
{\tt Force *}型。

{\tt (func\_ep\_sp)}: 入力。返り値がvoid型の{EssentialParticleI}と
{SuperParticleJ}間の相互作用計算用関数ポインタ、もしくは関数オブジェ
クト。関数の引数は第一引数から順に{\tt const EssentialParticleI *}型、
{\tt PS::S32}型、{\tt const SuperParticleJ *}型、{\tt PS::S32}型、
{\tt Force *}型。

{\tt psys}: 入力。{\tt PS::ParticleSystem \&}型。

\item{{\bf 返り値}}

なし。

\item{{\bf 機能}}

粒子移動、LT作り、LET交換、GT作り、相互作用計算まで行い、結果を\\
{\tt PS::TreeForForce::setParticleFromFullParticle()}で読み込んだ時の粒
子の並びと同じ順番で格納する。

\end{itemize}

%%%%%%%%%%%%%%%
\begin{screen}
\begin{verbatim}
template<class Tfunc_ep_ep, class Tfunc_ep_sp>
void PS::TreeForForce::calcForceAllAndWriteBack( Tfunc_ep_ep func_ep_ep,  Tfunc_ep_sp func_ep_sp, 
                                                 PS::ParticleSystem & psys) 

\end{verbatim}
\end{screen}

\begin{itemize}

\item{{\bf 引数}}

{\tt (func\_ep\_ep)}: 入力。返り値がvoid型の{EssentialParticleI}と
{EssentialParticleJ}間の相互作用計算用関数ポインタ、もしくは関数オブジェ
クト。関数の引数は第一引数から順に{\tt const EssentialParticleI *}型、
{\tt PS::S32}型、{\tt const EssentialParticleJ *}型、{\tt PS::S32}型、
{\tt Force *}型。

{\tt (func\_ep\_sp)}: 入力。返り値がvoid型の{EssentialParticleI}と
{SuperParticleJ}間の相互作用計算用関数ポインタ、もしくは関数オブジェ
クト。関数の引数は第一引数から順に{\tt const EssentialParticleI *}型、
{\tt PS::S32}型、{\tt const SuperParticleJ *}型、{\tt PS::S32}型、
{\tt Force *}型。

{\tt psys}: 入力。{\tt PS::ParticleSystem \&}型。

\item{{\bf 返り値}}

なし。

\item{{\bf 機能}}

粒子移動、LT作り、LET交換、GT作り、相互作用計算まで行い、結果を\\ {\tt
PS::TreeForForce::setParticleFromFullParticle()}で読み込んだ時の粒子の
並びと同じ順番で格納し、さらに{\tt psys}にも書き戻す。

\end{itemize}



%%%%%%%%%%%%%%%
\begin{screen}
\begin{verbatim}
void PS::TreeForForce::makeIPGroup()
\end{verbatim}
\end{screen}
\begin{itemize}

\item{{\bf 引数}}

なし。

\item{{\bf 返り値}}

なし。

\item{{\bf 機能}}

i粒子をツリー構造にしたがってグルーピングする。グループ内の最大粒子数は
\\{\tt PS::TreeForForce::initialize()}の第四引数でセットされた値である。

\end{itemize}

%%%%%%%%%%%%%%%
\begin{screen}
\begin{verbatim}
PS::S32 & PS::TreeForForce::getNumberOfIPG()
\end{verbatim}
\end{screen}

\begin{itemize}

\item{{\bf 引数}}

なし。

\item{{\bf 返り値}}

{PS::S32}型。i粒子グループの数を返す。

\end{itemize}

%%%%%%%%%%%%%%%
%\begin{screen}
%\begin{verbatim}
%void PS::TreeForForce::calcForce(
%                       void (*pfunc)(const EssentialParticleI * , 
%                                     const PS::S32 , 
%                                     const EssentialParticleJ *, 
%                                     const PS::S32, 
%                                     Force *) )
%\end{verbatim}
%\end{screen}
%
%\begin{itemize}
%
%\item{{\bf 引数}}
%
%{\tt (*pfunc)}: 入力。返り値はvoid型の相互作用計算用関数ポインタ。関数の引数
%は第一引数から順に{\tt const EssentialParticleI *}型、{\tt PS::S32}型、
%{\tt const EssentialParticleJ *}型、{\tt PS::S32}型、{\tt Force *}型。
%
%\item{{\bf 返り値}}
%
%なし。
%
%\item{{\bf 機能}}
%
%関数ポインタを用いて全粒子に及ぼされる力を計算し、結果を\\ {\tt
%PS::TreeForForce::setParticleFromFullParticle()}で読み込んだ時の粒子の
%並びと同じ順番で格納する。
%
%\end{itemize}

%%%%%%%%%%%%%%%%%%%%%%%%%
\paragraph{ネイバーリスト}
\mbox{}
%%%%%%%%%%%%%%%%%%%%%%%%%

%%%%%%%%%%%%%%%
\begin{screen}
\begin{verbatim}
template<class Tsearchmode,
         class Tptcl>
void PS::TreeForForce::getNeighborListOneParticle
             (const Tptcl & ptcl,
              PS::S32 & nnp,
              Tepj * (& epj));
\end{verbatim}
\end{screen}

\begin{itemize}

\item{{\bf テンプレート引数}}

{\tt Tsearchmode}: 入力。{\tt SEARCH\_MODE}型。

{\tt Tptcl}: 入力。{\tt Tptcl}型。入力しなくてOK。

\item{{\bf 引数}}

{\tt ptcl}: 入力。{\tt const Tptcl \&}型。

{\tt nnp}: 出力。{\tt PS::S32 \&}型。

{\tt epj}: 出力。{\tt Tepj * (\&)}型。

\item{{\bf 返り値}}

なし

\item 機能

ある1つの粒子のネイバーリストを返す。ネイバーリストの探し方をテンプレー
ト第1引数{\tt Tsearchmode}で指定する。この相互作用ツリークラスの{\tt
SEARCH\_MODE}型が収集モードである場合に、{\tt Tsearchmode}に散乱モード
や対称モードを選んだら、コンパイルエラーとなる。テンプレート第2引数
{\tt Tptcl}には入力の必要はないが、もし入力する場合は第1引数の型と同じ
でないと、コンパイルエラーとなる。粒子の指定を第1引数{\tt ptcl}で行う。
この型はメンバ関数に{\tt getPos}を含む必要がある。第2引数{\tt nnp}にネ
イバー粒子の数を返す。第3引数{\tt epj}にネイバーリストを返す。{\tt
epj}の型はこの相互作用ツリークラスのEssentialParticleJ型と同じ型でなけ
ればならず、そうでない場合はコンパイルエラーとなる。{\tt epj}のメモリの
確保はこの関数内で行う。このリストの寿命は、次にネイバーリストを作る関
数を呼ぶまでとする。

\end{itemize}

%%%%%%%%%%%%%%%
\begin{screen}
\begin{verbatim}
void PS::TreeForForce::getNeighborListOneIPGroup
              (const PS::S32 iipg,
               PS::S32 & nip,
               const Tepi * epi,
               PS::S32 & nnp,
               Tepj * (& epj));
\end{verbatim}
\end{screen}

\begin{itemize}

\item{{\bf 引数}}

{\tt iipg}: 入力。{\tt const PS::S32}型。

{\tt nip}: 出力。{\tt PS::S32 \&}型。

{\tt epi}: 出力。{\tt const epi *}型。

{\tt nnp}: 出力。{\tt PS::S32 \&}型。

{\tt epj}: 出力。{\tt Tepj * (\&)}型。

\item{{\bf 返り値}}

なし

\item 機能

この相互作用ツリークラスのある1つのiグループのネイバーリストの和集合
(以下ネイバーリストと省略)を返す。iグループの指定を第1引数{\tt iipg}で
行う。ユーザーはiグループの数を{\tt PS::TreeForForce::getNumberOfIPG}で
知ることはできるので、{\tt for}ループでまわせば、全iグループのネイバー
リストを得ることができる。第2引数{\tt nip}に指定したiグループの粒子の
数を返す。第3引数{\tt epi}にこのi グループの粒子リストを返す。この型は
この相互作用ツリークラスのEssentialParticleI型と同じである必要があり、
そうでないとコンパイルエラーとなる。第4引数{\tt nnp}にこのiグループの
ネイバー粒子の数を返す。第5引数{\tt epj}にこのiグループのネイバーリス
トを返す。{\tt epj}の型はこの相互作用ツリークラスのEssentialParticleJ型
と同じ型でなければならず、そうでない場合はコンパイルエラーとなる。{\tt
epi}と{\tt epj}のメモリの確保は、この関数内で行う。これらの寿命は{\tt
getNeighborListOneParticle}のネイバーリストの寿命と同じ。

\end{itemize}

%%%%%%%%%%%%%%%
\begin{screen}
\begin{verbatim}
template <class Tsearchmode,
          class TTreeForForce,
          class Tepi2>
void PS::TreeForForce::getNeighborListOneIPGroup
              (const PS::S32 iipg,
               const TTreeForForce & ttf,
               PS::S32 & nip,
               const Tepi2 * epi,
               PS::S32 & nnp,
               Tepj * (& epj));
\end{verbatim}
\end{screen}

\begin{itemize}

\item{{\bf テンプレート引数}}

{\tt Tsearchmode}: 入力。{\tt Tsearchmode}型。

{\tt TTreeForForce}: 入力。{\tt TTreeForForce}型。入力しなくてもOK。

{\tt Tepi2}: 入力。{\tt Tepi2}型。入力しなくてもOK。

\item{{\bf 引数}}

{\tt iipg}: 入力。{\tt const PS::S32}型。

{\tt ttf}: 入力。{\tt const TTreeForForce \&}型。

{\tt nip}: 出力。{\tt PS::S32 \&}型。

{\tt epi}: 出力。{\tt const epi2 *}型。

{\tt nnp}: 出力。{\tt PS::S32 \&}型。

{\tt epj}: 出力。{\tt Tepj * (\&)}型。

\item{{\bf 返り値}}

なし

\item 機能

相互作用クラスのインスタンス{\tt ttf}に属すある1つのiグループのネイバー
リストの和集合(以下ネイバーリストと省略)を返す。ネイバーリストの探し方
をテンプレート第1引数{\tt Tsearchmode}で指定する。ネイバーリストを返す
側の相互作用クラスの{\tt SEARCH\_MODE}型が収集モードである場合に、テン
プレート第1引数に散乱モードや対称モードを選んだら、コンパイルエラーと
なる。テンプレート第2({\tt TTreeForForce})、3引数({\tt Tepi2})に入力
の必要はないが、入力する場合、それぞれ第2引数{\tt ttf}と第4引数{\tt
epi}と同じでなければ、コンパイルエラーとなる。iグループの指定を第1引数
{\tt iipg} で行う。ユーザーはiグループの数を{\tt ttf.getNumberOfIPG}で
知ることはできるので、{\tt for}ループでまわせば、全iグループのネイバー
リストを得ることができる。第2引数{\tt ttf}でこのiグループの属す相互作
用ツリークラスのインスタンスを指定する。第3引数{\tt nip}にiグループの
粒子の数を返す。第4引数{\tt epi}にこのi グループの粒子リストを返す。
{\tt epi}の型は{\tt ttf}のEssentialParticleI型と同じである必要があり、
そうでないとコンパイルエラーとなる。第5引数{\tt nnp}にこのiグループの
ネイバー粒子の数を返す。第6引数{\tt epj}にこのiグループのネイバーリス
トを返す。{\tt epj}の型はネイバーリストを返す側の相互作用ツリークラスの
EssentialParticleJ 型と同じ型でなければならず、そうでない場合はコンパイ
ルエラーとなる。{\tt epi}と{\tt epj}のメモリの確保は、この関数内で行う。
これらの寿命は{\tt getNeighborListOneParticle}のネイバーリストの寿命と
同じ。

\end{itemize}

%%%%%%%%%%%%%%%
\begin{screen}
\begin{verbatim}
void PS::TreeForForce::getNeighborListOneIPGroupEachParticle
              (const PS::S32 iipg,
               PS::S32 & nip,
               const Tepi * epi,
               PS::S32 * (& nnp),
               Tepj * (& epj));
\end{verbatim}
\end{screen}

\begin{itemize}

\item{{\bf 引数}}

{\tt iipg}: 入力。{\tt const PS::S32}型。

{\tt nip}: 出力。{\tt PS::S32 \&}型。

{\tt epi}: 出力。{\tt const epi *}型。

{\tt nnp}: 出力。{\tt PS::S32 * (\&)}型。

{\tt epj}: 出力。{\tt Tepj * (\&)}型。

\item{{\bf 返り値}}

なし

\item 機能

この相互作用ツリークラスのある1つのiグループの粒子それぞれのネイバーリ
ストを返す。iグループの指定を第1引数{\tt iipg}で行う。ユーザーはiグルー
プの数を{\tt PS::TreeForForce::getNumberOfIPG}で知ることはできるので、
{\tt for}ループでまわせば、全iグループのネイバーリストを得ることができ
る。第2引数{\tt nip}に指定したiグループの粒子の数を返す。第3引数{\tt
epi}にこのiグループの粒子リストを返す。この型はこの相互作用ツリークラス
のEssentialParticleI型と同じである必要があり、そうでないとコンパイルエ
ラーとなる。第4引数{\tt nnp}と第5引数{\tt epj}に返すものは以下の通り
である。{\tt epj}には、リスト{\tt epi}の粒子順にネイバーリストを返す。
ネイバーリスト内の変位を第4引数{\tt nnp}に返す。{\tt epj}の型はこの相
互作用ツリークラスのEssentialParticleJ型と同じ型でなければならず、そう
でない場合はコンパイルエラーとなる。{\tt epi}、{\tt nnp}、{\tt epj}のメ
モリの確保は、この関数内で行う。これらのリストの寿命は{\tt
getNeighborListOneParticle}のネイバーリストの寿命と同じ。

\end{itemize}

%%%%%%%%%%%%%%%
\begin{screen}
\begin{verbatim}
template <class Tsearchmode,
          class TTreeForForce,
          class Tepi2>
void PS::TreeForForce::getNeighborListOneIPGroupEachParticle
              (const PS::S32 iipg,
               const TTreeForForce & ttf,
               PS::S32 & nip,
               const Tepi2 * epi,
               PS::S32 * (& nnp),
               Tepj * (& epj));
\end{verbatim}
\end{screen}

\begin{itemize}

\item{{\bf テンプレート引数}}

{\tt Tsearchmode}: 入力。{\tt Tsearchmode}型。

{\tt TTreeForForce}: 入力。{\tt TTreeForForce}型。入力しなくてもOK。

{\tt Tepi2}: 入力。{\tt Tepi2}型。入力しなくてもOK。

\item{{\bf 引数}}

{\tt iipg}: 入力。{\tt const PS::S32}型。

{\tt ttf}: 入力。{\tt const TTreeForForce \&}型。

{\tt nip}: 出力。{\tt PS::S32 \&}型。

{\tt epi}: 出力。{\tt const epi2 *}型。

{\tt nnp}: 出力。{\tt PS::S32 * (\&)}型。

{\tt epj}: 出力。{\tt Tepj * (\&)}型。

\item{{\bf 返り値}}

なし

\item 機能

相互作用クラスのインスタンス{\tt ttf}に属すある1つのiグループの粒子そ
れぞれのネイバーリストを返す。ネイバーリストの探し方をテンプレート第1
引数{\tt Tsearchmode}で指定する。ネイバーリストを返す側の相互作用クラス
の{\tt SEARCH\_MODE}型が収集モードである場合に、{\tt Tsearchmode}に散乱
モードや対称モードを選んだら、コンパイルエラーとなる。テンプレート第2
({\tt TTreeForForce})、3引数({\tt Tepi2})に入力の必要はないが、入力す
る場合、それぞれ第2引数{\tt ttf}と第4引数{\tt epi}と同じでなければ、
コンパイルエラーとなる。iグループの指定を第1引数{\tt iipg} で行う。ユー
ザーはiグループの数を{\tt ttf.getNumberOfIPG}で知ることはできるので、
{\tt for}ループでまわせば、全iグループのネイバーリストを得ることができ
る。第2引数{\tt ttf}でこのiグループの属す相互作用ツリークラスのインス
タンスを指定する。第3引数{\tt nip}にiグループの粒子の数を返す。第4引
数{\tt epi}にこのi グループの粒子リストを返す。{\tt epi}の型は{\tt
ttf}のEssentialParticleI型と同じである必要があり、そうでないとコンパイ
ルエラーとなる。第5引数{\tt nnp}と第6引数{\tt epj}に返すものは以下の
通りである。{\tt epj}には、リスト{\tt epi}の粒子順にネイバーリストを返
す。ネイバーリスト内の変位を{\tt nnp}に返す。{\tt epj}の型はこの相互作
用ツリークラスのEssentialParticleJ型と同じ型でなければならず、そうでな
い場合はコンパイルエラーとなる。{\tt epi}、{\tt nnp}、{\tt epj}のメモリ
の確保は、この関数内で行う。これらの寿命は{\tt
getNeighborListOneParticle}のネイバーリストの寿命と同じ。

\end{itemize}

%%%%%%%%%%%%%%%%%%%%%%%%%
\paragraph{その他}
\mbox{}
%%%%%%%%%%%%%%%%%%%%%%%%%

%%%%%%%%%%%%%%%
\begin{screen}
\begin{verbatim}
FORCE & PS::TreeForForce::getForce(const PS::S32 id)
\end{verbatim}
\end{screen}

\begin{itemize}

\item{{\bf 引数}}

{\tt id}: 入力。{\tt const PS::S32}。粒子インデクス。

\item{{\bf 返り値}}

{\tt Force \&}型。{\tt id}番目に挿入した粒子にかかる力を返す。

\end{itemize}

%%%%%%%%%%%%%%%
\begin{screen}
\begin{verbatim}
template<class TTreeForForce>
void PS::TreeForForce::copyLocalTreeStructure(const TTreeForForce & tff)
\end{verbatim}
\end{screen}

\begin{itemize}

\item{{\bf 引数}}

{\tt tff}: 入力。すでにローカルツリーを持っている相互作用ツリークラスの
インスタンス。

\item{{\bf 返り値}}

なし。

\item 機能

{\tt tff}の中にあるローカルツリーを\redtext{自分に}コピーする。

\end{itemize}

%%%%%%%%%%%%%%%
\begin{screen}
\begin{verbatim}
bool PS::TreeForForce::repeatLocalCalcForce()
\end{verbatim}
\end{screen}

\begin{itemize}

\item{{\bf 引数}}

なし。

\item{{\bf 返り値}}

{\tt bool}型。

\item 機能

ユーザーが相互作用計算カーネル内で{\tt EssentialParticleI}型のサーチ半
径を変更した後、もう一度ローカルだけで相互作用の計算が可能かどうかを判
定する。可能な場合は{\tt true}を返し、不可能な場合は{\tt false}を返す。

\end{itemize}


%%%%%%%%%%%%%%%


\subsection{通信クラス({\tt PS::Comm})}
\label{sec:parameter}
\subsubsection{概要}

並列化手法(MPI, OpenMP, SIMD)には様々な最適化パラメータが存在する。MPI
関連のパラメータやコミュニケータはシングルトンパターンを使って管理する。

以下のように定義する。

\begin{lstlisting}[caption={\tt Comm}の定義]
namespace ParticleSimulator {
    class Comm{
    public:
        static S32 getRank();
        static S32 getNumberOfProc();
        static S32 getRankMultiDim(const S32 id);
        static S32 getNumberOfProcMultiDim(const S32 id);
        static bool synchronizeConditionalBranchAND(const bool local);        
        static bool synchronizeConditionalBranchOR(const bool local);        
        template<class T>
        static T getMinValue(const T val);
        template<class T>
        static T getMaxValue(const T val);
        template<class Tfloat, class Tint>
        static void getMinValue(const Tfloat f_in, const Tint i_in, Tfloat & f_out, Tint & i_out);
        template<class Tfloat, class Tint>
        static void getMaxValue(const Tfloat f_in, const Tint i_in, Tfloat & f_out, Tint & i_out);
        template<class T>
        static T getSum(const T val);
}
\end{lstlisting}

\subsubsection{API}

%%%%%%%%%%%%%%%%%%%%%%%%%%%
\begin{screen}
\begin{verbatim}
static PS::S32 getRank();
\end{verbatim}
\end{screen}

\begin{itemize}

\item{{\bf 引数}}

なし。

\item{{\bf 返り値}}

{\tt PS::S32}型。全プロセス中でのランクを返す。

\end{itemize}

%%%%%%%%%%%%%%%%%%%%%%%%%%%
\begin{screen}
\begin{verbatim}
static PS::S32 PS::Comm::getNumberOfProc();
\end{verbatim}
\end{screen}

\begin{itemize}

\item{{\bf 引数}}

なし。

\item{{\bf 返り値}}

{\tt PS::S32}型。全プロセス数を返す。

\end{itemize}

%%%%%%%%%%%%%%%%%%%%%%%%%%%
\begin{screen}
\begin{verbatim}
static PS::S32 PS::Comm::getRankMultiDim(const PS::S32 id);
\end{verbatim}
\end{screen}

\begin{itemize}

\item{{\bf 引数}}

{\tt id}: 入力。{\tt const PS::S32}型。軸の番号。x軸:0, y軸:1, z軸:2。

\item{{\bf 返り値}}

{\tt PS::S32}型。id番目の軸でのランクを返す。2次元の場合、id=2は1を返す。

\end{itemize}

%%%%%%%%%%%%%%%%%%%%%%%%%%%
\begin{screen}
\begin{verbatim}
static PS::S32 PS::Comm::getNumberOfProcMultiDim(const PS::S32 id);
\end{verbatim}
\end{screen}

\begin{itemize}

\item{{\bf 引数}}

{\tt id}: 入力。{\tt const PS::S32}型。軸の番号。x軸:0, y軸:1, z軸:2。

\item{{\bf 返り値}}

{\tt PS::S32}型。id番目の軸のプロセス数を返す。2次元の場合、id=2は1を返す。

\end{itemize}

%%%%%%%%%%%%%%%%%%%%%%%%%%%
\begin{screen}
\begin{verbatim}
static bool PS::Comm::synchronizeConditionalBranchAND(const bool local)
\end{verbatim}
\end{screen}

\begin{itemize}

\item{{\bf 引数}}

{\tt local}: 入力。{\tt const bool}型。

\item{{\bf 返り値}}

{\tt bool}型。全プロセスで{\tt local}の{\tt AND}を取り、結果を返す。

\end{itemize}

%%%%%%%%%%%%%%%%%%%%%%%%%%%
\begin{screen}
\begin{verbatim}
static bool PS::Comm::synchronizeConditionalBranchOR(const bool local);
\end{verbatim}
\end{screen}

\begin{itemize}

\item{{\bf 引数}}

{\tt local}: 入力。{\tt const bool}型。

\item{{\bf 返り値}}

{\tt bool}型。全プロセスで{\tt local}の{\tt OR}を取り、結果を返す。

\end{itemize}

%%%%%%%%%%%%%%%%%%%%%%%%%%%
\begin{screen}
\begin{verbatim}
template <class T>
static T PS::Comm::getMinValue(const T val);
\end{verbatim}
\end{screen}

\begin{itemize}

\item{{\bf 引数}}

{\tt val}: 入力。{\tt const T}型。

\item{{\bf 返り値}}

{\tt T}型。全プロセスで{\tt val}の最小値を取り、結果を返す。

\end{itemize}

%%%%%%%%%%%%%%%%%%%%%%%%%%%
\begin{screen}
\begin{verbatim}
template <class T>
static T PS::Comm::getMaxValue(const T val);
\end{verbatim}
\end{screen}

\begin{itemize}

\item{{\bf 引数}}

{\tt val}: 入力。{\tt const T}型。

\item{{\bf 返り値}}

{\tt T}型。全プロセスで{\tt val}の最大値を取り、結果を返す。

\end{itemize}

%%%%%%%%%%%%%%%%%%%%%%%%%%%
\begin{screen}
\begin{verbatim}
template <class Tfloat, class Tint>
static void PS::Comm::getMinValue(const Tfloat f_in, const Tint i_in,
                                  Tfloat & f_out, Tint & i_out);
\end{verbatim}
\end{screen}

\begin{itemize}

\item{{\bf 引数}}

{\tt f\_in}: 入力。{\tt const Tfloat}型。

{\tt i\_in}: 入力。{\tt const Tint}型。

{\tt f\_out}: 出力。{\tt Tfloat}型。全プロセスで{\tt f\_in}の最小値を取
り、結果を返す。

{\tt i\_out}: 出力。{\tt Tint}型。{\tt f\_out}に伴うIDを返す。

\item{{\bf 返り値}}

なし。

\end{itemize}

%%%%%%%%%%%%%%%%%%%%%%%%%%%
\begin{screen}
\begin{verbatim}
template <class Tfloat, class Tint>
static void PS::Comm::getMaxValue(const Tfloat f_in, const Tint i_in,
                                  Tfloat & f_out, Tint & i_out);
\end{verbatim}
\end{screen}

\begin{itemize}

\item{{\bf 引数}}

{\tt f\_in}: 入力。{\tt const Tfloat}型。

{\tt i\_in}: 入力。{\tt const Tint}型。

{\tt f\_out}: 出力。{\tt Tfloat}型。全プロセスで{\tt f\_in}の最大値を取
り、結果を返す。

{\tt i\_out}: 出力。{\tt Tint}型。{\tt f\_out}に伴うIDを返す。

\item{{\bf 返り値}}

なし。

\end{itemize}

%%%%%%%%%%%%%%%%%%%%%%%%%%%
\begin{screen}
\begin{verbatim}
template <class T>
static T PS::Comm::getSum(const T val);
\end{verbatim}
\end{screen}

\begin{itemize}

\item{{\bf 引数}}

{\tt val}: 入力。{\tt const T}型。

\item{{\bf 返り値}}

{\tt T}型。全プロセスで{\tt val}の総和を取り、結果を返す。

\end{itemize}


%%%%%%%%%%%%%%%%%%%%%%%%%%%
%\begin{screen}
%\begin{verbatim}
%static void PS::Comm::setNProcXD(const PS::S32 nx, const PS::S32 ny, const PS::S32 nz);
%\end{verbatim}
%\end{screen}

%\begin{itemize}

%\item{{\bf 引数}}

%{\tt nx}: 入力。{\tt const PS::S32}型。x軸方向のプロセスの分割数。

%{\tt ny}: 入力。{\tt const PS::S32}型。y軸方向のプロセスの分割数。

%{\tt nz}: 入力。{\tt const PS::S32}型。z軸方向のプロセスの分割数。

%\item{{\bf 返り値}}

%なし。

%\item{{\bf 機能}}

%プロセッサの分割数を設定する。nx,ny,nzの積が全プロセス数にならない場合
%は例外を送出する。
%\end{itemize}


\section{使用例}
\label{sec:sample}
In this section, we present the complete working example of a
simulation code written using FDPS, to illustrate how a user actually
uses FDPS. As the target problem, we use the gravitational $N$-body
problem with an open boundary.  Within the terminology of FDPS, the
interaction between particles in the gravitational $N$-body problem is
of the ``long-range'' type. Therefore, we need to specify the function
to calculate interactions for both the ordinary particles and
superparticles. For the sake of brevity, we use the center-of-mass
approximation for superparticles, which means we can actually use the
same function for both types of particles.

The physical quantity vector $\myvec{u}_i$ and interaction functions
$\myvec{f}$, $\myvec{f'}$, and $\myvec{g}$ for the gravitational
$N$-body problem is now given by:
\begin{align}
  \myvec{u}_i &= (\myvec{r}_i,
  \myvec{v}_i,m_i) \label{eq:PhysicalVectorNbody} \\
%%  
  \myvec{f} (\myvec{u}_i, \myvec{u}_j) &= \frac{Gm_j \left(
    \myvec{r}_j - \myvec{r}_i \right)}{ \left( |\myvec{r}_j -
    \myvec{r}_i|^2 + \epsilon_i^2
    \right)^{3/2}} \label{eq:ParticleParticleNbody} \\
%%
  \myvec{f'} (\myvec{u}_i, \myvec{u'}_j) &= \frac{Gm_j' \left(
    \myvec{r}_j - \myvec{r'}_i \right)}{ \left( |\myvec{r}_j -
    \myvec{r'}_i|^2 + \epsilon_i^2
    \right)^{3/2}} \label{eq:ParticleSuperparticleNbody} \\
%%
  \myvec{g}(\myvec{F},\myvec{u}_i)  &= (\myvec{v}_i,\myvec{F},0),
\label{eq:ConversionNbody}
\end{align}
where $m_i$, $\myvec{r}_i$, $\myvec{v}_i$, and $\epsilon_i$ are, the
mass, position, velocity, and gravitational softening of particle $i$,
$m_j'$ and $\myvec{r'}_j$ are, the mass and position of a
superparticle $j$, and $G$ is the gravitational constant.  Note that
the shapes of the functions $\myvec{f}$ and $\myvec{f'}$ are the same.

Listing~\ref{code:samplecode} shows the complete code which can be
actually compiled and run, not only on a single-core machine but also
massively-parallel, distributed-memory machines such as the full-node
configuration of the K computer. The total number of lines is only
117.


\begin{lstlisting}[label=code:samplecode,numbers=left,numbersep=5pt,frame=single,basicstyle=\ttfamily,caption=A sample code of $N$-body simulation]
#include <particle_simulator.hpp>
using namespace PS;

class Nbody{
public:
    F64    mass, eps;
    F64vec pos, vel, acc;
    F64vec getPos() const {return pos;}
    F64 getCharge() const {return mass;}
    void copyFromFP(const Nbody &in){ 
        mass = in.mass;
        pos  = in.pos;
        eps  = in.eps;
    }
    void copyFromForce(const Nbody &out) {
        acc = out.acc;
    }    
    void clear() {
        acc = 0.0;
    }
    void readAscii(FILE *fp) {
        fscanf(fp,
               "%lf%lf%lf%lf%lf%lf%lf%lf",
               &mass, &eps,
               &pos.x, &pos.y, &pos.z,
               &vel.x, &vel.y, &vel.z);
    }
    void predict(F64 dt) {
        vel += (0.5 * dt) * acc;
        pos += dt * vel;
    }
    void correct(F64 dt) {
        vel += (0.5 * dt) * acc;
    }
};

template <class TPJ>
struct CalcGrav{
    void operator () (const Nbody * ip,
                      const S32 ni,
                      const TPJ * jp,
                      const S32 nj,
                      Nbody * force) {
        for(S32 i=0; i<ni; i++){
            F64vec xi  = ip[i].pos;
            F64    ep2 = ip[i].eps
                * ip[i].eps;
            F64vec ai = 0.0;
            for(S32 j=0; j<nj;j++){
                F64vec xj = jp[j].pos;
                F64vec dr = xi - xj;
                F64 mj  = jp[j].mass;
                F64 dr2 = dr * dr + ep2;
                F64 dri = 1.0 / sqrt(dr2);                
                ai -= (dri * dri * dri
                       * mj) * dr;
            }
            force[i].acc += ai;
        }
    }
};

template<class Tpsys>
void predict(Tpsys &p,
             const F64 dt) {
    S32 n = p.getNumberOfParticleLocal();
    for(S32 i = 0; i < n; i++)
        p[i].predict(dt);
}

template<class Tpsys>
void correct(Tpsys &p,
             const F64 dt) {
    S32 n = p.getNumberOfParticleLocal();
    for(S32 i = 0; i < n; i++)
        p[i].correct(dt);
}

template <class TDI, class TPS, class TTFF>
void calcGravAllAndWriteBack(TDI &dinfo,
                             TPS &ptcl,
                             TTFF &tree) {
    dinfo.decomposeDomainAll(ptcl);
    ptcl.exchangeParticle(dinfo);    
    tree.calcForceAllAndWriteBack
        (CalcGrav<Nbody>(),
         CalcGrav<SPJMonopole>(),
         ptcl, dinfo);    
}

int main(int argc, char *argv[]) {
    F32 time  = 0.0;
    const F32 tend  = 10.0;
    const F32 dtime = 1.0 / 128.0;
    PS::Initialize(argc, argv);
    PS::DomainInfo dinfo;
    dinfo.initialize();
    PS::ParticleSystem<Nbody> ptcl;
    ptcl.initialize();
    PS::TreeForForceLong<Nbody, Nbody,
        Nbody>::Monopole grav;
    grav.initialize(0);
    ptcl.readParticleAscii(argv[1]);
    calcGravAllAndWriteBack(dinfo,
                            ptcl,
                            grav);
    while(time < tend) {
        predict(ptcl, dtime);        
        calcGravAllAndWriteBack(dinfo,
                                ptcl,
                                grav);
        correct(ptcl, dtime);        
        time += dtime;
    }
    PS::Finalize();
    return 0;
}
\end{lstlisting}


Now let us explain how this sample code works. This code consists of
four parts: The declaration to use FDPS (lines 1 and 2), the
definition of the particle (the vector $\myvec{u}_i$) (lines 4 to 35),
the definition of the gravitational force (the functions $\myvec{f}$
and $\myvec{f'}$) (lines 37 to 61), and the actual user program,
comprising a user-defined main routine and user-defined functions from
which library functions of FDPS are called (lines 63 to line 117). In
the following, we explain them step by step.

In order to declare to use FDPS, the only thing the user program need
to do is to include the header file ``particle\_simulator.hpp''. This
file and other source library files of FDPS should be in the include
path of the compiler. Everything in the standard FDPS library is
provided as the header source library, since they are implemented as
template libraries which need to receive particle class and
interaction functions. Everything in FDPS is provided in the namespace
``PS''. Therefore in this sample program, we declare it as the default
namespace to simplify the code. (For simplicity's sake, we do not omit
the namespace ``PS'' of FDPS functions and class templates in the main
routine.)

Before going to the 2nd parts, let us list the data types and classes
defined in FDPS. \texttt{F32/F64} are data types of 32-bit and 64-bit
floating points. \texttt{S32} is a data type of 32-bit signed integer.
\texttt{F64vec} is a class of a vector consisting of three 64-bit
floating points. This class provides several operators, such as the
addition, subtraction and the inner product indicated by ``$*$''.
Users need not use these data types in their own program, but some of
the functions which users should define should return the values in
these data types.

In the 2nd part, we define the particle, i.e. the vector
$\myvec{u}_i$, as a class \texttt{Nbody}. This class has member
variables: \texttt{mass} ($m_i$), \texttt{eps}
($\epsilon_i$), \texttt{pos} ($\myvec{r}_i$), \texttt{vel}
($\myvec{v}_i$), and \texttt{acc} ($d\myvec{v}_i/dt$). Although the
member variable \texttt{acc} does not appear in
equation~(\ref{eq:PhysicalVectorNbody}) -- (\ref{eq:ConversionNbody}),
we need this variable to store the result of the gravitational force
calculation. A particle class for FDPS must provide public member
functions \texttt{getPos}, \texttt{getCharge}, \texttt{copyFromFP},
\texttt{copyFromForce}, \texttt{clear},
and \texttt{readAscii}, in these names, so that the internal functions
of FDPS can access the data within the particle class.  For the name
of the particle class itself and the names of the member variables, a
user can use whatever names allowed by the C++ syntax.  The member
functions \texttt{predict} and \texttt{correct} are used in the
user-defined part of the code to integrate the orbits of particles.
Note that since the interaction used here is of $1/r$ type, the
definition and construction method of the superparticle are given as
the default in FDPS and not shown here.

In the 3rd part, the interaction functions $\myvec{f}$ and
$\myvec{f'}$ are defined. Since the shapes of the functions
$\myvec{f}$ and $\myvec{f'}$ are the same, we give one as a template
function.  The interaction function used in FDPS should have the
following five arguments. The first argument \texttt{ip} is the
pointer to the array of variables of particle
class \texttt{Nbody}. This argument specifies $i$-particles which
receive the interaction. The second argument \texttt{ni} is the number
of $i$-particles. The third argument \texttt{jp} is the pointer to the
array of variable of a template data type \texttt{TPJ}. This argument
specifies $j$-particles or superparticles which exert the
interaction. The fourth argument \texttt{nj} is the number of
$j$-particles or super-particles. The fifth argument \texttt{force} is
the pointer to the array of a variable of a user-defined class to
which the calculated interaction on an $i$-particle can be stored. In
this example, we used the particle class itself, but this can be
another class or a simple array.

%
The interaction function should be defined as a function object, so
that it can be passed to other functions as argument. Thus, it is
declared as a \texttt{struct}, with the only member
function \texttt{operator ()}.  In this example, the interaction is
calculated through a simple double loop. In order to make full
advantage of the SIMD unit in modern processors,
architecture-dependent tuning may be necessary, but only to this
single function.

In the 4th part, we give the main routine and functions called from
the main routine. In the following, we describe the main routine in
detail, and briefly discuss other functions. The main routine consists
of the following seven steps:
\begin{enumerate}
\item Set simulation time and timestep (lines 92 to 94). \label{proc:literal}
\item Initialize FDPS (line 95). \label{proc:init}
\item Create and initialize objects of FDPS classes (lines 96 to 102). \label{proc:construct}
\item Read in particle data from a file (line 103). \label{proc:input}
\item Calculate the gravitational forces of all the particles at the
  initial time (lines 104 to 106). \label{proc:calcinteraction}
\item Integrate the orbits of all the particles with Leap-Frog method
  (lines 107 to 114). \label{proc:integration}
\item Finish the use of  FDPS (line 115). \label{proc:fin}
\end{enumerate}

In the following, we describe  steps~\ref{proc:init},
\ref{proc:construct}, \ref{proc:input}, \ref{proc:calcinteraction},
and \ref{proc:fin}, and skip steps~\ref{proc:literal}
and \ref{proc:integration}.  In step~\ref{proc:literal}, we do not
call FDPS libraries.  Although we call FDPS libraries in
step~\ref{proc:integration}, the usage is the same as in
step~\ref{proc:calcinteraction}.

In step~\ref{proc:init}, the FDPS function \texttt{Initialize} is
called. In this function, MPI and OpenMP libraries are initialized. If
neither of them are used, this function does nothing.  All functions
of FDPS must be called between this function and the
function \texttt{Finalize}.

In step~\ref{proc:construct}, we create and initialize three objects
of the FDPS classes:
\begin{itemize}
\item \texttt{dinfo}: An object of class \texttt{DomainInfo}. It is
  used for domain decomposition.
\item \texttt{ptcl}: An object of class template \texttt{ParticleSystem}.
It takes the user-defined particle class (in this
example, \texttt{Nbody}) as the template argument. From the user
program, this object looks as an array of $i$-particles.
\item \texttt{grav}: An object of a data type \texttt{Monopole} defined in
a class template \texttt{TreeForForceLong}. This object is used for
the calculation of long-range interaction using the tree algorithm.
It receives three user-defined classes template arguments: the class
to store the calculated interaction, the class for $i$-particles and
the class for $j$-particles. In this example, all three are the same
as the original class of particles.  It is possible to define classes
with minimal data for these purposes and use them here, in order to
optimize the cache usage. The data type \texttt{Monopole} indicates
that the center-of-mass approximation is used for superparticles.
\end{itemize}

In step~\ref{proc:input}, the data of particles are read from a file
into the object \texttt{ptcl}, using the FDPS
function \texttt{readParticleAscii}. In the function, a member
function of class \texttt{Nbody}, \texttt{readAscii}, is called.

In step~\ref{proc:calcinteraction}, the forces on all particles are
calculated through the function \texttt{calcGravAllAndWriteBack}, which
is defined in lines 79 to 89. In this function,
steps~\ref{proc:decompose}, \ref{proc:exchange}, and
\ref{proc:interaction} in section~\ref{sec:view} are performed. In
other words, all of the actual work of FDPS libraries to calculate
interaction between particles takes place here. For
step~\ref{proc:decompose}, \texttt{decomposeDomainAll}, a member function
of class \texttt{DomainInfo} is called. This function takes the object
\texttt{ptcl} as an argument to use the positions of particles to
determine the domain decomposition.  Step~\ref{proc:exchange} was
performed in \texttt{exchangeParticle}, a member function of
class \texttt{ParticleSystem}. This function takes the
object \texttt{dinfo} as an argument and redistributes particles among
MPI processes.  Step~\ref{proc:interaction} was performed
in \texttt{calcForceAllAndWriteBack}, a member function of
class \texttt{TreeForForceLong}. This function takes the user-defined
function object \texttt{CalcGrav} as the first and second arguments,
and calculates particle-particle and particle-superparticle
interactions using them.

In step~\ref{proc:fin}, the FDPS function \texttt{Finalize} is
called. It calls the \texttt{MPI\_finalize} function.

In this section, we have described in detail how a user program
written using FDPS looks like. As we stated earlier, this program can
be compiled with or without parallelization using MPI and/or OpenMP,
without any change in the user program. The executable parallelized
with MPI is generated by using an appropriate compiler with MPI
support and a compile-time flag.  Thus, a user need not worry about
complicated bookkeeping necessary for parallelization using MPI.
%
In the next section, we describe how FDPS provides a generic
framework which takes care of parallelization
and bookkeeping for particle-based simulations. 

% LocalWords:  monopole superparticle FDPS hpp namespace nd th vec Nbody eps dt
% LocalWords:  pos vel acc getPos getCharge copyFromFP copyFromForce readAscii
% LocalWords:  ip const ni jp TPJ nj MPI OpenMP DomainInfo dinfo subdomains
% LocalWords:  subdomain ParticleSystem ptcl TreeForForceLong readParticleAscii
% LocalWords:  calcGravAllAndWriteBack decomposeDomainAll exchangeParticle SIMD
% LocalWords:  calcForceAllAndWriteBack CalcGrav superparticles struct grav
% LocalWords:  parallelization parallelized timestep


\section{ロードマップ}
\label{sec:roadmap}

6月末にver0シリアルバージョン完成予定。
12月末にmpiバージョン完成予定。

\appendix

\section{ユーザー定義粒子クラス、Forceクラス}
\label{sec:particle_subclass}
以下では、相互作用計算や通信を効率的に行う為にユーザーが定義出来る粒子
クラスおよび、結果を格納するためのForceクラスについて説明する。これらの
クラスは名前空間{\tt PS}の下にある必要はない。

\subsection{{\tt FullParticle}型}

シミュレーションを行う上で必要な粒子が持つべき情報が入っているクラスで
絶対に必要。ParticleSystemがこのクラスの配列を持つ。ノード間での粒子交
換はこのクラスが交換される。メンバ変数はprivateでもpublicでもよい。
{\tt TreeParticle}はこのクラスから生成される。{\tt TreeParticle}は{\tt
FullParticle}から必要な情報をもらう為、{\tt FullParticle}は決まった名前
のアクセサを持つ必要がある(メンバ変数名は自由)。

\begin{itemize} 
\item {\tt getPos()} 相互作用の形によらず絶対必要。
\item {\tt getCharge()} 長距離力の計算をする場合。
\item {\tt getRSearch()} cutoff付長距離力、短距離力の計算をする場合。
\item {\tt getEps()} 対称化ソフトニングを用いる場合。
\end{itemize}

また、{\tt TreeForForce}が直接{\tt Force}の値を{\tt FullParticle}に書き
込めるように{\tt copyFromForce()}というメソッドも必要。

\subsection{{\tt EssentialParticleI}型}

粒子-粒子間相互作用を計算するのに必要なI粒子が持っている情報を持つクラ
ス。TreeForForceがこのクラスの配列を持つ。相互作用に必要な{\tt
FullParticle}のメンバをコピーするので、{\tt copyFromFP}という名前のメン
バ関数が必要。

\subsection{{\tt EssentialParticleJ}型}

粒子-粒子間相互作用を計算するのに必要なJ粒子が持っている情報を持つクラ
ス。TreeForForceがこのクラスの配列を持つ。LET交換で通信される。相互作
用に必要な{\tt FullParticle}のメンバをコピーするので、{\tt copyFromFP}
という名前のメンバ関数が必要。

\subsection{{\tt SuperParticleJ}型}

セル-粒子間相互作用を計算するのに必要なセルの情報を持ったクラス。
TreeForForce型がこのクラスの配列を持つ。短距離力の場合は{\tt size}この
クラスは必要ない。長距離力の場合は、{\tt charge}と{\tt pos}が必要である。
さらに双極子、四重極子を計算する場合はそれぞれ{\tt di}と{\tt quad}を加
えることができる。対称化されたソフトニングを用いる場合は{\tt eps}を加え
ることもできるが、その場合は{\tt FullParticle}型のメンバ変数に{\tt
eps}が必要となる。あらかじめ、以下の場合については最初から用意されてい
る。

\begin{itemize}
\item 重心周りで展開した単極子からの力の計算。
\item 重心周りで展開した四重極子までの力の計算。
\item ソフトニングが対称化された重心周りで展開した単極子からの力の計算
\item 幾何中心周りで展開した単極子からの力の計算。
\item 幾何中心周りで展開した双極子までの力の計算。
\item 幾何中心周りで展開した四重極子までの力の計算。
\end{itemize}

\subsection{{\tt Force}型}

相互作用の結果を格納するクラス。TreeForForceがこのクラスの配列を持つ。
TreeForForceが値の初期化等をしなければならないので、初期化をするメンバ
関数{\tt clear()}が必要。

\section{Treeを構築しているサブクラス。}
\label{sec:tree_subclass}

\subsection{TreeParticle}

{\tt TreeForForce}が持つクラスで、ユーザーからは見えない。{\tt
FullParticle}から粒子情報をもらいツリーの構築、モーメントの計算等に用い
る。これらのクラスは{\tt FullParticle}にアクセスする必要があり、{\tt
FullParticle}のユーザ定義メソッド({\tt FullParticle::getPos()}等)を用い
てアクセスする。

\begin{lstlisting}[caption=TreeParticle]
namespace ParticleSimulator{
    template<class Tprop>
    class TreeParticle{
    public:
        U64 key_;
        S32 next_adr_tp_;
        Tprop prop_;
        template<class Trp>
        void setFromFP(const Trp & rp){
            key_ = MortonKey<DIMENSION>::getKey( rp.getPos() );
            prop_.setFromFP(rp);
        }
    };
}
\end{lstlisting}

メンバ{\tt prop\_}はツリーセルのモーメントや情報等を作る為に使われる。こ
れはユーザーがツリークラスのオブジェクトを作るときに、自動的に適したク
ラスが採用される。

以下は重力用のクラスである。

\begin{lstlisting}[caption=PropertyLong]
namespace ParticleSimulator{
    class PropertyLong{
    public:
        F32 charge_;
        F32vec pos_;
    };
}
\end{lstlisting}



\subsection{TreeCell}

{\tt TreeForForce}が持つセルクラスで、ユーザーからは見えない。{\tt
Tree Particle}から粒子情報をもらいセルのモーメントの計算を行い値を格納
する。

以下に短距離力用の{\tt TreeCell}を記述する。
\begin{lstlisting}[caption=TreeCell]
namespace ParticleSimulator{
    template<class Tmom>
    class TreeCell{
    public:
        S32 n_ptcl_;
        S32 adr_tc_;
        S32 adr_tp_;
        S32 lev_ni_;
        Tmom mom_;
    };
}
\end{lstlisting}


メンバ{\tt mom\_}はツリーセルのモーメント等の情報を保持するモーメントク
ラス。ユーザーがツリークラスのオブジェクトを作るときに、自動的に適した
モーメントクラスが採用される。以下は短距離力用のモーメントクラスである。

\begin{lstlisting}[caption=MomentSearch]
namespace ParticleSimulator{
    class MomentSearch{
    public:
        F32ort vertex_in_;
        F32ort vertex_out_;
        template<class Tprop> void accumulateAtLeaf(const Tprop & _prop);
        void accumulate(const MomentSearch & _mom);
    };
}
\end{lstlisting}

ここで、{\tt PS::F32ort}が出てくるが、これは{\tt PS::Orthotope}クラスと
いわれるもので、内部に二つの{\tt PS::F32vec}型を持ち、それぞれが位置座
標の小さい頂点と大きい頂点の位置座標を持つ。

\section{相互作用関数の定義方法}
ユーザーは任意の相互作用関数を書くことができる。関数名は任意であるが、
引数は第一引数から順に、I粒子へのポインタ、I粒子の個数、J粒子へのポイン
タ、J粒子の個数、forceへのポインタとなる。粒子の型と力の型は任意である
が、相互作用関数内で定義されている型と対応していなければならない。
以下は重力の場合の例である。

\begin{lstlisting}
void CalcForceEpEp(const EssentialPtclI * ep_i,
                   const PS::S32 n_ip,
                   const EssentialPtclJ * ep_j,
                   const PS::S32 n_jp,
                   Force * force){
    for(PS::S32 i=0; i<n_ip; i++){
        PS::F64vec xi = ep_i[i].pos;
        PS::F64vec ai = 0.0;
        PS::F64 poti = 0.0;
        PS::S32 idi = ep_i[i].id;
        for(PS::S32 j=0; j<n_jp; j++){
            if(idi == ep_j[j].id) continue;
            PS::F64vec rij = xi - ep_j[j].pos;
            PS::F64 r3_inv = rij * rij;
            PS::F64 r_inv = 1.0/sqrt(r3_inv);
            r3_inv = r_inv * r_inv;
            r_inv *= ep_j[j].mass;
            r3_inv *= r_inv;
            ai -= r3_inv * rij;
            poti -= r_inv;
        }
        force[i].acc += ai;
        force[i].pot += poti;
    }
}
\end{lstlisting}


\section{実装}
\label{sec:implementation}
\subsection{粒子サンプリング}

以下は重み付けを使った粒子のサンプル方法のコードである。重みにはプロセ
スの計算時間や相互作用数等が適切と考えられる。各プロセスで同じ粒子数に
したい場合は、第三引数が各プロセスで同じ値を入れる(デフォルト引数を使
えばよい)。

\begin{lstlisting}
    template<class Tpsys>
    void DomainInfo::sampleParticle(const Tpsys & psys, const bool clear=true, const PS::F32 wgh=1.0){
        static S32 n_sample_ = 0;
        if(clear==true){ n_sample_ = 0; }
        PS::F32 wgh_tot;
        MPI::COMM_WORLD.Allreduce(&wgh, &wgh_tot, 1, MPI::FLOAT, MPI_SUM);
        S32 n_sample_tmp = n_sample_max_ * (wgh / wgh_tot);
        S32 interval = ( (this->getNumberOfParticleLocal()) - 1) / n_sample_tmp;
        interval = (interval <= 0) ? 1 : interval;
        for(S32 i=0; i<psys.getNumberOfParticleLocal(); i += interval){
            pos_sample_[n_sample_] = psys[i].getPos();
            n_sample_++;
        }
    }
\end{lstlisting}

\subsection{ルートドメインの分割}

\subsubsection{ルートドメインの分割}

\subsubsection{ドメイン、ツリーセルの番号付け}

各プロセスの持つドメインやツリー構造は2、もしくは3次元構造をしているが
1次元の番号を持っている。この番号付けのルールはz->y->x(2次元の場合は
y->x)の順で付ける。例えば、直方体をx,y,zにnx,ny,nz個に分割し、それぞれ
の箱に軸方向のID番号(idx,idy,idz)を小さい方から順につけていくとした場合。
1次元のidは
\begin{equation}
id = idx \times ny \times nz + idy \times nz + idz.
\end{equation}
となる。
ツリーセルの番号付けもこれと同じである。

\subsubsection{ルートドメインの分割の順番}

ルートドメインの分割の方法はMakino(2004)の方法に従う。切り方はx方向から
先に切り、y、zと切っていく。

\subsubsection{実装例}

以下にルートドメインの分割のサンプルコードを記述する。ここでの方法はサ
ンプルした粒子をルートノードに集めて全てのドメインの座標をルートノード
で計算するようになっているが、下にそれを回避する方法も記しておく。

サンプル粒子の少なさが原因となる境界のポアソンノイズを抑えるために、過
去の境界を用いて移動平均を使う。ここでは記述の簡単さから指数移動平均を
用いることとする(平滑化係数は\verb|alpha|とする)。

\begin{lstlisting}[caption=ルートドメインの分割]
void DomainInfo::decomposeDomain(){
    PS::S32 n_proc = MPI::COMM_WORLD.Get_size();
    std::vector<PS::S32> n_sample_array;
    std::vector<PS::S32> n_sample_array_disp;
    n_sample_array.reserve(n_proc);
    n_sample_array_disp.reserve(n_proc+1);
    MPI::COMM_WORLD.Gather(&n_sample_, 1, MPI::INT, 
                           &n_sample_array[0], 1, MPI::INT, 0);
    n_sample_array[0] = 0;
    for(PS::S32 i=0; i<n_proc; i++){
        n_sample_array_disp[i+1] = n_sample_array_disp[i] + n_sample_array[i];
    }
    MPI::COMM_WORLD.Gatherv(pos_sample_, n_sample_, MPI_F32VEC, 
                            pos_sample_tot_, &n_sample_array[0], &n_sample_array_disp[0],
                            MPI_F32VEC, 0);
    PS::S32 rank = Comm::getRank();
    PS::S32 np[PARTICLE_SIMULATOR_DIMENSION];
    for(int k=0; k<PARTICLE_SIMULATOR_DIMENSION; k++){ np[k] = Comm::getNProcXD(k); }
    if(rank == 0){
        std::vector<PS::F32rec> domain_old;
        domain_old.reserve(n_proc);
        for(PS::S32 i=0; i<n_proc; i++) domain_old[i] = domain_rem_[i];
        std::vector<PS::S32> istart, iend;
        istart.reserve(n_proc);
        iend.reserve(n_proc);
        quickSort(pos_sample_tot_, 0, n_sample_-1, 0);
        for(PS::S32 i=0; i<n_proc; i++){
           istart[i] = (i*n_sample_)/n_proc;
            if(i>0)
                iend[i-1] = istart[i] - 1;
        }
        iend[n_proc-1] = n_sample_-1;

        for(PS::S32 ix=0; ix<np[0]; ix++){
            PS::F32 xlow = 0.0; 
            PS::F32 xhigh = 0.0;
            PS::S32 ix0 = ix*np[1]*np[2];
            PS::S32 ix1 = (ix+1)*np[1]*np[2];
            getBoxCoordinate(n_sample_, pos_sample_tot_, 0, istart[ix0], iend[ix1-1], half_length_root_, xlow, xhigh);
            for(PS::S32 i=ix0; i<ix1; i++){
                domain_rem_[i].low_[0] = alpha_*xlow + (1.0-alpha_)*domain_old[i].low_[0];
                domain_rem_[i].high_[0] = alpha_*xhigh + (1.0-alpha_)*domain_old[i].high_[0];
            }
        }

        for(PS::S32 ix=0; ix<np[0]; ix++){
            PS::S32 ix0 = ix*np[1]*np[2];
            PS::S32 ix1 = (ix+1)*np[1]*np[2];
            PS::S32 nsample_y = iend[ix1-1] - istart[ix0] + 1;
            quickSort(pos_sample_tot_, istart[ix0], iend[ix1-1], 1);
            for(PS::S32 iy=0; iy<np[1]; iy++){
                PS::F32 ylow = 0.0; 
                PS::F32 yhigh = 0.0;
                PS::S32 iy0 = ix0+iy*np[2];
                PS::S32 iy1 = ix0+(iy+1)*np[2];
                getBoxCoordinate(nsample_y, pos_sample_tot_+istart[ix0], 1, istart[iy0]-istart[ix0], 
                                 iend[iy1-1]-istart[ix0], half_length_root_, ylow, yhigh);
                for(int i=iy0; i<iy1; i++){
                    domain_rem_[i].low_[1] = alpha_*ylow + (1.0-alpha_)*domain_old[i].low_[1];
                    domain_rem_[i].high_[1] = alpha_*yhigh + (1.0-alpha_)*domain_old[i].high_[1];
                }
            }
        }
#if (PARTICLE_SIMULATOR_DIMENSION == 3)
        for(PS::S32 ix=0; ix<np[0]; ix++){
            PS::S32  ix0 = ix*np[1]*np[2];
            for(PS::S32 iy=0; iy<np[1]; iy++){
                PS::S32 iy0 = ix0 + iy*np[2];
                PS::S32 iy1 = ix0 + (iy+1)*np[2];
                PS::S32 nsample_z = iend[iy1-1] - istart[iy0] + 1;
                quickSort(pos_sample_tot_, istart[iy0], iend[iy1-1], 2);
                for(PS::S32 iz=0; iz<np[2]; iz++){
                    PS::F32 zlow = 0.0; 
                    PS::F32 zhigh = 0.0;
                    int iz0 = iy0 + iz;
                    getBoxCoordinate
		    (nsample_z, pos_sample_tot_+istart[iy0], 2, istart[iz0]-istart[iy0], 
                    iend[iz0]-istart[iy0], half_length_root_, zlow, zhigh);
                    domain_rem_[iz0].low_[2] = alpha_*zlow 
		                             + (1.0-alpha_)*domain_old[iz0].low_[2];
                    domain_rem_[iz0].high_[2] = alpha_*zhigh 
		                              + (1.0-alpha_)*domain_old[iz0].high_[2];
                }
            }
        }
#endif
    }
    MPI::COMM_WORLD.Bcast(domain_rem_, n_proc, MPI_F32VEC, 0);
}
\end{lstlisting}

%%-------------------

\paragraph{ルートノード以外も使う方法}

上の場合だと、全ての計算をルートノードにやらせているため、サンプリング
や一番最初のソートがやや重いと考えられる。なので、以下にルートノードに
全ての粒子を集めない方法を記述する。

\begin{enumerate}
\item 同じxランクを持つ(y-z)スラブ内で粒子のサンプリングを行う。
\item 前回求めたドメインのx座標のみを考えて、収まるべきスラブに粒子を移
  動させる。
\item プロセスの粒子数のprefix sumを求めておいて、スラブの分割点になり
  そうな粒子をクイックセレクトで求める。この時平滑化係数を使っていれば、
  スラブの境界と粒子が重なる事はあまりないと思われる。
\item 新しく決まったスラブに収まるように粒子を移動させる。
\item 各スラブ内で、y,zのドメインを決める。
\end{enumerate}


\subsection{粒子交換}

粒子交換の実装について述べる。やり方は大きく分けて2通り考えられる。一つ
はプロセス数でループを回し、さらに粒子方向にもループを回して、粒子の行
き先のプロセスを探して、送信バッファに入れていく方法。もう一つは粒子方
向でループを回し、各粒子がどのプロセスに行くかを探す方法である。後者で
は粒子の行き先を探すのにルートドメインの分割のツリー構造が使えるため、
高速であるが、各送信バッファに入る粒子数は最後までわからない。しかし、
送信バッファのサイズは前回のサイズから推定する事も出来るし、そうでなく
てもリンクトリストを使ったり、ループを二回回せば良い。

以下にループを2回回した場合のサンプルコードを示す。
\begin{lstlisting}[caption=粒子交換]
PS::PS::S32 PS::ParticleSystem::whereToGo(const PS::F64vec & pos, const PS::DomainInfo & dinfo){
    PS::S32 id_node = 0;
#if (PARTICLE_SIMULATOR_DIMENSION == 3)
    static const PS::S32 nz = Comm::getNProcXD[2];
    static const PS::S32 ny = Comm::getNProcXD[1];
    static const PS::S32 nynz = ny * nz;
    PS::F32rec * dm_tmp = dinfo.domain_rem_;
    while( dm_tmp[id_node].high_.x <= pos.x ) 
        id_node += nynz;
    while( dm_tmp[id_node].high_.y <= pos.y )
        id_node += nz;
    while( dm_tmp[id_node].high_.z <= pos.z )
        id_node++;
#elif (PARTICLE_SIMULATOR_DIMENSION == 2)
    static const PS::S32 ny = Comm::getNProcXD[1];
    while( dm_tmp[id_node].high_.x <= pos.x)
        id_node += ny;
    while( dm_tmp[id_node].high_.y <= pos.y)
        id_ ++;
#endif
    return id_node;
}

void PS::ParticleSystem::exchangeParticle(const PS::DomainInfo & dinfo){
    static const PS::S32 n_proc = Comm::getNProc();
    static const PS::S32 rank = Comm::getRank();
    std::vector<PS::S32> n_send;
    n_send.reserve(n_proc);
    std::vector<PS::S32> n_send_disp;
    n_send_disp.reserve(n_proc+1);
    for(PS::S32 i=0; i<n_proc; i++) n_send[i] = 0;
    PS::S32 nsend_net = 0;
    for(PS::S32 i=0; i<n_ptcl_loc_; i++){
        PS::S32 id_send = whereToGo(ptcl_[i].pos, dinfo);
        n_send[id_send]++;
    }
    n_send_disp[0] = 0;
    for(PS::S32 i=0; i<n_proc; i++){
        n_send_disp[i+1] = n_send_disp[i] + n_send[i];
    }
    std::vector<Tptcl> ptcl_send;
    ptcl_send.reserve(n_send_disp[n_proc]);
    for(PS::S32 i=0; i<n_ptcl_loc_; i++){
        PS::S32 id_send = whereToGo(ptcl_[i].pos, dinfo);
        ptcl_send[n_send_disp[id_send] + n_send[id_send]] = ptcl_[i];
    }
    std::vector<PS::S32> n_recv;
    n_recv.reserve(n_proc);
    MPI::COMM_WORLD.Alltoall(&n_send[0], 1, MPI::INT, &n_recv[0], 1, MPI::INT );
    // do something to exchange particle
}
\end{lstlisting}



\end{document}
