In this section, we describe the implementations of interaction
information exchange and actual interaction calculation.
In the interaction information exchange step,
each process sends the  data required by other nodes. In the
interaction calculation step, actual interaction calculation is done
using the received data.
For both steps, the Barnes-Hut octree structure is used, for both of
long- and short-range interactions.

First, each process constructs the tree of its local particles. Then
this tree is used to determine the data to be sent to other
processes. For the long-range interaction, the procedure is the same
as that for usual tree traversal\citep{1986Natur.324..446B,
1990JCoPh..87..161B}.  The tree traversal is used also for short-range
interactions.  FDPS can currently handle four different types of the
cutoff length for the short-range interactions: fixed, $j$-dependent,
$i$-dependent and symmetric.  For $i$-dependent and symmetric cutoffs,
the tree traversal should be done twice.
%% Figure~\ref{fig:exchangeLET}
%% illustrates what data are sent.


Using the received data, each process performs the force
calculation. To do so, it first constructs the tree of all data
received and local particles, and then uses it to calculate  the
interaction on local particles.

%% The interaction calculation is performed using this new tree. The
%% procedure is the same as described in detail in the literature
%% \citep{1990JCoPh..87..161B, 1991PASJ...43..859M}, except for the
%% following two differences.  First, this part is fully multithreaded
%% using OpenMP, to achieve very good parallel performance. Second, for
%% the interaction calculation the user-provided functions are used, to
%% achieve the flexibility and high performance at the same time.

% LocalWords:  FDPS TreeForForceLong TreeForForceShort calcForceAllAndWriteBack
% LocalWords:  subdomains substeps octree multipole superparticles nd OpenMP
% LocalWords:  substep multithreading multithreaded
