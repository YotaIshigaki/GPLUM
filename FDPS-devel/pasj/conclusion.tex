
In this paper, we present the basic idea, implementation, measured
performance and performance model of FDPS, a framework for developing
efficient parallel particle-based simulation codes.  FDPS provides all
of these necessary functions for the parallel execution of
particle-based simulations. By using FDPS, researchers can easily
develop their programs which run on large-scale parallel
supercomputers. For example, a simple gravitational $N$-body program
can be written in around 120 lines.

We implemented three astrophysical applications using FDPS and
measured their performances. All applications showed good performance
and scalability. In the case of the disk galaxy simulation, the
achieved efficiency is around 50\% of the theoretical peak, for the
cosmological simulation 7\%, and for the giant impact simulation 40\%.

We constructed the performance model of FDPS and analyzed the
performance of applications using FDPS. We found that the performance
for small number of particles would be limited by the time for the
calculation necessary for the domain decomposition and communication
necessary for the interaction calculation. 



%%% original PASJ submission version
%We have developed a framework for large-scale parallel particle-based
%simulations, FDPS.  Users of FDPS need not care about complex
%implementations of domain decomposition, exchange of particles,
%communication of data for the interaction calculation, or optimization
%for multi-core processors.  Using FDPS, particle simulation codes
%which achieve high performance and high scalability on massively
%parallel distributed-memory machines can be easily developed for a
%variety of problems.  As we have shown in
%section~\ref{sec:samplecode}, a parallel $N$-body simulation code can
%be written in less than 120 lines.  Example implementations of
%gravitational $N$-body simulation and SPH simulation showed excellent
%scalability and performance. We hope FDPS will help researchers to
%concentrate on their research, by removing the burden of complex code
%development for parallization and architecture-dependent tuning.
%%%%%%%%%%%%%%


%In this paper, we report the performance of applications developed
%using our framework for particle-based simulations, FDPS. We have
%demonstrated that, by using FDPS, multiple applications achieve very
%good scalability without the need for spending years of time in code
%development. We believe our approach will be quite useful for those
%who want to develop high-performance application code for
%particle-based simulations in many different fields.

%Not only good scalability, but also the high absolute performance can
%be achieved for applications developed with FDPS, because the design
%concept of FDPS to provide highly efficient implementation for the
%parallelization part, such as domain decomposition, particle exchange
%and exchange of information necessary to calculate interaction. It is
%still necessary for application developers to provide highly optimized
%functions for interaction calculation. However, compared to all the
%works necessary to develop,  debug and tune large-scale parallel
%applications, this part is much 
%easier, in particular because development and test can be done on
%single-processor machines.

%We demonstrated that applications with very high performance can be
%developed by using FDPS. In the case of gravitational $N$-body
%problem, the achieved efficiency is around 50\% of the theoretical
%peak, and for SPH 10\%. These numbers are comparable or better than the
%best numbers in the literature.

%Thus, we believe that it is not unfair to say that FDPS offers the
%researchers and/or programmers  a new and easy way to
%develop large-scale particle-based simulation programs. We believe
%similar approach can be applied to other types of large-scale
%simulations, such as those based on regular or irregular grids.


%In this paper, we report the concept, implementation, and performance
%of FDPS. We found that useres can develop the particle-based
%simulation code easily.






