\def\badot{{\bf \dot a}}
\def\batwo{{\bf a}^{(2)}}
\def\bathree{{\bf a}^{(3)}}
\def\bx{\mbox{\boldmath $x$}}
\def\bk{{\bf k}}
\def\bv{\mbox{\boldmath $v$}}
\def\br{\mbox{\boldmath $r$}}
\def\ba{\mbox{\boldmath $a$}}
\def\bbf{\mbox{\boldmath $f$}}
\def\calE{{\cal E}}
\def\sub#1{_{\rm #1}}
\def\sup#1{^{\rm #1}}

%% \def\mum{\mu {\rm m}}
%% \def\bx{{\bf x}}
%% \def\bv{{\bf v}}
%% \def\barv{{\bar{ v}}}
%% \def\bP{{\bf P}}
%% \def\calE{{\cal E}}
\def\erf{{\rm erf}}
%% \def\sub#1{_{\rm #1}}
%% \def\sup#1{^{\rm #1}}

\input ../lmak_ohp

\documentclass[12pt,dvipdfmx]{article}
\usepackage[dvips]{color}
\usepackage[dvipdfm]{hyperref}
\usepackage{mathabx}
\usepackage{graphics}
\usepackage{graphicx}
\usepackage{epsf}
\usepackage{psfig}
\usepackage{ascmac_ntt}
%\documentstyle[12pt,psfig,epsf,eclcolor,ascmac_ntt]{article}
%\documentstyle[12pt,epsf,eclcolor,ascmac]{article}
\pagestyle{empty}

\raggedright

%set dimensions of columns, gap between columns, and space between paragraphs
\setlength{\textheight}{12cm}
\setlength{\columnsep}{2.0pc}
\setlength{\textwidth}{16cm}
%\setlength{\footheight}{0.0in}
\setlength{\topmargin}{0.0in}
\setlength{\headheight}{0.0in}
\setlength{\headsep}{0.0in}
\setlength{\oddsidemargin}{-.19in}
\setlength{\parindent}{1pc}

%I copied stuff out of art10.sty and modified them to conform to IEEE format

\makeatletter
%as Latex considers descenders in its calculation of interline spacing,
%to get 12 point spacing for normalsize text, must set it to 10 points
\def\@normalsize{\@setsize\normalsize{12pt}\xpt\@xpt
\abovedisplayskip 10pt plus2pt minus5pt\belowdisplayskip \abovedisplayskip
\abovedisplayshortskip \z@ plus3pt\belowdisplayshortskip 6pt plus3pt
minus3pt\let\@listi\@listI} 

%need an 11 pt font size for subsection and abstract headings
\def\subsize{\@setsize\subsize{12pt}\xipt\@xipt}

%make section titles bold and 12 point, 2 blank lines before, 1 after
\def\section{\@startsection {section}{1}{\z@}{24pt plus 2pt minus 2pt}
{12pt plus 2pt minus 2pt}{\LARGE\bf}}

%make subsection titles bold and 11 point, 1 blank line before, 1 after
\def\subsection{\@startsection {subsection}{2}{\z@}{12pt plus 2pt minus 2pt}
{12pt plus 2pt minus 2pt}{\LARGE\bf}}
\makeatother
\def\black{\color{black}}
\def\red{\color{red}}
\def\blue{\color{blue}}
\def\green{\color{green}}
\def\mossgreen{\color{green}} 
\def\purple{\color{purple}}
\def\yellow{\color{yellow}}

\input ../lmak_ohp

\begin{document}

\LARGE

\baselineskip 32 pt
\lineskip 10 pt

%don't want date printed
\date{}



\pagehead{FDPS 講習会 (FDPS で使っている)Fortran について}


{\large

\begin{center}
牧野淳一郎\\
理化学研究所 計算科学研究機構\\
コデザイン推進チーム\\
兼 粒子系シミュレータ開発チーム\\
(本務は神戸大学理学惑星学専攻)

\leavevmode


\end{center}

}

\vfill

\normalsize\bf
\hfill 2017/03/08 AICS/FOCUS 共催 FDPS 講習会 Fortran篇


\pageheadl{概要}

\begin{itemize}

\item F77 ユーザーを対象に、FDPS Fortran API で使っている Fortran 2003
のみなれない機能、文法を概説します。

\item F77 は知っていること想定。

\item まず質問: F77 以外に使っている言語ありますか?

\end{itemize}


\pagehead{使っている新機能一覧}

大きなもの




\begin{enumerate}

\item module, use
\item type (構造型、C++ でいう構造体とクラス)
\item iso\_c\_binding

\end{enumerate}

\pagehead{使っている新機能一覧(続き)}

細々したもの

\begin{enumerate}

\item 変数宣言、サブルーチンの引数宣言の形式、値渡し、パラメータ文の形式
\item do ... enddo
\item コメントの形式
\item 比較演算子
\end{enumerate}

まだもうちょっとあったかもしれません。

\pageheadl{参考書?}

http://www.cutt.co.jp/book/978-4-87783-399-2.html

Fortran 2008入門

\begin{itemize}

\item module, use とかの解説もあった。構造体も全くないわけではない

\end{itemize}
\pageheadl{module, use}

モジュール宣言文法

\begin{verbatim}
module モジュール名
    [宣言部]
 [contains
    モジュール副プログラム部]
end [module [モジュール名]]
\end{verbatim}

モジュール使用文法

\begin{verbatim}
use モジュール名
\end{verbatim}

モジュールが定義されているソースファイルを先にコンパイルすると、何か中間形式のファイルができる。使っているほうのコンパイルではコンパイラがそれを参照する

\pageheadl{module, use(続き)}

\begin{itemize}

\item 複数のプログラム単位で使う様々なものをまとめられる。

\item パラメータ宣言、データ(common block の代わりになる)、ユーザー定義型、ユーザー定義の関数やサブルーチン等。

\item 構造型はモジュール内で定義して、使うサブルーチンでモジュー
ルを use するのが基本。
\item ちなみに C/C++ には相変わらずモジュールにあたるものはない。

\end{itemize}

\pageheadl{module, use(簡単な例)}
\begin{verbatim}
module sample
  integer n
  parameter (n=10)
end
program main
  use sample
  write(*,*) n
end
\end{verbatim}
コンパイル、実行:
\begin{verbatim}
% gfortran module.F90; ;./a.out
          10
\end{verbatim}


\pagehead{type (derived type、構造型)}
文法 型宣言
\begin{verbatim}
type student
  character(32) name
  integer  age
end
\end{verbatim}

変数宣言、使用
\begin{verbatim}
type(student)  a
a%name="Sato"
a%age=18
\end{verbatim}


\pagehead{type (derived type、構造型)}

\begin{itemize}

\item いわゆる構造体。Fortran 90 からあるのでまあそろそろどうでしょうみたいな。

\item FDPS では、3次元ベクトル型、ユーザーが定義する「粒子型」等を使用。

\item プログラミングスタイル云々という話もあるが、キャッシュに確実に載るようにするとかにも有用。


\end{itemize}

\pagehead{type 続き(型束縛手続き)}

\begin{itemize}

\item いつのまにか Fortran も「オブジェクト志向」に

\item 雑にいうと、ある構造型の変数を第一引数にする手続きを foo(x) の代
わりに x\%foo と書けるというだけ。こういうのを言語によってメッセージと
かメンバー関数とかいう。
        
\item 但し、同じ名前でも別の構造体のメンバー関数なら別の関数になる。演
算子も関数にできるので、構造体同士の演算を定義できる。

\item 以下では「メンバー関数(手続き)」と呼ぶことに。

\end{itemize}

\pageheadl{メンバー関数の文法}

\large \bf

\begin{minipage}[t]{7cm}
\begin{verbatim}
module studentmodule
  type, public:: student
     character(32)  name
     integer age
   contains
     procedure :: print
  end type
contains
  subroutine print(self)
    class(student) self
    write(*,*) self%age
  end
end

\end{verbatim}
\end{minipage}
\begin{minipage}[t]{7cm}
\begin{verbatim}
program main
  use studentmodule
  type (student)  a
  a%name="Sato"
  a%age=18
  call a%print
end
\end{verbatim}


Fortran でもオブジェクト志向

\medskip


関数のオーバーロード、演算子のオーバーロードができる(ベクトル型を定義
して、ベクトル同士の加算とかする演算子を定義できる)
(FDPS 側で提供してます)



\end{minipage}

\pagehead{iso\_c\_binding}

\begin{itemize}

\item 処理系とか OS 依存ではなく言語定義として公式に Fortran と C の
相互運用性を保証する仕掛け

\item Fortran の側で、C側で使える変数型とか関数の宣言のしかたを用意

\item 文法はなんか面倒だけど、とにかくそれに従っておけば Cから (従って
C++からも) Fortran で宣言した構造型や関数が使える

\item というわけで FDPS の Fortran API は全面的にこの仕掛けを利用


\end{itemize}

\pageheadl{iso\_c\_binding 例(FDPSの中から)}
\begin{verbatim}
   type, public, bind(c) :: full_particle
      integer(kind=c_long_long) :: id
      real(kind=c_double)  mass !$fdps charge
      ....
      type(fdps_f64vec) :: pos !$fdps position
   end type full_particle
\end{verbatim}
\begin{itemize}
\item {\tt bind(c) } で構造体をCからもアクセスできるように(C側では別に宣言す
る必要あり)
\item {\tt (kind=c\_double)} とかで、基本データ型をCとコンパチブルに
\item ({\tt fdps\_f64vec} は FDPS で提供している倍精度3次元ベクトル型)
\end{itemize}

\pagehead{細々したこと}

\begin{enumerate}

\item 変数宣言、サブルーチンの引数宣言の形式、値渡し、パラメータ文の形式
\item do ... enddo
\item コメントの形式
\item 比較演算子
\end{enumerate}

\pageheadl{変数宣言}

古代(F77)
\begin{verbatim}
   real a(50)
   real c
   parameter (c=1.0)
\end{verbatim}

現代
\begin{verbatim}
   real, dimension ::a(50)
   real, parameter :: c=1.0
\end{verbatim}

\begin{itemize}
\item dimension, parameter の他に色々属性をつけられる。つける時には変数名の
前に "::" を。
\item 古代語でもコンパイラは文句いわない(他の新機能も基本的にそう)

\end{itemize}

\pageheadl{do ... enddo}

古代(F77)
\begin{verbatim}
   DO 10 i=1,50
     ...
50 CONTUNUE   

\end{verbatim}

現代
\begin{verbatim}
   do i=1,50
      ...
   enddo
\end{verbatim}

\pageheadl{コメントの形式}

古代(F77)
\begin{verbatim}
c  この行はコメントです
   x = x + 1
\end{verbatim}

現代
\begin{verbatim}
!  この行はコメントです
   x = x + 1  ! ここにもコメントかけます
\end{verbatim}

\pageheadl{比較演算子}

古代(F77)
\begin{verbatim}
       if (a .lt. b) then
\end{verbatim}

現代
\begin{verbatim}
       if (a < b) then
\end{verbatim}
{\tt ==, /=, <, <=, >, >=} がある。

\pagehead{まとめ}

\begin{itemize}

\item FDPS Fortran API で使っている Fortran 77 にない機能を概説した。

\item module, 構造体、 iso\_c\_binding が主。



\item 他にも配列演算とか便利そうな機能があるが省略。

\end{itemize}
\end{document}
