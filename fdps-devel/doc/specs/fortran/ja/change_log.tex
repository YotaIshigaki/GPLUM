本章では、本仕様書の変更履歴を記述する。

\begin{itemize}[leftmargin=*,itemsep=-1ex]

\item 2016/12/26
\begin{itemize}
\item Fortran インターフェース 初リリース (FDPS 3.0として)
\end{itemize}

\item 2017/08/23
\begin{itemize}
\item FDPSに予め用意された超粒子型のデフォルトの精度を64ビットに変更 (FDPS 3.0a)。
\end{itemize}

\item 2017/11/01
\begin{itemize}
\item 粒子群オブジェクト用APIに粒子の並び替えを行うAPI \texttt{sort\_particle} を追加。
\item ツリーオブジェクト用API \texttt{calc\_force\_all\_and\_write\_back} と \texttt{calc\_force\_all} に相互作用リストを再利用する機能を追加。
\item ツリーオブジェクト用APIに粒子IDからそれに対応するEssentialParticleJを取得するAPI \texttt{get\_epj\_from\_id} を追加。
\end{itemize}

\item 2017/11/08
\begin{itemize}
\item FDPS 4.0 リリース
\end{itemize}

\item 2017/11/17
\begin{itemize}
\item API \texttt{broadcast} の不具合を修正 (FDPS 4.0a)
\end{itemize}

\item 2018/8/1
\begin{itemize}
\item Fortranインターフェス生成スクリプト\texttt{get\_ftn\_if.py}の以下の不具合を修正(FDPS 4.1b)
\begin{itemize}
\item 従来のスクリプトでは、\texttt{copyFromForce}指示文の処理を正しく行っていなかった。具体的には、\texttt{\$!fdps copyFromForce (src\_mbr, dst\_mbr) ...}と処理すべきところを、\texttt{\$!fdps copyFromForce (dst\_mbr, src\_mbr)...} として処理していた。このバグのため、スクリプトがエラーで停止する場合があった。
\item 従来のスクリプトでは、内部の処理で、与えられたユーザ定義型からは生成できないはずのtreeクラスを生成する場合があった。この場合、コンパイルエラーが発生する問題があった。
\end{itemize}
\end{itemize}


\item 2018/8/2
\begin{itemize}
\item API \texttt{get\_boundary\_condition} を追加
\item API \texttt{collect\_sample\_particle}の引数\texttt{weight}のデフォルト値を1からローカル粒子数に変更。
\item API \texttt{decompose\_domain\_all}の引数\texttt{weight}のデフォルト値が1と記述されていたが、実際にはローカル粒子数だったため、記述を修正。
\end{itemize}

\item 2018/8/31
\begin{itemize}
\item API \texttt{barrier}, \texttt{set\_particle\_local\_tree}, \texttt{get\_force} を追加
\item スレッドセーフな疑似乱数生成用 API を追加 (APIに``mtts''がつくもの)
\end{itemize}

\item 2018/11/8
\begin{itemize}
\item C言語インターフェースの記述を追加 (FDPS 5.0としてリリース)
\end{itemize}

\item 2018/12/7
\begin{itemize}
\item 長距離力用ツリーの種別にMonopoleWithSymmetrySearch 型 及び QuadrupoleWithSymmetrySearch 型を追加 (FDPS 5.0aとしてリリース)
\end{itemize}

\item 2019/1/25
\begin{itemize}
\item FDPS v5.0a で\path{gen_ftn_if.py}に入ったバグを修正 (FDPS 5.0cとしてリリース)
\end{itemize}

\item 2019/3/1
\begin{itemize}
\item FDPS 5.0d リリース
\begin{itemize}
\item EssentialParticleI 型が探索半径を保持している場合に、 \path{gen_ftn_if.py}が停止してしまうバグを修正。
\item 今回のリリースからFDPSのC++コア部分の実装でC++11の機能を使用している。\textcolor{red}{したがって、使用しているC++コンパイラに適切なオプション(\texttt{gcc}の場合、\texttt{-std=c++11})をつける必要がある。}
\end{itemize}
\end{itemize}

\item 2019/3/7
\begin{itemize}
\item Long-MonopoleWithCutoff型に関する記述を改善
\end{itemize}

\item 2019/7/11
\begin{itemize}
\item API \texttt{remove\_particle}の仕様を明確化し、このAPIの実装を仕様に沿ったものに修正
\end{itemize}

\item 2019/9/06
\begin{itemize}
\item FDPS 5.0f リリース
\begin{itemize}
\item コンパイル時にマクロ\texttt{PARTICLE\_SIMULATOR\_TWO\_DIMENSION}を定義した場合、コンパイルエラーになる問題を修正
\item C言語からFDPSを使う場合、ユーザ定義型のメンバ変数名が1文字だと\path{gen_c_if.py}が正しく動作しない問題を修正
\item シンボリックリンク \path{doc/doc_specs_c_ja.pdf} 及び \path{doc/doc_specs_c_en.pdf} を追加
\end{itemize}
\end{itemize}

\item 2019/9/10
\begin{itemize}
\item FDPS 5.0g リリース
\begin{itemize}
\item コンパイル時にマクロ\texttt{PARTICLE\_SIMULATOR\_TWO\_DIMENSION}を定義した場合、実行時エラーになる問題を修正
\end{itemize}
\end{itemize}


\item 2020/8/16
\begin{itemize}
\item FDPS 6.0 リリース
\begin{itemize}
\item PIKGを導入
\end{itemize}
\end{itemize}

\item 2020.8.18
\begin{itemize}
\item FDPS 6.0a リリース
\begin{itemize}
\item 付属のPIKGのバージョンをv0.1bに更新
\end{itemize}
\end{itemize}

\item 2020.8.19
\begin{itemize}
\item FDPS 6.0b リリース
\begin{itemize}
\item $N$体シミュレーションサンプルコード(\path{sample/*/nbody})の実装をPIKGで生成したカーネルを使った場合に性能が出るように改善
\end{itemize}
\end{itemize}

\item 2020.8.28
\begin{itemize}
\item FDPS 6.0b1 リリース
\begin{itemize}
\item サンプルコードで使用する初期条件配布先が変更になったため、チュートリアルを修正
\end{itemize}
\end{itemize}

\item 2020.9.02
\begin{itemize}
\item FDPS 6.0b2 リリース
\begin{itemize}
\item サンプルコードのツリーオブジェクトを初期化する関数の第一引数を修正
\item 対応するチュートリアルの記述も修正
\end{itemize}
\end{itemize}

\end{itemize}
