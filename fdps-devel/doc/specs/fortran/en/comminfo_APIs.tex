%
% comminfo_APIs.tex
%
% FDPS development team June 2021
%

In this section we describe the APIs to manipulate MPI communicators.

The list of the APIs explained here is shown below:

\begin{screen}
  \begin{spverbatim}
(fdps_)ci_initialize
(fdps_)ci_set_communicator
(fdps_)ci_delete
(fdps_)ci_create
(fdps_)ci_split
\end{spverbatim}
\end{screen}

These functions share a table of MPI communicators and
manipulate them through the index of the table.
By giving the index to communication functions with prefix
{\tt ci\_} and by giving the MPI communicator to dinfo, psys and tree
by functions such as  {\tt   set\_dinfo\_comm\_info}, a user program
can use multiple MPI communicators.

Using these functions, a single MIP program can handle multiple FDPS
instances 

In the following we describe the specifications of APIs. 

%=============================================================
\subsection{ci\_initialize}
\subsubsection*{Fortran syntax}
\begin{screen}
\begin{spverbatim}
integer(kind=c_int) fdps_ctrl%ci_initialize(comm)
\end{spverbatim}
\end{screen}

\subsubsection*{C syntax}
\begin{screen}
\begin{spverbatim}
int fdps_ci_initialize(int comm);
\end{spverbatim}
\end{screen}

\subsubsection*{Dummy argument specification}
\begin{table}[h]
\begin{tabularx}{\linewidth}{cXcX}
\toprule
\rowcolor{Snow2}
Name & Data type & I/O characteristics & Definition \\
\midrule
\verb|comm| & integer(kind=c\_int) & Input & MPI communicator(Fortran API)\\
\bottomrule
\end{tabularx}
\end{table}


\subsubsection*{Returned value}
Type integer(kind=c\_int). Returns the index corresponding to the
given communicator.

\subsubsection*{Function}
Returns the index corresponding to the
given communicator.
Note that the input MPI communicator is of Fortran API.
In other words, the second argument is not an MPI communicator of
type {\tt MPI\_Comm} in MPI C language binding and thus
a C program shoud pass the communicator in Fortran binding, which
can be obtained by using {\tt MPI\_Comm\_c2f} function.


%=============================================================
\subsection{ci\_set\_communicator}
\subsubsection*{Fortran syntax}
\begin{screen}
\begin{spverbatim}
subroutine fdps_ctrl%ci_set_communicator(ci, comm)
\end{spverbatim}
\end{screen}

\subsubsection*{C syntax}
\begin{screen}
\begin{spverbatim}
void fdps_ci_set_communicator(int ci, int comm);
\end{spverbatim}
\end{screen}

\subsubsection*{Dummy argument specification}
\begin{table}[h]
\begin{tabularx}{\linewidth}{cXcX}
\toprule
\rowcolor{Snow2}
Name & Data type & I/O characteristics & Definition \\
\midrule
\verb|ci| & integer(kind=c\_int) & Input & The index of the communicator\\
\verb|comm| & integer(kind=c\_int) & Input & MPI communicator communicator(Fortran API)\\
\bottomrule
\end{tabularx}
\end{table}


\subsubsection*{Returned value}
なし。

\subsubsection*{Function}
あるインデックスのMPIcommunicatorを変更する。

%=============================================================
\subsection{ci\_delete}
\subsubsection*{Fortran syntax}
\begin{screen}
\begin{spverbatim}
subroutine fdps_ctrl%ci_delete(ci, comm)
\end{spverbatim}
\end{screen}

\subsubsection*{C syntax}
\begin{screen}
\begin{spverbatim}
void fdps_ci_delete(int ci, int comm);
\end{spverbatim}
\end{screen}

\subsubsection*{Dummy argument specification}
\begin{table}[h]
\begin{tabularx}{\linewidth}{cXcX}
\toprule
\rowcolor{Snow2}
Name & Data type & I/O characteristics & Definition \\
\midrule
\verb|ci| & integer(kind=c\_int) & Input & communicator index\\
\verb|comm| & integer(kind=c\_int) & Input & MPIcommunicator(Fortran API)\\
\bottomrule
\end{tabularx}
\end{table}


\subsubsection*{Returned value}
なし。

\subsubsection*{Function}
Delete the MPIcommunicator of an index ({\tt MPI\_Comm\_free}).
This index will become the unused state and cannot be used until
a new communicator is assigned by {\tt ci\_initialize}.


%=============================================================
\subsection{ci\_create}
\subsubsection*{Fortran syntax}
\begin{screen}
\begin{spverbatim}
integer(kind=c_int) fdps_ctrl%ci_create(ci, n, rank)
\end{spverbatim}
\end{screen}

\subsubsection*{C syntax}
\begin{screen}
\begin{spverbatim}
int fdps_ci_create(int ci, int n, int rank[]);
\end{spverbatim}
\end{screen}

\subsubsection*{Dummy argument specification}
\begin{table}[h]
\begin{tabularx}{\linewidth}{cXcX}
\toprule
\rowcolor{Snow2}
Name & Data type & I/O characteristics & Definition \\
\midrule
\verb|ci| & integer(kind=c\_int) & Input & communicator index\\
\verb|n| & integer(kind=c\_int) & Input &Number of processes to belong
to the new communicator. \\
\verb|rank| & integer(kind=c\_int), dimension(n) & Input &Ranks of processes processes to belong to the new communicator.\\
\bottomrule
\end{tabularx}
\end{table}


\subsubsection*{Returned value}
Type integer(kind=c\_int). Returns the index of the new  communicator.

\subsubsection*{Function}

Create a new communicator from the one spcified by {\tt ci}.
The new commmunicator contains the processes in the array {\tt rank}.



%=============================================================
\subsection{ci\_split}
\subsubsection*{Fortran syntax}
\begin{screen}
\begin{spverbatim}
integer(kind=c_int) fdps_ctrl%ci_split(ci, n, rank)
\end{spverbatim}
\end{screen}

\subsubsection*{C syntax}
\begin{screen}
\begin{spverbatim}
int fdps_ci_split(int ci, int color, int key);
\end{spverbatim}
\end{screen}

\subsubsection*{Dummy argument specification}
\begin{table}[h]
\begin{tabularx}{\linewidth}{cXcX}
\toprule
\rowcolor{Snow2}
Name & Data type & I/O characteristics & Definition \\
\midrule
\verb|ci| & integer(kind=c\_int) & Input & communicator index\\
\verb|color| & integer(kind=c\_int) & Input & processes with the same
color will belong to the same communicator\\
\verb|key| & integer(kind=c\_int) & Input &Order of processes in one
communicatoris given by this argument\\
\bottomrule
\end{tabularx}
\end{table}


\subsubsection*{Returned value}
Type integer(kind=c\_int). Returns the index of the new  communicator.


\subsubsection*{Function}
Split the communicator specified by  {\tt ci}.
Processes with the same color will belong to the same
communicator. Their ranks will be ordered in the order of their keys.




\clearpage






