In this section, we describe the specifications of other APIs. The list of the APIs explained here is shown below:
\begin{screen}
\begin{spverbatim}
(fdps_)create_mtts
(fdps_)delete_mtts
(fdps_)mtts_init_genrand
(fdps_)mtts_genrand_int31
(fdps_)mtts_genrand_real1
(fdps_)mtts_genrand_real2
(fdps_)mtts_genrand_res53
(fdps_)mt_init_genrand
(fdps_)mt_genrand_int31
(fdps_)mt_genrand_real1
(fdps_)mt_genrand_real2
(fdps_)mt_genrand_res53
\end{spverbatim}  
\end{screen}
In this list, APIs whose name contains a word \texttt{mt} are the APIs to manipulate the pseudorandom number generator Mersenne twister. 

In the following, we describe the specification of each API in the order above.

\clearpage

%=============================================================
\subsection{create\_mtts}
\subsubsection*{Fortran syntax}
\begin{screen}
\begin{spverbatim}  
subroutine fdps_ctrl%create_mtts(mtts_num)
\end{spverbatim}
\end{screen}

\subsubsection*{C syntax}
\begin{screen}
\begin{spverbatim}  
void fdps_create_mtts(int * mtts_num);
\end{spverbatim}
\end{screen}

\subsubsection*{Dummy argument specification}
\begin{table}[h]
\begin{tabularx}{\linewidth}{cccX}
\toprule
\rowcolor{Snow2}
Name & Data type & I/O Characteristics & Definition \\
\midrule
\verb|mtts_num| & integer(kind=c\_int) & Input and Output & Variable receiving the identification number of a pseudorandom number generator object. {\setnoko\Euc{Note that users need to pass the address of the variable in C}}.\\
\bottomrule
\end{tabularx}
\end{table}

\subsubsection*{Returned value}
None.

\subsubsection*{Function}
Create a object to manipulate the pseudorandom number generator ``Mersenne twister'' and returns its identification number.
\clearpage

%=============================================================
\subsection{delete\_mtts}
\subsubsection*{Fortran syntax}
\begin{screen}
\begin{spverbatim}  
subroutine fdps_ctrl%delete_mtts(mtts_num)
\end{spverbatim}
\end{screen}

\subsubsection*{C syntax}
\begin{screen}
\begin{spverbatim}
void fdps_delete_mtts(const int mtts_num);
\end{spverbatim}
\end{screen}

\subsubsection*{Dummy argument specification}
\begin{table}[h]
\begin{tabularx}{\linewidth}{cccX}
\toprule
\rowcolor{Snow2}
Name & Data type & I/O Characteristics & Definition \\
\midrule
\verb|mtts_num| & integer(kind=c\_int) & Input & Variable giving the identification number of a pseudorandom number generator object.\\
\bottomrule
\end{tabularx}
\end{table}

\subsubsection*{Returned value}
None.

\subsubsection*{Function}
Delete a pseudorandom number generator object indicated by the identification number. 
\clearpage

%=============================================================
\subsection{mtts\_init\_genrand}
\subsubsection*{Fortran syntax}
\begin{screen}
\begin{spverbatim}  
subroutine fdps_ctrl%mtts_init_genrand(mtts_num,s)
\end{spverbatim}
\end{screen}

\subsubsection*{C syntax}
\begin{screen}
\begin{spverbatim}  
void fdps_mtts_init_genrand(const int mtts_num,
                            const int s);
\end{spverbatim}
\end{screen}

\subsubsection*{Dummy argument specification}
\begin{table}[h]
\begin{tabularx}{\linewidth}{cccX}
\toprule
\rowcolor{Snow2}
Name & Data type & I/O Characteristics & Definition \\
\midrule
\verb|mtts_num| & integer(kind=c\_int) & Input & Variable giving the identification number of a pseudorandom number generator object.\\
\verb|s| & integer(kind=c\_int) & Input & a seed used to generate pseudorandom number.\\
\bottomrule
\end{tabularx}
\end{table}

\subsubsection*{Returned value}
None.

\subsubsection*{Function}
Initialize a pseudorandom number generator object indicated by the identification number.
\clearpage

%=============================================================
\subsection{mtts\_genrand\_int31}
\subsubsection*{Fortran syntax}
\begin{screen}
\begin{spverbatim}  
function fdps_ctrl%mtts_genrand_int31(mtts_num)
\end{spverbatim}
\end{screen}

\subsubsection*{C syntax}
\begin{screen}
\begin{spverbatim}  
int fdps_mtts_genrand_int31(const int mtts_num);
\end{spverbatim}
\end{screen}

\subsubsection*{Dummy argument specification}
\begin{table}[h]
\begin{tabularx}{\linewidth}{cccX}
\toprule
\rowcolor{Snow2}
Name & Data type & I/O Characteristics & Definition \\
\midrule
\verb|mtts_num| & integer(kind=c\_int) & Input & Variable giving the identification number of a pseudorandom number generator object.\\
\bottomrule
\end{tabularx}
\end{table}

\subsubsection*{Returned value}
Type integer(kind=c\_int).

\subsubsection*{Function}
Using the pseudorandom number generator object indicated by \texttt{mtts\_num}, generate an uniform integer random number on [0,0x7fffffff]-interval.
\clearpage

%=============================================================
\subsection{mtts\_genrand\_real1}
\subsubsection*{Fortran syntax}
\begin{screen}
\begin{spverbatim}  
function fdps_ctrl%mtts_genrand_real1(mtts_num)
\end{spverbatim}
\end{screen}

\subsubsection*{C syntax}
\begin{screen}
\begin{spverbatim}  
double fdps_mtts_genrand_real1(const int mtts_num);
\end{spverbatim}
\end{screen}

\subsubsection*{Dummy argument specification}
\begin{table}[h]
\begin{tabularx}{\linewidth}{cccX}
\toprule
\rowcolor{Snow2}
Name & Data type & I/O Characteristics & Definition \\
\midrule
\verb|mtts_num| & integer(kind=c\_int) & Input & Variable giving the identification number of a pseudorandom number generator object.\\
\bottomrule
\end{tabularx}
\end{table}

\subsubsection*{Returned value}
Type real(kind=c double).

\subsubsection*{Function}
Using the pseudorandom number generator object indicated by \texttt{mtts\_num}, generate a floating-point random number on [0,1]-interval.
\clearpage

%=============================================================
\subsection{mtts\_genrand\_real2}
\subsubsection*{Fortran syntax}
\begin{screen}
\begin{spverbatim}  
function fdps_ctrl%mtts_genrand_real2(mtts_num)
\end{spverbatim}
\end{screen}

\subsubsection*{C syntax}
\begin{screen}
\begin{spverbatim}  
double fdps_mtts_genrand_real2(const int mtts_num);
\end{spverbatim}
\end{screen}

\subsubsection*{Dummy argument specification}
\begin{table}[h]
\begin{tabularx}{\linewidth}{cccX}
\toprule
\rowcolor{Snow2}
Name & Data type & I/O Characteristics & Definition \\
\midrule
\verb|mtts_num| & integer(kind=c\_int) & Input & Variable giving the identification number of a pseudorandom number generator object.\\
\bottomrule
\end{tabularx}
\end{table}

\subsubsection*{Returned value}
Type real(kind=c double).

\subsubsection*{Function}
Using the pseudorandom number generator object indicated by \texttt{mtts\_num}, generate a floating-point random number on [0,1)-interval.
\clearpage

%=============================================================
\subsection{mtts\_genrand\_real3}
\subsubsection*{Fortran synax}
\begin{screen}
\begin{spverbatim}  
function fdps_ctrl%mtts_genrand_real3(mtts_num)
\end{spverbatim}
\end{screen}

\subsubsection*{C syntax}
\begin{screen}
\begin{spverbatim}  
double fdps_mtts_genrand_real3(const int mtts_num);
\end{spverbatim}
\end{screen}

\subsubsection*{Dummy argument specification}
\begin{table}[h]
\begin{tabularx}{\linewidth}{cccX}
\toprule
\rowcolor{Snow2}
Name & Data type & I/O Characteristics & Definition \\
\midrule
\verb|mtts_num| & integer(kind=c\_int) & Input & Variable giving the identification number of a pseudorandom number generator object.\\
\bottomrule
\end{tabularx}
\end{table}

\subsubsection*{Returned value}
Type real(kind=c double).

\subsubsection*{Function}
Using the pseudorandom number generator object indicated by \texttt{mtts\_num}, generate a floating-point random number on (0,1)-interval.
\clearpage

%=============================================================
\subsection{mtts\_genrand\_res53}
\subsubsection*{Fortran syntax}
\begin{screen}
\begin{spverbatim}  
function fdps_ctrl%mtts_genrand_res53(mtts_num)
\end{spverbatim}
\end{screen}

\subsubsection*{C syntax}
\begin{screen}
\begin{spverbatim}  
double fdps_mtts_genrand_res53(const int mtts_num);
\end{spverbatim}
\end{screen}

\subsubsection*{Dummy argument specification}
\begin{table}[h]
\begin{tabularx}{\linewidth}{cccX}
\toprule
\rowcolor{Snow2}
Name & Data type & I/O Characteristics & Definition \\
\midrule
\verb|mtts_num| & integer(kind=c\_int) & Input & Variable giving the identification number of a pseudorandom number generator object.\\
\bottomrule
\end{tabularx}
\end{table}

\subsubsection*{Returned value}
Type real(kind=c double).

\subsubsection*{Function}
Using the pseudorandom number generator object indicated by \texttt{mtts\_num}, generate a floating-point random number on [0,1)-interval with 53 bit resolution. Note that the prescribed APIs \texttt{mtts\_genrand\_real}$x$ ($x$=1-3) use a 32-bit integer to generate a floating-point random number.
\clearpage

%=============================================================
% API名::MT_init_genrand()
\subsection{mt\_init\_genrand}
\subsubsection*{Fortran syntax}
\begin{screen}
\begin{spverbatim}
subroutine fdps_ctrl%mt_init_genrand(s)
\end{spverbatim}
\end{screen}

\subsubsection*{C syntax}
\begin{screen}
\begin{spverbatim}
void fdps_mt_init_genrand(const int s);
\end{spverbatim}
\end{screen}

\subsubsection*{Dummy argument specification}
\begin{table}[h]
\begin{tabularx}{\linewidth}{cccX}
\toprule
\rowcolor{Snow2}
Name & Data type & I/O Characteristics & Definition \\
\midrule
\texttt{s} & integer(kind=c\_int) & Input and Output & a seed used to generate pseudorandom number. \\
\bottomrule
\end{tabularx}
\end{table}

\subsubsection*{Returned value}
None.

\subsubsection*{Function}
Create an object for the pseudorandom number generator "Mersenne twister" and initialize it.
\clearpage

%=============================================================
% API名::MT_genrand_int31()
\subsection{mt\_genrand\_int31}
\subsubsection*{Fortran syntax}
\begin{screen}
\begin{spverbatim}
function fdps_ctrl%mt_genrand_int31()
\end{spverbatim}
\end{screen}

\subsubsection*{C syntax}
\begin{screen}
\begin{spverbatim}
int fdps_mt_genrand_int31();
\end{spverbatim}
\end{screen}

\subsubsection*{Dummy argument specification}
None.

\subsubsection*{Returned value}
Type integer(kind=c\_int).

\subsubsection*{Function}
Using the pseudorandom number generator "Mersenne twister", generate an uniform integer random number on [0,0x7fffffff]-interval.
\clearpage

%=============================================================
% API名::MT_genrand_real1()
\subsection{mt\_genrand\_real1}
\subsubsection*{Fortran syntax}
\begin{screen}
\begin{spverbatim}
function fdps_ctrl%mt_genrand_real1()
\end{spverbatim}
\end{screen}

\subsubsection*{C syntax}
\begin{screen}
\begin{spverbatim}
double fdps_mt_genrand_real1();
\end{spverbatim}
\end{screen}


\subsubsection*{Dummy argument specification}
None.

\subsubsection*{Returned value}
Type real(kind=c\_double).

\subsubsection*{Function}
Using the pseudorandom number generator "Mersenne twister", generate a floating-point random number on [0,1]-interval.
\clearpage

%=============================================================
% API名::MT_genrand_real2()
\subsection{mt\_genrand\_real2}
\subsubsection*{Fortran syntax}
\begin{screen}
\begin{spverbatim}
function fdps_ctrl%mt_genrand_real2()
\end{spverbatim}
\end{screen}

\subsubsection*{C syntax}
\begin{screen}
\begin{spverbatim}
double fdps_mt_genrand_real2();
\end{spverbatim}
\end{screen}

\subsubsection*{Dummy argument specification}
None.

\subsubsection*{Returned value}
Type real(kind=c\_double).

\subsubsection*{Function}
Using the pseudorandom number generator "Mersenne twister", generate a floating-point random number on [0,1)-interval.
\clearpage

%=============================================================
% API名::MT_genrand_real3()
\subsection{mt\_genrand\_real3}
\subsubsection*{Fortran syntax}
\begin{screen}
\begin{spverbatim}
function fdps_ctrl%mt_genrand_real3()
\end{spverbatim}
\end{screen}

\subsubsection*{C syntax}
\begin{screen}
\begin{spverbatim}
double fdps_mt_genrand_real3();
\end{spverbatim}
\end{screen}

\subsubsection*{Dummy argument specification}
None.

\subsubsection*{Returned value}
Type real(kind=c\_double).

\subsubsection*{Function}
Using the pseudorandom number generator "Mersenne twister", generate a floating-point random number on (0,1)-interval.
\clearpage

%=============================================================
% API名::MT_genrand_res53()
\subsection{mt\_genrand\_res53}
\subsubsection*{Fortran syntax}
\begin{screen}
\begin{spverbatim}
function fdps_ctrl%mt_genrand_res53()
\end{spverbatim}
\end{screen}

\subsubsection*{C syntax}
\begin{screen}
\begin{spverbatim}
double fdps_mt_genrand_res53();
\end{spverbatim}
\end{screen}

\subsubsection*{Dummy argument specification}
None.

\subsubsection*{Returned value}
Type real(kind=c\_double).

\subsubsection*{Function}
Using the pseudorandom number generator "Mersenne twister", generate a floating-point random number on [0,1)-interval with 53 bit resolution. Note that the prescribed APIs \verb|mt_genrand_real|$x$ ($x$=1-3) use a 32-bit integer to generate a floating-point random number.
\clearpage








