\subsection{概要}

本節では、名前空間の構造について述べる。FDPSはParticleSimulatorという
名前空間で囲まれている。以下では、ParticleSimulator直下にある機能と、
ParticleSimulatorにネストされている名前空間について述べる。

\subsection{ParticleSimulator}

FDPSの標準機能すべては名前空間ParticleSimulatorの直下にある。

名前空間ParticleSimulatorは以下のように省略されており、この文書におけ
るあとの記述でもこの省略形を採用する。
\begin{screen}
\begin{verbatim}
namespace PS = ParticleSimulator;
\end{verbatim}
\end{screen}

名前空間ParticleSimulatorの下にはいくつかの名前空間が拡張機能毎にネストされている。拡
張機能にはParticleMeshがある。以下では拡張機能の名前空間について記述す
る。

\subsubsection{ParticleMesh}

Particle Meshの機能は名前空間PartcielMeshに囲まれており、名前空間
ParticleMeshは名前空間ParticleSimulatorの直下にネストされている。また、
ParticleMeshはPMと省略されている。これらをまとめると以下のようになって
いる。
\begin{screen}
\begin{verbatim}
ParticleSimulator {
    ParticleMesh {
    }
    namespace PM = ParticleMesh;
}
\end{verbatim}
\end{screen}

以後、この文書では省略形のPMを用いて記述する。
