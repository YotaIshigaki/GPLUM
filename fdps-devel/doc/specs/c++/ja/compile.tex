\subsection{概要}

FDPSでは、座標系、並列処理の有無、浮動小数点数型の精度、ツリーの最大の
深さ、粒子ソートのアルゴリズム等を選択できる。この選択はコンパイル時の
マクロの定義によってなされる。以下、それぞれの選択の方法を記述する。

\subsection{座標系}
\label{sec:compile_coordinate}

\subsubsection{概要}

座標系は直角座標系3次元と直角座標系2次元の選択ができる。以下、それら
の選択方法について述べる。

\subsubsection{直角座標系3次元}

デフォルトは直角座標系3次元である。なにも行わなくても直角座標系3次元
となる。

\subsubsection{直角座標系2次元}

コンパイル時にPARTICLE\_SIMULATOR\_TWO\_DIMENSIONをマクロ定義すると直
交座標系2次元となる。

\subsection{並列処理}

\subsubsection{概要}

並列処理に関しては、OpenMPの使用/不使用、MPIの使用/不使用を選択でき
る。以下、選択の仕方について記述する。

\subsubsection{OpenMPの使用}

デフォルトはOpenMP不使用である。使用する場合は、
\\ PARTICLE\_SIMULATOR\_THREAD\_PARALLELをマクロ定義すればよい。GCCコ
ンパイラの場合はコンパイラオプションに-fopenmpをつける必要がある。

\subsubsection{MPIの使用}

デフォルトはMPI不使用である。使用する場合は、
PARTICLE\_SIMULATOR\_THREAD\_PARALLELをマクロ定義すればよい。

\subsection{データ型の精度}
\subsubsection{概要}
FDPS側で用意したMomentクラス(第7.5.2節参照)とSuperParticleJクラス(第7.6.2節参照)のデータ型の精度を選択できる。以下、選択の仕方について記述する。

\subsubsection{既存のSuperParticleJクラスとMomentクラスの精度}
既存のSuperParticleJクラスとMomentクラスのメンバ変数の精度はデフォルトで64ビットである。32ビットにしたい場合、\\
PARTICLE\_SIMULATOR\_SPMOM\_F32 \\
をマクロ定義すればよい。


%\subsubsection{Vector型の範囲チェック}
%\label{sec:compile:vector_invalid_access}

%コンパイル時にPARTICLE\_SIMULATOR\_VECTOR\_RANGE\_CHECKをマクロ定義す
%るとVector型の範囲外の成分にアクセスするとエラーメッセージを出力するこ
%とが出来る。詳しくは節\ref{sec:errormessage:vector_invalid_access}を参
%照。


\subsection{ツリーの最大深さ(レベル)の変更}
\label{sec:compile_tree_level}


ツリー構造は粒子の位置座標のモートンキーを用いて作られているため、ツリー
の最大レベルはモートンキーのビット長に依存する.FDPSでは3次元シミュレー
ションの場合にのみ,適切なマクロを指定することにより、粒子のモートンキー
を64bit(レベル21),96bit(最大レベル31),128bit(最大レベル42)の3つから
選択することができる.特にマクロを指定しなければ,キーの長さは128bitに
なる.キーを96bitにする場合はPARTICLE\_SIMULATOR\_USE\_96BIT\_KEYを,
64bitにする場合はPARTICLE\_SIMULATOR\_USE\_64BIT\_KEYをマクロ定義す
ればよい.

アーキテクチャによっては短いキーとすることで若干の高速化やメモリ節約が
実現できるが、特に64ビットキーでは粒子分布によってはレベルが不足する。

\subsection{粒子のソートの方法の変更}
\label{sec:compile_sort_method}

{\tt TreeForForce}クラスの内部では粒子はモートンキーの順でソートされて
いる.デフォルトではソートアルゴリズムとしてマージソートが使われている
が,PARTICLE\_SIMULATOR\_USE\_RADIX\_SORTをマクロ定義することでソート
アルゴリズムを基数ソートに変更できる.

アーキテクチャによっては基数ソートが若干速いかもしれない。



