この文書は大規模並列粒子シミュレーションの開発を支援するFramework for
Developing Particle Simulator (FDPS)の仕様書である。この文書は理化学研
究所計算科学研究機構粒子系シミュレータ研究チームの谷川衝、岩澤全規、細
野七月、似鳥啓吾、村主崇行、牧野淳一郎によって記述された。

この文書は以下のような構成となっている。

節\ref{sec:overview}、\ref{sec:configuration}、\ref{sec:compile}には、
FDPSを使ってプログラムを書く際に前提となる情報が記述されている。
節\ref{sec:overview}には、FDPSの概要として、FDPSの基本的な考えかたや動
作が記述されている。節\ref{sec:configuration}には、FDPSのファイル構成
が記述されている。節\ref{sec:compile}には、FDPSのAPIを使用したコードを
コンパイルする時にどのようなマクロを用いればよいかが記述されている。

節\ref{sec:namespace}、\ref{sec:datatype}、\ref{sec:userdefined}、
\ref{sec:initfin}、\ref{sec:module}には、FDPSを使ってプログラムを書く
際に必要となる情報が提供されている。節\ref{sec:namespace}には、FDPS内
での名前空間の構造についてが記述されている。節\ref{sec:datatype}には、
FDPSで独自に定義されているデータ型が記述されている。
節\ref{sec:userdefined}には、FDPSのAPIを使用する際にユーザーが定義する
必要があるクラスや関数オブジェクトについて記述されている。
節\ref{sec:initfin}には、FDPSを開始するときと終了するときに呼ぶ必要の
あるAPIについて記述されている。節\ref{sec:module}には、FDPSにあるモジュー
ルとそのAPIについて記述されている。

節\ref{sec:errormessage}、\ref{sec:knownbug}、\ref{sec:limitation}、
\ref{sec:usersupport}には、FDPSのAPIを使用したコードを記述したがコード
が思ったように動作しない場合に有用な情報が記載されている。
節\ref{sec:errormessage}にはエラーメッセージについてが記述されている。
節\ref{sec:knownbug}には、よく知られているバグについて記述されている。
節\ref{sec:limitation}には、FDPSの限界について記述されている。
節\ref{sec:usersupport}には、ユーザーサポートに関する情報が記述されて
いる。

最後に節\ref{sec:license}にはFDPSのライセンスに関する情報が、
節\ref{sec:changelog}にはこの文書の変更履歴が記述されている。
