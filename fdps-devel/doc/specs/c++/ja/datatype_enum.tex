\subsubsection{概要}

本節ではFDPSで定義されている列挙型について記述する。列挙型には
BOUNDARY\_CONDITION型とINTERACTION\_LIST\_MODE型が存在する。以下、各列
挙型について記述する。

\subsubsection{PS::BOUNDARY\_CONDITION型}
\label{sec:datatype_enum_boundarycondition}

\subsubsubsection{概要}

BOUNDARY\_CONDITION型は境界条件を指定するためのデータ型である。これは
以下のように定義されている。
\begin{lstlisting}[caption=boundarycondition]
namespace ParticleSimulator{
    enum BOUNDARY_CONDITION{
        BOUNDARY_CONDITION_OPEN,
        BOUNDARY_CONDITION_PERIODIC_X,
        BOUNDARY_CONDITION_PERIODIC_Y,
        BOUNDARY_CONDITION_PERIODIC_Z,
        BOUNDARY_CONDITION_PERIODIC_XY,
        BOUNDARY_CONDITION_PERIODIC_XZ,
        BOUNDARY_CONDITION_PERIODIC_YZ,
        BOUNDARY_CONDITION_PERIODIC_XYZ,
        BOUNDARY_CONDITION_SHEARING_BOX,
        BOUNDARY_CONDITION_USER_DEFINED,
    };
}
\end{lstlisting}

以下にどの変数がどの境界条件に対応するかを記述する。

\subsubsubsection{PS::BOUNDARY\_CONDITION\_OPEN}

開放境界となる。

\subsubsubsection{PS::BOUNDARY\_CONDITION\_PERIODIC\_X}

x軸方向のみ周期境界、その他の軸方向は開放境界となる。周期の境界の下限
は閉境界、上限は開境界となっている。この境界の規定はすべての軸方向にあ
てはまる。

\subsubsubsection{PS::BOUNDARY\_CONDITION\_PERIODIC\_Y}

y軸方向のみ周期境界、その他の軸方向は開放境界となる。

\subsubsubsection{PS::BOUNDARY\_CONDITION\_PERIODIC\_Z}

z軸方向のみ周期境界、その他の軸方向は開放境界となる。

\subsubsubsection{PS::BOUNDARY\_CONDITION\_PERIODIC\_XY}

x, y軸方向のみ周期境界、その他の軸方向は開放境界となる。

\subsubsubsection{PS::BOUNDARY\_CONDITION\_PERIODIC\_XZ}

x, z軸方向のみ周期境界、その他の軸方向は開放境界となる。

\subsubsubsection{PS::BOUNDARY\_CONDITION\_PERIODIC\_YZ}

y, z軸方向のみ周期境界、その他の軸方向は開放境界となる。

\subsubsubsection{PS::BOUNDARY\_CONDITION\_PERIODIC\_XYZ}

x, y, z軸方向すべてが周期境界となる。

\subsubsubsection{PS::BOUNDARY\_CONDITION\_SHEARING\_BOX}

未実装。

\subsubsubsection{PS::BOUNDARY\_CONDITION\_USER\_DEFINED}

未実装。

%%%%%%%%%%%%%%%%%%%%%%%%%%%%%%%%%%%
\subsubsection{PS::INTERACTION\_LIST\_MODE型}
\label{sec:datatype_enum_interaction_list_mode}

\subsubsubsection{概要}

INTERACTION\_LIST\_MODE型は相互作用リストを再利用するかどうかを決定するためのデータ型である。これは以下のように定義されている。

\begin{lstlisting}[caption=interaction\_list\_mode]
namespace ParticleSimulator{
    enum INTERACTION_LIST_MODE{
        MAKE_LIST,
        MAKE_LIST_FOR_REUSE,
        REUSE_LIST,
    };
}
\end{lstlisting}

このデータ型はcalcForceAllAndWriteBack()等の関数の引数として使われる(詳しくはセクション\ref{sec:treeForForceHighLevelAPI}を参照)。

\subsubsubsection{PS::MAKE\_LIST}
相互作用リストを毎回作り相互作用計算を行う場合に用いる。FDPS は相互作
用リストを記憶しないため、相互作用リストの再利用はできない。これはデフォ
ルト動作である。


\subsubsubsection{PS::MAKE\_LIST\_FOR\_REUSE}
相互作用リストを再利用し相互作用計算を行いたい場合に用いる。このオプションを選択する事でFDPSは相互作用リストを作りそれを保持する。作成した相互作用リストはPS::MAKE\_LIST\_FOR\_REUSEもしくはPS::MAKE\_LISTを用いて相互作用計算を行った際に破棄される。

\subsubsubsection{PS::REUSE\_INTERACTION\_LIST}
相互作用リストを再利用し相互作用計算を行う。再利用される相互作用リスト
は PS::MAKE\_LIST\_FOR\_REUSE を選択時に作成した相互作用リストである。
この時、超粒子のモーメント情報は自動的にアップデートされる。


%%%%%%%%%%%%%%%%%%%%%%%%%%%%%%%%%%%
\subsubsection{PS::EXCHANGE\_LET\_MODE型}
\label{sec:datatype_enum_exchange_let_mode}

\subsubsubsection{概要}

EXCHANGE\_LET\_MODE型はLET交換の方法を決定するためのデータ型である。こ
れは以下のように定義されている。

\begin{lstlisting}[caption=EXCHANGE\_LET\_MODE]
namespace ParticleSimulator{
    enum EXCHANGE_LET_MODE{
        EXCHANGE_LET_A2A,
        EXCHANGE_LET_P2P_EXACT,
        EXCHANGE_LET_P2P_FAST,
    };
}
\end{lstlisting}

このデータ型は{\tt PS::TreeForForce::setExchangeLet}の引数として使われる.
(詳しくはセクション\ref{sec:module_standard_treeforce_setexchangeletmode}
を参照).

\subsubsubsection{PS::EXCHANGE\_LET\_A2A}

LETの交換に{\tt MPI\_Alltoall}を使用する。

\subsubsubsection{PS::EXCHANGE\_LET\_P2P\_EXACT}

LETの交換に{\tt MPI\_Alltoall}を使用せず、{\tt MPI\_Allgather}と{\tt
MPI\_Isend/recv}を使用する。{\tt MPI\_Alltoall}が効率的に動かない計算
機では、こちらの方が速い場合がある。結果は丸め誤差の範囲で
PS::EXCHANGE\_LET\_A2Aを用いた場合と一致する。

\subsubsubsection{EXCHANGE\_LET\_P2P\_FAST}

LETの交換に{\tt MPI\_Alltoall}を使用せず、{\tt MPI\_Allgather}と{\tt
MPI\_Isend/recv}を使用する。結果は{\tt PS::EXCHANGE\_LET\_P2P\_EXACT}
を用いた場合と異なるが、より通信量が減っているため、高速に動作する可能
性がある。



%%%%%%%%%%%%%%%%%%%%%%%%%%%%%%%%%%%
\subsubsection{PS::CALC\_DISTANCE\_TYPE型}
\label{sec:datatype_enum_calc_distance_type}

\subsubsubsection{概要}

{\tt CALC\_DISTANCE\_TYPE}型は粒子間の距離の計算方法を変更するための型
である.

\begin{lstlisting}[caption=CALC\_DISTANCE\_TYPE]
namespace ParticleSimulator{
    enum CALC_DISTANCE_TYPE{
        CALC_DISTANCE_TYPE_NORMAL = 0,
        CALC_DISTANCE_TYPE_NEAREST_X = 1,
	CALC_DISTANCE_TYPE_NEAREST_Y = 2,
        CALC_DISTANCE_TYPE_NEAREST_XY = 3,
	CALC_DISTANCE_TYPE_NEAREST_Z = 4,
	CALC_DISTANCE_TYPE_NEAREST_XZ = 5,
	CALC_DISTANCE_TYPE_NEAREST_YZ = 6,
	CALC_DISTANCE_TYPE_NEAREST_XYZ = 7,
    };
}
\end{lstlisting}

このデータ型は{\tt PS::TreeForForce}のテンプレート引数として与えられる.
(セクション\ref{sec:module_standard_treeforce_object}を参照).

\subsubsubsection{CALC\_DISTANCE\_TYPE\_NORMAL}

領域内の粒子のL2ノルムを計算する.特に指定のない場合は、この値が使われる.

\subsubsubsection{CALC\_DISTANCE\_TYPE\_NEAREST\_X}

x方向に周期境界だとした場合に最も近い粒子とのL2ノルムを計算する.FDPS
のx方向の境界条件が開放境界の場合でも、最近接粒子との距離を計算する。

\subsubsubsection{CALC\_DISTANCE\_TYPE\_NEAREST\_Y}

Y方向に周期境界だとした場合に最も近い粒子とのL2ノルムを計算する.FDPS
のy方向の境界条件が開放境界の場合でも、最近接粒子との距離を計算する。

\subsubsubsection{CALC\_DISTANCE\_TYPE\_NEAREST\_Z}

Z方向に周期境界だとした場合に最も近い粒子とのL2ノルムを計算する.FDPS
のz方向の境界条件が開放境界の場合でも、最近接粒子との距離を計算する。

\subsubsubsection{CALC\_DISTANCE\_TYPE\_NEAREST\_XY}

X,Y方向に周期境界だとした場合に最も近い粒子とのL2ノルムを計算する.
FDPSのx,y方向の境界条件が開放境界の場合でも、最近接粒子との距離を計算
する。

\subsubsubsection{CALC\_DISTANCE\_TYPE\_NEAREST\_XZ}

X,Z方向に周期境界だとした場合に最も近い粒子とのL2ノルムを計算する.
FDPSのx,z方向の境界条件が開放境界の場合でも、最近接粒子との距離を計算す
る。

\subsubsubsection{CALC\_DISTANCE\_TYPE\_NEAREST\_YZ}

Y,Z方向に周期境界だとした場合に最も近い粒子とのL2ノルムを計算する.
FDPSのy,z方向の境界条件が開放境界の場合でも、最近接粒子との距離を計算
する。

\subsubsubsection{CALC\_DISTANCE\_TYPE\_NEAREST\_XYZ}

X,Y,Z方向に周期境界だとした場合に最も近い粒子とのL2ノルムを計算する.
FDPSのx,y,z方向の境界条件が開放境界の場合でも、最近接粒子との距離を計
算する。
