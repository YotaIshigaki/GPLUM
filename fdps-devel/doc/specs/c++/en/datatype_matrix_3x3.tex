\texttt{PS::MatrixSym3} has six components: \texttt{xx}, \texttt{yy}, \texttt{zz}, \texttt{xy}, \texttt{xz}, and \texttt{yz}.
We define various APIs and operators for these components.
In the following, we list them.
%%PS::MatrixSym3はxx, yy, zz, xy, xz, yzの6要素を持つ。これらに対する様々
%%なAPIや演算子を定義した。それらの宣言を以下に記述する。この節ではこれ
%%らについて詳しく記述する。
\begin{lstlisting}[caption=MatrixSym3]
namespace ParticleSimulator{
    template<class T>
    class MatrixSym3{
    public:
        // Six member variables
        T xx, yy, zz, xy, xz, yz;

        // Constructors
        MatrixSym3() : xx(T(0)), yy(T(0)), zz(T(0)),
                       xy(T(0)), xz(T(0)), yz(T(0)) {}
        MatrixSym3(const T _xx, const T _yy, const T _zz,
                   const T _xy, const T _xz, const T _yz )
                       : xx(_xx), yy(_yy), zz(_zz),
                       xy(_xy), xz(_xz), yz(_yz) {}
        MatrixSym3(const T s) : xx(s), yy(s), zz(s),
                                xy(s), xz(s), yz(s) {}
        MatrixSym3(const MatrixSym3 & src) :
            xx(src.xx), yy(src.yy), zz(src.zz),
            xy(src.xy), xz(src.xz), yz(src.yz) {}

        // Assignment operator
        const MatrixSym3 & operator = (const MatrixSym3 & rhs);

        // Addition and subtraction
        MatrixSym3 operator + (const MatrixSym3 & rhs) const;
        const MatrixSym3 & operator += (const MatrixSym3 & rhs) const;
        MatrixSym3 operator - (const MatrixSym3 & rhs) const;
        const MatrixSym3 & operator -= (const MatrixSym3 & rhs) const;

        // Trace
        T getTrace() const;

        // Typecast to MatrixSym3<U>
        template <typename U>
        operator MatrixSym3<U> () const;
    }
}
namespace PS = ParticleSimulator;
\end{lstlisting}

%%%%%%%%%%%%%%%%%%%%%%%%%%%%%%%%%%%%%%%%%%%%%%%%%%%%%
\subsubsubsection{Constructor}

\begin{screen}
\begin{verbatim}
template<typename T>
PS::MatrixSym3<T>::MatrixSym3();
\end{verbatim}
\end{screen}

\begin{itemize}

\item{{\bf Argument}}

  None.

\item{{\bf Feature}}

Default constructor. All the member variables are initialized as 0.
%デフォルトコンストラクタ。6要素は0で初期化される。

\end{itemize}

\begin{screen}
\begin{verbatim}
template<typename T>
PS::MatrixSym3<T>::MatrixSym3(const T _xx,
                              const T _yy,
                              const T _zz,
                              const T _xy,
                              const T _xz,
                              const T _yz);
\end{verbatim}
\end{screen}

\begin{itemize}

\item{{\bf Argument}}

\texttt{\_xx}: Input. Type \texttt{const T}.
%%{\_xx}: 入力。{const T}型。

\texttt{\_yy}: Input. Type \texttt{const T}.
%{\_yy}: 入力。{const T}型。

\texttt{\_zz}: Input. Type \texttt{const T}.
%{\_zz}: 入力。{const T}型。

\texttt{\_xy}: Input. Type \texttt{const T}.
%{\_xy}: 入力。{const T}型。

\texttt{\_xz}: Input. Type \texttt{const T}.
%{\_xz}: 入力。{const T}型。

\texttt{\_yz}: Input. Type \texttt{const T}.
%{\_yz}: 入力。{const T}型。

\item{{\bf Feature}}

Values of members \texttt{xx}, \texttt{yy}, \texttt{zz}, \texttt{xy},
\texttt{xz} and \texttt{yz} are set to values of arguments \texttt{\_xx},
\texttt{\_yy}, \texttt{\_zz}, \texttt{\_xy}, \texttt{xz} and \texttt{\_yz},
respectively.
%%メンバ{xx}、{yy}、{zz}、{xy}、{xz}、{yz}をそれぞれ{\_xx}、{\_yy}、
%%{\_zz}、{\_xy}、{\_xz}、{\_yz}で初期化する。

\end{itemize}

\begin{screen}
\begin{verbatim}
template<typename T>
PS::MatrixSym3<T>::MatrixSym3(const T s);
\end{verbatim}
\end{screen}

\begin{itemize}

\item{{\bf Argument}}

\texttt{s}: Input. Type \texttt{const T}.
%{s}: 入力。{const T}型。

\item{{\bf Feature}}

Values of members are set to the value of argument \texttt{s}.
%6要素すべてを{s}の値で初期化する。

\end{itemize}

%%%%%%%%%%%%%%%%%%%%%%%%%%%%%%%%%%%%%%%%%%%%%%%%%%%%%
\subsubsubsection{Copy constructor}

\begin{screen}
\begin{verbatim}
template<typename T>
PS::MatrixSym3<T>::MatrixSym3(const PS::MatrixSym3<T> & src)
\end{verbatim}
\end{screen}

\begin{itemize}

\item{{\bf Argument}}

\texttt{src}: Input. Type const \texttt{PS::MatrixSym3$<$T$>$ \&}.
%{src}: 入力。{const PS::MatrixSym3$<$T$>$ \&}型。

\item{{\bf Feature}}

Copy constructor. The new variable will have the same value as \texttt{src}.
%コピーコンストラクタ。{src}で初期化する。

\end{itemize}

%%%%%%%%%%%%%%%%%%%%%%%%%%%%%%%%%%%%%%%%%%%%%%%%%%%%%
\subsubsubsection{Assignment operator}

\begin{screen}
\begin{verbatim}
template<typename T>
const PS::MatrixSym3<T> & PS::MatrixSym3<T>::operator = 
                       (const PS::MatrixSym3<T> & rhs);
\end{verbatim}
\end{screen}

\begin{itemize}

\item{{\bf Argument}}

\texttt{rhs}: Input. Type \texttt{const PS::MatrixSym3$<$T$>$ \&}.
%{rhs}: 入力。{const PS::MatrixSym3$<$T$>$ \&}型。

\item{{\bf Return value}}

Type \texttt{const PS::MatrixSym3$<$T$>$ \&}. Assigns the values of components of \texttt{rhs}
to its components, and returns the vector itself.
%%{const PS::MatrixSym3$<$T$>$ \&}型。{rhs}の6要素それぞれの値を自身の
%%6要素それぞれに代入し自身の参照を返す。代入演算子。

\end{itemize}

%%%%%%%%%%%%%%%%%%%%%%%%%%%%%%%%%%%%%%%%%%%%%%%%%%%%%
\subsubsubsection{Addition and subtraction}

\begin{screen}
\begin{verbatim}
template<typename T>
PS::MatrixSym3<T> PS::MatrixSym3<T>::operator + 
               (const PS::MatrixSym3<T> & rhs) const;
\end{verbatim}
\end{screen}

\begin{itemize}

\item{{\bf Argument}}

\texttt{rhs}: Input. Type \texttt{const PS::MatrixSym3$<$T$>$ \&}.
%{rhs}: 入力。{const PS::MatrixSym3$<$T$>$ \&}型。

\item{{\bf Return value}}

Type \texttt{PS::MatrixSym3$<$T$>$}. Add the components of \texttt{rhs} and its
components, and return the results.
%%{PS::MatrixSym3$<$T$>$ }型。{rhs}の6要素それぞれの値と自身の6要素の
%%値の和を取った値を返す。

\end{itemize}

\begin{screen}
\begin{verbatim}
template<typename T>
const PS::MatrixSym3<T> & PS::MatrixSym3<T>::operator += 
                       (const PS::MatrixSym3<T> & rhs);
\end{verbatim}
\end{screen}

\begin{itemize}

\item{{\bf Argument}}

\texttt{rhs}: Input. Type \texttt{const PS::MatrixSym3$<$T$>$ \&}.
%{rhs}: 入力。{const PS::MatrixSym3$<$T$>$ \&}型。

\item{{\bf Return value}}

Type \texttt{const PS::MatrixSym3$<$T$>$ \&}. Add the components of \texttt{rhs} to its
components, and return itself (lhs is changed).
%%{const PS::MatrixSym3$<$T$>$ \&}型。{rhs}の6要素それぞれの値を自身の
%%6要素それぞれに足し、自身を返す。

\end{itemize}

\begin{screen}
\begin{verbatim}
template<typename T>
PS::MatrixSym3<T> PS::MatrixSym3<T>::operator - 
               (const PS::MatrixSym3<T> & rhs) const;
\end{verbatim}
\end{screen}

\begin{itemize}

\item{{\bf Argument}}

\texttt{rhs}: Input. Type \texttt{const PS::MatrixSym3$<$T$>$ \&}.
%{rhs}: 入力。{const PS::MatrixSym3$<$T$>$ \&}型。

\item{{\bf Return value}}

Type \texttt{PS::MatrixSym3$<$T$>$}. Subtract the components of \texttt{rhs} from its
components, and return the results.
%%{PS::MatrixSym3$<$T$>$}型。{rhs}の6要素それぞれの値と自身の6要素そ
%%れぞれの値の差を取った値を返す。

\end{itemize}

\begin{screen}
\begin{verbatim}
template<typename T>
const PS::MatrixSym3<T> & PS::MatrixSym3<T>::operator -= 
                       (const PS::MatrixSym3<T> & rhs);
\end{verbatim}
\end{screen}

\begin{itemize}

\item{{\bf Argument}}

\texttt{rhs}: Input. Type \texttt{const PS::MatrixSym3$<$T$>$ \&}.
%{rhs}: 入力。{const PS::MatrixSym3$<$T$>$ \&}型。

\item{{\bf Return value}}

Type \texttt{const PS::MatrixSym3$<$T$>$ \&}. Subtract the components of \texttt{rhs} from
its components, and return itself (lhs is changed).
%%{const PS::MatrixSym3$<$T$>$ \&}型。自身の6要素それぞれから{rhs}の6
%%要素それぞれを引き自身を返す。

\end{itemize}

%%%%%%%%%%%%%%%%%%%%%%%%%%%%%%%%%%%%%%%%%%%%%%%%%%%%%
\subsubsubsection{Trace}

\begin{screen}
\begin{verbatim}
template<typename T>
T PS::MatrixSym3<T>::getTrace() const;
\end{verbatim}
\end{screen}

\begin{itemize}

\item{{\bf Argument}}

  None.

\item{{\bf Return value}}

  Type \texttt{T}.

\item{{\bf Feature}}

  Calculate the trace, and return the result.
%  トレースを計算し、その結果を返す。

\end{itemize}

%%%%%%%%%%%%%%%%%%%%%%%%%%%%%%%%%%%%%%%%%%%%%%%%%%%%%
\subsubsubsection{Typecast to MatrixSym3$<$U$>$}

\begin{screen}
\begin{verbatim}
template<typename T>
template<typename U>
PS::MatrixSym3<T>::operator PS::MatrixSym3<U> () const;
\end{verbatim}
\end{screen}

\begin{itemize}

\item{{\bf Argument}}

  None.

\item{{\bf Return value}}

  Type \texttt{const PS::MatrixSym3$<$U$>$}.

%{const PS::MatrixSym3$<$U$>$}型。

\item{{\bf Feature}}

  Typecast from type \texttt{const PS::MatrixSym3$<$T$>$} to \texttt{const
  PS::MatrixSym3$<$U$>$}.
%%  {const PS::MatrixSym3$<$T$>$}型を{const PS::MatrixSym3$<$U$>$}型に
%%  キャ ストする

\end{itemize}

