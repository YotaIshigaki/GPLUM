\subsection{Summary}
In this chapter we define data types used in FDPS.The data types include
integer types, floating point types, vector types, symmetric matrix types,
PS::SEARCH\_MODE type, and enumerated types. We recommend users to
use these data types. The integer and floating point types can be
replaced with data types available in C/C++
languages. PS::SEARCH\_MODE and enumerated types must be used. In the
following, we describe these data types.

%%FDPSでは独自の整数型、実数型、ベクトル型、対称行列型、
%%PS::SEARCH\_MODE型、列挙型が定義されている。整数型、実数型、ベクトル
%%型、対称行列型に関しては必ずしもここに挙げるものを用いる必要はないが、
%%これらを用いることを推奨する。PS::SEARCH\_MODE型、列挙型は必ず用いる
%%必要がある。以下、整数型、実数型、ベクトル型、対称行列型、
%%PS::SEARCH\_MODE型、列挙型の順に記述する。

\subsection{Integer type}

%\subsubsection{概要}
\subsubsection{Overview}

The integer types are PS::S32, PS::S64, PS::U32, and PS::U64. We
describe these data types in this section.

%整数型にはPS::S32, PS::S64, PS::U32, PS::U64がある。以下、順にこれらを
%記述する。

\subsubsection{PS::S32}

PS::S32, which is 32-bit signed integer, is defined as follows.
%PS::S32は以下のように定義されている。すなわち32bitの符号付き整数である。
\begin{lstlisting}[caption=S32]
namespace ParticleSimulator {
    typedef int S32;
}
\end{lstlisting}

%Safe performance is ensured, only when users adopt GCC and K compilers.
%ただし、GCCコンパイラとKコンパイラでのみ32bitであることが保証されている。

\subsubsection{PS::S64}

PS::S64, which is 64-bit signed integer, is defined as follows.
%PS::S64は以下のように定義されている。すなわち64bitの符号付き整数である。
\begin{lstlisting}[caption=S64]
namespace ParticleSimulator {
    typedef long long int S64;
}
\end{lstlisting}

%Safe performance is ensured, only when users adopt GCC and K compilers.
%ただし、GCCコンパイラとKコンパイラでのみ64bitであることが保証されている。

\subsubsection{PS::U32}

PS::U32, which is 32-bit unsigned integer, is defined as follows.
%PS::U32は以下のように定義されている。すなわち32bitの符号なし整数である。
\begin{lstlisting}[caption=U32]
namespace ParticleSimulator {
    typedef unsigned int U32;
}
\end{lstlisting}

%Safe performance is ensured, only when users adopt GCC and K compilers.
%ただし、GCCコンパイラとKコンパイラでのみ32bitであることが保証されている。

\subsubsection{PS::U64}

PS::U64, which is 64-bit unsigned integer, is defined as follows.
%PS::U64は以下のように定義されている。すなわち64bitの符号なし整数である。
\begin{lstlisting}[caption=U64]
namespace ParticleSimulator {
    typedef unsigned long long int U64;
}
\end{lstlisting}

%Safe performance is ensured, only when users adopt GCC and K compilers.
%ただし、GCCコンパイラとKコンパイラでのみ64bitであることが保証されている。



\subsection{Floating point type}

\subsubsection{Abstract}

The floating point types are PS::F32 and PS::F64. We described
these data types in this section.
%実数型にはPS::F32, PS::F64がある。以下、順にこれらを記述する。

\subsubsection{PS::F32}

PS::F32, which is 32-bit floating point number, is defined as follows.
%PS::F32は以下のように定義されている。すなわち32bitの浮動小数点数である。
\begin{lstlisting}[caption=F32]
namespace ParticleSimulator {
    typedef float F32;
}
\end{lstlisting}

\subsubsection{PS::F64}

PS::F64, which is 64-bit floating point number, is defined as follows.
%PS::F64は以下のように定義されている。すなわち64bitの浮動小数点数である。
\begin{lstlisting}[caption=F64]
namespace ParticleSimulator {
    typedef double F64;
}
\end{lstlisting}


\subsection{Vector type}
\label{sec:datatype_vector}
\subsubsection{Abstract}

The vector types are \texttt{PS::Vector2} (2D vector) and
\texttt{PS::Vector3} (3D vector). We describe these vector types first.
These are template classes, which can take basic datatypes
as \texttt{F32} or \texttt{F64} as template arguments. We then present
wrappers for these vector types.

%%ベクトル型には2次元ベクトル型PS::Vector2と3次元ベクトル型PS::Vector3
%%がある。まずこれら2つを記述する。最後にこれらベクトル型のラッパーに
%%ついて記述する。

\subsubsection{PS::Vector2}

PS::Vector2はx, yの2要素を持つ。これらに対する様々なAPIや演算子を定義
した。それらの宣言を以下に記述する。この節ではこれらについて詳しく記述
する。
\begin{lstlisting}[caption=Vector2]
namespace ParticleSimulator{
    template <typename T>
    class Vector2{
    public:
        //メンバ変数2要素
        T x, y;

        //コンストラクタ
        Vector2();
        Vector2(const T _x, const T _y) : x(_x), y(_y) {}
        Vector2(const T s) : x(s), y(s) {}
        Vector2(const Vector2 & src) : x(src.x), y(src.y) {}

        //代入演算子
        const Vector2 & operator = (const Vector2 & rhs);

        //[]演算子
        const T & opertor[](const int i);
        T & operator[](const int i);

        //加減算
        Vector2 operator + (const Vector2 & rhs) const;
        const Vector2 & operator += (const Vector2 & rhs);
        Vector2 operator - (const Vector2 & rhs) const;
        const Vector2 & operator -= (const Vector2 & rhs);

        //ベクトルスカラ積
        Vector2 operator * (const T s) const;
        const Vector2 & operator *= (const T s);
        friend Vector2 operator * (const T s,
                                   const Vector2 & v);
        Vector2 operator / (const T s) const;
        const Vector2 & operator /= (const T s);

        //内積
        T operator * (const Vector2 & rhs) const;

        //外積(返り値はスカラ!!)
        T operator ^ (const Vector2 & rhs) const;

        //Vector2<U>への型変換
        template <typename U>
        operator Vector2<U> () const;
    };
}
namespace PS = ParticleSimulator;
\end{lstlisting}

\subsubsubsection{コンストラクタ}

\begin{screen}
\begin{verbatim}
template<typename T>
PS::Vector2<T>::Vector2()
\end{verbatim}
\end{screen}

\begin{itemize}

\item{{\bf 引数}}

なし。

\item{{\bf 機能}}

デフォルトコンストラクタ。メンバx,yは0で初期化される。

\end{itemize}

%%%%%%%%%%%%%%%%%%%%%%%%%%%%%
\begin{screen}
\begin{verbatim}
template<typename T>
PS::Vector2<T>::Vector2(const T _x, const T _y)
\end{verbatim}
\end{screen}

\begin{itemize}

\item{{\bf 引数}}

{\_x}: 入力。{const T}型。

{\_y}: 入力。{const T}型。

\item{{\bf 機能}}

メンバ{x}、{y}をそれぞれ{\_x}、{\_y}で初期化する。

\end{itemize}

%%%%%%%%%%%%%%%%%%%%%%%%%%%%%
\begin{screen}
\begin{verbatim}
template<typename T>
PS::Vector2<T>::Vector2(const T s);
\end{verbatim}
\end{screen}

\begin{itemize}

\item{{\bf 引数}}

{s}: 入力。{const T}型。

\item{{\bf 機能}}

メンバ{x}、{y}を両方とも{s}の値で初期化する。

\end{itemize}

%%%%%%%%%%%%%%%%%%%%%%%%%%%%%
\subsubsubsection{コピーコンストラクタ}

\begin{screen}
\begin{verbatim}
template<typename T>
PS::Vector2<T>::Vector2(const PS::Vector2<T> & src)
\end{verbatim}
\end{screen}

\begin{itemize}

\item{{\bf 引数}}

{src}: 入力。{const PS::Vector2$<$T$>$ \&}型。

\item{{\bf 機能}}

コピーコンストラクタ。{src}で初期化する。

\end{itemize}

%%%%%%%%%%%%%%%%%%%%%%%%%%%%%
\subsubsubsection{メンバ変数}

\begin{screen}
\begin{verbatim}
template<typename T>
T PS::Vector2<T>::x;

template<typename T>
T PS::Vector2<T>::y;
\end{verbatim}
\end{screen}

\begin{itemize}
  
\item{{\bf 機能}}
  
  メンバ{x}、{y}を直接操作出来る。
  
\end{itemize}

%%%%%%%%%%%%%%%%%%%%%%%%%%%%%
\subsubsubsection{代入演算子}

\begin{screen}
\begin{verbatim}
template<typename T>
const PS::Vector2<T> & PS::Vector2<T>::operator = 
                       (const PS::Vector2<T> & rhs);
\end{verbatim}
\end{screen}

\begin{itemize}

\item{{\bf 引数}}

{rhs}: 入力。{const PS::Vector2$<$T$>$ \&}型。

\item{{\bf 返り値}}

{const PS::Vector2$<$T$>$ \&}型。{rhs}のx,yの値を自身のメンバx,yに
代入し自身の参照を返す。代入演算子。

\end{itemize}

\subsubsubsection{[]演算子}

\begin{screen}
\begin{verbatim}
template<typename T>
const T & PS::Vector2<T>::operator[]
                       (const int i);
\end{verbatim}
\end{screen}

\begin{itemize}

\item{{\bf 引数}}

  {i}: 入力。{const int}型。

\item{{\bf 返り値}}

  {const $<$T$>$ \&}型。ベクトルのi成分を返す。
  
\item{{\bf 備考}}

  直接メンバ変数を指定する場合に比べ、処理が遅くなることがある。

\end{itemize}

\begin{screen}
\begin{verbatim}
template<typename T>
T & PS::Vector2<T>::operator[]
                       (const int i);
\end{verbatim}
\end{screen}

\begin{itemize}

\item{{\bf 引数}}

{i}: 入力。{const int}型。

\item{{\bf 返り値}}

{$<$T$>$ \&}型。ベクトルのi成分を返す。

\item{{\bf 備考}}

  直接メンバ変数を指定する場合に比べ、処理が遅くなることがある。

\end{itemize}

\subsubsubsection{加減算}

\begin{screen}
\begin{verbatim}
template<typename T>
PS::Vector2<T> PS::Vector2<T>::operator + 
               (const PS::Vector2<T> & rhs) const;
\end{verbatim}
\end{screen}

\begin{itemize}

\item{{\bf 引数}}

{rhs}: 入力。{const PS::Vector2$<$T$>$ \&}型。

\item{{\bf 返り値}}

{PS::Vector2$<$T$>$}型。{rhs}のx,yの値と自身のメンバx,yの値の和を
取った値を返す。

\end{itemize}

\begin{screen}
\begin{verbatim}
template<typename T>
const PS::Vector2<T> & PS::Vector2<T>::operator += 
                       (const PS::Vector2<T> & rhs);
\end{verbatim}
\end{screen}

\begin{itemize}

\item{{\bf 引数}}

{rhs}: 入力。{const PS::Vector2$<$T$>$ \&}型。

\item{{\bf 返り値}}

{const PS::Vector2$<$T$>$ \&}型。{rhs}のx,yの値を自身のメンバx,yに足し、自
身を返す。

\end{itemize}

\begin{screen}
\begin{verbatim}
template<typename T>
PS::Vector2<T> PS::Vector2<T>::operator - 
               (const PS::Vector2<T> & rhs) const;
\end{verbatim}
\end{screen}

\begin{itemize}

\item{{\bf 引数}}

{rhs}: 入力。{const PS::Vector2$<$T$>$ \&}型。

\item{{\bf 返り値}}

{PS::Vector2$<$T$>$}型。{rhs}のx,yの値と自身のメンバx,yの値の差を
取った値を返す。

\end{itemize}

\begin{screen}
\begin{verbatim}
template<typename T>
const PS::Vector2<T> & PS::Vector2<T>::operator -= 
                       (const PS::Vector2<T> & rhs);
\end{verbatim}
\end{screen}

\begin{itemize}

\item{{\bf 引数}}

{rhs}: 入力。{const PS::Vector2$<$T$>$ \&}型。

\item{{\bf 返り値}}

{const PS::Vector2$<$T$>$ \&}型。自身のメンバx,yから{rhs}のx,yを引
き自身を返す。

\end{itemize}

\subsubsubsection{ベクトルスカラ積}

\begin{screen}
\begin{verbatim}
template<typename T>
PS::Vector2<T> PS::Vector2<T>::operator * (const T s) const;
\end{verbatim}
\end{screen}

\begin{itemize}

\item{{\bf 引数}}

{s}: 入力。{const T}型。

\item{{\bf 返り値}}

{PS::Vector2$<$T$>$}型。自身のメンバx,yそれぞれに{s}をかけた値を返
す。

\end{itemize}

\begin{screen}
\begin{verbatim}
template<typename T>
const PS::Vector2<T> & PS::Vector2<T>::operator *= (const T s);
\end{verbatim}
\end{screen}

\begin{itemize}

\item{{\bf 引数}}

{rhs}: 入力。{const T}型。

\item{{\bf 返り値}}

{const PS::Vector2$<$T$>$ \&}型。自身のメンバx,yそれぞれに{s}をかけ
自身を返す。

\end{itemize}

\begin{screen}
\begin{verbatim}
template<typename T>
PS::Vector2<T> PS::Vector2<T>::operator / (const T s) const;
\end{verbatim}
\end{screen}

\begin{itemize}

\item{{\bf 引数}}

{s}: 入力。{const T}型。

\item{{\bf 返り値}}

{PS::Vector2$<$T$>$}型。自身のメンバx,yそれぞれを{s}で割った値を返
す。

\end{itemize}

\begin{screen}
\begin{verbatim}
template<typename T>
const PS::Vector2<T> & PS::Vector2<T>::operator /= (const T s);
\end{verbatim}
\end{screen}

\begin{itemize}

\item{{\bf 引数}}

{rhs}: 入力。{const T}型。

\item{{\bf 返り値}}

{const PS::Vector2$<$T$>$ \&}型。自身のメンバx,yそれぞれを{s}で割り
自身を返す。

\end{itemize}

\subsubsubsection{内積、外積}

\begin{screen}
\begin{verbatim}
template<typename T>
T PS::Vector2<T>::operator * (const PS::Vector2<T> & rhs) const;
\end{verbatim}
\end{screen}

\begin{itemize}

\item{{\bf 引数}}

{rhs}: 入力。{const PS::Vector2$<$T$>$ \&}型。

\item{{\bf 返り値}}

{T}型。自身と{rhs}の内積を取った値を返す。

\end{itemize}

\begin{screen}
\begin{verbatim}
template<typename T>
T PS::Vector2<T>::operator ^ (const PS::Vector2<T> & rhs) const;
\end{verbatim}
\end{screen}

\begin{itemize}

\item{{\bf 引数}}

{rhs}: 入力。{const PS::Vector2$<$T$>$ \&}型。

\item{{\bf 返り値}}

{T}型。自身と{rhs}の外積を取った値を返す。

\end{itemize}

\subsubsubsection{{Vector2$<$U$>$}への型変換}

\begin{screen}
\begin{verbatim}
template<typename T>
template <typename U>
PS::Vector2<T>::operator PS::Vector2<U> () const;
\end{verbatim}
\end{screen}

\begin{itemize}

\item{{\bf 引数}}

  なし。

\item{{\bf 返り値}}

  {const PS::Vector2$<$U$>$}型。

\item{{\bf 機能}}

  {const PS::Vector2$<$T$>$}型を{const PS::Vector2$<$U$>$}型にキャ
  ストする。

\end{itemize}




\subsubsection{PS::Vector3}

\texttt{PS::Vector3} has three components: \texttt{x}, \texttt{y}, and \texttt{z}. We define various APIs
and operators for these components. In the following, we list them.
%%PS::Vecotr3はx, y, zの3要素を持つ。これらに対する様々なAPIや演算子を
%%定義した。それらの宣言を以下に記述する。この節ではこれらについて詳し
%%く記述する。
\begin{lstlisting}[caption=Vector3]
namespace ParticleSimulator{
    template <typename T>
    class Vector3{
    public:
        // Three member variables
        T x, y, z;

        // Constructor
        Vector3() : x(T(0)), y(T(0)), z(T(0)) {}
        Vector3(const T _x, const T _y, const T _z) : x(_x), y(_y), z(_z) {}
        Vector3(const T s) : x(s), y(s), z(s) {}
        Vector3(const Vector3 & src) : x(src.x), y(src.y), z(src.z) {}

        // Assignment operator
        const Vector3 & operator = (const Vector3 & rhs);

        //[] operators
        const T & opertor[](const int i);
        T & operator[](const int i);

        // Addition and subtraction
        Vector3 operator + (const Vector3 & rhs) const;
        const Vector3 & operator += (const Vector3 & rhs);
        Vector3 operator - (const Vector3 & rhs) const;
        const Vector3 & operator -= (const Vector3 & rhs);

        // Scalar multiplication
        Vector3 operator * (const T s) const;
        const Vector3 & operator *= (const T s);
        friend Vector3 operator * (const T s, const Vector3 & v);
        Vector3 operator / (const T s) const;
        const Vector3 & operator /= (const T s);

        // Inner product (scalar product)
        T operator * (const Vector3 & rhs) const;

        // Outer product (vector product).
        Vector3 operator ^ (const Vector3 & rhs) const;

        // Typecast to Vector3<U>
        template <typename U>
        operator Vector3<U> () const;
    };
}
\end{lstlisting}
%%%%%%%%%%%%%%%%%%%%%%%%%%%%%
\subsubsubsection{Constructor}
\mbox{}
%%%%%%%%%%%%%%%%%%%%%%%%%%%%%
%%%%%%%%%%%%%%%%%%%%%%%%%%%%%
\begin{screen}
\begin{verbatim}
template<typename T>
PS::Vector3<T>::Vector3()
\end{verbatim}
\end{screen}

\begin{itemize}

\item{{\bf Argument}}

  None.

\item{{\bf Feature}}

Default constructor. The member variables \texttt{x}, \texttt{y} and \texttt{z} are initialized as 0.
%デフォルトコンストラクタ。メンバx,yは0で初期化される。

\end{itemize}

%%%%%%%%%%%%%%%%%%%%%%%%%%%%%
\begin{screen}
\begin{verbatim}
template<typename T>
PS::Vector3<T>::Vector3(const T _x, const T _y, const T _z)
\end{verbatim}
\end{screen}

\begin{itemize}

\item{{\bf Argument}}

\texttt{\_x}: Input. Type \texttt{const T}.
%{\_x}: 入力。{const T}型。

\texttt{\_y}: Input. Type \texttt{const T}.
%{\_y}: 入力。{const T}型。

\texttt{\_z}: Input. Type \texttt{const T}.
%{\_z}: 入力。{const T}型。

\item{{\bf Feature}}

Values of members \texttt{x}, \texttt{y} and \texttt{z} are set to values of arguments \texttt{\_x}, \texttt{\_y} and \texttt{\_z}, respectively.
%メンバ{x}、{y}をそれぞれ{\_x}、{\_y}で初期化する。

\end{itemize}

%%%%%%%%%%%%%%%%%%%%%%%%%%%%%
\begin{screen}
\begin{verbatim}
template<typename T>
PS::Vector3<T>::Vector3(const T s);
\end{verbatim}
\end{screen}

\begin{itemize}

\item{{\bf Argument}}

\texttt{s}: Input. Type \texttt{const T}.
%{s}: 入力。{const T}型。

\item{{\bf Feature}}

Values of members \texttt{x}, \texttt{y} and \texttt{z} are set to the value of argument \texttt{s}.
%メンバ{x}、{y}を両方とも{s}の値で初期化する。

\end{itemize}





%%%%%%%%%%%%%%%%%%%%%%%%%%%%%
\subsubsubsection{Copy constructor}
\mbox{}
%%%%%%%%%%%%%%%%%%%%%%%%%%%%%

%%%%%%%%%%%%%%%%%%%%%%%%%%%%%
\begin{screen}
\begin{verbatim}
template<typename T>
PS::Vector3<T>::Vector3(const PS::Vector3<T> & src)
\end{verbatim}
\end{screen}

\begin{itemize}

\item{{\bf Argument}}

\texttt{src}: Input. Type \texttt{const PS::Vector3$<$T$>$ \&}.
%{src}: 入力。{const PS::Vector3$<$T$>$ \&}型。

\item{{\bf Feature}}

Copy constructor. The new variable will have the same value as \texttt{src}.
%コピーコンストラクタ。{src}で初期化する。

\end{itemize}


%%%%%%%%%%%%%%%%%%%%%%%%%%%%%
\subsubsubsection{Member variables}

\begin{screen}
\begin{verbatim}
template<typename T>
T PS::Vector3<T>::x;

template<typename T>
T PS::Vector3<T>::y;

template<typename T>
T PS::Vector3<T>::z;
\end{verbatim}
\end{screen}

\begin{itemize}
  
\item{{\bf function}}
  
  Member variables, {x}, {y} and {z} can be directly handled.
  
\end{itemize}


%%%%%%%%%%%%%%%%%%%%%%%%%%%%%
\subsubsubsection{Assignment operator}
\mbox{}
%%%%%%%%%%%%%%%%%%%%%%%%%%%%%

%%%%%%%%%%%%%%%%%%%%%%%%%%%%%
\begin{screen}
\begin{verbatim}
template<typename T>
const PS::Vector3<T> & PS::Vector3<T>::operator = 
                       (const PS::Vector3<T> & rhs);
\end{verbatim}
\end{screen}

\begin{itemize}

\item{{\bf Argument}}

\texttt{rhs}: Input. Type \texttt{const PS::Vector3$<$T$>$ \&}.
%{rhs}: 入力。{const PS::Vector3$<$T$>$ \&}型。

\item{{\bf Return value}}
Type \texttt{const PS::Vector3$<$T$>$ \&}. Assigns the values of components of \texttt{rhs}
to its components, and returns the vector itself.
%%{const PS::Vector3$<$T$>$ \&}型。{rhs}のx,yの値を自身のメンバx,yに代
%%入し自身の参照を返す。代入演算子。

\end{itemize}


\subsubsubsection{[] operators}

\begin{screen}
\begin{verbatim}
template<typename T>
const T & PS::Vector3<T>::operator[]
                       (const int i);
\end{verbatim}
\end{screen}

\begin{itemize}

\item{{\bf Argument}}

\texttt{i}: input. Type \texttt{const int}.

\item{{\bf Return value}}

Type \texttt{const $<$T$>$ \&}. It returns $i$th-component of the vector.

\item{{\bf Remark}}

This operator would be slower than the direct access of the member variables.

\end{itemize}

\begin{screen}
\begin{verbatim}
template<typename T>
T & PS::Vector3<T>::operator[]
                       (const int i);
\end{verbatim}
\end{screen}

\begin{itemize}

\item{{\bf Argument}}

\texttt{i}: input. Type \texttt{const int}.

\item{{\bf Return value}}

Type \texttt{$<$T$>$ \&}. It returns $i$th-component of the vector.

\item{{\bf Remark}}

This operator would be slower than the direct access of the member variables.

\end{itemize}

%%%%%%%%%%%%%%%%%%%%%%%%%%%%%
\subsubsubsection{Addition and subtraction}
\mbox{}
%%%%%%%%%%%%%%%%%%%%%%%%%%%%%

%%%%%%%%%%%%%%%%%%%%%%%%%%%%%
\begin{screen}
\begin{verbatim}
template<typename T>
PS::Vector3<T> PS::Vector3<T>::operator + 
               (const PS::Vector3<T> & rhs) const;
\end{verbatim}
\end{screen}

\begin{itemize}

\item{{\bf Argument}}

\texttt{rhs}: Input. Type \texttt{const PS::Vector3$<$T$>$ \&}.
%{rhs}: 入力。{const PS::Vector3$<$T$>$ \&}型。

\item{{\bf Return value}}

Type \texttt{PS::Vector3$<$T$>$}. Add the components of \texttt{rhs} and its
components, and return the results.
%%{PS::Vector3$<$T$>$}型。{rhs}のx,yの値と自身のメンバx,yの値の和を取っ
%%た値を返す。

\end{itemize}


%%%%%%%%%%%%%%%%%%%%%%%%%%%%%
\begin{screen}
\begin{verbatim}
template<typename T>
const PS::Vector3<T> & PS::Vector3<T>::operator += 
                       (const PS::Vector3<T> & rhs);
\end{verbatim}
\end{screen}

\begin{itemize}

\item{{\bf Argument}}

\texttt{rhs}: Input. Type \texttt{const PS::Vector3$<$T$>$ \&}.
%{rhs}: 入力。{const PS::Vector3$<$T$>$ \&}型。

\item{{\bf Return value}}

Type \texttt{const PS::Vector3$<$T$>$ \&}. Add the components of \texttt{rhs} to its
components, and return itself (lhs is changed).
%%{const PS::Vector3$<$T$>$ \&}型。{rhs}のx,yの値を自身のメンバx,yに足
%%し、自身を返す。

\end{itemize}


%%%%%%%%%%%%%%%%%%%%%%%%%%%%%
\begin{screen}
\begin{verbatim}
template<typename T>
PS::Vector3<T> PS::Vector3<T>::operator - 
               (const PS::Vector3<T> & rhs) const;
\end{verbatim}
\end{screen}

\begin{itemize}

\item{{\bf 引数}}

\texttt{rhs}: Input. Type \texttt{const PS::Vector3$<$T$>$ \&}.
%{rhs}: 入力。{const PS::Vector3$<$T$>$ \&}型。

\item{{\bf Return value}}

Type \texttt{PS::Vector3$<$T$>$}. Subtract the components of \texttt{rhs} from its
components, and return the results.
%%{PS::Vector3$<$T$>$}型。{rhs}のx,yの値と自身のメンバx,yの値の差を取っ
%%た値を返す。

\end{itemize}


%%%%%%%%%%%%%%%%%%%%%%%%%%%%%
\begin{screen}
\begin{verbatim}
template<typename T>
const PS::Vector3<T> & PS::Vector3<T>::operator -= 
                       (const PS::Vector3<T> & rhs);
\end{verbatim}
\end{screen}

\begin{itemize}

\item{{\bf Argument}}

\texttt{rhs}: Input. Type \texttt{const PS::Vector3$<$T$>$ \&}.
%{rhs}: 入力。{const PS::Vector3$<$T$>$ \&}型。

\item{{\bf Return value}}

Type \texttt{const PS::Vector3$<$T$>$ \&}. Subtract the components of \texttt{rhs} from
its components, and return itself (lhs is changed).
%%{const PS::Vector3$<$T$>$ \&}型。自身のメンバx,yから{rhs}のx,yを引き
%%自身を返す。

\end{itemize}

%%%%%%%%%%%%%%%%%%%%%%%%%%%%%
\subsubsubsection{Scalar multiplication}
\mbox{}
%%%%%%%%%%%%%%%%%%%%%%%%%%%%%

%%%%%%%%%%%%%%%%%%%%%%%%%%%%%
\begin{screen}
\begin{verbatim}
template<typename T>
PS::Vector3<T> PS::Vector3<T>::operator * (const T s) const;
\end{verbatim}
\end{screen}

\begin{itemize}

\item{{\bf Argument}}

\texttt{s}: Input. Type \texttt{const T}.
%{s}: 入力。{const T}型。

\item{{\bf Return value}}

Type \texttt{PS::Vector3$<$T$>$}. Multiply its components by \texttt{s}, and return
the results.
%%{PS::Vector3$<$T$>$}型。自身のメンバx,yそれぞれに{s}をかけた値を返す。

\end{itemize}


%%%%%%%%%%%%%%%%%%%%%%%%%%%%%
\begin{screen}
\begin{verbatim}
template<typename T>
const PS::Vector3<T> & PS::Vector3<T>::operator *= (const T s);
\end{verbatim}
\end{screen}

\begin{itemize}

\item{{\bf Argument}}

\texttt{rhs}: Input. Type \texttt{const T}.
%{rhs}: 入力。{const T}型。

\item{{\bf Return value}}

Type \texttt{const PS::Vector3$<$T$>$ \&}. Multiply its components by \texttt{s},
and return itself (lhs is changed).
%%{const PS::Vector3$<$T$>$ \&}型。自身のメンバx,yそれぞれに{s}をかけ自
%%身を返す。

\end{itemize}


%%%%%%%%%%%%%%%%%%%%%%%%%%%%%
\begin{screen}
\begin{verbatim}
template<typename T>
PS::Vector3<T> PS::Vector3<T>::operator / (const T s) const;
\end{verbatim}
\end{screen}

\begin{itemize}

\item{{\bf Argument}}

\texttt{s}: Input. Type \texttt{const T}.
%{s}: 入力。{const T}型。

\item{{\bf Return value}}

Type \texttt{PS::Vector3$<$T$>$}. Divide its components by \texttt{s}, and return
the results.
%%{PS::Vector3$<$T$>$}型。自身のメンバx,yそれぞれを{s}で割った値を返す。

\end{itemize}


%%%%%%%%%%%%%%%%%%%%%%%%%%%%%
\begin{screen}
\begin{verbatim}
template<typename T>
const PS::Vector3<T> & PS::Vector3<T>::operator /= (const T s);
\end{verbatim}
\end{screen}

\begin{itemize}

\item{{\bf Argument}}

\texttt{rhs}: Input. Type \texttt{const T}.
%{rhs}: 入力。{const T}型。

\item{{\bf Return value}}

Type \texttt{const PS::Vector3$<$T$>$ \&}. Divide its components by \texttt{s}, and
return itself (lhs is changed).
%%{const PS::Vector3$<$T$>$ \&}型。自身のメンバx,yそれぞれを{s}で割り自
%%身を返す。

\end{itemize}


%%%%%%%%%%%%%%%%%%%%%%%%%%%%%
\subsubsubsection{Inner and outer product}
\mbox{}
%%%%%%%%%%%%%%%%%%%%%%%%%%%%%

%%%%%%%%%%%%%%%%%%%%%%%%%%%%%
\begin{screen}
\begin{verbatim}
template<typename T>
T PS::Vector3<T>::operator * (const PS::Vector3<T> & rhs) const;
\end{verbatim}
\end{screen}

\begin{itemize}

\item{{\bf Argument}}

\texttt{rhs}: Input. Type \texttt{const PS::Vector3$<$T$>$ \&}.
%{rhs}: 入力。{const PS::Vector3$<$T$>$ \&}型。

\item{{\bf Return value}}

Type \texttt{T}. Take inner product of it and \texttt{rhs}, and return the result.
%{T}型。自身と{rhs}の内積を取った値を返す。

\end{itemize}

%%%%%%%%%%%%%%%%%%%%%%%%%%%%%
\begin{screen}
\begin{verbatim}
template<typename T>
T PS::Vector3<T>::operator ^ (const PS::Vector3<T> & rhs) const;
\end{verbatim}
\end{screen}

\begin{itemize}

\item{{\bf Argument}}

\texttt{rhs}: Input. Type \texttt{const PS::Vector3$<$T$>$ \&}.
%{rhs}: 入力。{const PS::Vector3$<$T$>$ \&}型。

\item{{\bf Return value}}

Type \texttt{T}. Take outer product of it and \texttt{rhs}, and return the results.
%{T}型。自身と{rhs}の外積を取った値を返す。

\end{itemize}


%%%%%%%%%%%%%%%%%%%%%%%%%%%%%
\subsubsubsection{Typecast to Vector3$<$U$>$}
%\subsubsubsection{{Vector3$<$U$>$}への型変換}
\mbox{}
%%%%%%%%%%%%%%%%%%%%%%%%%%%%%

%%%%%%%%%%%%%%%%%%%%%%%%%%%%%
\begin{screen}
\begin{verbatim}
template<typename T>
template <typename U>
PS::Vector3<T>::operator PS::Vector3<U> () const;
\end{verbatim}
\end{screen}

\begin{itemize}

\item{{\bf Argument}}

  None.

\item{{\bf Return value}}

  Type \texttt{const PS::Vector3$<$U$>$}.
%  {const PS::Vector3$<$U$>$}型。

\item{{\bf Feature}}

  Typecast from type \texttt{const PS::Vector3$<$T$>$} to \texttt{const
  PS::Vector3$<$U$>$}.
%%  {const PS::Vector3$<$T$>$}型を{const PS::Vector3$<$U$>$}型にキャ ス
%%  トする。

\end{itemize}


\subsubsection{Wrappers}

The wrappers of vector types are defined as follows.
%ベクトル型のラッパーの定義を以下に示す。
\begin{lstlisting}[caption=vectorwrapper]
namespace ParticleSimulator{
    typedef Vector2<F32> F32vec2;
    typedef Vector3<F32> F32vec3;
    typedef Vector2<F64> F64vec2;
    typedef Vector3<F64> F64vec3;
#ifdef PARTICLE_SIMULATOR_TWO_DIMENSION
    typedef F32vec2 F32vec;
    typedef F64vec2 F64vec;
#else
    typedef F32vec3 F32vec;
    typedef F64vec3 F64vec;
#endif
}
\end{lstlisting}

PS::F32vec2, PS::F32vec3, PS::F64vec2, and PS::F64vec3 are,
respectively, 2D vector in single precision, 3D vector in single
presicion, 2D vector in double precision, and 3D vector in double
precision. If users set 2D (3D) coordinate system, PS::F32vec and
PS::F64vec is wrappers of PS::F32vec2 and PS::F64vec2 (PS::F32vec3 and
PS::F64vec3).
%%すなわちPS::F32vec2, PS::F32vec3, PS::F64vec2, PS::F64vec3はそれぞれ
%%単精度2次元ベクトル、倍精度2次元ベクトル、単精度3次元ベクトル、倍精度
%%3次元ベクトルである。FDPSで扱う空間座標系を2次元とした場合、
%%PS::F32vecとPS::F64vecはそれぞれ単精度2次元ベクトル、倍精度2次元ベク
%%トルとなる。一方、FDPSで扱う空間座標系を3次元とした場合、PS::F32vecと
%%PS::F64vecはそれぞれ単精度3次元ベクトル、倍精度3次元ベクトルとなる。





\subsection{Orthotope type}
\label{sec:datatype_orthotope}
\subsubsection{Abstract}

The orthotope types are \texttt{PS::Orthotope2} (rectangle) and
\texttt{PS::Orthotope3} (cuboid). We describe these orthotope types first.
These are template classes, which can take basic datatypes
as \texttt{F32} or \texttt{F64} as template arguments. We then present
wrappers for these orthotope types.


\subsubsection{PS::Orthotope2}

\texttt{PS::Orthotope2}は\texttt{PS::Vector2}型のメンバ変数\texttt{low\_}, \texttt{high\_}を持つ。
これらに対する様々なAPIや演算子を定義した。それらの宣言を以下に記述する。
この節ではこれらについて詳しく記述する。
\begin{lstlisting}[caption=Orthotope2]
namespace ParticleSimulator{
    template<class T>
    class Orthotope2{
    public:
        Vector2<T> low_;
        Vector2<T> high_;

        Orthotope2(): low_(9999.9), high_(-9999.9){}
        
        Orthotope2(const Vector2<T> & _low, const Vector2<T> & _high)
            : low_(_low), high_(_high){}
        
        Orthotope2(const Orthotope2 & src) : low_(src.low_), high_(src.high_){}

        Orthotope2(const Vector2<T> & center, const T length) :
            low_(center-(Vector2<T>)(length)), high_(center+(Vector2<T>)(length)) {
        }

        void initNegativeVolume(){
            low_ = std::numeric_limits<float>::max() / 128;
            high_ = -low_;
        }

        void init(){
            initNegativeVolume();
        }

        void merge( const Orthotope2 & ort ){
            this->high_.x = ( this->high_.x > ort.high_.x ) ? this->high_.x : ort.high_.x;
            this->high_.y = ( this->high_.y > ort.high_.y ) ? this->high_.y : ort.high_.y;
            this->low_.x = ( this->low_.x <= ort.low_.x ) ? this->low_.x : ort.low_.x;
            this->low_.y = ( this->low_.y <= ort.low_.y ) ? this->low_.y : ort.low_.y;
        }

        void merge( const Vector2<T> & vec ){
            this->high_.x = ( this->high_.x > vec.x ) ? this->high_.x : vec.x;
            this->high_.y = ( this->high_.y > vec.y ) ? this->high_.y : vec.y;
            this->low_.x = ( this->low_.x <= vec.x ) ? this->low_.x : vec.x;
            this->low_.y = ( this->low_.y <= vec.y ) ? this->low_.y : vec.y;
        }

        void merge( const Vector2<T> & vec, const T size){
            this->high_.x = ( this->high_.x > vec.x + size ) ? this->high_.x : vec.x + size;
            this->high_.y = ( this->high_.y > vec.y + size ) ? this->high_.y : vec.y + size;
            this->low_.x = ( this->low_.x <= vec.x - size) ? this->low_.x : vec.x - size;
            this->low_.y = ( this->low_.y <= vec.y - size) ? this->low_.y : vec.y - size;
        }
    };
}
namespace PS = ParticleSimulator;
\end{lstlisting}

%%%%%%%%%%%%%%%%%%%%%%%%%%%%%
\subsubsubsection{メンバ変数}

\begin{screen}
\begin{verbatim}
template<typename T>
PS::Vector2<T> PS::Orthotope2<T>::low_;

template<typename T>
PS::Vector2<T> PS::Orthotope2<T>::high_;
\end{verbatim}
\end{screen}

\begin{itemize}
  
\item{{\bf 機能}}
  
メンバ変数\texttt{low\_}、\texttt{high\_}を直接操作出来る。
  
\end{itemize}


%%%%%%%%%%%%%%%%%%%%%%%%%%%%%%%%
\subsubsubsection{コンストラクタ}
%---------------
\begin{screen}
\begin{verbatim}
template<typename T>
PS::Orthotope2<T>::Orthotope2();
\end{verbatim}
\end{screen}

\begin{itemize}

\item{{\bf 引数}}

なし。ただし、テンプレート引数Tは \texttt{PS::F32} または \texttt{PS::F64} でなければならない。

\item{{\bf 機能}}

デフォルトコンストラクタ。
メンバ変数\texttt{low\_}, \texttt{high\_}は、それぞれ、(9999.9, 9999.9), (-9999.9, -9999.9)で初期化される。

\end{itemize}
%---------------
\begin{screen}
\begin{verbatim}
template<typename T>
PS::Orthotope2<T>::Orthotope2(const Vector2<T> _low, const Vector2<T> _high);
\end{verbatim}
\end{screen}

\begin{itemize}

\item{{\bf 引数}}

\texttt{\_low}: 入力。\texttt{const Vector2$<$T$>$}型。

\texttt{\_high}: 入力。\texttt{const Vector2$<$T$>$}型。

ここで、\texttt{T} は \texttt{PS::F32} または \texttt{PS::F64} でなければならない。

\item{{\bf 機能}}

メンバ変数\texttt{low\_}、\texttt{high\_}を、それぞれ\texttt{\_low}、\texttt{\_high}で初期化する。

\end{itemize}
%---------------
\begin{screen}
\begin{verbatim}
template<typename T>
PS::Orthotope2<T>::Orthotope2(const Vector2<T> & center, const T length);
\end{verbatim}
\end{screen}

\begin{itemize}

\item{{\bf 引数}}

\texttt{center}: 入力。\texttt{const Vector2$<$T$>$ \&}

\texttt{length}: 入力。\texttt{const T}型。

\item{{\bf 機能}}

メンバ変数\texttt{low\_}、\texttt{high\_}を、それぞれ、\texttt{center-(Vector2$<$T$>$)(length)}、\texttt{center+(Vector2$<$T$>$)(length)}で初期化する。

\end{itemize}


%%%%%%%%%%%%%%%%%%%%%%%%%%%%%%%%
\subsubsubsection{コピーコンストラクタ}
%---------------
\begin{screen}
\begin{verbatim}
template<typename T>
PS::Orthotope2<T>::Orthotope2(const Orthotope2<T> & src);
\end{verbatim}
\end{screen}

\begin{itemize}

\item{{\bf 引数}}

\texttt{src}: 入力。\texttt{const Orthotope2$<$T$>$ \&}型。

\item{{\bf 機能}}

メンバ変数\texttt{low\_}、\texttt{high\_}を、それぞれ、\texttt{src.low\_}、\texttt{src.high\_}で初期化する。

\end{itemize}


%%%%%%%%%%%%%%%%%%%%%%%%%%%%%
\subsubsubsection{初期化}
%---------------
\begin{screen}
\begin{verbatim}
template<typename T>
PS::Orthotope2<T>::initNegativeVolume();
\end{verbatim}
\end{screen}

\begin{itemize}

\item{{\bf 引数}}

なし。

\item{{\bf 機能}}

メンバ変数\texttt{low\_}、\texttt{high\_}を、それぞれ、($a$, $a$)、(-$a$, -$a$)で初期化する。
ここで、$a$=\texttt{std::numeric\_limits$<$float$>$::max() / 128}である。

\end{itemize}

%---------------
\begin{screen}
\begin{verbatim}
template<typename T>
PS::Orthotope2<T>::init();
\end{verbatim}
\end{screen}

\begin{itemize}

\item{{\bf 引数}}

なし。

\item{{\bf 機能}}

メンバ関数 \texttt{initNegativeVolume} と同じ機能を提供する。

\end{itemize}

%%%%%%%%%%%%%%%%%%%%%%%%%%%%%
\subsubsubsection{結合操作}
%---------------
\begin{screen}
\begin{verbatim}
template<typename T>
void PS::Orthotope2<T>::merge( const Orthotope2 & ort )();
\end{verbatim}
\end{screen}

\begin{itemize}

\item{{\bf 引数}}

\texttt{ort}: 入力。\texttt{const Orthotope2 \&}型。

\item{{\bf 機能}}

メンバ変数\texttt{low\_}、\texttt{high\_}を、当該長方形と長方形\texttt{ort}を包含する最小の長方形を記述するように更新する。

\end{itemize}
%---------------
\begin{screen}
\begin{verbatim}
template<typename T>
void PS::Orthotope2<T>::merge( const Vector2<T> & vec )();
\end{verbatim}
\end{screen}

\begin{itemize}

\item{{\bf 引数}}

\texttt{vec}: 入力。\texttt{const Vector2$<$T$>$ \&}型。

\item{{\bf 機能}}

メンバ変数\texttt{low\_}、\texttt{high\_}を、点\texttt{vec}を包含するように更新する。

\end{itemize}
%---------------
\begin{screen}
\begin{verbatim}
template<typename T>
void PS::Orthotope2<T>::merge( const Vector2<T> & vec, const T size )();
\end{verbatim}
\end{screen}

\begin{itemize}

\item{{\bf 引数}}

\texttt{vec}: 入力。\texttt{const Vector2$<$T$>$ \&}型。

\texttt{size}: 入力。\texttt{const T}型。

\item{{\bf 機能}}

メンバ変数\texttt{low\_}、\texttt{high\_}を、中心\texttt{vec}、半径\texttt{size}の円を包含するように更新する。

\end{itemize}



\subsubsection{PS::Orthotope3}

\texttt{PS::Orthotope3} has two components: \texttt{low\_} and \texttt{high\_}, both of which are \texttt{PS::Vector3} class.
We define various APIs and operators for these components.
In the following, we describe them.

\begin{lstlisting}[caption=Orthotope3]
namespace ParticleSimulator{
    template<class T>
    class Orthotope3{
    public:
        Vector3<T> low_;
        Vector3<T> high_;

        Orthotope3(): low_(9999.9), high_(-9999.9){}

        Orthotope3(const Vector3<T> & _low, const Vector3<T> & _high)
            : low_(_low), high_(_high) {}

        Orthotope3(const Orthotope3 & src) : low_(src.low_), high_(src.high_){}

        Orthotope3(const Vector3<T> & center, const T length) :
            low_(center-(Vector3<T>)(length)), high_(center+(Vector3<T>)(length)) {
        }

        void initNegativeVolume(){
            low_ = std::numeric_limits<float>::max() / 128;
            high_ = -low_;
        }

        void init(){
            initNegativeVolume();
        }

        void merge( const Orthotope3 & ort ){
            this->high_.x = ( this->high_.x > ort.high_.x ) ? this->high_.x : ort.high_.x;
            this->high_.y = ( this->high_.y > ort.high_.y ) ? this->high_.y : ort.high_.y;
            this->high_.z = ( this->high_.z > ort.high_.z ) ? this->high_.z : ort.high_.z;
            this->low_.x = ( this->low_.x <= ort.low_.x ) ? this->low_.x : ort.low_.x;
            this->low_.y = ( this->low_.y <= ort.low_.y ) ? this->low_.y : ort.low_.y;
            this->low_.z = ( this->low_.z <= ort.low_.z ) ? this->low_.z : ort.low_.z;
        }

        void merge( const Vector3<T> & vec ){
            this->high_.x = ( this->high_.x > vec.x ) ? this->high_.x : vec.x;
            this->high_.y = ( this->high_.y > vec.y ) ? this->high_.y : vec.y;
            this->high_.z = ( this->high_.z > vec.z ) ? this->high_.z : vec.z;
            this->low_.x = ( this->low_.x <= vec.x ) ? this->low_.x : vec.x;
            this->low_.y = ( this->low_.y <= vec.y ) ? this->low_.y : vec.y;
            this->low_.z = ( this->low_.z <= vec.z ) ? this->low_.z : vec.z;
        }

        void merge( const Vector3<T> & vec, const T size){
            this->high_.x = ( this->high_.x > vec.x + size ) ? this->high_.x : vec.x + size;
            this->high_.y = ( this->high_.y > vec.y + size ) ? this->high_.y : vec.y + size;
            this->high_.z = ( this->high_.z > vec.z + size) ? this->high_.z : vec.z + size;
            this->low_.x = ( this->low_.x <= vec.x - size) ? this->low_.x : vec.x - size;
            this->low_.y = ( this->low_.y <= vec.y - size) ? this->low_.y : vec.y - size;
            this->low_.z = ( this->low_.z <= vec.z - size) ? this->low_.z : vec.z - size;
        }
    };
}
namespace PS = ParticleSimulator;
\end{lstlisting}

%%%%%%%%%%%%%%%%%%%%%%%%%%%%%
\subsubsubsection{Member variables}

\begin{screen}
\begin{verbatim}
template<typename T>
PS::Vector3<T> PS::Orthotope3<T>::low_;

template<typename T>
PS::Vector3<T> PS::Orthotope3<T>::high_;
\end{verbatim}
\end{screen}

\begin{itemize}
  
\item{{\bf Feature}}
  
Member variables, \texttt{low\_} and \texttt{high\_} can be directly handled.
  
\end{itemize}


%%%%%%%%%%%%%%%%%%%%%%%%%%%%%%%%
\subsubsubsection{Constructors}
%---------------
\begin{screen}
\begin{verbatim}
template<typename T>
PS::Orthotope3<T>::Orthotope3();
\end{verbatim}
\end{screen}

\begin{itemize}

\item{{\bf Arguments}}

None. But, the template argument \texttt{T} must be either \texttt{PS::F32} or \texttt{PS::F64}.

\item{{\bf Feature}}

Default constructor.
Member variables \texttt{low\_} and \texttt{high\_} are initialized by (9999.9, 9999.9, 9999.9) and (-9999.9, -9999.9, -9999.9), respectively.

\end{itemize}
%---------------
\begin{screen}
\begin{verbatim}
template<typename T>
PS::Orthotope3<T>::Orthotope3(const Vector3<T> _low, const Vector3<T> _high);
\end{verbatim}
\end{screen}

\begin{itemize}

\item{{\bf Arguments}}

\texttt{\_low}: Input. Type \texttt{const Vector3$<$T$>$}.

\texttt{\_high}: Input. Type \texttt{const Vector3$<$T$>$}.

where \texttt{T} must be either \texttt{PS::F32} or \texttt{PS::F64}.

\item{{\bf Feature}}

Member variables \texttt{low\_} and \texttt{high\_} are initialized by \texttt{\_low} and \texttt{\_high}, respectively.

\end{itemize}
%---------------
\begin{screen}
\begin{verbatim}
template<typename T>
PS::Orthotope3<T>::Orthotope3(const Vector3<T> & center, const T length);
\end{verbatim}
\end{screen}

\begin{itemize}

\item{{\bf Arguments}}

\texttt{center}: Input. Type \texttt{const Vector3$<$T$>$ \&}.
\texttt{length}: Input. Type \texttt{const T}.

\item{{\bf Feature}}

Member variables \texttt{low\_} and \texttt{high\_} are initialized by \texttt{center-(Vector3$<$T$>$)(length)} and \texttt{center+(Vector3$<$T$>$)(length)}, respectively.

\end{itemize}


%%%%%%%%%%%%%%%%%%%%%%%%%%%%%%%%
\subsubsubsection{Copy constructor}
%---------------
\begin{screen}
\begin{verbatim}
template<typename T>
PS::Orthotope3<T>::Orthotope3(const Orthotope3<T> & src);
\end{verbatim}
\end{screen}

\begin{itemize}

\item{{\bf Argument}}

\texttt{src}: Input. Type \texttt{const Orthotope3$<$T$>$ \&}.

\item{{\bf Feature}}

Member variables \texttt{low\_} and \texttt{high\_} are initialized by \texttt{src.low\_} and \texttt{src.high\_}, respectively.

\end{itemize}


%%%%%%%%%%%%%%%%%%%%%%%%%%%%%
\subsubsubsection{Initialize}
%---------------
\begin{screen}
\begin{verbatim}
template<typename T>
PS::Orthotope3<T>::initNegativeVolume();
\end{verbatim}
\end{screen}

\begin{itemize}

\item{{\bf Arguments}}

None.

\item{{\bf Feature}}

Member variables \texttt{low\_} and \texttt{high\_} are initialized by ($a$, $a$, $a$) and (-$a$, -$a$, -$a$), respectively,
where $a$=\texttt{std::numeric\_limits$<$float$>$::max() / 128}.

\end{itemize}

%---------------
\begin{screen}
\begin{verbatim}
template<typename T>
PS::Orthotope3<T>::init();
\end{verbatim}
\end{screen}

\begin{itemize}

\item{{\bf Arguments}}

None.

\item{{\bf Feature}}

This member function is the same as \texttt{initNegativeVolume}.

\end{itemize}

%%%%%%%%%%%%%%%%%%%%%%%%%%%%%
\subsubsubsection{Merge operations}
%---------------
\begin{screen}
\begin{verbatim}
template<typename T>
void PS::Orthotope3<T>::merge( const Orthotope3 & ort )();
\end{verbatim}
\end{screen}

\begin{itemize}

\item{{\bf Arguments}}

\texttt{ort}: Input. Type \texttt{const Orthotope3 \&}.

\item{{\bf Feature}}

Member variables \texttt{low\_} and \texttt{high\_} are updated so that they represents the smallest cuboid that contains both the original cuboid and cuboid \texttt{ort}.

\end{itemize}
%---------------
\begin{screen}
\begin{verbatim}
template<typename T>
void PS::Orthotope3<T>::merge( const Vector3<T> & vec )();
\end{verbatim}
\end{screen}

\begin{itemize}

\item{{\bf Arguments}}

\texttt{vec}: Input. Type \texttt{const Vector3$<$T$>$ \&}.

\item{{\bf Feature}}

Member variables \texttt{low\_} and \texttt{high\_} are updated so that they represents the smallest cuboid that contains the point \texttt{vec}.

\end{itemize}
%---------------
\begin{screen}
\begin{verbatim}
template<typename T>
void PS::Orthotope3<T>::merge( const Vector3<T> & vec, const T size )();
\end{verbatim}
\end{screen}

\begin{itemize}

\item{{\bf Arguments}}

\texttt{vec}: Input. Type \texttt{const Vector3$<$T$>$ \&}.
\texttt{size}: Input. Type \texttt{const T}.

\item{{\bf Feature}}

Member variables \texttt{low\_} and {high\_} are updated so that they represents the smallest cuboid that contains a sphere whose center is \texttt{vec} and radius is \texttt{size}.

\end{itemize}



\subsubsection{Wrappers}

オルソトープ型のラッパーの定義を以下に示す。
\begin{lstlisting}[caption=Orthotope Wrapper]
namespace ParticleSimulator{
    typedef Orthotope2<F32> F32ort2;
    typedef Orthotope3<F32> F32ort3;
    typedef Orthotope2<F64> F64ort2;
    typedef Orthotope3<F64> F64ort3;
#ifdef PARTICLE_SIMULATOR_TWO_DIMENSION
    typedef F32ort2 F32ort;
    typedef F64ort2 F64ort;
#else
    typedef F32ort3 F32ort;
    typedef F64ort3 F64ort;
#endif
}
\end{lstlisting}

すなわちPS::F32ort2, PS::F32ort3, PS::F64ort2, PS::F64ort3はそれぞれ単
精度2次元オルソトープ、倍精度2次元オルソトープ、単精度3次元オルソトープ、
倍精度3次元オルソトープである。FDPSで扱う空間座標系を2次元とした場合、
PS::F32ortとPS::F64ortはそれぞれ単精度2次元オルソトープ、倍精度2次元オルソトープとなる。
一方、FDPSで扱う空間座標系を3次元とした場合、PS::F32ortとPS::F64ortは
それぞれ単精度3次元オルソトープ、倍精度3次元オルソトープとなる。






\subsection{Symmetric matrix type}

\subsubsection{Abstract}

Symmetric matrix types are \texttt{PS::MatrixSym2} (2x2 matrix) and
\texttt{PS::MatrixSym3} (3x3 matrix). We describe these matrix types
first and then, we present wrappers for these matrix types.

%%対称行列型には2x2対称行列型PS::MatrixSym2と3x3対称行列型
%%PS::MatrixSym3がある。まずこれら2つを記述する。最後にこれら対称行列
%%型のラッパーについて記述する。

\subsubsection{PS::MatrixSym2}

PS::MatrixSym2はxx, yy, xyの3要素を持つ。これらに対する様々なAPIや演算
子を定義した。それらの宣言を以下に記述する。この節ではこれらについて詳
しく記述する。
\begin{lstlisting}[caption=MatrixSym2]
namespace ParticleSimulator{
    template<class T>
    class MatrixSym2{
    public:
        // メンバ変数3要素
        T xx, yy, xy;

        // コンストラクタ
        MatrixSym2() : xx(T(0)), yy(T(0)), xy(T(0)) {}
        MatrixSym2(const T _xx, const T _yy, const T _xy)
            : xx(_xx), yy(_yy), xy(_xy) {}
        MatrixSym2(const T s) : xx(s), yy(s), xy(s){}
        MatrixSym2(const MatrixSym2 & src) : xx(src.xx), yy(src.yy), xy(src.xy) {}

        // 代入演算子
        const MatrixSym2 & operator = (const MatrixSym2 & rhs);

        // 加減算
        MatrixSym2 operator + (const MatrixSym2 & rhs) const;
        const MatrixSym2 & operator += (const MatrixSym2 & rhs) const;
        MatrixSym2 operator - (const MatrixSym2 & rhs) const;
        const MatrixSym2 & operator -= (const MatrixSym2 & rhs) const;

        // トレースの計算
        T getTrace() const;

        // MatrixSym2<U>への型変換
        template <typename U>
        operator MatrixSym2<U> () const;
    }
}
namespace PS = ParticleSimulator;
\end{lstlisting}

%%%%%%%%%%%%%%%%%%%%%%%%%%%%%%%%%%%%%%%%%%%%%%%%%%%%%
\subsubsubsection{コンストラクタ}

\begin{screen}
\begin{verbatim}
template<typename T>
PS::MatrixSym2<T>::MatrixSym2();
\end{verbatim}
\end{screen}

\begin{itemize}

\item{{\bf 引数}}

なし。

\item{{\bf 機能}}

デフォルトコンストラクタ。メンバxx,yy,xyは0で初期化される。

\end{itemize}

\begin{screen}
\begin{verbatim}
template<typename T>
PS::MatrixSym2<T>::MatrixSym2
            (const T _xx,
             const T _yy,
             const T _xy);
\end{verbatim}
\end{screen}

\begin{itemize}

\item{{\bf 引数}}

{\_xx}: 入力。{const T}型。

{\_yy}: 入力。{const T}型。

{\_xy}: 入力。{const T}型。

\item{{\bf 機能}}

メンバ{xx}、{yy}、{xy}をそれぞれ{\_xx}、{\_yy}、
{\_xy}で初期化する。

\end{itemize}

\begin{screen}
\begin{verbatim}
template<typename T>
PS::MatrixSym2<T>::MatrixSym2(const T s);
\end{verbatim}
\end{screen}

\begin{itemize}

\item{{\bf 引数}}

{s}: 入力。{const T}型。

\item{{\bf 機能}}

メンバ{xx}、{yy}、{xy}すべてを{s}の値で初期化する。

\end{itemize}

%%%%%%%%%%%%%%%%%%%%%%%%%%%%%%%%%%%%%%%%%%%%%%%%%%%%%
\subsubsubsection{コピーコンストラクタ}

\begin{screen}
\begin{verbatim}
template<typename T>
PS::MatrixSym2<T>::MatrixSym2(const PS::MatrixSym2<T> & src)
\end{verbatim}
\end{screen}

\begin{itemize}

\item{{\bf 引数}}

{src}: 入力。{const PS::MatrixSym2$<$T$>$ \&}型。

\item{{\bf 機能}}

コピーコンストラクタ。{src}で初期化する。

\end{itemize}

%%%%%%%%%%%%%%%%%%%%%%%%%%%%%%%%%%%%%%%%%%%%%%%%%%%%%
\subsubsubsection{代入演算子}

\begin{screen}
\begin{verbatim}
template<typename T>
const PS::MatrixSym2<T> & PS::MatrixSym2<T>::operator = 
                       (const PS::MatrixSym2<T> & rhs);
\end{verbatim}
\end{screen}

\begin{itemize}

\item{{\bf 引数}}

{rhs}: 入力。{const PS::MatrixSym2$<$T$>$ \&}型。

\item{{\bf 返り値}}

{const PS::MatrixSym2$<$T$>$ \&}型。{rhs}のxx,yy,xyの値を自身のメ
ンバxx,yy,xyに代入し自身の参照を返す。代入演算子。

\end{itemize}

%%%%%%%%%%%%%%%%%%%%%%%%%%%%%%%%%%%%%%%%%%%%%%%%%%%%%
\subsubsubsection{加減算}

\begin{screen}
\begin{verbatim}
template<typename T>
PS::MatrixSym2<T> PS::MatrixSym2<T>::operator + 
               (const PS::MatrixSym2<T> & rhs) const;
\end{verbatim}
\end{screen}

\begin{itemize}

\item{{\bf 引数}}

{rhs}: 入力。{const PS::MatrixSym2$<$T$>$ \&}型。

\item{{\bf 返り値}}

{PS::MatrixSym2$<$T$>$}型。{rhs}のxx,yy,xyの値と自身のメンバ
xx,yy,xyの値の和を取った値を返す。

\end{itemize}

\begin{screen}
\begin{verbatim}
template<typename T>
const PS::MatrixSym2<T> & PS::MatrixSym2<T>::operator += 
                       (const PS::MatrixSym2<T> & rhs);
\end{verbatim}
\end{screen}

\begin{itemize}

\item{{\bf 引数}}

{rhs}: 入力。{const PS::MatrixSym2$<$T$>$ \&}型。

\item{{\bf 返り値}}

{const PS::MatrixSym2$<$T$>$ \&}型。{rhs}のxx,yy,xyの値を自身のメ
ンバxx,yy,xyに足し、自身を返す。

\end{itemize}

\begin{screen}
\begin{verbatim}
template<typename T>
PS::MatrixSym2<T> PS::MatrixSym2<T>::operator - 
               (const PS::MatrixSym2<T> & rhs) const;
\end{verbatim}
\end{screen}

\begin{itemize}

\item{{\bf 引数}}

{rhs}: 入力。{const PS::MatrixSym2$<$T$>$ \&}型。

\item{{\bf 返り値}}

{PS::MatrixSym2$<$T$>$}型。{rhs}のxx,yy,xyの値と自身のメンバ
xx,yy,xyの値の差を取った値を返す。

\end{itemize}

\begin{screen}
\begin{verbatim}
template<typename T>
const PS::MatrixSym2<T> & PS::MatrixSym2<T>::operator -= 
                       (const PS::MatrixSym2<T> & rhs);
\end{verbatim}
\end{screen}

\begin{itemize}

\item{{\bf 引数}}

{rhs}: 入力。{const PS::MatrixSym2$<$T$>$ \&}型。

\item{{\bf 返り値}}

{const PS::MatrixSym2$<$T$>$ \&}型。自身のメンバxx,yy,xyから{rhs}
のxx,yy,xyを引き自身を返す。

\end{itemize}

%%%%%%%%%%%%%%%%%%%%%%%%%%%%%%%%%%%%%%%%%%%%%%%%%%%%%
\subsubsubsection{トレースの計算}

\begin{screen}
\begin{verbatim}
template<typename T>
T PS::MatrixSym2<T>::getTrace() const;
\end{verbatim}
\end{screen}

\begin{itemize}

\item{{\bf 引数}}

なし

\item{{\bf 返り値}}

{T}型。

\item{{\bf 機能}}

  トレースを計算し、その結果を返す。

\end{itemize}

%%%%%%%%%%%%%%%%%%%%%%%%%%%%%%%%%%%%%%%%%%%%%%%%%%%%%
\subsubsubsection{{MatrixSym2$<$U$>$}への型変換}

\begin{screen}
\begin{verbatim}
template<typename T>
template<typename U>
PS::MatrixSym2<T>::operator PS::MatrixSym2<U> () const;
\end{verbatim}
\end{screen}

\begin{itemize}

\item{{\bf 引数}}

  なし。

\item{{\bf 返り値}}

{const PS::MatrixSym2$<$U$>$}型。

\item{{\bf 機能}}

  {const PS::MatrixSym2$<$T$>$}型を{const PS::MatrixSym2$<$U$>$}型にキャ
  ストする

\end{itemize}



\subsubsection{PS::MatrixSym3}

\texttt{PS::MatrixSym3} has six components: \texttt{xx}, \texttt{yy}, \texttt{zz}, \texttt{xy}, \texttt{xz}, and \texttt{yz}.
We define various APIs and operators for these components.
In the following, we list them.
%%PS::MatrixSym3はxx, yy, zz, xy, xz, yzの6要素を持つ。これらに対する様々
%%なAPIや演算子を定義した。それらの宣言を以下に記述する。この節ではこれ
%%らについて詳しく記述する。
\begin{lstlisting}[caption=MatrixSym3]
namespace ParticleSimulator{
    template<class T>
    class MatrixSym3{
    public:
        // Six member variables
        T xx, yy, zz, xy, xz, yz;

        // Constructors
        MatrixSym3() : xx(T(0)), yy(T(0)), zz(T(0)),
                       xy(T(0)), xz(T(0)), yz(T(0)) {}
        MatrixSym3(const T _xx, const T _yy, const T _zz,
                   const T _xy, const T _xz, const T _yz )
                       : xx(_xx), yy(_yy), zz(_zz),
                       xy(_xy), xz(_xz), yz(_yz) {}
        MatrixSym3(const T s) : xx(s), yy(s), zz(s),
                                xy(s), xz(s), yz(s) {}
        MatrixSym3(const MatrixSym3 & src) :
            xx(src.xx), yy(src.yy), zz(src.zz),
            xy(src.xy), xz(src.xz), yz(src.yz) {}

        // Assignment operator
        const MatrixSym3 & operator = (const MatrixSym3 & rhs);

        // Addition and subtraction
        MatrixSym3 operator + (const MatrixSym3 & rhs) const;
        const MatrixSym3 & operator += (const MatrixSym3 & rhs) const;
        MatrixSym3 operator - (const MatrixSym3 & rhs) const;
        const MatrixSym3 & operator -= (const MatrixSym3 & rhs) const;

        // Trace
        T getTrace() const;

        // Typecast to MatrixSym3<U>
        template <typename U>
        operator MatrixSym3<U> () const;
    }
}
namespace PS = ParticleSimulator;
\end{lstlisting}

%%%%%%%%%%%%%%%%%%%%%%%%%%%%%%%%%%%%%%%%%%%%%%%%%%%%%
\subsubsubsection{Constructor}

\begin{screen}
\begin{verbatim}
template<typename T>
PS::MatrixSym3<T>::MatrixSym3();
\end{verbatim}
\end{screen}

\begin{itemize}

\item{{\bf Argument}}

  None.

\item{{\bf Feature}}

Default constructor. All the member variables are initialized as 0.
%デフォルトコンストラクタ。6要素は0で初期化される。

\end{itemize}

\begin{screen}
\begin{verbatim}
template<typename T>
PS::MatrixSym3<T>::MatrixSym3(const T _xx,
                              const T _yy,
                              const T _zz,
                              const T _xy,
                              const T _xz,
                              const T _yz);
\end{verbatim}
\end{screen}

\begin{itemize}

\item{{\bf Argument}}

\texttt{\_xx}: Input. Type \texttt{const T}.
%%{\_xx}: 入力。{const T}型。

\texttt{\_yy}: Input. Type \texttt{const T}.
%{\_yy}: 入力。{const T}型。

\texttt{\_zz}: Input. Type \texttt{const T}.
%{\_zz}: 入力。{const T}型。

\texttt{\_xy}: Input. Type \texttt{const T}.
%{\_xy}: 入力。{const T}型。

\texttt{\_xz}: Input. Type \texttt{const T}.
%{\_xz}: 入力。{const T}型。

\texttt{\_yz}: Input. Type \texttt{const T}.
%{\_yz}: 入力。{const T}型。

\item{{\bf Feature}}

Values of members \texttt{xx}, \texttt{yy}, \texttt{zz}, \texttt{xy},
\texttt{xz} and \texttt{yz} are set to values of arguments \texttt{\_xx},
\texttt{\_yy}, \texttt{\_zz}, \texttt{\_xy}, \texttt{xz} and \texttt{\_yz},
respectively.
%%メンバ{xx}、{yy}、{zz}、{xy}、{xz}、{yz}をそれぞれ{\_xx}、{\_yy}、
%%{\_zz}、{\_xy}、{\_xz}、{\_yz}で初期化する。

\end{itemize}

\begin{screen}
\begin{verbatim}
template<typename T>
PS::MatrixSym3<T>::MatrixSym3(const T s);
\end{verbatim}
\end{screen}

\begin{itemize}

\item{{\bf Argument}}

\texttt{s}: Input. Type \texttt{const T}.
%{s}: 入力。{const T}型。

\item{{\bf Feature}}

Values of members are set to the value of argument \texttt{s}.
%6要素すべてを{s}の値で初期化する。

\end{itemize}

%%%%%%%%%%%%%%%%%%%%%%%%%%%%%%%%%%%%%%%%%%%%%%%%%%%%%
\subsubsubsection{Copy constructor}

\begin{screen}
\begin{verbatim}
template<typename T>
PS::MatrixSym3<T>::MatrixSym3(const PS::MatrixSym3<T> & src)
\end{verbatim}
\end{screen}

\begin{itemize}

\item{{\bf Argument}}

\texttt{src}: Input. Type const \texttt{PS::MatrixSym3$<$T$>$ \&}.
%{src}: 入力。{const PS::MatrixSym3$<$T$>$ \&}型。

\item{{\bf Feature}}

Copy constructor. The new variable will have the same value as \texttt{src}.
%コピーコンストラクタ。{src}で初期化する。

\end{itemize}

%%%%%%%%%%%%%%%%%%%%%%%%%%%%%%%%%%%%%%%%%%%%%%%%%%%%%
\subsubsubsection{Assignment operator}

\begin{screen}
\begin{verbatim}
template<typename T>
const PS::MatrixSym3<T> & PS::MatrixSym3<T>::operator = 
                       (const PS::MatrixSym3<T> & rhs);
\end{verbatim}
\end{screen}

\begin{itemize}

\item{{\bf Argument}}

\texttt{rhs}: Input. Type \texttt{const PS::MatrixSym3$<$T$>$ \&}.
%{rhs}: 入力。{const PS::MatrixSym3$<$T$>$ \&}型。

\item{{\bf Return value}}

Type \texttt{const PS::MatrixSym3$<$T$>$ \&}. Assigns the values of components of \texttt{rhs}
to its components, and returns the vector itself.
%%{const PS::MatrixSym3$<$T$>$ \&}型。{rhs}の6要素それぞれの値を自身の
%%6要素それぞれに代入し自身の参照を返す。代入演算子。

\end{itemize}

%%%%%%%%%%%%%%%%%%%%%%%%%%%%%%%%%%%%%%%%%%%%%%%%%%%%%
\subsubsubsection{Addition and subtraction}

\begin{screen}
\begin{verbatim}
template<typename T>
PS::MatrixSym3<T> PS::MatrixSym3<T>::operator + 
               (const PS::MatrixSym3<T> & rhs) const;
\end{verbatim}
\end{screen}

\begin{itemize}

\item{{\bf Argument}}

\texttt{rhs}: Input. Type \texttt{const PS::MatrixSym3$<$T$>$ \&}.
%{rhs}: 入力。{const PS::MatrixSym3$<$T$>$ \&}型。

\item{{\bf Return value}}

Type \texttt{PS::MatrixSym3$<$T$>$}. Add the components of \texttt{rhs} and its
components, and return the results.
%%{PS::MatrixSym3$<$T$>$ }型。{rhs}の6要素それぞれの値と自身の6要素の
%%値の和を取った値を返す。

\end{itemize}

\begin{screen}
\begin{verbatim}
template<typename T>
const PS::MatrixSym3<T> & PS::MatrixSym3<T>::operator += 
                       (const PS::MatrixSym3<T> & rhs);
\end{verbatim}
\end{screen}

\begin{itemize}

\item{{\bf Argument}}

\texttt{rhs}: Input. Type \texttt{const PS::MatrixSym3$<$T$>$ \&}.
%{rhs}: 入力。{const PS::MatrixSym3$<$T$>$ \&}型。

\item{{\bf Return value}}

Type \texttt{const PS::MatrixSym3$<$T$>$ \&}. Add the components of \texttt{rhs} to its
components, and return itself (lhs is changed).
%%{const PS::MatrixSym3$<$T$>$ \&}型。{rhs}の6要素それぞれの値を自身の
%%6要素それぞれに足し、自身を返す。

\end{itemize}

\begin{screen}
\begin{verbatim}
template<typename T>
PS::MatrixSym3<T> PS::MatrixSym3<T>::operator - 
               (const PS::MatrixSym3<T> & rhs) const;
\end{verbatim}
\end{screen}

\begin{itemize}

\item{{\bf Argument}}

\texttt{rhs}: Input. Type \texttt{const PS::MatrixSym3$<$T$>$ \&}.
%{rhs}: 入力。{const PS::MatrixSym3$<$T$>$ \&}型。

\item{{\bf Return value}}

Type \texttt{PS::MatrixSym3$<$T$>$}. Subtract the components of \texttt{rhs} from its
components, and return the results.
%%{PS::MatrixSym3$<$T$>$}型。{rhs}の6要素それぞれの値と自身の6要素そ
%%れぞれの値の差を取った値を返す。

\end{itemize}

\begin{screen}
\begin{verbatim}
template<typename T>
const PS::MatrixSym3<T> & PS::MatrixSym3<T>::operator -= 
                       (const PS::MatrixSym3<T> & rhs);
\end{verbatim}
\end{screen}

\begin{itemize}

\item{{\bf Argument}}

\texttt{rhs}: Input. Type \texttt{const PS::MatrixSym3$<$T$>$ \&}.
%{rhs}: 入力。{const PS::MatrixSym3$<$T$>$ \&}型。

\item{{\bf Return value}}

Type \texttt{const PS::MatrixSym3$<$T$>$ \&}. Subtract the components of \texttt{rhs} from
its components, and return itself (lhs is changed).
%%{const PS::MatrixSym3$<$T$>$ \&}型。自身の6要素それぞれから{rhs}の6
%%要素それぞれを引き自身を返す。

\end{itemize}

%%%%%%%%%%%%%%%%%%%%%%%%%%%%%%%%%%%%%%%%%%%%%%%%%%%%%
\subsubsubsection{Trace}

\begin{screen}
\begin{verbatim}
template<typename T>
T PS::MatrixSym3<T>::getTrace() const;
\end{verbatim}
\end{screen}

\begin{itemize}

\item{{\bf Argument}}

  None.

\item{{\bf Return value}}

  Type \texttt{T}.

\item{{\bf Feature}}

  Calculate the trace, and return the result.
%  トレースを計算し、その結果を返す。

\end{itemize}

%%%%%%%%%%%%%%%%%%%%%%%%%%%%%%%%%%%%%%%%%%%%%%%%%%%%%
\subsubsubsection{Typecast to MatrixSym3$<$U$>$}

\begin{screen}
\begin{verbatim}
template<typename T>
template<typename U>
PS::MatrixSym3<T>::operator PS::MatrixSym3<U> () const;
\end{verbatim}
\end{screen}

\begin{itemize}

\item{{\bf Argument}}

  None.

\item{{\bf Return value}}

  Type \texttt{const PS::MatrixSym3$<$U$>$}.

%{const PS::MatrixSym3$<$U$>$}型。

\item{{\bf Feature}}

  Typecast from type \texttt{const PS::MatrixSym3$<$T$>$} to \texttt{const
  PS::MatrixSym3$<$U$>$}.
%%  {const PS::MatrixSym3$<$T$>$}型を{const PS::MatrixSym3$<$U$>$}型に
%%  キャ ストする

\end{itemize}



\subsubsection{Wrappers}

The wrappers of symmetric matrix types are defined as follows.
%対称行列型のラッパーの定義を以下に示す。
\begin{lstlisting}[caption=matrixsymwrapper]
namespace ParticleSimulator{
    typedef MatrixSym2<F32> F32mat2;
    typedef MatrixSym3<F32> F32mat3;
    typedef MatrixSym2<F64> F64mat2;
    typedef MatrixSym3<F64> F64mat3;
#ifdef PARTICLE_SIMULATOR_TWO_DIMENSION
    typedef F32mat2 F32mat;
    typedef F64mat2 F64mat;
#else
    typedef F32mat3 F32mat;
    typedef F64mat3 F64mat;
#endif
}
namespace PS = ParticleSimulator;
\end{lstlisting}

\texttt{PS::F32mat2}, \texttt{PS::F32mat3}, \texttt{PS::F64mat2}, and \texttt{PS::F64mat3} are,
respectively, 2x2 symmetric matrix in single precision, 3x3 symmetric
matrix in single presicion, 2x2 symmetric matrix in double precision,
and 3x3 symmetric matrix in double precision. If users set 2D (3D)
coordinate system, \texttt{PS::F32mat} and \texttt{PS::F64mat} is wrappers of
\texttt{PS::F32mat2} and \texttt{PS::F64mat2} (\texttt{PS::F32mat3} and \texttt{PS::F64mat3}).
%%すなわちPS::F32mat2, PS::F32mat3, PS::F64mat2, PS::F64mat3はそれぞれ
%%単精度2x2対称行列、倍精度2x2対称行列、単精度3x3対称行列、倍精度3x3対
%%称行列である。FDPSで扱う空間座標系を2次元とした場合、PS::F32matと
%%PS::F64matはそれぞれ単精度2x2対称行列、倍精度2x2対称行列となる。一方、
%%FDPSで扱う空間座標系を3次元とした場合、PS::F32matとPS::F64matはそれぞ
%%れ単精度3x3対称行列、倍精度3x3対称行列となる。





\subsection{PS::SEARCH\_MODE type}

\subsubsection{Summary}

In this section, we describe data type \texttt{PS::SEARCH\_MODE}.
It is used only as template arguments of class \texttt{PS::TreeForForce}.
This data type determines the interaction mode of the class.
\texttt{PS::SEARCH\_MODE} can take the following values:
\begin{itemize}[leftmargin=*,itemsep=-1ex]
\item PS::SEARCH\_MODE\_LONG
\item PS::SEARCH\_MODE\_LONG\_CUTOFF
\item PS::SEARCH\_MODE\_GATHER
\item PS::SEARCH\_MODE\_SCATTER
\item PS::SEARCH\_MODE\_SYMMETRY
\item PS::SEARCH\_MODE\_LONG\_SCATTER
\item PS::SEARCH\_MODE\_LONG\_SYMMETRY
\item PS::SEARCH\_MODE\_LONG\_CUTOFF\_SCATTER
\end{itemize}

Each of them corresponds to a mode for interaction calculation.
In the following, we describe them.

\subsubsection{PS::SEARCH\_MODE\_LONG}

This type is used when a group of distant particles is regarded as a
superparticle as in the standard Barnes-Hut tree code.
This type is for gravitational force and Coulomb's force under open boundary condition (Not usable in periodic boundary condition).

\subsubsection{PS::SEARCH\_MODE\_LONG\_CUTOFF}

This type is used when a group of distant particles is regarded as a
superparticle, and when its force does not reach to infinity.  This
type is for gravitational force and Coulomb's force under
periodic boundary condition.

\subsubsection{PS::SEARCH\_MODE\_GATHER}

This type is used when its force decays to zero at a finite distance,
and when the distance is determined by the size of $i$-particle.

\subsubsection{PS::SEARCH\_MODE\_SCATTER}

This type is used when its force decays to zero at a finite distance,
and when the distance is determined by the size of $j$-particle.

\subsubsection{PS::SEARCH\_MODE\_SYMMETRY}

This type is used when its force decays to zero at a finite distance,
and when the distance is determined by the larger of the sizes of $i$-
and $j$-particles.

\subsubsection{PS::SEARCH\_MODE\_LONG\_SCATTER}

Almost the same as {\tt SEARCH\_MODE\_LONG}, but if the distance
between $i$- and $j$-particles is smaller than the search radius of
$j$-particle, the $j$-particle is not included in superparticle.


\subsubsection{PS::SEARCH\_MODE\_LONG\_SYMMETRY}

Almost the same as {\tt SEARCH\_MODE\_LONG}, but if the distance
between $i$- and $j$-particles is smaller than the larger of the search radii of
$i$- and $j$-particle, the $j$-particle is not included in superparticle.


\subsubsection{PS::SEARCH\_MODE\_LONG\_CUTOFF\_SCATTER}

Not implemented yet.

%未実装。


\subsection{Enumerated type}
\label{sec:datatype_enum}

\subsubsection{Summary}

In this section, we describe enumerated types defined in FDPS.
Currently, there is just one datatype. We describe it below.
%%本節ではFDPSで定義されている列挙型について記述する。列挙型には
%%BOUNDARY\_CONDITION型が存在する。以下、各列挙型について記述する。

\subsubsection{PS::BOUNDARY\_CONDITION type}
\label{sec:datatype_enum_boundarycondition}

\subsubsubsection{Summary}

Type BOUNDARY\_CONDITION specifies boundary conditions. The definition
is as follows.
%%BOUNDARY\_CONDITION型は境界条件を指定するためのデータ型である。これは
%%以下のように定義されている。
\begin{lstlisting}[caption=boundarycondition]
namespace ParticleSimulator{
    enum BOUNDARY_CONDITION{
        BOUNDARY_CONDITION_OPEN,
        BOUNDARY_CONDITION_PERIODIC_X,
        BOUNDARY_CONDITION_PERIODIC_Y,
        BOUNDARY_CONDITION_PERIODIC_Z,
        BOUNDARY_CONDITION_PERIODIC_XY,
        BOUNDARY_CONDITION_PERIODIC_XZ,
        BOUNDARY_CONDITION_PERIODIC_YZ,
        BOUNDARY_CONDITION_PERIODIC_XYZ,
        BOUNDARY_CONDITION_SHEARING_BOX,
        BOUNDARY_CONDITION_USER_DEFINED,
    };
}
\end{lstlisting}

We explain each value below.
%以下にどの変数がどの境界条件に対応するかを記述する。

\subsubsubsection{PS::BOUNDARY\_CONDITION\_OPEN}

This specifies the open boundary condition.
%開放境界となる。

\subsubsubsection{PS::BOUNDARY\_CONDITION\_PERIODIC\_X}

This specifies the periodic boundary condition in the direction of x-axis,
and open boundary condition in other directions. The interval is
left-bounded and right-unbounded. This is true for all periodic
boundary conditions.
%%x軸方向のみ周期境界、その他の軸方向は開放境界となる。周期の境界の下限
%%は閉境界、上限は開境界となっている。この境界の規定はすべての軸方向に
%%あてはまる。

\subsubsubsection{PS::BOUNDARY\_CONDITION\_PERIODIC\_Y}

This specifies the periodic boundary condition in the direction of y-axis,
and open boundary condition in other directions.
%y軸方向のみ周期境界、その他の軸方向は開放境界となる。

\subsubsubsection{PS::BOUNDARY\_CONDITION\_PERIODIC\_Z}

This specifies the periodic boundary condition in the direction of z-axis,
and open boundary condition in other directions.
%z軸方向のみ周期境界、その他の軸方向は開放境界となる。

\subsubsubsection{PS::BOUNDARY\_CONDITION\_PERIODIC\_XY}

This specifies the periodic boundary condition in the directions of x- and
y-axes, and open boundary condition in the direction of z-axis.
%x, y軸方向のみ周期境界、その他の軸方向は開放境界となる。

\subsubsubsection{PS::BOUNDARY\_CONDITION\_PERIODIC\_XZ}

This specifies the periodic boundary condition in the directions of x- and
z-axes, and open boundary condition in the direction of y-axis.
%x, z軸方向のみ周期境界、その他の軸方向は開放境界となる。

\subsubsubsection{PS::BOUNDARY\_CONDITION\_PERIODIC\_YZ}

This specifies the periodic boundary condition in the directions of y-
and z-axes, and open boundary condition in the direction of x-axis.
%y, z軸方向のみ周期境界、その他の軸方向は開放境界となる。

\subsubsubsection{PS::BOUNDARY\_CONDITION\_PERIODIC\_XYZ}

This specifies the periodic boundary condition in all three directions.
%x, y, z軸方向すべてが周期境界となる。

\subsubsubsection{PS::BOUNDARY\_CONDITION\_SHEARING\_BOX}

Not implemented yet.
%未実装。

\subsubsubsection{PS::BOUNDARY\_CONDITION\_USER\_DEFINED}

Not implemented yet.
%未実装。

%%%%%%%%%%%%%%%%%%%%%%%%%%%%%%%%%%%
\subsubsection{PS::INTERACTION\_LIST\_MODE type}
\label{sec:datatype_enum_interaction_list_mode}

%\subsubsubsection{概要}
\subsubsubsection{Summary}


Type INTERACTION\_LIST\_MODE is used to determine if user program
reuse the interaction list or not. This type is defined as follows.


%INTERACTION\_LIST\_MODE型は相互作用リストを再利用するかどうかを決定す
%るためのデータ型である。これは以下のように定義されている。

\begin{lstlisting}[caption=interaction\_list\_mode]
namespace ParticleSimulator{
    enum INTERACTION_LIST_MODE{
        MAKE_LIST,
        MAKE_LIST_FOR_REUSE,
        REUSE_LIST,
    };
}
\end{lstlisting}

This data type is used as the last argument of the function
calcForceAllAndWriteBack(). For more detail, plese see
section \ref{sec:treeForForceHighLevelAPI}.

%このデータ型はcalcForceAllAndWriteBack()等の関数の引数として使われる
%(詳しくはセクション\ref{sec:treeForForceHighLevelAPI}を参照)。

\subsubsubsection{PS::MAKE\_LIST}

FDPS (re)makes interaction lists for each interaction calculation
(each call of the APIs described above). In this case, we cannot reuse
interaction list in the next interaction calculation because FDPS does
not store the information of interaction list. \textbf{This is the
default operation mode in FDPS}.


%% 相互作用リストを毎回作り相互作用計算を行う場合に用いる。FDPS は相互作
%% 用リストを記憶しないため、相互作用リストの再利用はできない。これはデフォ
%% ルト動作である。


\subsubsubsection{PS::MAKE\_LIST\_FOR\_REUSE}

FDPS (re)makes interaction lists and stores them internally. Then, it
performs interaction calculation. In this case, we can reuse these
interaction lists in the next interaction calculation if we call the
APIs with the flag PS::REUSE\_LIST. The interaction lists memorized in
FDPS are destroyed if we perform the interaction calculation with the
flags PS::MAKE\_LIST\_FOR\_REUSE or PS::MAKE\_LIST.\\

%相互作用リストを再利用し相互作用計算を行いたい場合に用いる。このオプショ
%ンを選択する事でFDPSは相互作用リストを作りそれを保持する。作成した相互
%作用リストはPS::MAKE\_LIST\_FOR\_REUSEもしくはPS::MAKE\_LISTを用いて相
%互作用計算を行った際に破棄される。

\subsubsubsection{PS::REUSE\_INTERACTION\_LIST}

FDPS performs interaction calculation using the previously-created
interaction lists, which are the lists that are created at the
previous call of the APIs with the flag PS::MAKE\_LIST\_FOR\_REUSE. In
this case, moment information in superparticles are automatically
updated using the latest particle information.\\

%% 相互作用リストを再利用し相互作用計算を行う。再利用される相互作用リスト
%% は PS::MAKE\_LIST\_FOR\_REUSE を選択時に作成した相互作用リストである。
%% この時、超粒子のモーメント情報は自動的にアップデートされる。

%%%%%%%%%%%%%%%%%%%%%%%%%%%%%%%%%%%
\subsubsection{PS::EXCHANGE\_LET\_MODE tyoe}
\label{sec:datatype_enum_exchange_let_mode}

\subsubsubsection{Summary}


EXCHANGE\_LET\_MODE type is used to determine the algorithm for LET
exchange. It is defined as follows. 

%% EXCHANGE\_LET\_MODE型はLET交換の方法を決定するためのデータ型である。こ
%% れは以下のように定義されている。

\begin{lstlisting}[caption=EXCHANGE\_LET\_MODE]
namespace ParticleSimulator{
    enum EXCHANGE_LET_MODE{
        EXCHANGE_LET_A2A,
        EXCHANGE_LET_P2P_EXACT,
        EXCHANGE_LET_P2P_FAST,
    };
}
\end{lstlisting}


This data type is used as an argument for {\tt PS::TreeForForce::setExchangeLet}.
(See section \ref{sec:module_standard_treeforce_setexchangeletmode} for details).


%% このデータ型は{\tt PS::TreeForForce::setExchangeLet}の引数として使われる.
%% (詳しくはセクション\ref{sec:treeForForceInitializeAPI}を参照).

\subsubsubsection{PS::EXCHANGE\_LET\_A2A}

Use {\tt MPI\_Alltoall}.
%% LETの交換に{\tt MPI\_Alltoall}を使用する。
\subsubsubsection{PS::EXCHANGE\_LET\_P2P\_EXACT}

Do not use {\tt MPI\_Alltoall} but use {\tt MPI\_Allgather} and {\tt
MPI\_Isend/recv}. On machines where {\tt MPI\_Alltoall} is not very
efficient, or if the total number of processes is very large, this
algorithm can be faster. The result should be identical to that for
the case of {\tt PS::EXCHANGE\_LET\_A2A} up to the roundoff error.

%% LETの交換に{\tt MPI\_Alltoall}を使用せず、{\tt MPI\_Allgather}と{\tt
%% MPI\_Isend/recv}を使用する。{\tt MPI\_Alltoall}が効率的に動かない計算
%% 機では、こちらの方が速い場合がある。結果は丸め誤差の範囲で
%% PS::EXCHANGE\_LET\_A2Aを用いた場合と一致する。

\subsubsubsection{EXCHANGE\_LET\_P2P\_FAST}

Do not use {\tt MPI\_Alltoall} but use {\tt MPI\_Allgather} and {\tt
MPI\_Isend/recv}. The result can be slightly different from that for
the case of PS::EXCHANGE\_LET\_A2A, but can be faster than
the case of {\tt PS::EXCHANGE\_LET\_P2P\_EXACT}


%% LETの交換に{\tt MPI\_Alltoall}を使用せず、{\tt MPI\_Allgather}と{\tt
%% MPI\_Isend/recv}を使用する。結果は{\tt PS::EXCHANGE\_LET\_P2P\_EXACT}
%% を用いた場合と異なるが、より通信量が減っているため、高速に動作する可能
%% 性がある。



%%%%%%%%%%%%%%%%%%%%%%%%%%%%%%%%%%%
\subsubsection{PS::CALC\_DISTANCE\_TYPE type}
\label{sec:datatype_enum_calc_distance_type}

\subsubsubsection{Summary}

The {\tt CALC\_DISTANCE\_TYPE} type is used to determine the way the
distances between particles are calculated.

%% {\tt CALC\_DISTANCE\_TYPE}型は粒子間の距離の計算方法を変更するための型
%% である.

\begin{lstlisting}[caption=CALC\_DISTANCE\_TYPE]
namespace ParticleSimulator{
    enum CALC_DISTANCE_TYPE{
        CALC_DISTANCE_TYPE_NORMAL = 0,
        CALC_DISTANCE_TYPE_NEAREST_X = 1,
	CALC_DISTANCE_TYPE_NEAREST_Y = 2,
        CALC_DISTANCE_TYPE_NEAREST_XY = 3,
	CALC_DISTANCE_TYPE_NEAREST_Z = 4,
	CALC_DISTANCE_TYPE_NEAREST_XZ = 5,
	CALC_DISTANCE_TYPE_NEAREST_YZ = 6,
	CALC_DISTANCE_TYPE_NEAREST_XYZ = 7,
    };
}
\end{lstlisting}

This data type is used as the template parameter for {\tt PS::TreeForForce}.
(See \ref{sec:module_standard_treeforce_object}).

%% このデータ型は{\tt PS::TreeForForce}のテンプレート引数として与えられる.
%% (セクション\ref{sec:module_standard_treeforce_object}を参照).

\subsubsubsection{CALC\_DISTANCE\_TYPE\_NORMAL}

Calculate the L2 norm. This is the defaultbehavior.

%% 領域内の粒子のL2ノルムを計算する.特に指定のない場合は、この値が使われる.


\subsubsubsection{CALC\_DISTANCE\_TYPE\_NEAREST\_X}

Assume the periodic boundary for the x coordinate, even when the
boundary condition is open.

%% x方向に周期境界だとした場合に最も近い粒子とのL2ノルムを計算する.FDPS
%% のx方向の境界条件が開放境界の場合でも、最近接粒子との距離を計算する。

\subsubsubsection{CALC\_DISTANCE\_TYPE\_NEAREST\_Y}

Assume the periodic boundary for the y coordinate, even when the
boundary condition is open.


%% Y方向に周期境界だとした場合に最も近い粒子とのL2ノルムを計算する.FDPS
%% のy方向の境界条件が開放境界の場合でも、最近接粒子との距離を計算する。

\subsubsubsection{CALC\_DISTANCE\_TYPE\_NEAREST\_Z}

Assume the periodic boundary for the z coordinate, even when the
boundary condition is open.

% Z方向に周期境界だとした場合に最も近い粒子とのL2ノルムを計算する.FDPS
% のz方向の境界条件が開放境界の場合でも、最近接粒子との距離を計算する。

\subsubsubsection{CALC\_DISTANCE\_TYPE\_NEAREST\_XY}

Assume the periodic boundary for the x and y coordinates, even when the
boundary condition is open.

%% X,Y方向に周期境界だとした場合に最も近い粒子とのL2ノルムを計算する.
%% FDPSのx,y方向の境界条件が開放境界の場合でも、最近接粒子との距離を計算
%% する。

\subsubsubsection{CALC\_DISTANCE\_TYPE\_NEAREST\_XZ}

Assume the periodic boundary for the x and z coordinates, even when the
boundary condition is open.

%% X,Z方向に周期境界だとした場合に最も近い粒子とのL2ノルムを計算する.
%% FDPSのx,z方向の境界条件が開放境界の場合でも、最近接粒子との距離を計算す
%% る。

\subsubsubsection{CALC\_DISTANCE\_TYPE\_NEAREST\_YZ}

Assume the periodic boundary for the y and z coordinates, even when the
boundary condition is open.

%% Y,Z方向に周期境界だとした場合に最も近い粒子とのL2ノルムを計算する.
%% FDPSのy,z方向の境界条件が開放境界の場合でも、最近接粒子との距離を計算
%% する。

\subsubsubsection{CALC\_DISTANCE\_TYPE\_NEAREST\_XYZ}

Assume the periodic boundary for the x, y and z coordinates, even when the
boundary condition is open.

%% X,Y,Z方向に周期境界だとした場合に最も近い粒子とのL2ノルムを計算する.
%% FDPSのx,y,z方向の境界条件が開放境界の場合でも、最近接粒子との距離を計
%% 算する。


\subsection{PS::TimeProfile type}
\label{sec:datatype_timeprofile}

\subsubsection{概要}

本節では、PS::TimeProfile型について記述する。PS::TimeProfile型はFDPSで
使われる3つのクラス、領域分割クラス、粒子群クラス、相互作用ツリークラ
ス、各メソッドの計算時間を格納するクラスである。これら3つのクラスには
{\tt PS::TimeProfile getTimeProfile()}というメソッドが存在し、このメソッ
ドをつかって、ユーザーは各メソッドの計算時間を取得出来る。

このクラスは以下のように記述されている。

\begin{lstlisting}[caption=TimeProfile]
namespace ParticleSimulator{
    class TimeProfile{
    publid:
        F64 collect_sample_particle;
        F64 decompose_domain;
        F64 exchange_particle;
        F64 make_local_tree;
        F64 make_global_tree;
        F64 calc_force;
        F64 calc_moment_local_tree;
        F64 calc_moment_global_tree;
        F64 make_LET_1st;
        F64 make_LET_2nd;        
        F64 exchange_LET_1st;
        F64 exchange_LET_2nd;
    };
}    
\end{lstlisting}

%%%%%%%%%%%%%%%%%%%%%%%%%%%%%
\subsubsubsection{加算}
\mbox{}
%%%%%%%%%%%%%%%%%%%%%%%%%%%%%

%%%%%%%%%%%%%%%%%%%%%%%%%%%%%
\begin{screen}
\begin{verbatim}
PS::TimeProfile PS::TimeProfile::operator + 
               (const PS::TimeProfile & rhs) const;
\end{verbatim}
\end{screen}

\begin{itemize}

\item{{\bf 引数}}

{rhs}: 入力。{const TimeProfile \&}型。

\item{{\bf 返り値}}

{PS::TimeProfile}型。{rhs}のすべてのメンバ変数の値と自身のメンバ変数の
値の和を取った値を返す。

\end{itemize}

%%%%%%%%%%%%%%%%%%%%%%%%%%%%%
\subsubsubsection{縮約}
\mbox{}
%%%%%%%%%%%%%%%%%%%%%%%%%%%%%

%%%%%%%%%%%%%%%%%%%%%%%%%%%%%
\begin{screen}
\begin{verbatim}
PS::F64 PS::TimeProfile::getTotalTime() const;
\end{verbatim}
\end{screen}

\begin{itemize}

\item{{\bf 引数}}

なし。

\item{{\bf 返り値}}

{PS::F64}型。すべてのメンバ変数の値の和を返す。

\end{itemize}

%%%%%%%%%%%%%%%%%%%%%%%%%%%%%
\subsubsubsection{初期化}
\mbox{}
%%%%%%%%%%%%%%%%%%%%%%%%%%%%%

%%%%%%%%%%%%%%%%%%%%%%%%%%%%%
\begin{screen}
\begin{verbatim}
void PS::TimeProfile::clear();
\end{verbatim}
\end{screen}

\begin{itemize}

\item{{\bf 引数}}

なし。

\item{{\bf 返り値}}

なし。

\item{{\bf 機能}}

すべてのメンバ変数に0を代入する。

\end{itemize}


%%\subsection{MPIデータ型}

%%\input{datatype_mpidata.tex}

