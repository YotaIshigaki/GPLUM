\subsubsection{Summary}

%EssentialParticleIクラスは相互作用の計算に必要なi粒子の情報を持つクラ
%スであり、相互作用の定義(節\ref{sec:overview_action}の手順0)に必要とな
%る。EssentialParticleIクラスはFullParticleクラス
%(節\ref{sec:fullparticle})のサブセットである。FDPSは、このクラスのデー
%タにアクセスする必要がある。そのため、EssentialParticleIクラスはいくつ
%かのメンバ関数を持つ必要がある。以下、この節の前提、常に必要なメンバ関
%数と、場合によっては必要なメンバ関数について記述する。
The \texttt{EssentialParticleI} class should contain all information of
an $i-$ particle which is necessary to calculate interaction
(see step 0 in Sec. \ref{sec:overview_action}).
Class \texttt{EssentialParticleI} should have required member functions
with specific names, as described below.

\subsubsection{Premise}

%この節の中では、EssentialParticleIクラスとしてEPIというクラスを一例と
%して使う。また、FullParticleクラスの一例としてFPというクラスを使う。
%EPI, FPというクラス名は変更可能である。
Let us take \texttt{EPI} and \texttt{FP} classes as examples of
\texttt{EssentialParticleI} and \texttt{FullParticle} as below.
Users can use arbitrary names in place of \texttt{EPI} and \texttt{FP}.

%EPIとFPの宣言は以下の通りである。
\begin{screen}
\begin{verbatim}
class FP;
class EPI;
\end{verbatim}
\end{screen}

\subsubsection{Required member functions}

\subsubsubsection{Summary}

%常に必要なメンバ関数はEPI::getPosとEPI::copyfromFPである。EPI::getPos
%はEPIクラスの位置情報をFDPSに読み込ませるための関数で、EPI::copyFromFP
%はFPクラスの情報をEPIクラスに書きこむ関数である。これらのメンバ関数の
%記述例と解説を以下に示す。
The member functions \texttt{EPI::getPos} and \texttt{EPI::copyFromForce} are required.
\texttt{EPI::getPos} returns the position of a particle to FDPS.
\texttt{EPI::copyFromFP} copies the information necessary for the interaction calculation from \texttt{FullParticle}.
The examples and descriptions for these member functions are listed below.

\subsubsubsection{EPI::getPos}

\begin{screen}
\begin{verbatim}
class EPI {
public:
    PS::F64vec getPos() const;
};
\end{verbatim}
\end{screen}

\begin{itemize}

\item {\bf Arguments}

  None.
  
\item {\bf Returns}

  %PS::F64vec型。EPIクラスの位置情報を保持したメンバ変数。
  \texttt{PS::F64vec}.
  Returns the position of a particle of \texttt{EPI} class.

\end{itemize}

\subsubsubsection{EPI::copyFromFP}

\begin{screen}
\begin{verbatim}
class FP;
class EPI {
public:
    void copyFromFP(const FP & fp);
};
\end{verbatim}
\end{screen}

\begin{itemize}

\item {\bf Arguments}

  %fp: 入力。const FP \&型。FPクラスの情報を持つ。
  \texttt{fp}: Input. \texttt{const FP \&} type.

\item {\bf Returns}

  None.
  
\item {\bf Behaviour}

  %FPクラスの持つ1粒子の情報の一部をEssnetialParticleIクラス
  %に書き込む。
  Copies the part of information of \texttt{FP} to \texttt{EssnetialParticleI}.
  
\end{itemize}

\subsubsection{Required member functions for specific case}

\subsubsubsection{Summary}

%本節では、場合によっては必要なメンバ関数について記述する。相互作用ツリー
%クラスのPS::SEARCH\_MODE型にPS::SEARCH\_MODE\_GATHERまたは
%PS::SEARCH\_MODE\_SYMMETRYを用いる場合に必要となるメンバ関数ついて記述
%する。
In this section we describe the member functions in the case that \texttt{PS::SEARCH\_MODE\_GATHER} or \texttt{PS::SEARCH\_MODE\_SYMMETRY} are used.

\subsubsubsection{Modes other than \texttt{PS::SEARCH\_MODE\_LONG} as \texttt{PS::SEARCH\_MODE} are used}

\subsubsubsubsection{EPI::getRSearch}

\begin{screen}
\begin{verbatim}
class EPI {
public:
    PS::F64 getRSearch() const;
};
\end{verbatim}
\end{screen}

\begin{itemize}

\item {\bf Arguments}

  None.
  
\item {\bf Returns}

  %PS::F32型またはPS::F64型。 EPIクラスの近傍粒子を探す
  %半径の大きさを保持したメンバ変数。
  \texttt{PS::F32vec} or \texttt{PS::F64vec}.
  Returns the value of the neighbour search radius in \texttt{EPI}.

\end{itemize}
