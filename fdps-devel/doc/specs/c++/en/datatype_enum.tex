\subsubsection{Summary}

In this section, we describe enumerated types defined in FDPS.
Currently, there is just one datatype. We describe it below.
%%本節ではFDPSで定義されている列挙型について記述する。列挙型には
%%BOUNDARY\_CONDITION型が存在する。以下、各列挙型について記述する。

\subsubsection{PS::BOUNDARY\_CONDITION type}
\label{sec:datatype_enum_boundarycondition}

\subsubsubsection{Summary}

Type BOUNDARY\_CONDITION specifies boundary conditions. The definition
is as follows.
%%BOUNDARY\_CONDITION型は境界条件を指定するためのデータ型である。これは
%%以下のように定義されている。
\begin{lstlisting}[caption=boundarycondition]
namespace ParticleSimulator{
    enum BOUNDARY_CONDITION{
        BOUNDARY_CONDITION_OPEN,
        BOUNDARY_CONDITION_PERIODIC_X,
        BOUNDARY_CONDITION_PERIODIC_Y,
        BOUNDARY_CONDITION_PERIODIC_Z,
        BOUNDARY_CONDITION_PERIODIC_XY,
        BOUNDARY_CONDITION_PERIODIC_XZ,
        BOUNDARY_CONDITION_PERIODIC_YZ,
        BOUNDARY_CONDITION_PERIODIC_XYZ,
        BOUNDARY_CONDITION_SHEARING_BOX,
        BOUNDARY_CONDITION_USER_DEFINED,
    };
}
\end{lstlisting}

We explain each value below.
%以下にどの変数がどの境界条件に対応するかを記述する。

\subsubsubsection{PS::BOUNDARY\_CONDITION\_OPEN}

This specifies the open boundary condition.
%開放境界となる。

\subsubsubsection{PS::BOUNDARY\_CONDITION\_PERIODIC\_X}

This specifies the periodic boundary condition in the direction of x-axis,
and open boundary condition in other directions. The interval is
left-bounded and right-unbounded. This is true for all periodic
boundary conditions.
%%x軸方向のみ周期境界、その他の軸方向は開放境界となる。周期の境界の下限
%%は閉境界、上限は開境界となっている。この境界の規定はすべての軸方向に
%%あてはまる。

\subsubsubsection{PS::BOUNDARY\_CONDITION\_PERIODIC\_Y}

This specifies the periodic boundary condition in the direction of y-axis,
and open boundary condition in other directions.
%y軸方向のみ周期境界、その他の軸方向は開放境界となる。

\subsubsubsection{PS::BOUNDARY\_CONDITION\_PERIODIC\_Z}

This specifies the periodic boundary condition in the direction of z-axis,
and open boundary condition in other directions.
%z軸方向のみ周期境界、その他の軸方向は開放境界となる。

\subsubsubsection{PS::BOUNDARY\_CONDITION\_PERIODIC\_XY}

This specifies the periodic boundary condition in the directions of x- and
y-axes, and open boundary condition in the direction of z-axis.
%x, y軸方向のみ周期境界、その他の軸方向は開放境界となる。

\subsubsubsection{PS::BOUNDARY\_CONDITION\_PERIODIC\_XZ}

This specifies the periodic boundary condition in the directions of x- and
z-axes, and open boundary condition in the direction of y-axis.
%x, z軸方向のみ周期境界、その他の軸方向は開放境界となる。

\subsubsubsection{PS::BOUNDARY\_CONDITION\_PERIODIC\_YZ}

This specifies the periodic boundary condition in the directions of y-
and z-axes, and open boundary condition in the direction of x-axis.
%y, z軸方向のみ周期境界、その他の軸方向は開放境界となる。

\subsubsubsection{PS::BOUNDARY\_CONDITION\_PERIODIC\_XYZ}

This specifies the periodic boundary condition in all three directions.
%x, y, z軸方向すべてが周期境界となる。

\subsubsubsection{PS::BOUNDARY\_CONDITION\_SHEARING\_BOX}

Not implemented yet.
%未実装。

\subsubsubsection{PS::BOUNDARY\_CONDITION\_USER\_DEFINED}

Not implemented yet.
%未実装。

%%%%%%%%%%%%%%%%%%%%%%%%%%%%%%%%%%%
\subsubsection{PS::INTERACTION\_LIST\_MODE type}
\label{sec:datatype_enum_interaction_list_mode}

%\subsubsubsection{概要}
\subsubsubsection{Summary}


Type INTERACTION\_LIST\_MODE is used to determine if user program
reuse the interaction list or not. This type is defined as follows.


%INTERACTION\_LIST\_MODE型は相互作用リストを再利用するかどうかを決定す
%るためのデータ型である。これは以下のように定義されている。

\begin{lstlisting}[caption=interaction\_list\_mode]
namespace ParticleSimulator{
    enum INTERACTION_LIST_MODE{
        MAKE_LIST,
        MAKE_LIST_FOR_REUSE,
        REUSE_LIST,
    };
}
\end{lstlisting}

This data type is used as the last argument of the function
calcForceAllAndWriteBack(). For more detail, plese see
section \ref{sec:treeForForceHighLevelAPI}.

%このデータ型はcalcForceAllAndWriteBack()等の関数の引数として使われる
%(詳しくはセクション\ref{sec:treeForForceHighLevelAPI}を参照)。

\subsubsubsection{PS::MAKE\_LIST}

FDPS (re)makes interaction lists for each interaction calculation
(each call of the APIs described above). In this case, we cannot reuse
interaction list in the next interaction calculation because FDPS does
not store the information of interaction list. \textbf{This is the
default operation mode in FDPS}.


%% 相互作用リストを毎回作り相互作用計算を行う場合に用いる。FDPS は相互作
%% 用リストを記憶しないため、相互作用リストの再利用はできない。これはデフォ
%% ルト動作である。


\subsubsubsection{PS::MAKE\_LIST\_FOR\_REUSE}

FDPS (re)makes interaction lists and stores them internally. Then, it
performs interaction calculation. In this case, we can reuse these
interaction lists in the next interaction calculation if we call the
APIs with the flag PS::REUSE\_LIST. The interaction lists memorized in
FDPS are destroyed if we perform the interaction calculation with the
flags PS::MAKE\_LIST\_FOR\_REUSE or PS::MAKE\_LIST.\\

%相互作用リストを再利用し相互作用計算を行いたい場合に用いる。このオプショ
%ンを選択する事でFDPSは相互作用リストを作りそれを保持する。作成した相互
%作用リストはPS::MAKE\_LIST\_FOR\_REUSEもしくはPS::MAKE\_LISTを用いて相
%互作用計算を行った際に破棄される。

\subsubsubsection{PS::REUSE\_INTERACTION\_LIST}

FDPS performs interaction calculation using the previously-created
interaction lists, which are the lists that are created at the
previous call of the APIs with the flag PS::MAKE\_LIST\_FOR\_REUSE. In
this case, moment information in superparticles are automatically
updated using the latest particle information.\\

%% 相互作用リストを再利用し相互作用計算を行う。再利用される相互作用リスト
%% は PS::MAKE\_LIST\_FOR\_REUSE を選択時に作成した相互作用リストである。
%% この時、超粒子のモーメント情報は自動的にアップデートされる。

%%%%%%%%%%%%%%%%%%%%%%%%%%%%%%%%%%%
\subsubsection{PS::EXCHANGE\_LET\_MODE tyoe}
\label{sec:datatype_enum_exchange_let_mode}

\subsubsubsection{Summary}


EXCHANGE\_LET\_MODE type is used to determine the algorithm for LET
exchange. It is defined as follows. 

%% EXCHANGE\_LET\_MODE型はLET交換の方法を決定するためのデータ型である。こ
%% れは以下のように定義されている。

\begin{lstlisting}[caption=EXCHANGE\_LET\_MODE]
namespace ParticleSimulator{
    enum EXCHANGE_LET_MODE{
        EXCHANGE_LET_A2A,
        EXCHANGE_LET_P2P_EXACT,
        EXCHANGE_LET_P2P_FAST,
    };
}
\end{lstlisting}


This data type is used as an argument for {\tt PS::TreeForForce::setExchangeLet}.
(See section \ref{sec:module_standard_treeforce_setexchangeletmode} for details).


%% このデータ型は{\tt PS::TreeForForce::setExchangeLet}の引数として使われる.
%% (詳しくはセクション\ref{sec:treeForForceInitializeAPI}を参照).

\subsubsubsection{PS::EXCHANGE\_LET\_A2A}

Use {\tt MPI\_Alltoall}.
%% LETの交換に{\tt MPI\_Alltoall}を使用する。
\subsubsubsection{PS::EXCHANGE\_LET\_P2P\_EXACT}

Do not use {\tt MPI\_Alltoall} but use {\tt MPI\_Allgather} and {\tt
MPI\_Isend/recv}. On machines where {\tt MPI\_Alltoall} is not very
efficient, or if the total number of processes is very large, this
algorithm can be faster. The result should be identical to that for
the case of {\tt PS::EXCHANGE\_LET\_A2A} up to the roundoff error.

%% LETの交換に{\tt MPI\_Alltoall}を使用せず、{\tt MPI\_Allgather}と{\tt
%% MPI\_Isend/recv}を使用する。{\tt MPI\_Alltoall}が効率的に動かない計算
%% 機では、こちらの方が速い場合がある。結果は丸め誤差の範囲で
%% PS::EXCHANGE\_LET\_A2Aを用いた場合と一致する。

\subsubsubsection{EXCHANGE\_LET\_P2P\_FAST}

Do not use {\tt MPI\_Alltoall} but use {\tt MPI\_Allgather} and {\tt
MPI\_Isend/recv}. The result can be slightly different from that for
the case of PS::EXCHANGE\_LET\_A2A, but can be faster than
the case of {\tt PS::EXCHANGE\_LET\_P2P\_EXACT}


%% LETの交換に{\tt MPI\_Alltoall}を使用せず、{\tt MPI\_Allgather}と{\tt
%% MPI\_Isend/recv}を使用する。結果は{\tt PS::EXCHANGE\_LET\_P2P\_EXACT}
%% を用いた場合と異なるが、より通信量が減っているため、高速に動作する可能
%% 性がある。



%%%%%%%%%%%%%%%%%%%%%%%%%%%%%%%%%%%
\subsubsection{PS::CALC\_DISTANCE\_TYPE type}
\label{sec:datatype_enum_calc_distance_type}

\subsubsubsection{Summary}

The {\tt CALC\_DISTANCE\_TYPE} type is used to determine the way the
distances between particles are calculated.

%% {\tt CALC\_DISTANCE\_TYPE}型は粒子間の距離の計算方法を変更するための型
%% である.

\begin{lstlisting}[caption=CALC\_DISTANCE\_TYPE]
namespace ParticleSimulator{
    enum CALC_DISTANCE_TYPE{
        CALC_DISTANCE_TYPE_NORMAL = 0,
        CALC_DISTANCE_TYPE_NEAREST_X = 1,
	CALC_DISTANCE_TYPE_NEAREST_Y = 2,
        CALC_DISTANCE_TYPE_NEAREST_XY = 3,
	CALC_DISTANCE_TYPE_NEAREST_Z = 4,
	CALC_DISTANCE_TYPE_NEAREST_XZ = 5,
	CALC_DISTANCE_TYPE_NEAREST_YZ = 6,
	CALC_DISTANCE_TYPE_NEAREST_XYZ = 7,
    };
}
\end{lstlisting}

This data type is used as the template parameter for {\tt PS::TreeForForce}.
(See \ref{sec:module_standard_treeforce_object}).

%% このデータ型は{\tt PS::TreeForForce}のテンプレート引数として与えられる.
%% (セクション\ref{sec:module_standard_treeforce_object}を参照).

\subsubsubsection{CALC\_DISTANCE\_TYPE\_NORMAL}

Calculate the L2 norm. This is the defaultbehavior.

%% 領域内の粒子のL2ノルムを計算する.特に指定のない場合は、この値が使われる.


\subsubsubsection{CALC\_DISTANCE\_TYPE\_NEAREST\_X}

Assume the periodic boundary for the x coordinate, even when the
boundary condition is open.

%% x方向に周期境界だとした場合に最も近い粒子とのL2ノルムを計算する.FDPS
%% のx方向の境界条件が開放境界の場合でも、最近接粒子との距離を計算する。

\subsubsubsection{CALC\_DISTANCE\_TYPE\_NEAREST\_Y}

Assume the periodic boundary for the y coordinate, even when the
boundary condition is open.


%% Y方向に周期境界だとした場合に最も近い粒子とのL2ノルムを計算する.FDPS
%% のy方向の境界条件が開放境界の場合でも、最近接粒子との距離を計算する。

\subsubsubsection{CALC\_DISTANCE\_TYPE\_NEAREST\_Z}

Assume the periodic boundary for the z coordinate, even when the
boundary condition is open.

% Z方向に周期境界だとした場合に最も近い粒子とのL2ノルムを計算する.FDPS
% のz方向の境界条件が開放境界の場合でも、最近接粒子との距離を計算する。

\subsubsubsection{CALC\_DISTANCE\_TYPE\_NEAREST\_XY}

Assume the periodic boundary for the x and y coordinates, even when the
boundary condition is open.

%% X,Y方向に周期境界だとした場合に最も近い粒子とのL2ノルムを計算する.
%% FDPSのx,y方向の境界条件が開放境界の場合でも、最近接粒子との距離を計算
%% する。

\subsubsubsection{CALC\_DISTANCE\_TYPE\_NEAREST\_XZ}

Assume the periodic boundary for the x and z coordinates, even when the
boundary condition is open.

%% X,Z方向に周期境界だとした場合に最も近い粒子とのL2ノルムを計算する.
%% FDPSのx,z方向の境界条件が開放境界の場合でも、最近接粒子との距離を計算す
%% る。

\subsubsubsection{CALC\_DISTANCE\_TYPE\_NEAREST\_YZ}

Assume the periodic boundary for the y and z coordinates, even when the
boundary condition is open.

%% Y,Z方向に周期境界だとした場合に最も近い粒子とのL2ノルムを計算する.
%% FDPSのy,z方向の境界条件が開放境界の場合でも、最近接粒子との距離を計算
%% する。

\subsubsubsection{CALC\_DISTANCE\_TYPE\_NEAREST\_XYZ}

Assume the periodic boundary for the x, y and z coordinates, even when the
boundary condition is open.

%% X,Y,Z方向に周期境界だとした場合に最も近い粒子とのL2ノルムを計算する.
%% FDPSのx,y,z方向の境界条件が開放境界の場合でも、最近接粒子との距離を計
%% 算する。
