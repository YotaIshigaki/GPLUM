\texttt{PS::Orthotope2} has two components: \texttt{low\_} and \texttt{high\_}, both of which are \texttt{PS::Vector2} class.
We define various APIs and operators for these components.
In the following, we describe them.

\begin{lstlisting}[caption=Orthotope2]
namespace ParticleSimulator{
    template<class T>
    class Orthotope2{
    public:
        Vector2<T> low_;
        Vector2<T> high_;

        Orthotope2(): low_(9999.9), high_(-9999.9){}
        
        Orthotope2(const Vector2<T> & _low, const Vector2<T> & _high)
            : low_(_low), high_(_high){}
        
        Orthotope2(const Orthotope2 & src) : low_(src.low_), high_(src.high_){}

        Orthotope2(const Vector2<T> & center, const T length) :
            low_(center-(Vector2<T>)(length)), high_(center+(Vector2<T>)(length)) {
        }

        void initNegativeVolume(){
            low_ = std::numeric_limits<float>::max() / 128;
            high_ = -low_;
        }

        void init(){
            initNegativeVolume();
        }

        void merge( const Orthotope2 & ort ){
            this->high_.x = ( this->high_.x > ort.high_.x ) ? this->high_.x : ort.high_.x;
            this->high_.y = ( this->high_.y > ort.high_.y ) ? this->high_.y : ort.high_.y;
            this->low_.x = ( this->low_.x <= ort.low_.x ) ? this->low_.x : ort.low_.x;
            this->low_.y = ( this->low_.y <= ort.low_.y ) ? this->low_.y : ort.low_.y;
        }

        void merge( const Vector2<T> & vec ){
            this->high_.x = ( this->high_.x > vec.x ) ? this->high_.x : vec.x;
            this->high_.y = ( this->high_.y > vec.y ) ? this->high_.y : vec.y;
            this->low_.x = ( this->low_.x <= vec.x ) ? this->low_.x : vec.x;
            this->low_.y = ( this->low_.y <= vec.y ) ? this->low_.y : vec.y;
        }

        void merge( const Vector2<T> & vec, const T size){
            this->high_.x = ( this->high_.x > vec.x + size ) ? this->high_.x : vec.x + size;
            this->high_.y = ( this->high_.y > vec.y + size ) ? this->high_.y : vec.y + size;
            this->low_.x = ( this->low_.x <= vec.x - size) ? this->low_.x : vec.x - size;
            this->low_.y = ( this->low_.y <= vec.y - size) ? this->low_.y : vec.y - size;
        }
    };
}
namespace PS = ParticleSimulator;
\end{lstlisting}

%%%%%%%%%%%%%%%%%%%%%%%%%%%%%
\subsubsubsection{Member variables}

\begin{screen}
\begin{verbatim}
template<typename T>
PS::Vector2<T> PS::Orthotope2<T>::low_;

template<typename T>
PS::Vector2<T> PS::Orthotope2<T>::high_;
\end{verbatim}
\end{screen}

\begin{itemize}
  
\item{{\bf Feature}}
  
Member variables, \texttt{low\_} and \texttt{high\_} can be directly handled.
  
\end{itemize}


%%%%%%%%%%%%%%%%%%%%%%%%%%%%%%%%
\subsubsubsection{Constructors}
%---------------
\begin{screen}
\begin{verbatim}
template<typename T>
PS::Orthotope2<T>::Orthotope2();
\end{verbatim}
\end{screen}

\begin{itemize}

\item{{\bf Arguments}}

None. But, the template argument \texttt{T} must be either \texttt{PS::F32} or \texttt{PS::F64}.

\item{{\bf Feature}}

Default constructor.
Member variables \texttt{low\_} and \texttt{high\_} are initialized by (9999.9, 9999.9) and (-9999.9, -9999.9), respectively.

\end{itemize}
%---------------
\begin{screen}
\begin{verbatim}
template<typename T>
PS::Orthotope2<T>::Orthotope2(const Vector2<T> _low, const Vector2<T> _high);
\end{verbatim}
\end{screen}

\begin{itemize}

\item{{\bf Arguments}}

\texttt{\_low}: Input. Type \texttt{const Vector2$<$T$>$}.

\texttt{\_high}: Input. Type \texttt{const Vector2$<$T$>$}.

where \texttt{T} must be either \texttt{PS::F32} or \texttt{PS::F64}.

\item{{\bf Feature}}

Member variables \texttt{low\_} and \texttt{high\_} are initialized by \texttt{\_low} and \texttt{\_high}, respectively.

\end{itemize}
%---------------
\begin{screen}
\begin{verbatim}
template<typename T>
PS::Orthotope2<T>::Orthotope2(const Vector2<T> & center, const T length);
\end{verbatim}
\end{screen}

\begin{itemize}

\item{{\bf Arguments}}

\texttt{center}: Input. Type \texttt{const Vector2$<$T$>$ \&}.
\texttt{length}: Input. Type \texttt{const T}.

\item{{\bf Feature}}

Member variables \texttt{low\_} and \texttt{high\_} are initialized by \texttt{center-(Vector2$<$T$>$)(length)} and \texttt{center+(Vector2$<$T$>$)(length)}, respectively.

\end{itemize}


%%%%%%%%%%%%%%%%%%%%%%%%%%%%%%%%
\subsubsubsection{Copy constructor}
%---------------
\begin{screen}
\begin{verbatim}
template<typename T>
PS::Orthotope2<T>::Orthotope2(const Orthotope2<T> & src);
\end{verbatim}
\end{screen}

\begin{itemize}

\item{{\bf Argument}}

\texttt{src}: Input. Type \texttt{const Orthotope2$<$T$>$ \&}.

\item{{\bf Feature}}

Member variables \texttt{low\_} and \texttt{high\_} are initialized by \texttt{src.low\_} and \texttt{src.high\_}, respectively.

\end{itemize}


%%%%%%%%%%%%%%%%%%%%%%%%%%%%%
\subsubsubsection{Initialize}
%---------------
\begin{screen}
\begin{verbatim}
template<typename T>
PS::Orthotope2<T>::initNegativeVolume();
\end{verbatim}
\end{screen}

\begin{itemize}

\item{{\bf Arguments}}

None.

\item{{\bf Feature}}

Member variables \texttt{low\_} and \texttt{high\_} are initialized by ($a$, $a$) and (-$a$, -$a$), respectively,
where $a$=\texttt{std::numeric\_limits$<$float$>$::max() / 128}.

\end{itemize}

%---------------
\begin{screen}
\begin{verbatim}
template<typename T>
PS::Orthotope2<T>::init();
\end{verbatim}
\end{screen}

\begin{itemize}

\item{{\bf Arguments}}

None.

\item{{\bf Feature}}

This member function is the same as \texttt{initNegativeVolume}.

\end{itemize}

%%%%%%%%%%%%%%%%%%%%%%%%%%%%%
\subsubsubsection{Merge operations}
%---------------
\begin{screen}
\begin{verbatim}
template<typename T>
void PS::Orthotope2<T>::merge( const Orthotope2 & ort )();
\end{verbatim}
\end{screen}

\begin{itemize}

\item{{\bf Arguments}}

\texttt{ort}: Input. Type \texttt{const Orthotope2 \&}.

\item{{\bf Feature}}

Member variables \texttt{low\_} and \texttt{high\_} are updated so that they represents the smallest rectangle that contains both the original rectangle and rectangle \texttt{ort}.

\end{itemize}
%---------------
\begin{screen}
\begin{verbatim}
template<typename T>
void PS::Orthotope2<T>::merge( const Vector2<T> & vec )();
\end{verbatim}
\end{screen}

\begin{itemize}

\item{{\bf Arguments}}

\texttt{vec}: Input. Type \texttt{const Vector2$<$T$>$ \&}.

\item{{\bf Feature}}

Member variables \texttt{low\_} and \texttt{high\_} are updated so that they represents the smallest rectangle that contains the point \texttt{vec}.

\end{itemize}
%---------------
\begin{screen}
\begin{verbatim}
template<typename T>
void PS::Orthotope2<T>::merge( const Vector2<T> & vec, const T size )();
\end{verbatim}
\end{screen}

\begin{itemize}

\item{{\bf Arguments}}

\texttt{vec}: Input. Type \texttt{const Vector2$<$T$>$ \&}.
\texttt{size}: Input. Type \texttt{const T}.

\item{{\bf Feature}}

Member variables \texttt{low\_} and {high\_} are updated so that they represents the smallest rectangle that contains a circle whose center is \texttt{vec} and radius is \texttt{size}.

\end{itemize}

