\texttt{PS::Vector3} has three components: \texttt{x}, \texttt{y}, and \texttt{z}. We define various APIs
and operators for these components. In the following, we list them.
%%PS::Vecotr3はx, y, zの3要素を持つ。これらに対する様々なAPIや演算子を
%%定義した。それらの宣言を以下に記述する。この節ではこれらについて詳し
%%く記述する。
\begin{lstlisting}[caption=Vector3]
namespace ParticleSimulator{
    template <typename T>
    class Vector3{
    public:
        // Three member variables
        T x, y, z;

        // Constructor
        Vector3() : x(T(0)), y(T(0)), z(T(0)) {}
        Vector3(const T _x, const T _y, const T _z) : x(_x), y(_y), z(_z) {}
        Vector3(const T s) : x(s), y(s), z(s) {}
        Vector3(const Vector3 & src) : x(src.x), y(src.y), z(src.z) {}

        // Assignment operator
        const Vector3 & operator = (const Vector3 & rhs);

        //[] operators
        const T & opertor[](const int i);
        T & operator[](const int i);

        // Addition and subtraction
        Vector3 operator + (const Vector3 & rhs) const;
        const Vector3 & operator += (const Vector3 & rhs);
        Vector3 operator - (const Vector3 & rhs) const;
        const Vector3 & operator -= (const Vector3 & rhs);

        // Scalar multiplication
        Vector3 operator * (const T s) const;
        const Vector3 & operator *= (const T s);
        friend Vector3 operator * (const T s, const Vector3 & v);
        Vector3 operator / (const T s) const;
        const Vector3 & operator /= (const T s);

        // Inner product (scalar product)
        T operator * (const Vector3 & rhs) const;

        // Outer product (vector product).
        Vector3 operator ^ (const Vector3 & rhs) const;

        // Typecast to Vector3<U>
        template <typename U>
        operator Vector3<U> () const;
    };
}
\end{lstlisting}
%%%%%%%%%%%%%%%%%%%%%%%%%%%%%
\subsubsubsection{Constructor}
\mbox{}
%%%%%%%%%%%%%%%%%%%%%%%%%%%%%
%%%%%%%%%%%%%%%%%%%%%%%%%%%%%
\begin{screen}
\begin{verbatim}
template<typename T>
PS::Vector3<T>::Vector3()
\end{verbatim}
\end{screen}

\begin{itemize}

\item{{\bf Argument}}

  None.

\item{{\bf Feature}}

Default constructor. The member variables \texttt{x}, \texttt{y} and \texttt{z} are initialized as 0.
%デフォルトコンストラクタ。メンバx,yは0で初期化される。

\end{itemize}

%%%%%%%%%%%%%%%%%%%%%%%%%%%%%
\begin{screen}
\begin{verbatim}
template<typename T>
PS::Vector3<T>::Vector3(const T _x, const T _y, const T _z)
\end{verbatim}
\end{screen}

\begin{itemize}

\item{{\bf Argument}}

\texttt{\_x}: Input. Type \texttt{const T}.
%{\_x}: 入力。{const T}型。

\texttt{\_y}: Input. Type \texttt{const T}.
%{\_y}: 入力。{const T}型。

\texttt{\_z}: Input. Type \texttt{const T}.
%{\_z}: 入力。{const T}型。

\item{{\bf Feature}}

Values of members \texttt{x}, \texttt{y} and \texttt{z} are set to values of arguments \texttt{\_x}, \texttt{\_y} and \texttt{\_z}, respectively.
%メンバ{x}、{y}をそれぞれ{\_x}、{\_y}で初期化する。

\end{itemize}

%%%%%%%%%%%%%%%%%%%%%%%%%%%%%
\begin{screen}
\begin{verbatim}
template<typename T>
PS::Vector3<T>::Vector3(const T s);
\end{verbatim}
\end{screen}

\begin{itemize}

\item{{\bf Argument}}

\texttt{s}: Input. Type \texttt{const T}.
%{s}: 入力。{const T}型。

\item{{\bf Feature}}

Values of members \texttt{x}, \texttt{y} and \texttt{z} are set to the value of argument \texttt{s}.
%メンバ{x}、{y}を両方とも{s}の値で初期化する。

\end{itemize}





%%%%%%%%%%%%%%%%%%%%%%%%%%%%%
\subsubsubsection{Copy constructor}
\mbox{}
%%%%%%%%%%%%%%%%%%%%%%%%%%%%%

%%%%%%%%%%%%%%%%%%%%%%%%%%%%%
\begin{screen}
\begin{verbatim}
template<typename T>
PS::Vector3<T>::Vector3(const PS::Vector3<T> & src)
\end{verbatim}
\end{screen}

\begin{itemize}

\item{{\bf Argument}}

\texttt{src}: Input. Type \texttt{const PS::Vector3$<$T$>$ \&}.
%{src}: 入力。{const PS::Vector3$<$T$>$ \&}型。

\item{{\bf Feature}}

Copy constructor. The new variable will have the same value as \texttt{src}.
%コピーコンストラクタ。{src}で初期化する。

\end{itemize}


%%%%%%%%%%%%%%%%%%%%%%%%%%%%%
\subsubsubsection{Member variables}

\begin{screen}
\begin{verbatim}
template<typename T>
T PS::Vector3<T>::x;

template<typename T>
T PS::Vector3<T>::y;

template<typename T>
T PS::Vector3<T>::z;
\end{verbatim}
\end{screen}

\begin{itemize}
  
\item{{\bf function}}
  
  Member variables, {x}, {y} and {z} can be directly handled.
  
\end{itemize}


%%%%%%%%%%%%%%%%%%%%%%%%%%%%%
\subsubsubsection{Assignment operator}
\mbox{}
%%%%%%%%%%%%%%%%%%%%%%%%%%%%%

%%%%%%%%%%%%%%%%%%%%%%%%%%%%%
\begin{screen}
\begin{verbatim}
template<typename T>
const PS::Vector3<T> & PS::Vector3<T>::operator = 
                       (const PS::Vector3<T> & rhs);
\end{verbatim}
\end{screen}

\begin{itemize}

\item{{\bf Argument}}

\texttt{rhs}: Input. Type \texttt{const PS::Vector3$<$T$>$ \&}.
%{rhs}: 入力。{const PS::Vector3$<$T$>$ \&}型。

\item{{\bf Return value}}
Type \texttt{const PS::Vector3$<$T$>$ \&}. Assigns the values of components of \texttt{rhs}
to its components, and returns the vector itself.
%%{const PS::Vector3$<$T$>$ \&}型。{rhs}のx,yの値を自身のメンバx,yに代
%%入し自身の参照を返す。代入演算子。

\end{itemize}


\subsubsubsection{[] operators}

\begin{screen}
\begin{verbatim}
template<typename T>
const T & PS::Vector3<T>::operator[]
                       (const int i);
\end{verbatim}
\end{screen}

\begin{itemize}

\item{{\bf Argument}}

\texttt{i}: input. Type \texttt{const int}.

\item{{\bf Return value}}

Type \texttt{const $<$T$>$ \&}. It returns $i$th-component of the vector.

\item{{\bf Remark}}

This operator would be slower than the direct access of the member variables.

\end{itemize}

\begin{screen}
\begin{verbatim}
template<typename T>
T & PS::Vector3<T>::operator[]
                       (const int i);
\end{verbatim}
\end{screen}

\begin{itemize}

\item{{\bf Argument}}

\texttt{i}: input. Type \texttt{const int}.

\item{{\bf Return value}}

Type \texttt{$<$T$>$ \&}. It returns $i$th-component of the vector.

\item{{\bf Remark}}

This operator would be slower than the direct access of the member variables.

\end{itemize}

%%%%%%%%%%%%%%%%%%%%%%%%%%%%%
\subsubsubsection{Addition and subtraction}
\mbox{}
%%%%%%%%%%%%%%%%%%%%%%%%%%%%%

%%%%%%%%%%%%%%%%%%%%%%%%%%%%%
\begin{screen}
\begin{verbatim}
template<typename T>
PS::Vector3<T> PS::Vector3<T>::operator + 
               (const PS::Vector3<T> & rhs) const;
\end{verbatim}
\end{screen}

\begin{itemize}

\item{{\bf Argument}}

\texttt{rhs}: Input. Type \texttt{const PS::Vector3$<$T$>$ \&}.
%{rhs}: 入力。{const PS::Vector3$<$T$>$ \&}型。

\item{{\bf Return value}}

Type \texttt{PS::Vector3$<$T$>$}. Add the components of \texttt{rhs} and its
components, and return the results.
%%{PS::Vector3$<$T$>$}型。{rhs}のx,yの値と自身のメンバx,yの値の和を取っ
%%た値を返す。

\end{itemize}


%%%%%%%%%%%%%%%%%%%%%%%%%%%%%
\begin{screen}
\begin{verbatim}
template<typename T>
const PS::Vector3<T> & PS::Vector3<T>::operator += 
                       (const PS::Vector3<T> & rhs);
\end{verbatim}
\end{screen}

\begin{itemize}

\item{{\bf Argument}}

\texttt{rhs}: Input. Type \texttt{const PS::Vector3$<$T$>$ \&}.
%{rhs}: 入力。{const PS::Vector3$<$T$>$ \&}型。

\item{{\bf Return value}}

Type \texttt{const PS::Vector3$<$T$>$ \&}. Add the components of \texttt{rhs} to its
components, and return itself (lhs is changed).
%%{const PS::Vector3$<$T$>$ \&}型。{rhs}のx,yの値を自身のメンバx,yに足
%%し、自身を返す。

\end{itemize}


%%%%%%%%%%%%%%%%%%%%%%%%%%%%%
\begin{screen}
\begin{verbatim}
template<typename T>
PS::Vector3<T> PS::Vector3<T>::operator - 
               (const PS::Vector3<T> & rhs) const;
\end{verbatim}
\end{screen}

\begin{itemize}

\item{{\bf 引数}}

\texttt{rhs}: Input. Type \texttt{const PS::Vector3$<$T$>$ \&}.
%{rhs}: 入力。{const PS::Vector3$<$T$>$ \&}型。

\item{{\bf Return value}}

Type \texttt{PS::Vector3$<$T$>$}. Subtract the components of \texttt{rhs} from its
components, and return the results.
%%{PS::Vector3$<$T$>$}型。{rhs}のx,yの値と自身のメンバx,yの値の差を取っ
%%た値を返す。

\end{itemize}


%%%%%%%%%%%%%%%%%%%%%%%%%%%%%
\begin{screen}
\begin{verbatim}
template<typename T>
const PS::Vector3<T> & PS::Vector3<T>::operator -= 
                       (const PS::Vector3<T> & rhs);
\end{verbatim}
\end{screen}

\begin{itemize}

\item{{\bf Argument}}

\texttt{rhs}: Input. Type \texttt{const PS::Vector3$<$T$>$ \&}.
%{rhs}: 入力。{const PS::Vector3$<$T$>$ \&}型。

\item{{\bf Return value}}

Type \texttt{const PS::Vector3$<$T$>$ \&}. Subtract the components of \texttt{rhs} from
its components, and return itself (lhs is changed).
%%{const PS::Vector3$<$T$>$ \&}型。自身のメンバx,yから{rhs}のx,yを引き
%%自身を返す。

\end{itemize}

%%%%%%%%%%%%%%%%%%%%%%%%%%%%%
\subsubsubsection{Scalar multiplication}
\mbox{}
%%%%%%%%%%%%%%%%%%%%%%%%%%%%%

%%%%%%%%%%%%%%%%%%%%%%%%%%%%%
\begin{screen}
\begin{verbatim}
template<typename T>
PS::Vector3<T> PS::Vector3<T>::operator * (const T s) const;
\end{verbatim}
\end{screen}

\begin{itemize}

\item{{\bf Argument}}

\texttt{s}: Input. Type \texttt{const T}.
%{s}: 入力。{const T}型。

\item{{\bf Return value}}

Type \texttt{PS::Vector3$<$T$>$}. Multiply its components by \texttt{s}, and return
the results.
%%{PS::Vector3$<$T$>$}型。自身のメンバx,yそれぞれに{s}をかけた値を返す。

\end{itemize}


%%%%%%%%%%%%%%%%%%%%%%%%%%%%%
\begin{screen}
\begin{verbatim}
template<typename T>
const PS::Vector3<T> & PS::Vector3<T>::operator *= (const T s);
\end{verbatim}
\end{screen}

\begin{itemize}

\item{{\bf Argument}}

\texttt{rhs}: Input. Type \texttt{const T}.
%{rhs}: 入力。{const T}型。

\item{{\bf Return value}}

Type \texttt{const PS::Vector3$<$T$>$ \&}. Multiply its components by \texttt{s},
and return itself (lhs is changed).
%%{const PS::Vector3$<$T$>$ \&}型。自身のメンバx,yそれぞれに{s}をかけ自
%%身を返す。

\end{itemize}


%%%%%%%%%%%%%%%%%%%%%%%%%%%%%
\begin{screen}
\begin{verbatim}
template<typename T>
PS::Vector3<T> PS::Vector3<T>::operator / (const T s) const;
\end{verbatim}
\end{screen}

\begin{itemize}

\item{{\bf Argument}}

\texttt{s}: Input. Type \texttt{const T}.
%{s}: 入力。{const T}型。

\item{{\bf Return value}}

Type \texttt{PS::Vector3$<$T$>$}. Divide its components by \texttt{s}, and return
the results.
%%{PS::Vector3$<$T$>$}型。自身のメンバx,yそれぞれを{s}で割った値を返す。

\end{itemize}


%%%%%%%%%%%%%%%%%%%%%%%%%%%%%
\begin{screen}
\begin{verbatim}
template<typename T>
const PS::Vector3<T> & PS::Vector3<T>::operator /= (const T s);
\end{verbatim}
\end{screen}

\begin{itemize}

\item{{\bf Argument}}

\texttt{rhs}: Input. Type \texttt{const T}.
%{rhs}: 入力。{const T}型。

\item{{\bf Return value}}

Type \texttt{const PS::Vector3$<$T$>$ \&}. Divide its components by \texttt{s}, and
return itself (lhs is changed).
%%{const PS::Vector3$<$T$>$ \&}型。自身のメンバx,yそれぞれを{s}で割り自
%%身を返す。

\end{itemize}


%%%%%%%%%%%%%%%%%%%%%%%%%%%%%
\subsubsubsection{Inner and outer product}
\mbox{}
%%%%%%%%%%%%%%%%%%%%%%%%%%%%%

%%%%%%%%%%%%%%%%%%%%%%%%%%%%%
\begin{screen}
\begin{verbatim}
template<typename T>
T PS::Vector3<T>::operator * (const PS::Vector3<T> & rhs) const;
\end{verbatim}
\end{screen}

\begin{itemize}

\item{{\bf Argument}}

\texttt{rhs}: Input. Type \texttt{const PS::Vector3$<$T$>$ \&}.
%{rhs}: 入力。{const PS::Vector3$<$T$>$ \&}型。

\item{{\bf Return value}}

Type \texttt{T}. Take inner product of it and \texttt{rhs}, and return the result.
%{T}型。自身と{rhs}の内積を取った値を返す。

\end{itemize}

%%%%%%%%%%%%%%%%%%%%%%%%%%%%%
\begin{screen}
\begin{verbatim}
template<typename T>
T PS::Vector3<T>::operator ^ (const PS::Vector3<T> & rhs) const;
\end{verbatim}
\end{screen}

\begin{itemize}

\item{{\bf Argument}}

\texttt{rhs}: Input. Type \texttt{const PS::Vector3$<$T$>$ \&}.
%{rhs}: 入力。{const PS::Vector3$<$T$>$ \&}型。

\item{{\bf Return value}}

Type \texttt{T}. Take outer product of it and \texttt{rhs}, and return the results.
%{T}型。自身と{rhs}の外積を取った値を返す。

\end{itemize}


%%%%%%%%%%%%%%%%%%%%%%%%%%%%%
\subsubsubsection{Typecast to Vector3$<$U$>$}
%\subsubsubsection{{Vector3$<$U$>$}への型変換}
\mbox{}
%%%%%%%%%%%%%%%%%%%%%%%%%%%%%

%%%%%%%%%%%%%%%%%%%%%%%%%%%%%
\begin{screen}
\begin{verbatim}
template<typename T>
template <typename U>
PS::Vector3<T>::operator PS::Vector3<U> () const;
\end{verbatim}
\end{screen}

\begin{itemize}

\item{{\bf Argument}}

  None.

\item{{\bf Return value}}

  Type \texttt{const PS::Vector3$<$U$>$}.
%  {const PS::Vector3$<$U$>$}型。

\item{{\bf Feature}}

  Typecast from type \texttt{const PS::Vector3$<$T$>$} to \texttt{const
  PS::Vector3$<$U$>$}.
%%  {const PS::Vector3$<$T$>$}型を{const PS::Vector3$<$U$>$}型にキャ ス
%%  トする。

\end{itemize}
