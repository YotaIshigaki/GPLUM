\subsubsection{Summary}

%関数オブジェクトcalcForceEpEpは粒子同士の相互作用を記述するものであり、
%相互作用の定義(節\ref{sec:overview_action}の手順0)に必要となる。以下、
%これの書き方の規定を記述する。
Functor \texttt{calcForceEpEp} defines the interaction between two particles.
This functor is required for the calculation of interactions (see, step 0 in Sec. \ref{sec:overview_action}).

\subsubsection{Premise}

%ここで示すのは重力N体シミュレーションの粒子間相互作用の記述の仕方であ
%る。関数オブジェクトcalcForceEpEpの名前はgravityEpEpとする。これは変更
%自由である。また、EssentialParitlceIクラスのクラス名をEPI,
%EssentialParitlceJクラスのクラス名をEPJ, Forceクラスのクラス名をResult
%とする。
Here one example of gravitational $N$-body problems are shown.
The name of functor \texttt{calcForceEpEp} is \texttt{gravityEpEp}, which is an arbitrary.
The class name of \texttt{EssentialParitlceI}, \texttt{EssentialParitlceJ} and \texttt{Force} are \texttt{EPI}, \texttt{EPJ} and \texttt{Result}.

\subsubsection{gravityEpEp::operator ()}

\begin{lstlisting}[caption=calcForceEpEp]
class Result;
class EPI;
class EPJ;
struct gravityEpEp {
    static PS::F32 eps2;
    void operator () (const EPI *epi,
                      const PS::S32 ni,
                      const EPJ *epj,
                      const PS::S32 nj,
                      Result *result)
    }
};
PS::F32 gravityEpEp::eps2 = 9.765625e-4;
\end{lstlisting}

\begin{itemize}

\item {\bf Arguments}

  %epi: 入力。const EPI *型またはEPI *型。i粒子情報を持つ配列。
  \texttt{epi}: Input. \texttt{const EPI *} type or \texttt{EPI *} type. Array of $i$ particles.

  %ni: 入力。const PS::S32型またはPS::S32型。i粒子数。
  \texttt{ni}: Input. \texttt{const PS::S32} type or \texttt{PS::S32}. The number of $i$ particles.

  %epj: 入力。const EPJ *型またはEPJ *型。j粒子情報を持つ配列。
  \texttt{epj}: Input. \texttt{const EPJ *} type or \texttt{EPJ *} type. Array of $j$ particles.

  %nj: 入力。const PS::S32型またはPS::S32型。j粒子数。
  \texttt{nj}: Input. \texttt{const PS::S32} type or \texttt{PS::S32} type. Number of $j$ particles.

  %result: 出力。Result *型。i粒子の相互作用結果を返す配列。
  \texttt{result}: Output. \texttt{Result *} type. Array of the results of interaction.

\item {\bf Returns}

  None.
  
\item {\bf Behaviour}

  %j粒子からi粒子への作用を計算する。
  Calculates the interaction to $i-$ particle from $j-$ particle.
  
\end{itemize}
