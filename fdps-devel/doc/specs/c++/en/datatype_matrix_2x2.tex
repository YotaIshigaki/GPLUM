\texttt{PS::MatrixSym2} has three components: \texttt{xx}, \texttt{yy}, and \texttt{xy}.
We define various APIs and operators for these components.
In the following, we list them.
%%PS::MatrixSym2はxx, yy, xyの3要素を持つ。これらに対する様々なAPIや演
%%算子を定義した。それらの宣言を以下に記述する。この節ではこれらについ
%%て詳しく記述する。
\begin{lstlisting}[caption=MatrixSym2]
namespace ParticleSimulator{
    template<class T>
    class MatrixSym2{
    public:
        // Three member variables
        T xx, yy, xy;

        // Constructors
        MatrixSym2() : xx(T(0)), yy(T(0)), xy(T(0)) {}
        MatrixSym2(const T _xx, const T _yy, const T _xy)
            : xx(_xx), yy(_yy), xy(_xy) {}
        MatrixSym2(const T s) : xx(s), yy(s), xy(s){}
        MatrixSym2(const MatrixSym2 & src) : xx(src.xx), yy(src.yy), xy(src.xy) {}

        // Assignment operator
        const MatrixSym2 & operator = (const MatrixSym2 & rhs);

        // Addition and subtraction
        MatrixSym2 operator + (const MatrixSym2 & rhs) const;
        const MatrixSym2 & operator += (const MatrixSym2 & rhs) const;
        MatrixSym2 operator - (const MatrixSym2 & rhs) const;
        const MatrixSym2 & operator -= (const MatrixSym2 & rhs) const;

        // Trace
        T getTrace() const;

        // Typecast to MatrixSym2<U>
        template <typename U>
        operator MatrixSym2<U> () const;
    }
}
namespace PS = ParticleSimulator;
\end{lstlisting}

%%%%%%%%%%%%%%%%%%%%%%%%%%%%%%%%%%%%%%%%%%%%%%%%%%%%%
\subsubsubsection{Constructor}

\begin{screen}
\begin{verbatim}
template<typename T>
PS::MatrixSym2<T>::MatrixSym2();
\end{verbatim}
\end{screen}

\begin{itemize}

\item{{\bf Argument}}

  None.

\item{{\bf Feature}}

Default constructor. The member variables \texttt{xx}, \texttt{yy} and \texttt{xy} are initialized as 0.
%デフォルトコンストラクタ。メンバxx,yy,xyは0で初期化される。

\end{itemize}

\begin{screen}
\begin{verbatim}
template<typename T>
PS::MatrixSym2<T>::MatrixSym2
            (const T _xx,
             const T _yy,
             const T _xy);
\end{verbatim}
\end{screen}

\begin{itemize}

\item{{\bf Argument}}

\texttt{\_xx}: Input. Type \texttt{const T}.
%{\_xx}: 入力。{const T}型。

\texttt{\_yy}: Input. Type \texttt{const T}.
%{\_yy}: 入力。{const T}型。

\texttt{\_xy}: Input. Type \texttt{const T}.
%{\_xy}: 入力。{const T}型。

\item{{\bf Feature}}

Values of members \texttt{xx}, \texttt{yy} and \texttt{xy} are set to values of arguments \texttt{\_xx}, \texttt{\_yy} and \texttt{\_xy}, respectively.
%%メンバ{xx}、{yy}、{xy}をそれぞれ{\_xx}、{\_yy}、{\_xy}で初期化する。

\end{itemize}

\begin{screen}
\begin{verbatim}
template<typename T>
PS::MatrixSym2<T>::MatrixSym2(const T s);
\end{verbatim}
\end{screen}

\begin{itemize}

\item{{\bf Argument}}

\texttt{s}: Input. Type \texttt{const T}.
%{s}: 入力。{const T}型。

\item{{\bf Feature}}

Values of members \texttt{xx}, \texttt{yy} and \texttt{xy} are set to the value of argument \texttt{s}.
%メンバ{xx}、{yy}、{xy}すべてを{s}の値で初期化する。

\end{itemize}

%%%%%%%%%%%%%%%%%%%%%%%%%%%%%%%%%%%%%%%%%%%%%%%%%%%%%
\subsubsubsection{Copy constructor}

\begin{screen}
\begin{verbatim}
template<typename T>
PS::MatrixSym2<T>::MatrixSym2(const PS::MatrixSym2<T> & src)
\end{verbatim}
\end{screen}

\begin{itemize}

\item{{\bf Argument}}

\texttt{src}: Input. Type \texttt{const PS::MatrixSym2$<$T$>$ \&}.
%{src}: 入力。{const PS::MatrixSym2$<$T$>$ \&}型。

\item{{\bf Feature}}

Copy constructor. The new variable will have the same value as \texttt{src}.
%コピーコンストラクタ。{src}で初期化する。

\end{itemize}

%%%%%%%%%%%%%%%%%%%%%%%%%%%%%%%%%%%%%%%%%%%%%%%%%%%%%
\subsubsubsection{Assignment operator}

\begin{screen}
\begin{verbatim}
template<typename T>
const PS::MatrixSym2<T> & PS::MatrixSym2<T>::operator = 
                       (const PS::MatrixSym2<T> & rhs);
\end{verbatim}
\end{screen}

\begin{itemize}

\item{{\bf Argument}}

\texttt{rhs}: Input. Type \texttt{const PS::MatrixSym2$<$T$>$ \&}.
%{rhs}: 入力。{const PS::MatrixSym2$<$T$>$ \&}型。

\item{{\bf Return value}}

Type \texttt{const PS::MatrixSym2$<$T$>$ \&}. Assigns the values of components of \texttt{rhs}
to its components, and returns the vector itself.
%%{const PS::MatrixSym2$<$T$>$ \&}型。{rhs}のxx,yy,xyの値を自身のメンバ
%%xx,yy,xyに代入し自身の参照を返す。代入演算子。

\end{itemize}

%%%%%%%%%%%%%%%%%%%%%%%%%%%%%%%%%%%%%%%%%%%%%%%%%%%%%
\subsubsubsection{Addition and subtraction}

\begin{screen}
\begin{verbatim}
template<typename T>
PS::MatrixSym2<T> PS::MatrixSym2<T>::operator + 
               (const PS::MatrixSym2<T> & rhs) const;
\end{verbatim}
\end{screen}

\begin{itemize}

\item{{\bf Argument}}

\texttt{rhs}: Input. Type \texttt{const PS::MatrixSym2$<$T$>$ \&}.
%{rhs}: 入力。{const PS::MatrixSym2$<$T$>$ \&}型。

\item{{\bf Return value}}

Type \texttt{PS::MatrixSym2$<$T$>$}. Add the components of \texttt{rhs} and its
components, and return the results.

%%{PS::MatrixSym2$<$T$>$}型。{rhs}のxx,yy,xyの値と自身のメンバxx,yy,xy
%%の値の和を取った値を返す。

\end{itemize}

\begin{screen}
\begin{verbatim}
template<typename T>
const PS::MatrixSym2<T> & PS::MatrixSym2<T>::operator += 
                       (const PS::MatrixSym2<T> & rhs);
\end{verbatim}
\end{screen}

\begin{itemize}

\item{{\bf Argument}}

\texttt{rhs}: Input. Type \texttt{const PS::MatrixSym2$<$T$>$ \&}.
%{rhs}: 入力。{const PS::MatrixSym2$<$T$>$ \&}型。

\item{{\bf Return value}}

Type \texttt{const PS::MatrixSym2$<$T$>$ \&}. Add the components of \texttt{rhs} to its
components, and return itself (lhs is changed).
%%{const PS::MatrixSym2$<$T$>$ \&}型。{rhs}のxx,yy,xyの値を自身のメンバ
%%xx,yy,xyに足し、自身を返す。

\end{itemize}

\begin{screen}
\begin{verbatim}
template<typename T>
PS::MatrixSym2<T> PS::MatrixSym2<T>::operator - 
               (const PS::MatrixSym2<T> & rhs) const;
\end{verbatim}
\end{screen}

\begin{itemize}

\item{{\bf Argument}}

\texttt{rhs}: Input. Type \texttt{const PS::MatrixSym2$<$T$>$ \&}.
%{rhs}: 入力。{const PS::MatrixSym2$<$T$>$ \&}型。

\item{{\bf Return value}}

Type \texttt{PS::MatrixSym2$<$T$>$}. Subtract the components of \texttt{rhs} from its
components, and return the results.

%%{PS::MatrixSym2$<$T$>$}型。{rhs}のxx,yy,xyの値と自身のメンバxx,yy,xy
%%の値の差を取った値を返す。

\end{itemize}

\begin{screen}
\begin{verbatim}
template<typename T>
const PS::MatrixSym2<T> & PS::MatrixSym2<T>::operator -= 
                       (const PS::MatrixSym2<T> & rhs);
\end{verbatim}
\end{screen}

\begin{itemize}

\item{{\bf Argument}}

\texttt{rhs}: Input. Type \texttt{const PS::MatrixSym2$<$T$>$ \&}.
%{rhs}: 入力。{const PS::MatrixSym2$<$T$>$ \&}型。

\item{{\bf Return value}}

Type \texttt{const PS::MatrixSym2$<$T$>$ \&}. Subtract the components of \texttt{rhs} from
its components, and return itself (lhs is changed).
%%{const PS::MatrixSym2$<$T$>$ \&}型。自身のメンバxx,yy,xyから{rhs}の
%%xx,yy,xyを引き自身を返す。

\end{itemize}

%%%%%%%%%%%%%%%%%%%%%%%%%%%%%%%%%%%%%%%%%%%%%%%%%%%%%
\subsubsubsection{Trace}

\begin{screen}
\begin{verbatim}
template<typename T>
T PS::MatrixSym2<T>::getTrace() const;
\end{verbatim}
\end{screen}

\begin{itemize}

\item{{\bf Argument}}

  None.

\item{{\bf Return value}}

  Type \texttt{T}.

\item{{\bf Feature}}

  Calculate the trace, and return the result.

\end{itemize}

%%%%%%%%%%%%%%%%%%%%%%%%%%%%%%%%%%%%%%%%%%%%%%%%%%%%%
\subsubsubsection{Typecast to MatrixSym2$<$U$>$}

\begin{screen}
\begin{verbatim}
template<typename T>
template<typename U>
PS::MatrixSym2<T>::operator PS::MatrixSym2<U> () const;
\end{verbatim}
\end{screen}

\begin{itemize}

\item{{\bf Argument}}

  None.

\item{{\bf Return value}}

  Type \texttt{const PS::MatrixSym2$<$U$>$}.

\item{{\bf Feature}}

  Typecast from type \texttt{const PS::MatrixSym2$<$T$>$} to \texttt{const
  PS::MatrixSym2$<$U$>$}.
%%  {const PS::MatrixSym2$<$T$>$}型を{const PS::MatrixSym2$<$U$>$}型に
%%  キャ ストする

\end{itemize}

