%\subsection{概要}
\subsection{Summary}

In this section, we describe the namespaces used in FDPS. All the FDPS
APIs are under namespace \texttt{ParticleSimulator}. In the
following, we show APIs directly under \texttt{ParticleSimulator}, and
namespaces nested under \texttt{ParticleSimulator}.
%%本節では、名前空間の構造について述べる。FDPSはParticleSimulatorという
%%名前空間で囲まれている。以下では、ParticleSimulator直下にある機能と、
%%ParticleSimulatorにネストされている名前空間について述べる。

\subsection{ParticleSimulator}

The standard features of FDPS are in namespace \texttt{ParticleSimulator}.
%FDPSの標準機能すべては名前空間ParticleSimulatorの直下にある。

Namespace \texttt{ParticleSimulator} is abbreviated to \texttt{PS} as follows.
%名前空間ParticleSimulatorは以下のように省略されており、この文書におけ
%るあとの記述でもこの省略形を採用する。
\begin{screen}
\begin{verbatim}
namespace PS = ParticleSimulator;
\end{verbatim}
\end{screen}
In this document, we use this abbreviation.

Extended features of FDPS are grouped under nested namespaces.
Currently, \texttt{ParticleMesh} features is under nested namespace
\texttt{ParticleMesh}.  In the following, we describe this namespace
containing the extended features.
%名前空間ParticleSimulatorの下にはいくつかの名前空間が拡張機能毎にネストされている。拡
%張機能にはParticleMeshがある。以下では拡張機能の名前空間について記述す
%る。

\subsubsection{ParticleMesh}

The features of Particle Mesh are under namespace \texttt{PartcielMesh}.
This namespace is abbreviated to \texttt{PM} as follows.
%Particle Meshの機能は名前空間PartcielMeshに囲まれており、名前空間
%ParticleMeshは名前空間ParticleSimulatorの直下にネストされている。また、
%ParticleMeshはPMと省略されている。これらをまとめると以下のようになって
%いる。
\begin{screen}
\begin{verbatim}
ParticleSimulator {
    ParticleMesh {
    }
    namespace PM = ParticleMesh;
}
\end{verbatim}
\end{screen}

In this document, we use this abbreviation.
%以後、この文書では省略形のPMを用いて記述する。
