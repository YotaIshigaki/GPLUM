
本節では、Framework for Developing Particle Simulator (FDPS)
\describeForEach{}{および FDPS Fortranインターフェース}{および FDPS C言語インターフェース}
の概要について述べる。FDPSは粒子シミュレーションのコード開発を支援するフレームワークである。FDPSが行うのは、計算コストの最も大きな粒子間相互作用の計算と、粒子間相互作用の計算のコストを負荷分散するための処理である。これらはマルチプロセス、マルチスレッドで並列に処理することができる。比較的計算コストが小さく、並列処理を必要としない処理(粒子の軌道計算など)はユーザーが行う。

FDPSが対応している座標系は、2次元直交座標系と3次元直交座標系である。また、境界条件としては、開放境界条件と周期境界条件に対応している。周期境界条件の場合、$x$、$y$、$z$軸方向の任意の組み合わせの周期境界条件を課すことができる。

ユーザーは粒子間相互作用の形を定義する必要がある。定義できる粒子間相互作用の形には様々なものがある。粒子間相互作用の形を大きく分けると2種類あり、1つは長距離力、 もう1つは短距離力である。この2つの力は、遠くの複数の粒子からの作用を1つの超粒子からの作用にまとめるか(長距離力)、まとめないか(短距離力)という基準でもって分類される。

長距離力には、小分類があり、無限遠に存在する粒子からの力も計算するカットオフなし長距離力と、ある距離以上離れた粒子からの力は計算しないカットオフあり長距離力がある。前者は開境界条件下における重力やクーロン力に対して、後者は周期境界条件下の重力やクーロン力に使うことができる。後者のためにはParticle Mesh法などが必要となるが、これはFDPSの拡張機能として用意されている。

短距離力には、小分類が4つ存在する。短距離力の場合、粒子はある距離より離れた粒子からの作用は受けない。すなわち必ずカットオフが存在する。このカットオフ長の決め方によって、小分類がなされる。すなわち、全粒子のカットオフ長が等しいコンスタントカーネル、 カットオフ長が作用を受ける粒子固有の性質で決まるギャザーカーネル、カットオフ長が作用を与える粒子固有の性質で決まるスキャッタカーネル、カットオフ長が作用を受ける粒子と作用を与える粒子の両方の性質で決まるシンメトリックカーネルである。コンスタントカーネルは分子動力学におけるLJ力に適用でき、その他のカーネルはSPHなどに適用できる。

ユーザーは、粒子間相互作用や粒子の軌道積分などを、
\describeForEach{C++}{Fortran 2003}{C言語}
を用いて記述する。