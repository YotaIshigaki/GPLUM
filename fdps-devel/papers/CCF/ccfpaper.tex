%%%%%%%%%%%%%%%%%%%%%%% file template.tex %%%%%%%%%%%%%%%%%%%%%%%%%
%
% This is a general template file for the LaTeX package SVJour3
% for Springer journals.          Springer Heidelberg 2010/09/16
%
% Copy it to a new file with a new name and use it as the basis
% for your article. Delete % signs as needed.
%
% This template includes a few options for different layouts and
% content for various journals. Please consult a previous issue of
% your journal as needed.
%
%%%%%%%%%%%%%%%%%%%%%%%%%%%%%%%%%%%%%%%%%%%%%%%%%%%%%%%%%%%%%%%%%%%
%
% First comes an example EPS file -- just ignore it and
% proceed on the \documentclass line
% your LaTeX will extract the file if required
%% \begin{filecontents*}{example.eps}
%% %!PS-Adobe-3.0 EPSF-3.0
%% %%BoundingBox: 19 19 221 221
%% %%CreationDate: Mon Sep 29 1997
%% %%Creator: programmed by hand (JK)
%% %%EndComments
%% gsave
%% newpath
%%   20 20 moveto
%%   20 220 lineto
%%   220 220 lineto
%%   220 20 lineto
%% closepath
%% 2 setlinewidth
%% gsave
%%   .4 setgray fill
%% grestore
%% stroke
%% grestore
%% \end{filecontents*}
%
%\RequirePackage{fix-cm}
%
%\documentclass{svjour3}                     % onecolumn (standard format)
%\documentclass[smallcondensed]{svjour3}     % onecolumn (ditto)
\documentclass[smallextended]{svjour3}       % onecolumn (second format)
%\documentclass[twocolumn]{svjour3}          % twocolumn
%
\smartqed  % flush right qed marks, e.g. at end of proof
%
\usepackage{graphicx}
\usepackage{amsmath}
\usepackage{bm} 
%
% \usepackage{mathptmx}      % use Times fonts if available on your TeX system
%
% insert here the call for the packages your document requires
%\usepackage{latexsym}
% etc.
%
% please place your own definitions here and don't use \def but
% \newcommand{}{}
%
% Insert the name of "your journal" with
% \journalname{myjournal}
%
\newcommand{\myvec}[1]{\vec{#1}}
\newcommand{\redtext}[1]{\textcolor{red}{#1}}
\newcommand{\icarus}{Icarus}
\newcommand{\pasa}{Publications of the Astronomical Society of Australia}

\def\araa{Annual Review of Astronomy and Astrophysics }
\def\aap{Astronomy and Astrophysics }
\def\aj{The Astronomical Journal }
\def\apj{The Astrophysical Journal }
\def\APJ{The Astrophysical Journal }
\def\apjl{The Astrophysical Journal Letters }
\def\apjs{The Astrophysical Journal Supplement Series }
\def\apss{Astrophysics and Space Science }
\def\ajl{The Astronomical Journal }
\def\pasj{Publications of the Astronomical Society of Japan }
\def\mn{Monthly Notices of Royal Astronomical Society }
\def\MN{Monthly Notices of Royal Astronomical Society  }
\def\mnras{Monthly Notices of Royal Astronomical Society }
\def\nat{Nature }
\def\jcp{Journal of Computational Physics }
\def\etal{{\frenchspacing\it et al.}}
\def \K         {{\rm\,K}}
\def \Msun      {{\rm\,M}_{\odot}}
\def \Lsun      {{\rm\,L}_{\odot}}
\def \kmsmpc    {{\rm\,km\ s^{-1}\ Mpc^{-1}}}
\def \kms       {{\rm\,km\ s^{-1}}}
\def \mpc       {{\rm\,Mpc}}
\def \hmpc      {{\rm\,h^{-1}\,Mpc}}
\def \yr        {{\rm\,yr}}
\def \kpc       {{\rm\,kpc}}
\def \pc        {{\rm\,pc}}
\def \cc        {{\rm\,cm^{-3}}}
\def \kev       {{\rm\,keV}}
\def \erg       {{\rm\,erg}}
\def \myr       {{\rm\,Myr}}
\def \s         {{\rm\,s}}

\begin{document}

\title{Extreme-scale particle-based simulations on advanced HPC platforms
  %\thanks{Grants or other notes
%about the article that should go on the front page should be
%placed here. General acknowledgments should be placed at the end of the article.}
}
\subtitle{Lessons from PEZY-SC2, Sunway Taihulight and NVIDIA Volta}

%\titlerunning{Short form of title}        % if too long for running head

\author{M. Iwasawa\and D. Namekata \and K. Nomura \and M. Tsubouchi \and J. Makino}


%\authorrunning{Short form of author list} % if too long for running head

\institute{
  %% F. Author \at
%%               first address \\
%%               Tel.: +123-45-678910\\
%%               Fax: +123-45-678910\\
%%               \email{fauthor@example.com}           %  \\
%% %             \emph{Present address:} of F. Author  %  if needed
%%            \and
%%            S. Author \at
%%            second address
  Masaki Iwasawa\at
Center for Planetary Science,   Graduate School of  Science, Kobe University
  \email{iwasawa@people.kobe-u.ac.jp}
\and 
Daisuke Namekata\at
  RIKEN Center for Computational Science
\email{daisuke.namekata@riken.jp}
\and
Kentaro Nomura\at
Center for Planetary Science, Graduate School of  Science, Kobe University
\email{kentaro.nomura@port.kobe-u.ac.jp}
\and
Miyuki Tsubouchi\at
  RIKEN Center for Computational Science
\email{ miyuki.tsubouchi@riken.jp}
\and
Junichiro Makino\at
Department of Planetology, Graduate School of  Science, Kobe University
\email{makino@mail.jmlab.jp}
}

\date{Received: date / Accepted: date}
% The correct dates will be entered by the editor


\maketitle

\begin{abstract}
  We overview the current status and future development directions of
  our FDPS (Framework for Developing Particle Simulator)
  framework. Many of particle-based simulation codes share the same
  characteristic that the most time-consuming part of the 
  simulation is the calculation of the interactions between particles,
  and a large fraction of programming effort is spent for procedures to
  make the force calculation efficient, such as the decomposition of
  computational domain, exchange of particles between domains,
  exchange of information necessary to calculate the interaction to
  particles in different domains, and efficient neighbor search.  The
  basic idea of FDPS is to provide generic and high-performance
  library for these procedures. Using these procedures, researchers or
  application programmers in various fields can write their programs
  without taking care of parallelization and performance tuning. In
  order to make FDPS useful on advanced HPC platforms at present and
  (near) future, we investigated its performance on several modern
  platforms and learned what can be the bottleneck.  In this paper we
  summarize what we learned.  \keywords{Particle-based simulations
    \and High-performance computing}
  % \PACS{PACS code1 \and PACS code2 \and more}
% \subclass{MSC code1 \and MSC code2 \and more}
\end{abstract}

\setcounter{tocdepth}{3}
\tableofcontents

\section{Introduction}
\label{sect:intro}

\subsection{Background}

Large-scale particle-based simulations are now used to model physical
systems of all scales, from molecular scale to cosmological scale. 
As a result, a number of simulation programs have been developed and
are being maintained. Just to give several examples, Amber\cite{Amber2012}, NAMD\cite{NAMD2005},
and GROMACS\cite{GROMACS2013} are used for classical molecular dynamics simulations of
biomolecules, LAMMPS\cite{LAMMPS1995}. For cosmological $N$-body or $N$-body+SPH
simulations, PKDGRAV\cite{PKDGRAV2017},  Gadget\cite{Springeletal2000}, and GreeM\cite{Ishiyamaetal2009b} have been
available. It is certainly the case that for each research field
several groups are developing their own  codes for particle-based
simulations to meet their specific needs. 

On the other hand, it has become quite difficult to
achieve good, or even moderate efficiency for large-scale parallel
particle-based simulation code on modern platforms, and will be even
more difficult in future.  There are many
reasons for this difficulty. To illustrate them, let us start with one
of the most efficient and scalable code, GreeM. It is designed for
large-scale cosmological $N$-body simulations on large HPC systems,
and it was used for the cosmological simulation on K computer which
was awarded the 2012 Gordon Bell Prize. It uses the multisection
method with sampling algorithm\cite{Makino2004} for the domain
decomposition, and parallel Barns-Hut treecode modified for periodic
boundary (TreePM) based on the LET(local essential tree\cite{Salmon1990}).
The force calculation within one MPI process is further parallelized
by Barnes' vectorization algorithm, to make use of the SIMD execution
units of modern computers. It achieved the efficiency of more than
50\%, for the cosmological $N$-body simulation of $10240^3$ particles
on the entire K computer with 82944 nodes.

Each node of K computer has one 8-core SPARC64 VIIIfx CPU with the
theoretical peak performance of 128 Gflops, and one MPI process runs
on one node. K computer has very fast memory (peak B/F number of 0.5)
and rich network of 6D torus ($24\times 18 \times 16\times 3\times 2
\times 2$). Thus, each node has 10 links, and the speed of each link
is 5GB/s bidirectional. Also, 
SPARC64 VIIIfx processor has two unique features. First one is that its L2
cache is physically shared by all eight cores, and the second one is
that the barrier synchronization of cores is supported by hardware and
thus extremely fast.

Many
of other modern HPC systems lack these features. Table
\ref{tab:hpcmachines} shows some of the biggest modern HPC systems. We
can see that K has exceptionally fast memory and rich network, and
other systems have relatively weak memory and network. Even in the
case of Fugaku, the relative network bandwidth is much less than that
of K. Thus, applications which works fine on K (or Fugaku) do not
necessarily run efficiently on other platforms, and we need to modify
the implementation or introduce new algorithms to reduce the necessary
memory/network bandwidth.

It is also necessary to rewrite the
code on systems which offer platform-specific programming
environment. For example, Cuda is used on NVIDIA GPGPUs. Though in principle
Sunway Taihulight can be programmed with OpenACC,  in many
cases we need to write codes using Athread library. GYOUKOU offers
PZCL, a dialect of OpenCL, as its only available programming
environment.

% 2019/12/13

The architectures, in particular the structures of the memory systems,
are also all different on these machines. NVIDIA V100 has rather
traditional host-accelerator architecture with cache-based memory
systems on both sides. PEZY-SC2, at least as
planned, would have single physical memory shared by both of its
MIPS64 cores and proprietary SC2 cores, and both have cache memories
(but non-coherent on SC2 side). The SW26010 processor of Sunway
Taihulight also have a single physical memory shared by MPE
(management processor element) and CPE (computing processor element).
MPE and CPE share the same ISA. However, CPEs do not have data cache
but local memories. In addition, processor architectures are also all
different. NVIDIA Volta have relatively small number of ``Streaming
Multiprocessors'', each with a number of floating-point arithmetic
units.  PEZY-SC2 is an MIMD processor with 2048 SC2 cores and 6 MIPS64
cores, and each of SC2 cores has one floating-point units. An SW26010
processor has four ``core groups'', each with one MPE with data cache
and 64 CPEs with local memories. A unique and very important feature
of CPEs is that they are organized as an $8\times8$ grid.  Within each rows
of columns of this grid, both of point-to-point communication and
broadcast are supported. 

Thus, programs developped for machines with different processors, such
as K/Fugaku, GYOUKOU, TaihuLight and Summits can easily become very
different, because (a) programming environments are different (usual
OpenMP, OpenCL, Athread and Cuda), (b) memory structures are all
different, and (c) ISA and many other things are different.



\begin{table}
\centering  
% table caption is above the table
\caption{Performance parameters of modern HPC systems}
\label{tab:hpcmachines}       % Give a unique label
% For LaTeX tables use
\begin{tabular}{llccc}
\hline\noalign{\smallskip}
System  & Processor & memory B/F & network B/F   \\
\noalign{\smallskip}\hline\noalign{\smallskip}
K  & SPARC64 VIIIfx & 0.5 &    0.12 \\
Fugaku  &  A64fx& 0.34 &       0.013   \\
GYOUKOU  & PEZY SC2& 0.03 &    0.0005   \\
Taihulight  & SW26010 & 0.03 & 0.003 \\
Summit  & NVIDIA V100 & 0.12 & 0.0006 \\
\noalign{\smallskip}\hline
\end{tabular}
\end{table}

With such a wide variety of both hardware and software, it is now very
difficult for one research group to develop efficient programs for
different architectures. Thus, almost all widely used programs for HPC
are optimized for Intel x86 processors, with some support for NVIDIA
GPGPUs but nothing else, and porting  them to  other architectures and
achieve reasonable performance requires years or work of a large group
of researchers.

This situation is clearly not ideal. From the viewpoint of 
developers of large-scale simulation sofware, it is unpractical to
develop and maintain multiple versions optimized for very different
architectures, and concentrating on one or two most widely used ones
makes  perfect sense. From the viewpoint of developers of processors
and HPC systems, it is necessary to change the architecture in many
different ways to improve the performance.  Thus, the ``best''
strategies are completely different.

The companies which develop the processors understand this
problem very well, and try to port and optimize as many important
application programs as possible, and in some cases such effort proved
to be successful. The problem of this approach is that it requires huge
amount of resources, in particular human resources, much more than
what is necessary to develop hardware. The success of an architecture
depends primarily on the amount of human resources available for
application development.

\subsection{Portability by language?}
In principle, if we could develop portable application programs, which
can achieve high efficiency on various architectures without
significant rewriting, that would be the solution for this
problem. The traditional approach in this direction is the standard
language and/or libraries for parallel execution. Unfortunately,
though the parallel languages have been and still is the target of the active
research, they have  never widely used with a handful of exceptions such
as CM-Fortran (and later HPF). CM-Fortran was the language for TMC
CM-2 and CM-5, and HPF was one of the practical choices on
distributed-memory vector-parallel machines such as Earth Simulator.

The problem with parallel language like HPF is that it's parallel
operation is defined at the  level of arithmetic operations on 
elements of distributed arrays. Thus, the generated machine code tends
to loop over large arrays, and there is no easy way to make use of the
cache hierarchy.


Modern approaches are limited to parallelization within one node,
and there are several proposed open standards such as OpenCL, OpenACC
and OpenMP. OpenCL is designed for  the host-accelerator architecture
with separate memory spaces, and the data transfer between the
host and the accelerator is controlled explicitly by the application
programmer. Both OpenACC and OpenMP support both shared- and
separate-memory architectures, and rely on directives to implicitly
specify the data transfer. Here, again, even when the application
program is written with an open standard supported by different
hardware platforms,  to achieve high efficiency
with a ``portable'' code is difficult, since in order to achieve high
efficiency architecture-specific optimizations are necessary.

\subsection{Portability by library/framework}


We have been working on a different approach. Instead of trying to
provide a solution for all application areas, we limit ourselves to
particle-based simulation programs, since they have a sufficiently wide
range of actual applications in many areas of science and
engineering. In many particle-based applications, the interaction
between particles can be expressed  as 
\begin{equation}
    \myvec{F}_{i} = \sum_j^N \myvec{f}  (\myvec{u}_{i}, \myvec{u}_{j}),
      \label{eq:basic}
\end{equation}
where  $N$ is the number of particles in the system, $\myvec{u}_i$ is
a vector which represents the state  of  particle $i$, 
$\myvec{f}$ is a function which describes
the contribution of particle $j$ to the total ``interaction'' on
particle $i$, and $\myvec{F}$ is the total
interaction used to update the state of particle $i$.

In the case of the gravitational $N$-body simulation, $\myvec{u}_i$
contains position, velocity, mass, and other parameters of particle
$i$, $\myvec{f}$ is the gravitational force from particle $j$ to
particle $i$, and $\myvec{F}_i$ represents the gravitational
acceleration on particle $i$ from all other particles in the system.

Our software, which we call FDPS (Framework for Developing
Particle Simulator\cite{Iwasawaetal2016}), provides functions necessary for large-scale
parallelization of particle-based simulations. The basic functions are
\begin{itemize}

  \item domain decomposition
  \item exchange of particles between nodes
  \item interaction calculation  
\end{itemize}  

FDPS is designed as a template library in C++ language, so that
it can accept the data structure of particle as a class definition 
and the function to calculate interaction between particles in written
in C++. We have extended the FDPS so that now the struct in C language
and interaction function written in C can be used. Thus, it is now
possible to use FDPS from any language which can call functions
written in C.

A parallel particle-based simulation program developped using FDPS works roughly as follows

\begin{enumerate}

\item Initial setup, including the generation of the initial condition
  for particles. Particles are stored in the usual array of particles
  allocated in the user memory. At this moment, particles can be in
  arbitrary MPI processes. 
\item Generate the domain decomposition by calling FDPS functions.
\item Exchange the particles so that they are in their appropriate MPI
  process by calling FDPS functions..  
\item Evaluate the interaction between particles by calling FDPS
  functions..
\item Update the data of particles and evaluate other necessary quantities such
  as total energy and  other diagnostics, next times etc.
\item Go back to step (2) if the termination time has not been reached.
  \end{enumerate}

Note that operations other than the three operations listed above are
done in the usual user-written code. Thus, I/O, time integration,
diagnostic output are all done in the user code, to which 
data of particles are accessible as  array elements.

%% Initially, user
%% program must be written in C++, but we have added interfaces to both C
%% and Fortran. The user can define particles as a struct of C or derived
%% date of Fortran, and the interaction calculation function can also be
%% written in C or Fortran. It is even possible to call FDPS from  any
%% language, if it has FFI through which C  struct can be manipulated.

At first sight, 
the requirement that the problem should be written in the form of
equation (\ref{eq:basic}) might look too restrictive. Contributions
from different particles must be linearly added, and interactions on
all particles should be calculated at each step.
In practice, these are not a severe restriction, since  many problems
for which large-scale parallel calculation is currently applied
satisfy the above conditions. FDPS is now being used by many
researchers to develop their own high-performance parallel codes for
their problems. 

In the rest of this paper, we discuss what should be done to achieve
high efficiency for particle-based simulations on modern and future
HPC systems, based on our experience on evaluating and optimizing
FDPS-based programs on TaihuLight,  GYOUKOU, and NVIDIA GPGPUs.
As discussed earlier, the potential bottlenecks for the performance
are the bandwidth of main memory, interconnect and host-accelerator
connection. 
In section \ref{sec:communicationcost}, we discuss the cost of
internode communication and  possibilities to further reduce them.
We will see that the limiting factor for the scalability to very large
number of processes is that there are operations whose cost {\it
  increase}  as we increase the number of processes $p$. Their cost is
negligible when $p$ is small, but becomes dominant when $p$ is large,
in particular when it becomes  much larger than the number of
particles per node $n$. We discuss if we can make the cost of
operations less than $O(p)$, ideally $O(1)$ but in the worst cases
$O(p^{1/d})$, where $d$ is the number of dimensions.
In section \ref{sect:singlenode}, we discuss the limiting
factors which come from the node architecture,  such as the main memory bandwidth
and host-accelerator communication bandwidth, and overview the
algorithms we developped and implemented.
In
section \ref{sec:performance} we overview the achieved performance on
several systems. Section
\ref{sec:summary} is for summary and future directions.


\section{Reduction of the communication cost}
\label{sec:communicationcost}

In this section, we discuss the communication cost of 
parallel particle-based simulation code. 
We first consider the  standard procedure used  in FDPS, in which the
Barnes-Hut tree algorithm  is used, and we discuss mainly the
long-range interactions. For simplicity, we assume that the
distribution of particles is not too far from uniform. One advantage
of FDPS is actually that it can handle highly inhomogenous particle
distributions without significant increase in the calculation time,
because of its use of the adaptive tree structure and adaptive domain
decomposition. Thus, estimation of calculation time under the
assumption of uniform particle distribution is not too bad.

\subsection{Standard procedure}
As discussed in Introduction, the basic steps for parallel
particle-based simulation code on distributed-memory parallel
platforms are


\begin{enumerate}
  

\item Domain decomposition
\item Particle exchange 
\item Interaction calculation
\item Miscellany operations such as I/O, time integration and so on.
  \end{enumerate}

In the following, we overview the algorithms and performance
characteristics of these procedures.


\subsection{Domain decomposition}
\label{sect:domaindecomposition}
FDPS uses the 3D multisection algorithm \cite{Makino2004} modified to
achieve better load balance \cite{Ishiyamaetal2009b}. The multisection
algorithm is the generalization of the orthogonal recursive bisection (ORB\cite{Salmon1990}). As its name suggests, ORB divides the computational
domain recursively to two subdomains, in $x$, $y$, and $z$
directions. In the multisection method, the division in one dimension can
be of arbitrary size, and we divide the domains only once per one
dimension. Thus, instead of recursively divide domains $\log_2 p$ times,
where $p$ is the number of MPI processes, we divide the domains three
times for three-dimensional tree. The main advantage of the
multisection method is that it can be used for number of MPI processes
not an integer powers of two.

Early versions of FDPS uses the sampling method
\cite{BlackstonSuel1997} to determine the coordinates of
subdomains. In the sampling method, all MPI processes send randomly
sampled subsets of their particles to the rank-0 process (or root
process), and the rank-0
process performs necessary calculations to determine the coordinates of
subdomains.

This sampling method works fine for thousands or even tens of
thousands of MPI processes, or  as far as the number of particles in
one MPI process is significantly larger than the number of MPI
processes. Since the root process receives sampled particles from all
other processes, if the number of particles per process becomes less
than the number of MPI processes, the amount of data the root
processes receives becomes more than the amount of the particle data
itself of the root process, and the time for the internode
communication would become visible.

In the current version of FDPS, we adopted the hierarchical sampling
\cite{Iwasawaetal2016,Iwasawaetal2019barxive} in which the communication cost
is  $O(p^{1/d})$, where $d$ is the number of
spatial dimensions. The basic idea here is to re-sample particles at
each dimension so that at each dimension the number of sample
particles received is proportional to $p^{1/d}$.

Even with this scheme, the communication cost is still $O(p)$, where
$p$ is the total number of MPI processes, since all processes receives
the physical dimensions of subdomains and for very large number of
process, this $O(p)$ term can be the limitation of the scalability.

We therefore need a new algorithm in which the amount of data received
by each process is less than $O(p)$. If we keep using the 3D
multisection algorithm, one possibility is that each process has only
one division data in each dimension. The division in x direction is
shared by all processes, but that in y direction is shared only by
processes in the same x coordinate in the processor grid, and z direction
only by in the same x-y coordinates. This change reduces  the amount of
data received by each process from $O(p)$ to $O(dp^{1/d})$, where $d$
is the number of dimensions. With this change, however, each no longer
knows the physical dimensions of other processes, and cannot directly
determine the destinations of particles moved or the LET.
This scheme is not implemented yet in
FDPS but in the following we also discuss how algorithms should be
changed and how the communication cost scales with this scheme.

%2019/12/13


\subsection{Particle exchange}

The standard algorithm used in FDPS to exchange particles is simple
and straightforward. After the new coordinates of all subdomains are
determined and shared by all MPI processes, each process determines
for each particle if it is still in its subdomain or not, and if not,
which process it should be sent. Then with one call to
MPI\_alltoall, each process tells all other processes how many
particles it wants to send, and then using MPI\_alltoallv sends
actual particle data. This scheme  works fine when the number of MPI
processes is small, and works reasonably well for around 100K nodes (K
computer) if (a) the hardware vendor
provides  rich interconnect and   fast
implementation of alltoall(v) and (b) calculation time
for one timestep is large, order of several seconds or more. However,
we want to solve problems of practical size on big machines, and thus
the calculation time per one timestep should be much shorter, much less
than one second. However, even with ideal implementation the
throughput of the alltoall operation is determined by the bisection
bandwidth, which does not scale on very large machines. Moreover, the
actual performance of alltoall for short messages can be extremely
low, while the total amount of data need to be sent/received is large
simply because the number of MPI process is large. Thus, it has become
critical to eliminate  alltoall(v) operation completely.

For the particle exchange operation, though in principle particles can
move from any node to any other node, in practice the actual movement
is between subdomains which are physically nearby, and thus with our
multisection method between nearby nodes in the three-dimensional
process grid. Our current implementation makes use of this fact. For
each of six directions ($\pm x, \pm y, \pm z$), we first determine the
largest distance particles move in the process grid, and then we loop
over all possible relative displacement within the process grid. If
the maximum distance is independent of the total number of the
processes, the communication time would not increase when the number
of processes increases.

There are many other possible implementations of
the particle exchange which might be more efficient on some types of
interconnect. For example, consider the following algorithm.
If process $(i_1, j_1, k_1)$ wants to send its particles
to process $(i_2, j_2, k_2)$, it first sends its data to
process $(i_2, j_1, k_1)$. Similarly, all processes sends all their
data to $\pm x$ directions, so that all data transfers in the direction
of the  $x$ axis are finished. Then all processors send data in $y$
direction, and then in $z$ direction. In practice, on machines with
direct torus network such as K and Fugaku, we should consider sending
data in multiple directions in parallel, to make more efficient use of
the communication links. With this algorithm the total number of point-to-point
communications is at the maximum $O(p^{1/3})$ and is usually much
smaller, and thus fairly good performance might be achieved.

When the scalable implementation of the domain decomposition in which
each node does not have all information of subdomains is used, the
above algorithm would still work without problem.

\subsection{Interaction calculation}
\label{sect:comm:interaction}

In FDPS (and also in many other parallel particle-based simulation
codes), Barnes-Hut tree algorithm\cite{BarnesHut1986} is used for the
calculation of long-range interaction. In some codes
FMM\cite{GreengardRokhlin1987} is used, but as far as the
parallelization strategy and algorithm to exchange the necessary data
between processes are concerned, there is no essential difference
between the tree algorithm and FMM. Therefore, what is discussed here
applies to FMM as well. In fact, we are currently working on the
integration of a variation FMM into our FDPS framework.

The parallel version of Barnes-Hut tree algorithm used in FDPS
consists of following steps

\begin{enumerate}

\item Each process constructs its local tree.

\item Each process (process $i$) determines, for each of all other
  processes (process $j$),  the information of the tree of  process
  $i$  that process $j$ needs to evaluate the total interaction on particles
  in process $j$. This can be done locally since each process already
  know the geometry of other subdomains. This information is called
  ``local essential tree'', or LET  \cite{WarrenSalmon1992}.

\item Each process receives the LETs from all other processors.

\item Each process constructs the ``global'' tree using LETs it received.

\item Each process evaluates the interaction on its particles using
  the global tree.

\end{enumerate}

We can see that the internode communication takes place only at step
3. In the following, we discuss the several possible implementations
of step 3 and their performance characteristics.

A simple and straightforward implementation would just use
MPI\_alltoallv for LET exchange. As in the case of particle exchange,
alltoall communication should be avoided since even with rich
interconnect and highly optimized software, it would be limiting factor
of the performance if the number of MPI processes is not much smaller
than the number of particles in one MPI process.

Unlike in  particle exchange, where processes would actually
exchange particles only with nearby processes, in LET exchange all
processes have to receive some information from all other processes. Thus,
at first sight alltoall communication looks unavoidable. However,
actually there are several approaches to avoid it. The following is
the list of known schemes. 

\begin{description}
\item{a)} Use TreePM 
\item{b)} Use allgather for distant communication
\item{c)} Construct higher-level inter-process tree
 \end{description}

We briefly discuss each of them below.


\subsubsection{TreePM }

When the periodic boundary condition is used, FFT is the natural way
to solve the Poisson equation, and the TreePM
algorithm\cite{Bagla2003} has been widely used. The basic idea of
the TreePM method is to replace the particle-particle part of the $\rm
P^3M$ scheme\cite{Eastwoodetal1984} by the tree algorithm, so that the
calculation cost would not increase significantly when the
distribution of particles becomes highly inhomogeneous. Also, since
the PP part is replaced by the tree algorithm, the calculation cost of
the tree part depends only weakly on the cutoff radius, and thus the
necessary number of grid points for FFT is relatively small. 

Consider the case where we have $p$ processors, with $n$ particles
each, and use $g$  grid points per process for FFT. Theoretically, the
communication efficiencies of both the LET exchange and FFT are
limited by the bisection bandwidth, and the amount of data passed in
one dimension is $O((np)^{2/3})$ and $O(pg)$. Therefore,
to make the cost of FFT smaller than LET exchange,  $g$ should satisfy
\begin{equation}
  g \le \frac{n^{2/3}}{p^{1/3}}.
  \end{equation}
Since we are dealing with very large systems with $10^5$ or even
$10^6$ MPI processes, for most of practical applications $n<p$, and
thus $g$ should be very small. The fact that $g$ is small implies the
cutoff length is significantly larger than the size of subdomains
assigned to MPI processes, and thus the processes should still
communicate with a fairly large number of processes. Even so, the
alltoall communication can be eliminated.


\subsubsection{Use allgather for distant communication}
\label{sect:allgatherlet}
The idea here is that when a very large number of processes is used,
many subdomains would actually be regarded as points (or single
multipole expansions). Thus, even in principle the LETs one process
should send to other nodes are different for different nodes, many of
them receive the single expansion, which is the same for all of
them. Thus, we can replace most of the transfers of LETs by single
allgather operation, through which each node receives the top-level
multipole expansions of all other nodes. Then, each node send LETs
only to nodes which require lower-level tree structures. Since in our
current algorithm all processes know the physical dimensions of all
other processes, we can use usual send/receive pairs to guarantee that
all data are exchanged correctly.

With this scheme, only global communication is through allgather, and
thus its bandwidth is not limited by the bisection bandwidth but by
the injection bandwidth, and thus the scaling in the case of ideal
implementation is much better. In the case of naive alltoall
implementation, the necessary time is $O(p^2/b_{\rm bisect})$, where
$b_{\rm bisect}$ is the bisection bandwidth. With allgather,
it is $O(p/b_{\rm inject})$, where $b_{\rm inject}$ is the effective bandwidth of the
network port of a single MPI process. Thus, even when we have the
full-bisection network where $ b_{\rm bisect} = pb_{\rm inject}$,
the allgather-based implementation would show comparable or better
performance than alltoall-based, and on large machines with
$ b_{bisect} \ll pb_{inject}$, the allgather-based algorithm is much
better.

However, this algorithm still has the problem that the amount of the
data received by each process is $O(p)$ and the cost of tree
construction is  $O(p \ln p)$. If $p\gg n$, which is likely the case
for large systems, the single allgather operation can be the limiting
factor of the scalability, as in the case of the domain decomposition.
We will address this issue in the subsections below.


\subsubsection{Construct higher-level inter-process tree}

As seen in section \ref{sect:allgatherlet}, the problem with the
LET scheme is that when the inter-process communication takes place,
there is no information concerning the structure higher than those of
local trees, and that information of higher levels of the tree  is
constructed at each process redundantly. Thus, if we construct the
global higher-level structure and let each process access them, in
principle the amount of data received by each process can be reduced.

With the multisection method, it is not clear how we can make
higher-level tree. One possibility is to map eight neighboring
subdomains to one higher-level tree node and apply that process
recursively to top level.

One problem with this approach is that if the data of one tree node in
this higher-level tree exists in one MPI process, that process will be
accessed by many processes, and the communication bottleneck is not
removed. This can be avoided if we use FMM or at least FMM-like
communication pattern, in which tree nodes communicate only with the
nodes in the same level.  for all nodes the amount of data transfer
can be independent (or only $O(\log p)$). In this scheme, LET is built
at each level of the tree, and the information of the distant nodes is
not directly sent but sent through high-level nodes.

\subsubsection{Combination with new domain decomposition scheme}
\label{sect:localizedlet}
If we use the scalable algorithm for the domain decomposition
discussed in section \ref{sect:domaindecomposition},
we need to modify the LET based scheme, since one process no longer
has the geometries of other nodes. We can use multi-stage LET
algorithms, in which the LETs are constructed step by step.

% 20191214

First, each process construct LETs, not for each processes but for all
``slabs'' along the x directions (slabs  extends to y and z
directions). Now each node $(i,j,k)$ has LETs from node $(1, j, k)$ to
$(n_x, j, k)$, where $n_x$, $n_y$, and $n_z$ are number of divisions
in x, y, and z directions, respectively. Then each node constructs the
tree from both its local particles and received LETs. Here, the
opening criterion of the tree cells must be for the slabs and not for
its subdomain.   Now, each slab has the
information of the entire system. So the communication in x direction
will not occur. In the second stage,
we will do the same thing for ``columns'', created by cutting the
slabs in y direction, and do the same thing: construct LETs for all
other columns in the same slab and send them, and reconstruct the tree
by adding new LETs. Finally, do the same thing in z direction. After
this procedure each process has the information of entire system.

This scheme does remove the $O(p)$ part of the LET algorithm and
reduce the total number of communications from $O(p)$ to
$O(dp^{1/d})$. One drawback of this scheme is that the tree need to be
constructed four times per timestep: initial local tree and 
communications in x, y, and z directions. However, the dominant part
of the calculation cost of the tree construction is sorting, and for
global trees the sorting cost can be made practically $O(n)$ since the
local particles are already sorted, or even smaller if we keep the
local tree and tree for the global information separate. Therefore, we
believe the additional cost is acceptable and the gain by removing the
$O(p)$ communication is larger.


\section{The effect of node architecture}
\label{sect:singlenode}

In this section we discuss the limiting factors of the node-level
performance of particle-based codes on present and near-future
machines and new algorithms to improve the performance. The factors we
need to consider include the following factors.
\begin{enumerate}

  \item Main memory bandwidth for sequential and random accesses.
 \item In the case of accelerators, communication bandwidth between
   the host and the accelerator.
   \item Also in the case of accelerators or heterogeneous
     architecture, the ratio of host performance
    and accelerator performance.
\end{enumerate}
In the following we discuss these factors.

\subsection{Main memory bandwidth}

For many large-scale simulation codes, the effective main memory
bandwidth tends to be the limiting factor of the efficiency.  Let us
first derive the theoretical minimum necessary bandwidth to achieve a
reasonable efficiency for particle-based simulation codes, assuming
that we are using explicit time stepping,.

As in section \ref{sec:communicationcost}, we consider the procedures
for domain decomposition, particle exchange, and interaction
calculation using Barnes-Hut tree algorithm.

\subsubsection{Domain decomposition}

As far as the memory access is concerned, this part is negligible
since the calculation time here is dominated by the communication
bandwidth which is much smaller than the memory bandwidth.

\subsubsection{Particle exchange}

For all particles we need to check if it is still in its
subdomain. Thus, we read the data of all particles a least once per
timestep.

\subsubsection{Interaction calculation}


As discussed in section \ref{sect:comm:interaction}, the LET-based
parallel Barnes-Hut tree algorithm with Barnes' vectorization for
interaction calculation consists of the following steps:

\begin{enumerate}

\item Local tree construction
\item LET exchange
\item Global tree construction
\item Creation of the interaction lists
\item Interaction calculation using the interaction lists  
\end{enumerate}  

The dominant part of the tree construction is the sorting of particles
using keys. With any algorithm, we need to read  each particle at
least once to make its key, and write once to store the sorted result.
With our current implementation the cost of sorting is much higher
\cite{v2plus}, and that implies there is rooms to improve algorithms,
We ignore the cost of LET exchange, since that part is smaller
compared to the actual interaction calculation.


The cost of the construction of the  interaction lists depends on many factors, in
particular the choice of the parameter $n_g$, the maximum number of
particles to share the same interaction list. We should
chose this parameter so that the total calculation time is minimized,
and that means the optimal choice of this parameter depends on the
processor architecture. To make rough estimate, let us assume that the
length of the interaction list is $n_{0}+n_g$, where $n_0$ is a constant
which depends on the required accuracy, and the ratio between
the calculation time of one interaction and that for adding one entry in the
interaction list is $X$. The total  cost of interaction list and
interaction calculation per particle in unit of the time for single
interaction calculation  is then given by
\begin{equation}
  C = X(n_{0}+n_g)/n_g + n_{0}+n_g,
\end{equation}
and the optimal value of $n_g$ is given by
\begin{equation}
 n_g = \sqrt{Xn_{0}}.
\end{equation}
If we assume that the data of one particle is around 30bytes and
single interaction calculation 30 floating-point operations,
we have
\begin{equation}
  X = 1/b,
\end{equation}
where $b$ is the B/F number  for the random access of the main memory, which on modern
machines around 1/30 or less of the B/F number for the sequential
access. Thus, for a machine with B/F=0.03, we have $X=1000$, and that
means the optimal value of $n_g$ would be around 1000. This is,
however, quite a bit too large, since such a large value of $n_g$
results in the significant increase of the total calculation cost. We thus
need some new approach to reduce the main memory access.

One method to reduce the main memory access is to use the same tree
structure and interaction lists for multiple time steps. This method,
which we call the reuse method, turned out to be quite effective on
all of modern platforms we have so far tested, and helped us to
achieve near-peak performance on all of them.

With the reuse method, the necessary  memory access per
particle are one for local tree (momentum update),
two for global tree (reordering and /momentum update),
and a few times for the construction of the interaction list and
interaction calculation. Thus, the amount of memory access per
particle per timestep is less than ten times, and most of them are
fast, near-sequential access. Thus, the ratio between the memory
access time and the calculation time is now given by
\begin{equation}
  R = \frac{10}{bn_0}.
\end{equation}
Since $n_0$ is of the order of 1,000, roughly speaking the reuse
method would work for systems with $\rm B/F > 0.01$.
Right now all HPC processors provide $\rm B/F > 0.02$, but in near
future we might see machines with $\rm B/F < 0.01$, and how we can
make use of such systems is an important question.

The absolute minimum of the amount of the memory access necessary per
timestep per particle is one read and one write. Therefore, there is
still rooms of improvement by a factor of five.

The reuse method is quite effective when we can use it. However, there
are physical systems for which the reuse method does not work. How we can
minimize the main memory access is an important research direction for
such systems.  One possibility is to make better use of on-chip
memories.  If we divide the systems not just by the number of
processes but to smaller blocks which can fit into on-chip memories of
processors, in principle we can reduce the main memory access by a
large factor by processing one block at a time. In a naive
implementation we would still need one read for particle exchange and
another read for LET, and yet another read/write for the interaction
calculation and time integration.  Thus, we could reduce the memory
access to three reads and one write per particle per timestep. Such
implementation would be advantageous on machines of near future.

\subsection{Host-Accelerator communication}

The standard way to use accelerators in FDPS or other large-scale
parallel particle-based simulation codes is to use them only for
the interaction calculation using the interaction lists. The amount of
data transferred between the host and accelerators is essentially one
read and one write per timestep per particle, since interaction list
can be the list of the indices of particles and thus its total size in
bytes is smaller  than that of particles. Very roughly, unless the
interaction is of very short-range nature and inexpensive, for
host-accelerator interface the necessary B/F number is around
0.002. Unfortunately, interface bandwidth of modern GPUs are
approaching to this value. For example, NVIDIA Tesla V100s, used with
PCIe gen3 interface, has the interface B/F number of around 0.003. 

If all  particles of one  process, or those processed by one
accelerator card, can fit to the memory of the accelerator card, we can
in principle reduce the necessary interface bandwidth in several ways,
at least when the reuse method is used.

When we use the reuse method, it is straightforward to move all
operations except for communications  in reuse steps to the accelerator
side, since they can be expressed in  simple loops without conditional
branches. In this case, the necessary communication is only for LETs,
and if the number of particles per accelerator card is large enough,
we can reduce the necessary bandwidth of host-accelerator
communication by a large factor.

Another possibility is to use some form of data compression for
the host-accelerator communication. Since the necessary bandwidth of
the host-accelerator communication so small, we can pay fairly large
calculation cost for data compression and restoration. On the other
hand, efficient compression is not easy for the data from numerical
simulation, in particular when it is being used for simulation. One
possibility is to construct the same ``predictor'' on both side, and
to send only the correction terms. Whether or not such a strategy
works or not depends very strongly on the nature of the system and
the methods used for the interaction calculation and time integration.
When we use the reusing method, we can expect that the orbits of
particles are generally smooth and predictable, since otherwise the
reuse method would not work. 

In the case simulation of galaxy formation, for typical galaxies, we
will use the timestep of around $10^5$ years for particles which
represent stars and gas. Their dominant motion is the orbital motion
within the galaxy with the timescale of $10^8$ years. Thus, if we can
construct second-order prediction polynomial, its local relative
truncation error is around $10^{-9}$. For the interaction calculation,
this accuracy might be already sufficient and we may be able to send
only a small number of bits as the correction terms of the prediction,
and we should be able to apply a similar procedure to the calculated
interaction. Also, similar consideration can be applied to other
systems like planetary rings and  protoplanetary nebulae.

In prediction we are using the data redundancy in the temporal
domain. It should also be possible to use the redundancy in the spatial
domain. For example, the difference of Morton keys of near neighbors
is usually small, and thus we can compress the sequence of sorted
Morton keys by taking the difference of two consecutive keys and use
variable-length integer format, or by using any general-purpose
compression algorithms. The advantage of this algorithm is that it
does not require that the orbits of particles are smooth.

\subsection{Host-Accelerator Performance Difference}

We can regard both of  machines with host-accelerator architecture and
heterogeneous manycore architecture as consisting of small numbers of
general-purpose cores and large number of specialized cores. In the
case of GPGPU systems, the general-purpose side is usually Intel x86
processors, and the difference of their theoretical peak performance
is usually not very large. Typically around ten or so and only in  some
extreme systems the ratio exceeds 30.  In the case of heterogeneous
many-core systems, This  ratio is generally much larger, such as 64
for Sunway 26010 processor, which has 64 CPEs per  MPE. 
When we ignore the limitation from the host-accelerator communication
bandwidth, if this ratio is not very large, like ten or less, it makes
sense to use accelerator only for the interaction calculation, since
the calculation cost of the rest of the calculation is not very large
and it can be performed on the host computer without affecting the
overall efficiency too much. However, if this ratio is very large,
such as a factor of 100 or more, practically all calculations must be
done on the accelerator side, even when we use the reusing method.

In many cases, we need to rewrite the application program to move part
of calculations from host to accelerator, or from general-purpose
cores to specialized cores.  With frameworks like FDPS, we can hide
these architecture-specific codes in the FDPS framework and keep the
application program machine independent. At present, we have not yet
reached this goal, but we are working in this direction.

%201912141733

\section{Achieved performance}
\label{sec:performance}

In this section, we briefly discuss the achieved performance of
FDPS-based applications on three platforms: Sunway Taihulight,
GYOUKOU, and NVIDIA V100 GPGPU. For the first two machines,
the performance numbers are obtained with several machine- and
problem-specific optimizations not included in the public release of
FDPS yet. The code used for NVIDIA V100 is in
github\footnote{https://github.com/FDPS/FDPS}.
The operation of GYOUKOU was terminated on March 31st, 2018. We have
made further optimization of our calculation code and measured the
final performance numbers on a smaller system, Shoubu B, with 512
PEZY-SC2 chips.



\subsection{Test problems and optimizations used}

For Taihulight and GYOUKOU, we developped  the code for large-scale
simulations of planetary rings, such as Saturn's ring. The realistic
simulation of inner rings of Saturn can be done using $10^{11-12}$
particles, and such simulation is becoming feasible with the peak
speed of 100 PF or more. For NVIDIA V100, we report the single-card
performance for the simple three-dimensional cold collapse problem.

\begin{table}
\caption{Optimizations used}
\label{tab:optimizationss}       %
\begin{tabular}{lccc}
\hline\noalign{\smallskip}
& Taihulight & GYOUKOU & V100\\
\noalign{\smallskip}\hline\noalign{\smallskip}
$O(p^{1/d})$ domain decomposition & Yes & Yes & Yes \\
Higher-level tree* & Yes & Yes & - \\
LET exchange without alltoallv  & Yes & Yes & (Yes) \\
Accelerator for all calculations & Yes & Yes & No \\
Interaction list reuse & Yes & Yes & Yes \\
\noalign{\smallskip}\hline
\end{tabular}
\end{table}

Table \ref{tab:optimizationss} shows the algorithms used on each
platform. Not all algorithms discussed in sections
\ref{sec:communicationcost} and  \ref{sect:singlenode} have been
actually implemented. 
The domain decomposition without $O(p)$ communication is not complete
yet, since the current implementations  for Taihulight and GYOUKOU are
specialized to simulations of planetary rings. Thus, improvements
related to this change are not available in the released versions
of FDPS yet. However, LET exchange without alltoallv is available in
the current FDPS. For the benchmark on V100 this feature is not used
since we measured single-node performance only.
Also, in the code for Taihulight and GYOUKOU, calculations other than
interaction calculation using the interaction list are moved to
accelerator (CPE in the case of Taihulight and PEZY-SC2 processors on
GYOUKOU). These porting are relatively easy but requires 
machine-specific codes, since we need to make use of special
features of these processors to achieve acceptable performance. Such
porting is a bit difficult on GPGPU and has not been done yet.



  
\subsection{Performance Results}

Detailed performance and scalability numbers are given in
\cite{v2plus,Iwasawaetal2019barxive}. Here we present ``best'' numbers achieved so far.
On Taihulight,
the measured performance for the run with $1.6\times 10^{12}$ particles on 160k
processes (40000 nodes) of TaihuLight is 47.9 PF, or 39.7\% of the
theoretical peak performance.  On
PEZY-SC2 based systems, we achieved 10.6PF for $8\times 10^{9}$
particles on 8K SC2 chips, or efficiency of 23.3\% of the theoretical
peak performance. On 512-chip Shoubu System B, we achieved the speed
of 1.01 PF, or 35.5\%.  In all cases, the number of particles per MPI
process is 10M. The calculation time per timestep is 2.88, 0.822 and
0.525 second, on Taihulight, GYOUKOU, and Shoubu B. Note that the
difference of performance between GYOUKOU and Shoubu B is not due to
the scalability limitation but purely due to the difference in the
calculation code used. As we stated earlier the operation of GYOUKOU
was terminated before we completed the optimization.


Table \ref{tab:time} shows the breakdown of calculation time. On both
platforms, we can see that the interaction calculation dominates the
total calculation time, and the fraction of time spent for
communication is relatively small, less than 1/10 of the total time.
We should admit that part of the reason why the communication time is
small is that we simulate planetary rings which is very thin, and thus
communication is effectively limited to x and y directions. Thus, the
surface area of one subdomain, which is usually proportional to
$n^{2/3}$ where $n$ is the number of particles per domain, is in this
case proportional to $n^{1/2}$, and the amount of communication is
small.  However, even so, efficient calculation on these machines
would have been impossible without the new algorithms discussed in
previous sections. Thus an important guideline for extreme-scale
parallel calculation is to avoid any global communication or anything
whose calculation time is $O(p)$, even if the coefficient is very
small.  In particular, the use of communication pattern of alltoall(v)
must be eliminated since it will result in $O(p)$ execution time, even
when the message size is very small, and should be replaced with
communications with execution time at the maximum $O(p^{1/d})$ or
ideally $O(1)$.



\begin{table}
\centering
  \caption{Breakdown of calculation time}
  \label{tab:time}
  \begin{tabular}{lcccc}
    \hline
 System  & \# processes & interaction & comm. & others\\
    \hline  
TaihuLight&160000 &  2.31& 0.090& 0.476\\
Gyoukou&8192   & 0.453& 0.147&     0.222\\
Shoubu B         &512& 0.360& 0.030&     0.135\\
\hline
\end{tabular}
\end{table}




In terms of the number of particles integrated per second, we have
achieved $5.5\times 10^{11}$ particles per second on Taihulight, which
is more than 10 times faster than the results of previous works on K
computer\cite{Ishiyamaetal2012} or ORNL Titan
\cite{Bedorfetal2014}. The number of particles used is similar to that
used on K computer\cite{Ishiyamaetal2012}, and that means the
calculation time per one timestep is reduced by more than a factor of
10. 

Table \ref{tab:gpgputime} shows the calculation time per timestep for
the cold collapse calculation with 4M particles on single NVIDIA V100
card with Intel Xeon E5-2670 v3 host CPU. Here, ``GPU without index''
denote the use of interaction list with physical quantities of
particles in the list. Thus, communication cost is large. 
With ``GPU with index'' algorithm, the data of particles are sent only
once per timestep. Thus, the communication time is reduced, but the
calculation time becomes somewhat longer. With reusing, however, the
calculation time becomes much shorter. The number of reusing in this
case is 16. In this case, the effect of reusing is not the reduction
of the communication cost but mostly the reduction in the calculation
cost of operations still done on CPU. The achieved performance is
around 40M particles per timestep. Currently, we are sending particles
for tree data (x,y,z and mass) and particles which receive force
(x,y,z only) all in single precision. Thus, the data sent per particle
is 40 bytes. Thus, the data sent to GPU in one second is only 1.6GB,
while the bandwidth of CPU-GPU connection is close to 15GB/s.




\begin{table}
\centering
  \caption{GPGPU performance}
  \label{tab:gpgputime}
  \begin{tabular}{lc}
    \hline
Algorithm  & time per timesteps(sec)\\
\hline
CPU only & 0.98\\
GPU without index & 0.36\\
GPU with index & 0.42\\
GPU with reuse & 0.095\\
\hline
\end{tabular}
\end{table}


\section{Summary and Future directions}
\label{sec:summary}

\subsection{Lessons Learned}

In this paper, we overviewed the current status of our effort to make
large-scale modern HPC platforms usable for large-scale particle-based
simulations. The difficulties include extremely large number of cores
and MPI processes, small memory bandwidth, even smaller network
bandwidth. Also, in the case of accelerator architecture and/or
heterogeneous many-core architectures, very large ratio of performance
of accelerator cores and general-purpose cores, and small
host-accelerator communication bandwidth.

What we have observed is that it is not impossible to design the
framework, not a specific application, for particle-based simulations
so that the applications developed using that framework can achieve
high efficiency on modern HPC platforms.

In order to deal with the extremely large number of cores, it is the most
important to eliminate global communications, in particular those with
the $O(p)$ term in the communication cost. Examples of such
communications are MPI\_alltoall(v) and MPI\_allgather(v).
This means no single process should communicate with all other
processes, even if it is the rank-0 node.

In our case, such $O(p)$ communications were originally used in domain
decomposition, particle exchange  and LET exchange. We have replaced them with
$O(p^{1/d})$ communications.

Compared to mesh-based simulations, where the memory bandwidth tends to
be the bottleneck, particle-based simulations do not require very
high memory bandwidth, since the calculation cost of particle-particle
interaction is large and thus calculation is not very memory
intensive. Even so, with B/F numbers of 0.01 or less, we need new
approaches including the interaction list reuse. With the reuse
algorithm, the number of memory access per particle per timestep is
still of the order of ten. On the other hand, the lower bound of the
main memory access is one read and one write per timestep. Therefore,
it is at least theoretically possible to reduce the necessary
memory bandwidth by another factor of 10, so that we can use
machines with B/F $\sim 0.001$.

To reduce the necessary communication bandwidth is also quite
important. The necessary communication bandwidth, however,  depends
strongly on the target physical systems and difficult to provide
universal solutions. Solutions specific to physical systems and also
to network architecture of the machine will be necessary. 

First let us consider the communication due to migration of particles.
For a nearly uniform distribution of particles, we can
expect that particles move a small fraction of the average
interparticle distance in one timestep. Thus, the number of particles
which migrate from one subdomain to other (usually
neighboring) subdomains is $O(p^{2/3})$ with relatively small
coefficient, and that means the required network bandwidth is much
smaller than the required main memory bandwidth. However, in many
astrophysical simulations the situation is quite different. 

For example, systems like planetary rings, protoplanetary disks, disk
galaxies have the nature that the global rotation velocity is much larger
than the local velocity dispersion. Thus, if we have the domain geometry
fixed to the inertial frame, particles moves the distance much larger
than the typical interparticle distance, and with very large number of
domains, it can occur that all particles in one domain moves to other
domains at every timestep. It is clear that the network bandwidth
would limit the scalability. In our simulation of planetary rings
discussed in section \ref{sec:performance}, we introduced rotating
reference frame so that the circular motion of particles do not cause
migration of particles.  In addition, we adopted the domain geometries
defined in cylindrical coordinates.  This approach is quite effective
for narrow rings, where the range of the angular velocity is
small. However, for wider rings or disks, this strategy does not work
since the angular velocity can be quite different. If the topology of
the interprocessor network is mesh, there is no simply way to reduce
the communication. However, in the case of networks with fat-tree or
similar topologies, we can reduce the communication by let each ring of
domains, rotate at their local rotation velocities. In the case of the
fat-tree network, it is possible to maintain the bandwidth of
communication between two rings of domains with different rotation
speeds, by assigning low-level trees to rings.

The communication for interaction calculation is currently more
expensive than that for the migration of particles. However,
here problem-independent strategies such as data compression will be
quite effective. 

\subsection{Future directions}

As we have summarized in the previous subsection, in order to realize 
particle-based simulations with high efficiency  on future machines,
we will have to further reduce the necessary bandwidths of main memory
and inter-node network. For the main memory bandwidth,
more efficient use of the on-chip memory (either cache or local
memory) will become more and more important. For network bandwidth,
it will be necessary to investigate  problem- and network-specific
strategies. 










\begin{acknowledgements}
This work was supported by The Large Scale Computational Sciences with
Heterogeneous Many-Core Computers in grant-in-aid for High Performance
Computing with General Purpose Computers in MEXT (Ministry of
Education, Culture, Sports, Science and Technology-Japan), by JSPS
KAKENHI Grant Number JP18K11334 and JP18H04596 and also by JAMSTEC,
RIKEN, and PEZY Computing, K.K./ExaScaler Inc. In this research
computational resources of the PEZY-SC2 based systems supercomputer,
developed under the Large-scale energy-efficient supercomputer with
inductive coupling DRAM interface project of NexTEP program of Japan
Science and Technology Agency, has been used.

\end{acknowledgements}


% Authors must disclose all relationships or interests that 
% could have direct or potential influence or impart bias on 
% the work: 
%
\section*{Conflict of interest}
The authors declare that they have no conflict of interest.

% BibTeX users please use one of
%\bibliographystyle{spbasic}      % basic style, author-year citations
%\bibliographystyle{spmpsci}      % mathematics and physical sciences
%\bibliographystyle{spphys}       % APS-like style for physics
%\bibliography{../bibtex/allrefs}   % name your BibTeX data base

\newcommand{\noopsort}[1]{} \newcommand{\printfirst}[2]{#1}
  \newcommand{\singleletter}[1]{#1} \newcommand{\switchargs}[2]{#2#1}
\begin{thebibliography}{10}
\providecommand{\url}[1]{{#1}}
\providecommand{\urlprefix}{URL }
\expandafter\ifx\csname urlstyle\endcsname\relax
  \providecommand{\doi}[1]{DOI~\discretionary{}{}{}#1}\else
  \providecommand{\doi}{DOI~\discretionary{}{}{}\begingroup
  \urlstyle{rm}\Url}\fi

\bibitem{Bagla2003}
{Bagla}, J.S.: {TreePM: A Code for Cosmological N-Body Simulations}.
\newblock Journal of Astrophysics and Astronomy \textbf{23}, 185--196 (2002).
\newblock \doi{10.1007/BF02702282}

\bibitem{BarnesHut1986}
{Barnes}, J., {Hut}, P.: A hiearchical o(nlogn) force calculation algorithm.
\newblock Nature \textbf{324}, 446--449 (1986)

\bibitem{Bedorfetal2014}
B{\'e}dorf, J., Gaburov, E., Fujii, M.S., Nitadori, K., Ishiyama, T., Zwart,
  S.P.: 24.77 pflops on a gravitational tree-code to simulate the milky way
  galaxy with 18600 gpus.
\newblock In: SC14: International Conference for High Performance Computing,
  Networking, Storage and Analysis, pp. 54--65 (2014).
\newblock \doi{10.1109/SC.2014.10}

\bibitem{BlackstonSuel1997}
Blackston, D., Suel, T.: Highly portable and efficient implementations of
  parallel adaptive n-body methods.
\newblock In: Proceedings of SC97, pp. (CD--ROM). ACM (1997)

\bibitem{Eastwoodetal1984}
{Eastwood}, J.W., {Hockney}, R.W., {Lawrence}, D.N.: {P3M3DP-the
  three-dimensional periodic particle-particle/particle-mesh program}.
\newblock Computer Physics Communications \textbf{35}, C--618 (1984).
\newblock \doi{10.1016/S0010-4655(84)82783-6}

\bibitem{GreengardRokhlin1987}
Greengard, L., Rokhlin, V.: A fast algorithm for particle simulations.
\newblock Journal of Computational Physics \textbf{73}, 325--348 (1987)

\bibitem{Ishiyamaetal2009b}
{Ishiyama}, T., {Fukushige}, T., {Makino}, J.: {GreeM: Massively Parallel
  TreePM Code for Large Cosmological N -body Simulations}.
\newblock \pasj \textbf{61}, 1319-- (2009)

\bibitem{Ishiyamaetal2012}
Ishiyama, T., Nitadori, K., Makino, J.: 4.45 pflops astrophysical n-body
  simulation on k computer: The gravitational trillion-body problem.
\newblock In: Proceedings of the International Conference on High Performance
  Computing, Networking, Storage and Analysis, SC '12, pp. 5:1--5:10. IEEE
  Computer Society Press, Los Alamitos, CA, USA (2012).
\newblock \urlprefix\url{http://dl.acm.org/citation.cfm?id=2388996.2389003}

\bibitem{v2plus}
{Iwasawa}, M., {Namekata}, D., {Nitadori}, K., {Nomura}, K., {Wang}, L.,
  {Tsubouchi}, M., {Makino}, J.: {Accelerated FDPS --- Algorithms to Use
  Accelerators with FDPS}.
\newblock arXiv e-prints arXiv:1907.02290 (2019)

\bibitem{Iwasawaetal2019barxive}
{Iwasawa}, M., {Namekata}, D., {Sakamoto}, R., {Nakamura}, T., {Kimura}, Y.,
  {Nitadori}, K., {Wang}, L., {Tsubouchi}, M., {Makino}, J., {Liu}, Z., {Fu},
  H., {Yang}, G.: {Implementation and Performance of Barnes-Hut N-body
  algorithm on Extreme-scale Heterogeneous Many-core Architectures}.
\newblock arXiv e-prints arXiv:1907.02289 (2019)

\bibitem{Iwasawaetal2016}
{Iwasawa}, M., {Tanikawa}, A., {Hosono}, N., {Nitadori}, K., {Muranushi}, T.,
  {Makino}, J.: {Implementation and performance of FDPS: a framework for
  developing parallel particle simulation codes}.
\newblock \pasj \textbf{68}, 54 (2016).
\newblock \doi{10.1093/pasj/psw053}

\bibitem{Makino2004}
{Makino}, J.: {A Fast Parallel Treecode with GRAPE}.
\newblock \pasj \textbf{56}, 521--531 (2004)

\bibitem{NAMD2005}
Phillips, J.C., Braun, R., Wang, W., Gumbart, J., Tajkhorshid, E., Villa, E.,
  Chipot, C., Skeel, R.D., Kale, L., Schulten, K.: { Scalable molecular
  dynamics with NAMD}.
\newblock ournal of Computational Chemistry \textbf{26}, 1781--1802 (2005)

\bibitem{LAMMPS1995}
Plimpton, S.J.: {Fast Parallel Algorithms for Short-Range Molecular Dynamics}.
\newblock Journal of Computational Physics \textbf{117}, 1--19 (1995)

\bibitem{PKDGRAV2017}
{Potter}, D., {Stadel}, J., {Teyssier}, R.: {PKDGRAV3: beyond trillion particle
  cosmological simulations for the next era of galaxy surveys}.
\newblock Computational Astrophysics and Cosmology \textbf{4}(1), 2 (2017).
\newblock \doi{10.1186/s40668-017-0021-1}

\bibitem{GROMACS2013}
Pronk, S., Páll, S., Schulz, R., Larsson, P., Bjelkmar, P., Apostolov, R.,
  Shirts, M.R., Smith, J.C., Kasson, P.M., van~der Spoel, D., Hess, B.,
  Lindahl, E.: {GROMACS 4.5: a high-throughput and highly parallel open source
  molecular simulation toolkit}.
\newblock Bioinformatics \textbf{29}(7), 845--854 (2013).
\newblock \doi{10.1093/bioinformatics/btt055}.
\newblock \urlprefix\url{https://doi.org/10.1093/bioinformatics/btt055}

\bibitem{Salmon1990}
Salmon, J., Quinn, P.J., Warren, M.: Using Parallel Computers for Very Large
  N-Body Simulations: Shell Formation Using 180 K Particles, pp. 216--218.
\newblock Springer Berlin Heidelberg, Berlin, Heidelberg (1990).
\newblock \doi{10.1007/978-3-642-75273-5\_51}.
\newblock \urlprefix\url{http://dx.doi.org/10.1007/978-3-642-75273-5\_51}

\bibitem{Amber2012}
{Salomon-Ferrer}, R., A., C.D., C., W.: {An overview of the Amber biomolecular
  simulation package}.
\newblock WIREs Comput Mol Sci  (2012).
\newblock \doi{10.1002/wcms.1121}

\bibitem{Springeletal2000}
Springel, V., Yoshida, N., White, S.D.: Gadget: A code for collisionless and
  gasdynamical cosmological simulations.
\newblock New Astronomy \textbf{6}, 79--117 (2001)

\bibitem{WarrenSalmon1992}
Warren, M.S., Salmon, J.K.: Astrophysical {N}-body simulations using
  hierarchical tree data structures.
\newblock In: Supercomputing '92, pp. 570--576. IEEE Comp. Soc., Los Alamitos
  (1992)

\end{thebibliography}

%% % Non-BibTeX users please use
%% \begin{thebibliography}{}
%% %
%% % and use \bibitem to create references. Consult the Instructions
%% % for authors for reference list style.
%% %
%% \bibitem{RefJ}
%% % Format for Journal Reference
%% Author, Article title, Journal, Volume, page numbers (year)
%% % Format for books
%% \bibitem{RefB}
%% Author, Book title, page numbers. Publisher, place (year)
%% % etc
%% \end{thebibliography}

\end{document}
% end of file template.tex

