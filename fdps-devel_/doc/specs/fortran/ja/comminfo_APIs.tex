%
% comminfo_APIs.tex
%
% FDPS development team June 2021
%

本節では、MPIコミュニケータを操作するAPIについて説明する。


コミュニケータを操作するAPIの名称の一覧を以下に示す:
\begin{screen}
  \begin{spverbatim}
(fdps_)ci_initialize
(fdps_)ci_set_communicator
(fdps_)ci_delete
(fdps_)ci_create
(fdps_)ci_split
\end{spverbatim}
\end{screen}

これらの関数群は、 MPI コミュニケータに対応したテーブルをもち、
そのテーブルインデックスを通してMPIコミュニケータを操作する。
このインデックスを {\tt ci\_}がついた通信関数、さらに {\tt
  set\_dinfo\_comm\_info} 等の関数でFDPSのデータクラスに与えることで、
FDPSで指定したコミュニケータを使うことができる。

この機能により、単一プログラムの中で複数の FDPS インスタンスの時間積分
を並行して行なうことができる。

以下、順に各APIの仕様を記述していく。

%=============================================================
\subsection{ci\_initialize}
\subsubsection*{Fortran 構文}
\begin{screen}
\begin{spverbatim}
integer(kind=c_int) fdps_ctrl%ci_initialize(comm)
\end{spverbatim}
\end{screen}

\subsubsection*{C言語 構文}
\begin{screen}
\begin{spverbatim}
int fdps_ci_initialize(int comm);
\end{spverbatim}
\end{screen}

\subsubsection*{仮引数仕様}
\begin{table}[h]
\begin{tabularx}{\linewidth}{cXcX}
\toprule
\rowcolor{Snow2}
仮引数名 & データ型 & 入出力属性 & 定義 \\
\midrule
\verb|comm| & integer(kind=c\_int) & 入力 & MPIコミュニケータ(Fortran API)\\
\bottomrule
\end{tabularx}
\end{table}


\subsubsection*{返り値}
integer(kind=c\_int) 型。コミュニケータに対応するインデックスを返す。

\subsubsection*{機能}
コミュニケータに対応するインデックスを返す。
入力の MPIコミュニケータは Fortran API であることに注意する。
すなわち、 引数は C 言語APIにおける MPI コミュニケータ({\tt MPI\_Comm}型)ではなく、
それを {\tt MPI\_Comm\_c2f} 関数で変換したFortran 言語でのコミュニケータでなければならない。


%=============================================================
\subsection{ci\_set\_communicator}
\subsubsection*{Fortran 構文}
\begin{screen}
\begin{spverbatim}
subroutine fdps_ctrl%ci_set_communicator(ci, comm)
\end{spverbatim}
\end{screen}

\subsubsection*{C言語 構文}
\begin{screen}
\begin{spverbatim}
void fdps_ci_set_communicator(int ci, int comm);
\end{spverbatim}
\end{screen}

\subsubsection*{仮引数仕様}
\begin{table}[h]
\begin{tabularx}{\linewidth}{cXcX}
\toprule
\rowcolor{Snow2}
仮引数名 & データ型 & 入出力属性 & 定義 \\
\midrule
\verb|ci| & integer(kind=c\_int) & 入力 & コミュニケータインデックス\\
\verb|comm| & integer(kind=c\_int) & 入力 & MPIコミュニケータ(Fortran API)\\
\bottomrule
\end{tabularx}
\end{table}


\subsubsection*{返り値}
なし。

\subsubsection*{機能}
あるインデックスのMPIコミュニケータを変更する。

%=============================================================
\subsection{ci\_delete}
\subsubsection*{Fortran 構文}
\begin{screen}
\begin{spverbatim}
subroutine fdps_ctrl%ci_delete(ci, comm)
\end{spverbatim}
\end{screen}

\subsubsection*{C言語 構文}
\begin{screen}
\begin{spverbatim}
void fdps_ci_delete(int ci, int comm);
\end{spverbatim}
\end{screen}

\subsubsection*{仮引数仕様}
\begin{table}[h]
\begin{tabularx}{\linewidth}{cXcX}
\toprule
\rowcolor{Snow2}
仮引数名 & データ型 & 入出力属性 & 定義 \\
\midrule
\verb|ci| & integer(kind=c\_int) & 入力 & コミュニケータインデックス\\
\verb|comm| & integer(kind=c\_int) & 入力 & MPIコミュニケータ(Fortran API)\\
\bottomrule
\end{tabularx}
\end{table}


\subsubsection*{返り値}
なし。

\subsubsection*{機能}
あるインデックスのMPIコミュニケータを削除 ({\tt MPI\_Comm\_free})する。
このインデックスは「未使用」状態となり、{\tt ci\_initialize} によって
新しいコミュニケータが割り当てられるまで使えない。


%=============================================================
\subsection{ci\_create}
\subsubsection*{Fortran 構文}
\begin{screen}
\begin{spverbatim}
integer(kind=c_int) fdps_ctrl%ci_create(ci, n, rank)
\end{spverbatim}
\end{screen}

\subsubsection*{C言語 構文}
\begin{screen}
\begin{spverbatim}
int fdps_ci_create(int ci, int n, int rank[]);
\end{spverbatim}
\end{screen}

\subsubsection*{仮引数仕様}
\begin{table}[h]
\begin{tabularx}{\linewidth}{cXcX}
\toprule
\rowcolor{Snow2}
仮引数名 & データ型 & 入出力属性 & 定義 \\
\midrule
\verb|ci| & integer(kind=c\_int) & 入力 & コミュニケータインデックス\\
\verb|n| & integer(kind=c\_int) & 入力 &生成されるコミュニケータに所属するプロセスの数\\
\verb|rank| & integer(kind=c\_int), dimension(n) & 入力 &生成されるコミュニケータに所属するプロセスのランクの配列。\\
\bottomrule
\end{tabularx}
\end{table}


\subsubsection*{返り値}
integer(kind=c\_int) 型。生成されたコミュニケータに対応するインデックスを返す。


\subsubsection*{機能}
呼び出しもとの{\tt ci}に対応するMPIコミュニケータから新たなコ
ミュニケータを作成する。配列rank で表されるプロセスが所属するコミュ
ニケータを作成し対応するインデックスを返す。

%=============================================================
\subsection{ci\_split}
\subsubsection*{Fortran 構文}
\begin{screen}
\begin{spverbatim}
integer(kind=c_int) fdps_ctrl%ci_split(ci, n, rank)
\end{spverbatim}
\end{screen}

\subsubsection*{C言語 構文}
\begin{screen}
\begin{spverbatim}
int fdps_ci_split(int ci, int color, int key);
\end{spverbatim}
\end{screen}

\subsubsection*{仮引数仕様}
\begin{table}[h]
\begin{tabularx}{\linewidth}{cXcX}
\toprule
\rowcolor{Snow2}
仮引数名 & データ型 & 入出力属性 & 定義 \\
\midrule
\verb|ci| & integer(kind=c\_int) & 入力 & コミュニケータインデックス\\
\verb|color| & integer(kind=c\_int) & 入力 &これが同じプロセスは同一のコミュニケータに属する\\
\verb|key| & integer(kind=c\_int) & 入力 &同一コミュニケータの中での順序を与える\\
\bottomrule
\end{tabularx}
\end{table}


\subsubsection*{返り値}
integer(kind=c\_int) 型。生成されたコミュニケータに対応するインデックスを返す。


\subsubsection*{機能}
呼び出しもとの{\tt ci}に対応するMPIコミュニケータを分割する.
同じcolorのプロセスは同一コミュニケータに所属し、keyの小さいものから順
にそのコミュニケータでのランクが割り振られる.




\clearpage






