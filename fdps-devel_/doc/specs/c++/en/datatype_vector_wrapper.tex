The wrappers of vector types are defined as follows.
%ベクトル型のラッパーの定義を以下に示す。
\begin{lstlisting}[caption=vectorwrapper]
namespace ParticleSimulator{
    typedef Vector2<F32> F32vec2;
    typedef Vector3<F32> F32vec3;
    typedef Vector2<F64> F64vec2;
    typedef Vector3<F64> F64vec3;
#ifdef PARTICLE_SIMULATOR_TWO_DIMENSION
    typedef F32vec2 F32vec;
    typedef F64vec2 F64vec;
#else
    typedef F32vec3 F32vec;
    typedef F64vec3 F64vec;
#endif
}
\end{lstlisting}

PS::F32vec2, PS::F32vec3, PS::F64vec2, and PS::F64vec3 are,
respectively, 2D vector in single precision, 3D vector in single
presicion, 2D vector in double precision, and 3D vector in double
precision. If users set 2D (3D) coordinate system, PS::F32vec and
PS::F64vec is wrappers of PS::F32vec2 and PS::F64vec2 (PS::F32vec3 and
PS::F64vec3).
%%すなわちPS::F32vec2, PS::F32vec3, PS::F64vec2, PS::F64vec3はそれぞれ
%%単精度2次元ベクトル、倍精度2次元ベクトル、単精度3次元ベクトル、倍精度
%%3次元ベクトルである。FDPSで扱う空間座標系を2次元とした場合、
%%PS::F32vecとPS::F64vecはそれぞれ単精度2次元ベクトル、倍精度2次元ベク
%%トルとなる。一方、FDPSで扱う空間座標系を3次元とした場合、PS::F32vecと
%%PS::F64vecはそれぞれ単精度3次元ベクトル、倍精度3次元ベクトルとなる。

