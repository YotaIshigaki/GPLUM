%% 本節では、Particle Meshクラスについて記述する。このクラスはParticle
%% Mesh法を用いて粒子の相互作用を計算するモジュールである。メッシュからの
%% 力はS-2型の関数によってカットオフされており、カットオフ半径はメッシュ
%% 間隔の3倍で固定されている。カットオフ関数径を変更する事は出来ない。ま
%% た、Particle Meshクラスに送る粒子の座標は0以上1未満の値に規格化して置
%% かなければならない。以下に、オブジェクトの生成方法、API、使用済マクロ、
%% 使いかたについて記述する。

This section describes {\tt ParticleMesh} class which is for the
evaluation of interactions between particles using the Particle Mesh
method. The cutoff of the S-2 type function is applied to the mesh
force, and the cutoff radius is fixed to three times the mesh
size. The coordinates of particles sent to {\tt ParticleMesh} class
should be in the range of [0,1).
The following subsections describe how to create the object
for Particle Mesh method, APIs, predefined macros and how to use the
class.

%%%%%%%%%%%%%%%%%%%%%%%%%%%%%%%%%%%%%%%%%%%%%%%%%%%%%%%%%%%%%%%%%
%\subsubsubsection{オブジェクトの生成}
\subsubsubsection{Creation of Object}

%Particle Meshクラスは以下のように宣言されている。

{\tt ParticleMesh} class is declared bellow.

\begin{lstlisting}[caption=ParticleMesh0]
namespace ParticleSimulator {
    namespace ParticleMesh {
        class ParticleMesh;
    }
}
\end{lstlisting}

%Particle Meshクラスのオブジェクトの生成は以下のように行う。ここではpm
%というオブジェクトを生成している。

Next example shows how to create an object of {\tt ParticleMesh} class.
Here, the object is named {\tt pm}.

\begin{screen}
\begin{verbatim}
PS::PM::ParticleMesh pm;
\end{verbatim}
\end{screen}

%%%%%%%%%%%%%%%%%%%%%%%%%%%%%%%%%%%%%%%%%%%%%%%%%%%%%%%%%%%%%%%%%
%%%%%%%%%%%%%%%%%%%%%%%%%%%%%%%%%%%%%%%%%%%%%%%%%%%%%%%%%%%%%%%%%
%%%%%%%%%%%%%%%%%%%%%%%%%%%%%%%%%%%%%%%%%%%%%%%%%%%%%%%%%%%%%%%%%
\subsubsubsection{API}

{\tt ParticleMesh} class has APIs for initialization and force
evaluation. The following subsections describe them.

%Particle Meshクラスには初期設定関連のAPI、低レベルAPI、高レベルAPIがあ
%る。以下、各節に分けて記述する。

%%%%%%%%%%%%%%%%%%%%%%%%%%%%%%%%%%%%%%%%%%%%%%%%%%%%%%
%\subsubsubsubsection{初期設定}
\subsubsubsubsection{Initial Setup}

%初期設定関連のAPIの宣言は以下のようになっている。このあと各APIについて
%記述する。

\begin{lstlisting}[caption=ParticleMesh1]
namespace ParticleSimulator {
    namespace ParticleMesh {
        class ParticleMesh{
            ParticleMesh();
        };
    }
}
\end{lstlisting}

%\subsubsubsubsubsection{コンストラクタ}
\subsubsubsubsubsection{Constructor}

\begin{screen}
\begin{verbatim}
void PS::PM::ParticleMesh::ParticleMesh();
\end{verbatim}
\end{screen}

\begin{itemize}

\item {\bf arguments}

void.

\item {\bf returned value}

void.

\item {\bf function}

Create an object of {\tt ParticleMesh} class.

%Particle Meshクラスのオブジェクトを生成する。

\end{itemize}

%%%%%%%%%%%%%%%%%%%%%%%%%%%%%%%%%%%%%%%%%%%
%\subsubsubsubsection{低レベルAPI}
\subsubsubsubsection{Low Level API}

%低レベルAPIの宣言は以下のようになっている。このあと各APIについて記述す
%る。

The low Level APIs are declared below.

\begin{lstlisting}[caption=ParticleMesh1]
namespace ParticleSimulator {
    namespace ParticleMesh {
        class ParticleMesh{
            template<class Tdinfo>
            void setDomainInfoParticleMesh
                        (const Tdinfo & dinfo);
            template<class Tpsys>
            void setParticleParticleMesh
                        (const Tpsys & psys,
                         const bool clear=true);
            void calcMeshForceOnly();
            F32vec getForce(F32vec pos);
            F32 getPotential(F32vec pos);
        };
    }
}
\end{lstlisting}

\subsubsubsubsubsection{PS::PM::ParticleMesh::setDomainInfoParticleMesh}

\begin{screen}
\begin{verbatim}
template<class Tdinfo>
void PS::PM::ParticleMesh::setDomainInfoParticleMesh
            (const Tdinfo & dinfo);
\end{verbatim}
\end{screen}

\begin{itemize}

\item {\bf arguments}

{\tt dinfo}: input. Type {\tt const DomainInfo \&}. An object of {\tt
DomainInfo} class.

%dinfo: 入力。Tdinfo \&型。領域クラスのオブジェクト。

\item {\bf returned value}

void.

\item {\bf function}

Read the data from {\tt dinfo}.
%領域情報を読み込む。

\end{itemize}

\subsubsubsubsubsection{PS::PM::ParticleMesh::setParticleParticleMesh}

\begin{screen}
\begin{verbatim}
template<class Tpsys>
void PS::PM::ParticleMesh::setParticleParticleMesh
            (const Tpsys & psys,
             const bool clear=true);
\end{verbatim}
\end{screen}

\begin{itemize}

\item {\bf arguments}

{\tt psys}: input. Type {\tt const ParticleSystem \&}. An object of
{\tt ParticleSystem} class.

{\tt clear}: input. Type {\tt const bool}. A flag to determine if
previous information of particles loaded is cleared.

%  psys: 入力。Tpsys \& 型。粒子群クラスのオブジェクト。

%  clear: 入力。const bool型。これまで読込んだ粒子情報をクリアするかど
%  うか決定するフラグ。trueならばクリアする。デフォルトはtrue。

\item {\bf returned value}

void.

\item {\bf function}

Read data of particles from {\tt psys}.
%  粒子情報を粒子群クラスのオブジェクトから読み込む。

\end{itemize}

\subsubsubsubsubsection{PS::PM::ParticleMesh::calcMeshForceOnly}

\begin{screen}
\begin{verbatim}
void PS::PM::ParticleMesh::calcMeshForceOnly();
\end{verbatim}
\end{screen}

\begin{itemize}

\item {\bf arguments}

void.

\item {\bf returned value}

void.

\item {\bf function}

Calculate forces on grid points.
%メッシュ上の力を計算する。

\end{itemize}

\subsubsubsubsubsection{PS::PM::ParticleMesh::getForce}

\begin{screen}
\begin{verbatim}
PS::F32vec PS::PM::ParticleMesh::getForce(PS::F32vec pos);
\end{verbatim}
\end{screen}

\begin{itemize}

\item {\bf arguments}

{\tt pos}: input. Type {\tt PS::F32vec}. %%

%  pos: 入力。PS::F32vec型。メッシュに課された粒子からの力を計算したい
%  位置。

\item {\bf returned value}

Type {\tt PS::F32vec}. PM force at the position {\tt pos}.
%  PS::F32vec型。メッシュに課された粒子からの力。

\item {\bf function}

Return a PM force exerted at the position ``pos".
This function is thread-safe.
%  位置posでのメッシュに課された粒子からの力を返す。

\end{itemize}

\subsubsubsubsubsection{PS::PM::ParticleMesh::getPotential}

\begin{screen}
\begin{verbatim}
PS::F32 PS::PM::ParticleMesh::getPotential(PS::F32vec pos);
\end{verbatim}
\end{screen}

\begin{itemize}

\item {\bf arguments}

{\tt pos}: input. Type {\tt PS::F32vec}. %%

%  pos: 入力。PS::F32vec型。メッシュに課された粒子からの力を計算したい
%  位置。

\item {\bf returned value}

Type {\tt PS::F32vec}. PM potential at the position {\tt pos}.
%  PS::F32vec型。メッシュに課された粒子からの力。

\item {\bf function}

Return a PM potential exerted at the position ``pos".
This function is thread-safe.
%  位置posでのメッシュに課された粒子からの力を返す。

\end{itemize}


%%%%%%%%%%%%%%%%%%%%%%%%%%%%%%%%%%%%%%%%%%%
%\subsubsubsubsection{高レベルAPI}
\subsubsubsubsection{High Level API}

%高レベルAPIの宣言は以下のようになっている。このあと各APIについて記述す
%る。

High level APIs are declared below.

\begin{lstlisting}[caption=ParticleMesh1]
namespace ParticleSimulator {
    namespace ParticleMesh {
        class ParticleMesh{
            template<class Tpsys,
                     class Tdinfo>
            void calcForceAllAndWriteBack
                        (Tpsys & psys,
                         const Tdinfo & dinfo);
        };
    }
}
\end{lstlisting}

\subsubsubsubsubsection{PS::PM::ParticleMesh::calcForceAllAndWriteBack}

\begin{screen}
\begin{verbatim}
template<class Tpsys,
         class Tdinfo>
void PS::PM::ParticleMesh::calcForceAllAndWriteBack
            (Tpsys & psys,
             const Tdinfo & dinfo);
\end{verbatim}
\end{screen}

\begin{itemize}

\item {\bf arguments}

{\tt psys}: input and output. Type {\tt ParticleSystem \&}. An object
of the {\tt ParticleSystem} class.

{\tt dinfo}: input. Type {\tt const DomainInfo \&}. An object of the
{\tt DomainInfo} class.

%  psys: 入力であり出力。Tpsys \& 型。粒子群クラスのオブジェクト。

%  dinfo: 入力。const Tdinfo \&型。領域クラスのオブジェクト。

\item {\bf returned value}

void.

\item {\bf function}

Calculate forces between particles of {\tt psys} and write back the
forces to {\tt psys}.

%  粒子群クラスのオブジェクトpsysに含まれる粒子間のメッシュ力を計算し、
%  その結果をpsysに返す。

\end{itemize}

%%%%%%%%%%%%%%%%%%%%%%%%%%%%%%%%%%%%%%%%%%%%%%%%%%%%%%%%%%%%%%%%%
%\subsubsubsection{使用済マクロ}
\subsubsubsection{Predefined Macros}

%%BINARY_BOUNDARY, BOUNDARY_COMM_NONBLOCKING, BOUNDARY_SMOOTHING,
%%BUFFER_FOR_TREE, CALCPOT, CLEAN_BOUNDARY_PARTICLE,
%%CONSTANT_TIMESTEP, EXCHANGE_COMM_NONBLOCKING, FFT3D, FFTW3_PARALLEL,
%%FFTW_DOUBLE, FIX_FFTNODE, GADGET_IO, GRAPE_OFF, KCOMPUTER, LONG_ID,
%%MAKE_LIST_PROF, MERGE_SNAPSHOT, MULTI_TIMESTEP, MY_MPI_BARRIER,
%%N128_2H, N256_2H, N256_H, N32_2H, N512_2H, NEW_DECOMPOSITION, NOACC,
%%NPART_DIFFERENT_DUMP, OMP_SCHDULE_DISABLE, PRINT_TANIKAWA,
%%REVERSE_ENDIAN_INPUT, REVERSE_ENDIAN_OUTPUT, RMM_PM,
%%SHIFT_INITIAL_BOUNDARY, STATIC_ARRAY, TREE2,
%%TREECONSTRUCTION_PARALLEL, TREE_PARTICLE_CACHE, UNIFORM, UNSTABLE,
%%USING_MPI_PARTICLE, VERBOSE_MODE, VERBOSE_MODE2

%このモジュールでは多くのマクロを使っている。これらを別のマクロとして使
%用した場合にプログラムが正しく動作する保証はない。ここでは使用されてい
%るマクロをアルファベティカルに列挙する。

There are many macros predefined in this module and users must not
define the same name macros. The predefined macros are listed bellow
in alphabetical order.


\begin{itemize}
\item BINARY\_BOUNDARY
\item BOUNDARY\_COMM\_NONBLOCKING
\item BOUNDARY\_SMOOTHING
\item BUFFER\_FOR\_TREE
\item CALCPOT
\item CLEAN\_BOUNDARY\_PARTICLE
\item CONSTANT\_TIMESTEP
\item EXCHANGE\_COMM\_NONBLOCKING
\item FFT3D
\item FFTW3\_PARALLEL
\item FFTW\_DOUBLE
\item FIX\_FFTNODE
\item GADGET\_IO
\item GRAPE\_OFF
\item KCOMPUTER
\item LONG\_ID
\item MAKE\_LIST\_PROF
\item MERGE\_SNAPSHOT
\item MULTI\_TIMESTEP
\item MY\_MPI\_BARRIER
\item N128\_2H
\item N256\_2H
\item N256\_H
\item N32\_2H
\item N512\_2H
\item NEW\_DECOMPOSITION
\item NOACC
\item NPART\_DIFFERENT\_DUMP
\item OMP\_SCHDULE\_DISABLE
\item PRINT\_TANIKAWA
\item REVERSE\_ENDIAN\_INPUT
\item REVERSE\_ENDIAN\_OUTPUT
\item RMM\_PM
\item SHIFT\_INITIAL\_BOUNDARY
\item STATIC\_ARRAY
\item TREE2
\item TREECONSTRUCTION\_PARALLEL
\item TREE\_PARTICLE\_CACHE
\item UNIFORM
\item UNSTABLE
\item USING\_MPI\_PARTICLE
\item VERBOSE\_MODE
\item VERBOSE\_MODE2。
\end{itemize}

%%%%%%%%%%%%%%%%%%%%%%%%%%%%%%%%%%%%%%%%%%%%%%%%%%%%%%%%%%%%%%%%%
%\subsubsubsection{Particle Meshクラスの使いかた}
\subsubsubsection{How To Use Particle Mesh Class}

%Particle Meshクラスを使うには以下の4つのことを行う必要がある。
%There are three steps listed below to use the Particle Mesh class.

Through the following steps, one can use this {\tt ParticleMesh} class.

\begin{enumerate}
\item Compile ParticleMesh class.
\item Write a program including FDPS.
\item Compile the program.
\end{enumerate}

The following subsections describe these steps.

%\begin{enumerate}
%\item Particle Meshクラスのコンパイル
%\item Particle Meshクラスを使ったFDPSコードの記述
%\item FDPSコードのコンパイル
%\end{enumerate}
%以下、詳細に記述する。

%\subsubsubsubsection{Particle Meshクラスのコンパイル}
\subsubsubsubsection{Compile of Particle Mesh Class}

%以下のように行う。ディレクトリsrcの下のディレクトリparticle\_meshの
%Makefileを適切に編集してmakeする。編集すべきことは以下の2点である。
%\begin{itemize}
%\item INCLUDE\_FFTWにFFTWのヘッダファイルがあるディレクトリを記述
%  する
%\item param\_fdps.hの中のSIZE\_OF\_MESH (1次元方向のメッシュの数)を設定。
%推奨値は $N^{1/3}/2$($N$は粒子数)。
%\end{itemize}
%うまく行けば、同じディレクトリにライブラリlibpm.aとヘッダファイル
%particle\_mesh.hppができる。

Users need to edit the {\tt Makefile} in
{\tt \$(FDPS)/src/particle\_mesh} appropriately as follows.

\begin{itemize}
\item Set the macro {\tt INCLUDE\_FFTW} (the directory in which the header file of FFTW is) in Makefile.
\item Set the macro {\tt SIZE\_OF\_MESH} (the number of meshes along one direction) in {\tt param\_fdps.h}.
Recommended value is $N^{1/3}/2$ ($N$ is the number of the particle).
\end{itemize}

After editing, users run make command.

If successful, the library {\tt libpm.a} and the header file {\tt
particle\_mesh.hpp} are created.

%\subsubsubsubsection{FDPSコードを記述}
\subsubsubsubsection{Writing Source Code}

Next, users write programs as follows.

\begin{itemize}
\item Include the header files created above.
\item Add member functions described below to {\tt FullParticle} class (here, this particle class called FP).
  \begin{itemize}
  \item void {\tt FP::copyFromForceParticleMesh(const PS::F32vec \& force)}.
     In this function, copy {\tt force} to arbitrary member variables in FP.
  \item {\tt PS::F64 FP::getChargeParticleMesh()}. This function returns mass.
  \end{itemize}
\item 

%以下のように行う。
%\begin{itemize}
%\item 上でできたヘッダファイルをincludeする
%\item PMを計算したい粒子クラスに以下のメンバ関数を加える(この粒子クラスのクラス名をFPとする)
%  \begin{itemize}
%  \item void FP::copyFromForceParticleMesh(const PS::F32vec \& force)。こ
%     の中でforceを好きなメンバ変数にセットする。
%  \item PS::F64 FP::getChargeParticleMesh()。この中で質量を返す。
%  \end{itemize}
%\item このクラスのオブジェクトを生成するときに、PS::PM::ParticleMeshと
%する
\end{itemize}

%\subsubsubsubsection{FDPSコードのコンパイル}
\subsubsubsubsection{Compile of Source Code}

To compile the programs written above, users need to follow the
following steps.

\begin{itemize}
\item Specify the path to the header file {\tt particle\_mesh.hpp}.
\item Link the library {\tt libpm.a}.
\item Specify the path to the directory in which the header files of FFTW are.
\item Link the FFTW libraries.
\end{itemize}

%上で記述したFDPSコードをコンパイルするには以下のことを行う必要がある。
%\begin{itemize}
%\item ヘッダファイルparticle\_mesh.hppのあるディレクトリへのパスを指定する
%\item ライブラリlibpm.aとリンクする
%\item FFTWのヘッダファイルがあるディレクトリへのパスを指定する
%\item FFTWのライブラリとリンクする
%\end{itemize}

%\subsubsubsubsection{注意事項}

\subsubsubsubsection{Note}

{\tt ParticleMesh} class does not work if the total number of
processes is one.

%Particle Meshクラスはプロセス数が2以上でないと、動かないことに注意。
