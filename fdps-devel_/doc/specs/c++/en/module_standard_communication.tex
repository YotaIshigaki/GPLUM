This section describes {\tt CommInfo} and {\tt PS::Comm} classes which
are used to communicate data among processes and keep the data for
communication.  An object of {\tt CommInfo} class has one MPI
communicator and its member functions operates under that
communicator.  It can be given to objects of {\tt  ParticleSystem},
{\tt TreeForForce}, and {\tt DomainInfo} classes so that they
communicate on a given communicator.


%% 本節では、通信用データクラス{\tt CommInfo}クラスと{\tt Comm}クラスにつ
%% いて記述する。これらのクラスはノード間通信のための情報の保持や実際の通
%% 信を行うモジュールである。{\tt CommInfo}クラスは1つのMPIのコミュニケー
%% タを持ち、そのメンバ関数はそのコミュニケータの元で動作する.{\tt
%% ParticleSystem}クラス、{\tt TreeForForce}クラス、{\tt DomainInfo}クラ
%% スはそれぞれ通信用データクラスをもっており,ユーザーは自分で作成した
%% {\tt CommInfo}クラスのオブジェクトをそれらのクラスのオブジェクトに与え
%% ることで、任意のコミュニケータ上で相互作用計算をさせることができる.

The {\tt Comm} class is a wrapper for the default behavior of using
{\tt MPI\_COMM\_WORLD} for all communications.  Since the singleton
pattern is applied in this class, users do not need to create an
object.


%% {\tt Comm}クラスはコミュニケータに{\tt MPI\_COMM\_WORLD}を指定した場合
%% のラッパークラスとなっている.{\tt Comm}クラスはシングルトンパターンで
%% 管理されており、ユーザーはオブジェクトの生成は必要としない.


\subsubsubsection{Object creation}
The {\tt CommInfo} class is decleared as follows.

%% {\tt CommInfo}クラスは以下のように宣言されている。
\begin{lstlisting}[caption=CommInfo0]
namespace ParticleSimulator {
    class CommInfo;
}
\end{lstlisting}

An object of {\tt CommInfo} class is created as follows. Here, an
object namede {\tt comm\_info} is created.


%% {\tt CommInfo}クラスのオブジェクトの生成は以下のように行う。ここでは
%% {\tt comm\_info}というオブジェクトを生成している。

\begin{screen}
\begin{verbatim}
PS::CommInfo comm_info;
\end{verbatim}
\end{screen}


%%%%%%%%%%%%%%%%%%%%%%%%%%%%%%%%%%%%%%%%%%%%%%%%%%%%%%%%%%%%%%%%%
\subsubsubsection{API}

The declaration of the API of {\tt CommInfo} class is as follows.

%%{\tt CommInfo}クラスのAPIの宣言は以下のようになっている。
\begin{lstlisting}[caption=CommunicationInformation]
namespace ParticleSimulator {
    class CommInfo{
    public:
        void setCommunicator(const MPI_Comm & comm);
        CommInfo create(const int n, const int rank[]);
        CommInfo split(int color, int key);
        S32 getRank();
        S32 getNumberOfProc();
        S32 getRankMultiDim(const S32 id);
        S32 getNumberOfProcMultiDim(const S32 id);
        bool synchronizeConditionalBranchAND
                    (const bool local);
        bool synchronizeConditionalBranchOR
                    (const bool local);
        template<class T>
        T getMinValue(const T val);
        template<class Tfloat, class Tint>
        void getMinValue(const Tfloat f_in,
                                const Tint i_in,
                                Tfloat & f_out,
                                Tint & i_out);
        template<class T>
        T getMaxValue(const T val);
        template<class Tfloat, class Tint>
        void getMaxValue(const Tfloat f_in,
                                const Tint i_in,
                                Tfloat & f_out,
                                Tint & i_out );
        template<class T>
        T getSum(const T val);
        template<class T>
        void broadcast(T * val,
                              const S32 n,
                              const S32 src=0);
        MPI_Comm getCommunicator();
    };
}
\end{lstlisting}


The declaration of the API of the  {\tt Comm} class is as follows.

%% {\tt Comm}クラスのAPIの宣言は以下のようになっている。
\begin{lstlisting}[caption=Communication]
namespace ParticleSimulator {
    class Comm{
    public:
        static S32 getRank();
        static S32 getNumberOfProc();
        static S32 getRankMultiDim(const S32 id);
        static S32 getNumberOfProcMultiDim(const S32 id);
        static bool synchronizeConditionalBranchAND
                    (const bool local);
        static bool synchronizeConditionalBranchOR
                    (const bool local);
        template<class T>
        static T getMinValue(const T val);
        template<class Tfloat, class Tint>
        static void getMinValue(const Tfloat f_in,
                                const Tint i_in,
                                Tfloat & f_out,
                                Tint & i_out);
        template<class T>
        static T getMaxValue(const T val);
        template<class Tfloat, class Tint>
        static void getMaxValue(const Tfloat f_in,
                                const Tint i_in,
                                Tfloat & f_out,
                                Tint & i_out );
        template<class T>
        static T getSum(const T val);
        template<class T>
        static void broadcast(T * val,
                              const S32 n,
                              const S32 src=0);
    };
}
\end{lstlisting}

%%%%%%%%%%%%%%%%%%%%%%%%%%%%%%%%%%%%%%%%%%%%%%%%%%%%%%
\subsubsubsubsection{Initialization}

%%%%%%%%%%%%%%%%%%%%%%%%%%%%%%%%%%%%%%%%%%%
\subsubsubsubsubsection{Constructor}

\begin{screen}
\begin{verbatim}
PS::CommInfo PS::CommInfo::CommInfo(const MPI_Comm & comm = MPI_COMM_NULL);
\end{verbatim}
\end{screen}

\begin{itemize}

\item{\bf arguments}

{\tt comm}: input.Type {\tt const MPI\_Comm \&}.

\item {\bf returned value}

Type CommInfo.The new communicator.

\item {\bf function}

Create an object of type {\tt CommInfo}.
The default comminicator is タは{\tt MPI\_COMM\_NULL}.

{\tt CommInfo}クラスのオブジェクトを生成する。デフォルトのコミュニケー
タは{\tt MPI\_COMM\_NULL}である.

%% \item {\bf 引数}

%% comm: 入力.{\tt const MPI\_Comm \&} 型.

%% \item {\bf 返値}

%% なし

%% \item {\bf 機能}

%% {\tt CommInfo}クラスのオブジェクトを生成する。デフォルトのコミュニケー
%% タは{\tt MPI\_COMM\_NULL}である.

\end{itemize}

\subsubsubsubsection{PS::CommInfo::setCommunicator}
\begin{screen}
\begin{verbatim}
void PS::CommInfo::setCommunicator(const MPI_Comm & comm = MPI_COMM_WORLD);
\end{verbatim}
\end{screen}

\begin{itemize}

\item{\bf arguments}

{\tt comm} : input.Type {\tt const MPI\_Comm \&}.

\item {\bf returned value}

None.

\item {\bf function}

Specify the MPI compunicator for an object of type {\tt CommInfo}.
The default communicator is {\tt MPI\_COMM\_WORLD}.


%% \item {\bf 引数}

%% comm: 入力.{\tt const MPI\_Comm \&} 型.

%% \item {\bf 返値}

%% なし

%% \item {\bf 機能}

%% {\tt CommInfo}クラスのオブジェクトにMPIコミュニケータを設定する.デフォ
%% ルトは{\tt MPI\_COMM\_WORLD}である.

\end{itemize}

\subsubsubsubsubsection{PS::CommInfo::creat}
\begin{screen}
\begin{verbatim}
CommInfo create(const int n, const int rank[]) const;
\end{verbatim}
\end{screen}

\begin{itemize}

\item {\bf arguments}

{\tt n}: input.Type int.Number of processes to belong to the new communicator.
{\tt rank}: input.Array of int.Ranks of processes processes to belong to the new communicator.

\item {\bf return value}

Type CommInfo.The new communicator.

\item {\bf function}

Create a new communicator from the one in the {\tt CommInfo} object.
The new commmunicator contains the processes in the array {\tt rank}.


%% \item {\bf 引数}

%% n: 入力.int型.生成されるコミュニケータに所属するプロセスの数.
%% rank: 入力.int型の配列.生成されるコミュニケータに所属するプロセスのランクの配列.

%% \item {\bf 返値}

%% CommInfo型.生成されたコミュニケータ.

%% \item {\bf 機能}

%% 呼び出しもとの{\tt CommInfo}クラスのもつMPIコミュニケータから新たなコ
%% ミュニケータを作成する.配列rank[]で表せられるプロセスが所属するコミュ
%% ニケータを作成し返す.

\end{itemize}

\subsubsubsubsubsection{PS::CommInfo::split}
\begin{screen}
\begin{verbatim}
CommInfo split(int color, int key) const;
\end{verbatim}
\end{screen}

\begin{itemize}

\item {\bf arguments}

{\tt color} : input.Type int.
{\tt key}: input.Type int.

\item {\bf return value}

Type CommInfo.Splitted communicator.

\item {\bf function}

Split the communicator of the {\tt CommInfo} class.
Processes with the same color will belong to the same
communicator. Their ranks will be ordered in the order of their keys.


%% \item {\bf 引数}

%% color: 入力.int型.
%% key: 入力.int型.

%% \item {\bf 返値}

%% CommInfo型.スプリットされたコミュニケータを返す.

%% \item {\bf 機能}

%% 呼び出しもとの{\tt CommInfo}クラスのもつMPIコミュニケータを分割する.
%% 同じcolorのプロセスは同一コミュニケータに所属し、keyの小さいものから順
%% にそのコミュニケータでのランクが割り振られる.

\end{itemize}


%%%%%%%%%%%%%%%%%%%%%%%%%%%%%%%%%%%%%%%%%%%%%%%%%%%%%%
\subsubsubsubsection{Information retrieval}

\subsubsubsubsubsection{PS::CommInfo::getRank}

\begin{screen}
\begin{verbatim}
PS::S32 PS::CommInfo::getRank();
\end{verbatim}
\end{screen}

\begin{itemize}

\item{{\bf arguments}}

None.

\item{{\bf return value}}

Tyoe {PS::S32}. The rank in the communicator.

%% \item{{\bf 引数}}

%% なし。

%% \item{{\bf 返り値}}

%% {PS::S32}型。コミュニケータ内でのランクを返す。

\end{itemize}

\subsubsubsubsubsection{PS::CommInfo::getNumberOfProc}

\begin{screen}
\begin{verbatim}
PS::S32 PS::CommInfo::getNumberOfProc();
\end{verbatim}
\end{screen}

\begin{itemize}

\item{{\bf arguments}}

None.

\item{{\bf returned value}}

Type PS::S32. The number of processes in this communicator.

%% \item{{\bf 引数}}

%% なし。

%% \item{{\bf 返り値}}

%% PS::S32型。全プロセス数を返す。


\end{itemize}

\subsubsubsubsubsection{PS::CommInfo::getRankMultiDim}

\begin{screen}
\begin{verbatim}
PS::S32 PS::CommInfo::getRankMultiDim(const PS::S32 id);
\end{verbatim}
\end{screen}

\begin{itemize}

\item{{\bf arguments}}

{\tt id}: input. Type const PS::S32. The index of the axis. 0, 1, 2
for x, y, and z.

\item{{\bf returned value}}

Type PS::S32. The rank in the specified axis. In the case of 2D
calculation, id=2 results in the return value of 1.


%% \item{{\bf 引数}}

%% id: 入力。const PS::S32型。軸の番号。x軸:0, y軸:1, z軸:2。

%% \item{{\bf 返り値}}

%% PS::S32型。id番目の軸でのランクを返す。2次元の場合、id=2は1を返す。

\end{itemize}

\subsubsubsubsubsection{PS::CommInfo::getNumberOfProcMultiDim}

\begin{screen}
\begin{verbatim}
PS::S32 PS::CommInfo::getNumberOfProcMultiDim(const PS::S32 id);
\end{verbatim}
\end{screen}

\begin{itemize}

\item{{\bf arguments}}

{\tt id}: input. TYpe const PS::S32. The index of the axis. 0, 1, 2
for x, y, and z.

\item{{\bf returned value}}

Type PS::S32. The number of processes in the specified axis. In the case of 2D
calculation, id=2 results in the return value of 1.


%% \item{{\bf 引数}}

%% id: 入力。const PS::S32型。軸の番号。x軸:0, y軸:1, z軸:2。

%% \item{{\bf 返り値}}

%% PS::S32型。id番目の軸のプロセス数を返す。2次元の場合、id=2は1を返す。

\end{itemize}

\subsubsubsubsubsection{PS::CommInfo::synchronizeConditionalBranchAND}

\begin{screen}
\begin{verbatim}
bool PS::CommInfo::synchronizeConditionalBranchAND(const bool local)
\end{verbatim}
\end{screen}

\begin{itemize}

\item{{\bf 引数}}

\item{\bf arguments}

{\tt local}: input. Type {\tt const bool}.

%local: 入力。const bool型。

\item{\bf returned value}

{\tt bool} type. Logical product of {\tt local} over all processes in
the communicator.

%% \item{{\bf 引数}}

%% local: 入力。const bool型。

%% \item{{\bf 返り値}}

%% bool型。全プロセスでlocalの論理積を取り、結果を返す。

\end{itemize}

\subsubsubsubsubsection{PS::CommInfo::synchronizeConditionalBranchOR}

\begin{screen}
\begin{verbatim}
bool PS::CommInfo::synchronizeConditionalBranchOR(const bool local);
\end{verbatim}
\end{screen}

\begin{itemize}

\item{\bf arguments}

{\tt local}: input. Type {\tt const bool}.
%local: 入力。const bool型。

\item{\bf returned value}

Type {\tt bool}.

\item{\bf function}

Return logical sum of {\tt local} over all processes in the communicator.

%% \item{{\bf 引数}}

%% local: 入力。const bool型。
%% \item{{\bf 返り値}}

%% bool型。全プロセスでlocalの論理和を取り、結果を返す。

\end{itemize}

\subsubsubsubsubsection{PS::CommInfo::getMinValue}

\begin{screen}
\begin{verbatim}
template <class T>
T PS::CommInfo::getMinValue(const T val);
\end{verbatim}
\end{screen}

\begin{itemize}

\item{\bf arguments}

{\tt val}: input. Type {\tt const T}. Type {\tt T} is allowed to be
{\tt PS::F32}, {\tt PS::F64}, {\tt PS::S32}, {\tt PS::S64}, {\tt
PS::U32} and {\tt PS::U64}

\item{\bf returned value}

Type {\tt T}. The minimum value of val of all processes in the communicator.

%% \item{{\bf 引数}}

%% val: 入力。const T型。

%% \item{{\bf 返り値}}

%% T型。全プロセスでvalの最小値を取り、結果を返す。

\end{itemize}

\begin{screen}
\begin{verbatim}
template <class Tfloat, class Tint>
void PS::CommInfo::getMinValue(const Tfloat f_in,
                               const Tint i_in,
                               Tfloat & f_out,
                               Tint & i_out);
\end{verbatim}
\end{screen}

\begin{itemize}

\item{\bf arguments}

{\tt f\_in}: input. Type {\tt const Tfloat}.

{\tt i\_in}: input. Type {\tt const Tint}.

{\tt f\_out}: output. Type {\tt Tfloat \&}. {\tt Tfloat} must be {\tt
PS::F64} or {\tt PS::F32}. The minimum value of {\tt f\_in} among all
processes in the communicator.

{\tt i\_out}: output. Type {\tt Tint \&}. {\tt Tint} must be {\tt
PS::S64}, {\tt PS::S32}, {\tt PS::U64} or {\tt PS::U32}. The rank of
the process of which {\tt f\_in} is the minimum.


%% \item{{\bf 引数}}

%% f\_in: 入力。const Tfloat型。

%% i\_in: 入力。const Tint型。

%% f\_out: 出力。Tfloat型。全プロセスでf\_inの最小値を取
%% り、結果を返す。

%% i\_out: 出力。Tint型。f\_outに伴うIDを返す。

%% \item{{\bf 返り値}}

%% なし。

\end{itemize}

\subsubsubsubsubsection{PS::CommInfo::getMaxValue}

\begin{screen}
\begin{verbatim}
template <class T>
static T PS::CommInfo::getMaxValue(const T val);
\end{verbatim}
\end{screen}

\begin{itemize}

\item{\bf arguments}

{\tt val}: input. Type {\tt const T}.

\item{\bf returned value}

Type {\tt T}. The maximum value of {\tt val} of all processes in the communicator.

%% \item{{\bf 引数}}

%% val: 入力。const T型。

%% \item{{\bf 返り値}}

%% T型。全プロセスでvalの最大値を取り、結果を返す。

\end{itemize}

\begin{screen}
\begin{verbatim}
template <class Tfloat, class Tint>
void PS::CommInfo::getMaxValue(const Tfloat f_in,
                               const Tint i_in,
                               Tfloat & f_out,
                               Tint & i_out);
\end{verbatim}
\end{screen}

\begin{itemize}

\item{\bf arguments}

{\tt f\_in}: input. Type {\tt const Tfloat}. Type {\tt Tfloat} is {\tt
PS::F32} or {\tt PS::F64}.

{\tt i\_in}: input. Type {\tt const Tint}. Type {\tt Tint} is {\tt
PS::S32}.

{\tt f\_out}: output. Type {\tt Tfloat \&}. The maximum value of {\tt
f\_in} among all processes in the communicator.

{\tt i\_out}: output. Type {\tt Tint \&}. The rank of the process of
which {\tt f\_in} is the maximum.

\item{\bf returned value}

None.


%% \item{{\bf 引数}}

%% f\_in: 入力。const Tfloat型。

%% i\_in: 入力。{const Tint}型。

%% {f\_out}: 出力。{Tfloat}型。全プロセスで{f\_in}の最大値を取
%% り、結果を返す。

%% {i\_out}: 出力。{Tint}型。{f\_out}に伴うIDを返す。

%% \item{{\bf 返り値}}

%% なし。

\end{itemize}

\subsubsubsubsubsection{PS::CommInfo::getSum}

\begin{screen}
\begin{verbatim}
template <class T>
T PS::Comm::getSum(const T val);
\end{verbatim}
\end{screen}

\begin{itemize}

\item{\bf arguments}

{\tt val}: input. Type {\tt const T}. Type T is allowed to be {\tt
     PS::S32}, {\tt PS::S64}, {\tt PS::F32}, {\tt PS::F64}, {\tt
     PS::U32}, {\tt PS::U64}, {\tt PS::F32vec} and {\tt F64vec}.
     
%{val}: 入力。{const T}型。

\item{\bf returned value}

Type {\tt T}. Return the global sum of {\tt val} in the communicator.

%% \item{{\bf 引数}}

%% {val}: 入力。{const T}型。

%% \item{{\bf 返り値}}

%% {T}型。全プロセスで{val}の総和を取り、結果を返す。

\end{itemize}

\subsubsubsubsubsection{PS::CommInfo::broadcast}

\begin{screen}
\begin{verbatim}
template <class T>
void PS::CommInfo::broadcast(T * val,
                             const PS::S32 n,
                             const PS::S32 src=0);
\end{verbatim}
\end{screen}

\begin{itemize}

\item{\bf arguments}

{\tt val}: input. Type {\tt T *}. Type {\tt T} is allowed to be any
types.

{\tt n}: input. Type {\tt const PS::S32}. The number of variables.

{\tt src}: input. Type {\tt const PS::S32}. The rank of source
process.


\item{\bf returned value}

None.

\item{\bf function}

Broadcast {\tt val} for the {\tt src}-th process.

%% \item{{\bf 引数}}


%% val: 入力。T *型。

%% n: 入力。const PS::S32型。T型変数の数。

%% src: 入力。const PS::S32型。放送するプロセスランク。デフォルトのランク
%% は0。

%% \item{{\bf 返り値}}

%% なし。

%% \item{{\bf 機能}}

%% プロセスランクsrcのプロセスがn個のT型変数を全プロセスに放送する。

\end{itemize}


%%%%%%%%%%%%%%%%%%%%%%%%%%%%%%%%%%%%%%%%%%%%%%%%%%%%%%%%%%%%%%%%%
%%%%%%%%%%%%%%%%%%%%%%%%%%%%%%%%%%%%%%%%%%%%%%%%%%%%%%%%%%%%%%%%%

%\subsubsubsection{Commクラス}

%%%%%%%%%%%%%%%%%%%%%%%%%%%%%%%%%%%%%%%%%%%%%%%%%%%%%%%%%%%%%%%%%
%\subsubsubsubsection{CommクラスのAPI}

%%%%%%%%%%%%%%%%%%%%%%%%%%%%%%%%%%%%%%%%%%%%%%%%%%%%%%%%%%%%%%%%%
%\subsubsubsection{API}

\subsubsubsubsubsection{PS::Comm::getRank}
\begin{screen}
\begin{verbatim}
static PS::S32 PS::Comm::getRank();
\end{verbatim}
\end{screen}

\begin{itemize}

\item{\bf arguments}

void.

\item{\bf returned value}

Type {\tt PS::S32}.

\item{\bf function}

Return the rank of the calling process.

\end{itemize}

\subsubsubsubsubsection{PS::Comm::getNumberOfProc}

\begin{screen}
\begin{verbatim}
static PS::S32 PS::Comm::getNumberOfProc();
\end{verbatim}
\end{screen}

\begin{itemize}

\item{\bf arguments}

void.

\item{\bf returned value}

Type {\tt PS::S32}. The total number of processes.

\item{\bf function}

Return the total number of processes.

\end{itemize}

\subsubsubsubsubsection{PS::Comm::getRankMultiDim}

\begin{screen}
\begin{verbatim}
static PS::S32 PS::Comm::getRankMultiDim(const PS::S32 id);
\end{verbatim}
\end{screen}

\begin{itemize}

\item{\bf arguments}

{\tt id}: input. Type {\tt const PS::S32}. Id of axes. x-axis:0,
y-axis:1, z-axis:2.

\item{\bf returned value}

Type {\tt PS::S32}. The rank of the calling process along {\tt id}-th
axis. In the case of two dimensional simulations, FDPS returns 0 for
{\tt id}=2.

%PS::S32型。id番目の軸でのランクを返す。2次元の場合、id=2は1を返す。

\end{itemize}

\subsubsubsubsubsection{PS::Comm::getNumberOfProcMultiDim}

\begin{screen}
\begin{verbatim}
static PS::S32 PS::Comm::getNumberOfProcMultiDim(const PS::S32 id);
\end{verbatim}
\end{screen}

\begin{itemize}

\item{\bf }

{\tt id}: input. Type const {\tt PS::S32}. id of axes. x-axis:0,
y-axis:1, z-axis:2.

%id: 入力。const PS::S32型。軸の番号。x軸:0, y軸:1, z軸:2。

\item{\bf returned value}

Type {\tt PS::S32}. The number of processes along {\tt id}-th axis.
In the case of two dimensional simulations, FDPS returns 1 for {\tt
id}=2.

%PS::S32型。id番目の軸のプロセス数を返す。2次元の場合、id=2は1を返す。

\end{itemize}

\subsubsubsubsubsection{PS::Comm::synchronizeConditionalBranchAND}

\begin{screen}
\begin{verbatim}
static bool PS::Comm::synchronizeConditionalBranchAND(const bool local)
\end{verbatim}
\end{screen}

\begin{itemize}

\item{\bf arguments}

{\tt local}: input. Type {\tt const bool}.

%local: 入力。const bool型。

\item{\bf returned value}

{\tt bool} type. Logical product of {\tt local} over all processes.
%bool型。全プロセスでlocalの論理積を取り、結果を返す。

\item{\bf function}

Return logical product of {\tt local} over all processes.

\end{itemize}

\subsubsubsubsubsection{PS::Comm::synchronizeConditionalBranchOR}

\begin{screen}
\begin{verbatim}
static bool PS::Comm::synchronizeConditionalBranchOR(const bool local);
\end{verbatim}
\end{screen}

\begin{itemize}

\item{\bf arguments}

{\tt local}: input. Type {\tt const bool}.
%local: 入力。const bool型。

\item{\bf returned value}

Type {\tt bool}.

\item{\bf function}

Return logical sum of {\tt local} over all processes.
%bool型。全プロセスでlocalの論理和を取り、結果を返す。

\end{itemize}

\subsubsubsubsubsection{PS::Comm::getMinValue}

\begin{screen}
\begin{verbatim}
template <class T>
static T PS::Comm::getMinValue(const T val);
\end{verbatim}
\end{screen}

\begin{itemize}

\item{\bf arguments}

{\tt val}: input. Type {\tt const T}. Type {\tt T} is allowed to be
{\tt PS::F32}, {\tt PS::F64}, {\tt PS::S32}, {\tt PS::S64}, {\tt
PS::U32} and {\tt PS::U64}

%val: 入力。const T型。

\item{\bf returned value}

Type {\tt T}. The minimum value of val of all processes.

%T型。全プロセスでvalの最小値を取り、結果を返す。

\end{itemize}

\begin{screen}
\begin{verbatim}
template <class Tfloat, class Tint>
static void PS::Comm::getMinValue(const Tfloat f_in,
                                  const Tint i_in,
                                  Tfloat & f_out,
                                  Tint & i_out);
\end{verbatim}
\end{screen}

\begin{itemize}

\item{\bf arguments}

{\tt f\_in}: input. Type {\tt const Tfloat}.

{\tt i\_in}: input. Type {\tt const Tint}.

{\tt f\_out}: output. Type {\tt Tfloat \&}. {\tt Tfloat} must be {\tt
PS::F64} or {\tt PS::F32}. The minimum value of {\tt f\_in} among all
processes.

{\tt i\_out}: output. Type {\tt Tint \&}. {\tt Tint} must be {\tt
PS::S64}, {\tt PS::S32}, {\tt PS::U64} or {\tt PS::U32}. The rank of
the process of which {\tt f\_in} is the minimum.


%f\_in: 入力。const Tfloat型。

%i\_in: 入力。const Tint型。

%f\_out: 出力。Tfloat型。全プロセスでf\_inの最小値を取
%り、結果を返す。

%i\_out: 出力。Tint型。f\_outに伴うIDを返す。

\item{\bf returned value}

void.

\end{itemize}

\subsubsubsubsubsection{PS::Comm::getMaxValue}

\begin{screen}
\begin{verbatim}
template <class T>
static T PS::Comm::getMaxValue(const T val);
\end{verbatim}
\end{screen}

\begin{itemize}

\item{\bf arguments}

{\tt val}: input. Type {\tt const T}.
%val: 入力。const T型。

\item{\bf returned value}

Type {\tt T}. The maximum value of {\tt val} of all processes.
%T型。全プロセスでvalの最大値を取り、結果を返す。

\end{itemize}

\begin{screen}
\begin{verbatim}
template <class Tfloat, class Tint>
static void PS::Comm::getMaxValue(const Tfloat f_in,
                                  const Tint i_in,
                                  Tfloat & f_out,
                                  Tint & i_out);
\end{verbatim}
\end{screen}

\begin{itemize}

\item{\bf arguments}

{\tt f\_in}: input. Type {\tt const Tfloat}. Type {\tt Tfloat} is {\tt
PS::F32} or {\tt PS::F64}.

{\tt i\_in}: input. Type {\tt const Tint}. Type {\tt Tint} is {\tt
PS::S32}.

{\tt f\_out}: output. Type {\tt Tfloat \&}. The maximum value of {\tt
f\_in} among all processes.

{\tt i\_out}: output. Type {\tt Tint \&}. The rank of the process of
which {\tt f\_in} is the maximum.

%{i\_out}: 出力。{Tint}型。{f\_out}に伴うIDを返す。

%f\_in: 入力。const Tfloat型。

%i\_in: 入力。{const Tint}型。

%{f\_out}: 出力。{Tfloat}型。全プロセスで{f\_in}の最大値を取
%り、結果を返す。

%{i\_out}: 出力。{Tint}型。{f\_out}に伴うIDを返す。

\item{\bf returned value}

void.

\end{itemize}

\subsubsubsubsubsection{PS::Comm::getSum}

\begin{screen}
\begin{verbatim}
template <class T>
static T PS::Comm::getSum(const T val);
\end{verbatim}
\end{screen}

\begin{itemize}

\item{\bf arguments}

{\tt val}: input. Type {\tt const T}. Type T is allowed to be {\tt
     PS::S32}, {\tt PS::S64}, {\tt PS::F32}, {\tt PS::F64}, {\tt
     PS::U32}, {\tt PS::U64}, {\tt PS::F32vec} and {\tt F64vec}.
     
%{val}: 入力。{const T}型。

\item{\bf returned value}

Type {\tt T}. Return the global sum of {\tt val}.

%{T}型。全プロセスで{val}の総和を取り、結果を返す。

\end{itemize}

\subsubsubsubsubsection{PS::Comm::broadcast}

\begin{screen}
\begin{verbatim}
template <class T>
static void PS::Comm::broadcast(T * val,
                                const PS::S32 n,
                                const PS::S32 src=0);
\end{verbatim}
\end{screen}

\begin{itemize}

\item{\bf arguments}

{\tt val}: input. Type {\tt T *}. Type {\tt T} is allowed to be any
types.

{\tt n}: input. Type {\tt const PS::S32}. The number of variables.

{\tt src}: input. Type {\tt const PS::S32}. The rank of source
process.

%val: 入力。T *型。

%n: 入力。const PS::F32型。T型変数の数。

%src: 入力。const PS::F32型。放送するプロセスランク。デフォルトのランク
%は0。

\item{\bf returned value}

void.

\item{\bf function}

Broadcast {\tt val} for the {\tt src}-th process.

%プロセスランクsrcのプロセスがn個のT型変数を全プロセスに放送する。

\end{itemize}
