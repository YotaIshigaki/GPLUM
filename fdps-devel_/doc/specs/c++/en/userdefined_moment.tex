\subsubsection{Summary}

The class \texttt{Moment} encapsulates the values of moment and related
quantities of a set of grouped particles
(see, step 0 in Sec. \ref{sec:overview_action}).
The examples of moment are monopole, dipole and radius of the largest
particle and so on. This class is used for intermediate variables to
make \texttt{SuperParticleJ} from \texttt{EssentialParticleJ}.
Thus, this class has a member function that calculates the moment from
\texttt{EssentialParticleJ} or \texttt{SuperParticleJ}.

Since these processes have prescribed form, FDPS has pre-defined classes.
In this section we first describe the pre-defined class and then describe
the rules in the case users define it.

\subsubsection{Pre-defined class}

\subsubsubsection{Summary}

FDPS has several pre-defined \texttt{Moment} classes, which are used if
specific case are chosen as \texttt{PS::SEARCH\_MODE} in \texttt{PS::TreeForForce}.
Below, we describe about the \texttt{Moment} class which can be chosen in each \texttt{PS::SEARCH\_MODE}.

\subsubsubsection{PS::SEARCH\_MODE\_LONG}

\subsubsubsubsection{PS::MomentMonopole}
\label{sec:MomentMonopole}

This class encapsulates the monopole moment.
The reference point for the calculation of monopole is set to center of mass or center of charge.

\begin{screen}
\begin{verbatim}
namespace ParticleSimulator {
    class MomentMonopole {
    public:
        F64    mass;
        F64vec pos;
    };
}
\end{verbatim}
\end{screen}

\begin{itemize}

\item Name of the class

  \texttt{PS::MomentMonopole}

\item Members and their information

  \texttt{mass}: accumulated mass or electron charge.

  \texttt{pos}: center of mass or center of charge.

\item Terms of use

  The class \texttt{EssentialParticleJ} (see, Sec. \ref{sec:essentialparticlej}) has \texttt{EssentialParticleJ::getCharge} and \texttt{EssentialParticleJ::getPos}
  and returns mass/electron charge and position.
  Users can use an arbitrary name instead of \texttt{EssentialParticleJ}.

\end{itemize}

\subsubsubsubsection{PS::MomentQuadrupole}
\label{sec:MomentQuadrupole}

This class encapsulates the monopole and quadrupole moment.
The reference point for the calculation of moments is set to the center of mass.

\begin{screen}
\begin{verbatim}
namespace ParticleSimulator {
    class MomentQuadrupole {
    public:
        F64    mass;    
        F64vec pos;
        F64mat quad;
    };
}
\end{verbatim}
\end{screen}

\begin{itemize}

\item Name of the class

  \texttt{PS::MomentQuadrupole}

\item Members and their information

  \texttt{mass}: accumulated mass.

  \texttt{pos}: center of mass.

  \texttt{quad}: accumulated quadrupole.

\item Terms of use

  The class \texttt{EssentialParticleJ} (see, Sec. \ref{sec:essentialparticlej}) has \texttt{EssentialParticleJ::getCharge} and \texttt{EssentialParticleJ::getPos}
  and returns mass/electron charge and position.
  Users can use an arbitrary name instead of \texttt{EssentialParticleJ}.

\end{itemize}

\subsubsubsubsection{PS::MomentMonopoleGeometricCenter}
\label{sec:MomentMonopoleGeometricCenter}


This class encapsulates the monopole moment.
The reference point for the calculation of monopole is set to the geometric center.

\begin{screen}
\begin{verbatim}
namespace ParticleSimulator {
    class MomentMonopoleGeometricCenter {
    public:
        F64    charge;    
        F64vec pos;
    };
}
\end{verbatim}
\end{screen}

\begin{itemize}

\item Name of the class

  \texttt{PS::MomentMonopoleGeometricCenter}

\item Members and their information

  \texttt{charge}: accumulated mass/charge.

  \texttt{pos}: geometric center.

\item Terms of use

  The class \texttt{EssentialParticleJ} (see, Sec. \ref{sec:essentialparticlej}) has \texttt{EssentialParticleJ::getCharge} and \texttt{EssentialParticleJ::getPos}
  and returns mass/electron charge and position.
  Users can use an arbitrary name instead of \texttt{EssentialParticleJ}.

\end{itemize}

\subsubsubsubsection{PS::MomentDipoleGeometricCenter}
\label{sec:MomentDipoleGeometricCenter}

This class encapsulates the moments up to dipole.
The reference point for the calculation of moments is set to geometric center.

\begin{screen}
\begin{verbatim}
namespace ParticleSimulator {
    class MomentDipoleGeometricCenter {
    public:
        F64    charge;    
        F64vec pos;
        F64vec dipole;
    };
}
\end{verbatim}
\end{screen}

\begin{itemize}

\item Name of the class

  \texttt{PS::MomentDipoleGeometricCenter}

\item Members and their information

  \texttt{charge}: accumulated mass/charge.

  \texttt{pos}: geometric center.

  \texttt{dipole}: dipole of the particle mass or electric charge.

\item Terms of use

  The class \texttt{EssentialParticleJ} (see, Sec. \ref{sec:essentialparticlej})
  has \texttt{EssentialParticleJ::getCharge} and \texttt{EssentialParticleJ::getPos}
  and returns mass/electron charge and position.
  Users can use an arbitrary name instead of \texttt{EssentialParticleJ}.

\end{itemize}

\subsubsubsubsection{PS::MomentQuadrupoleGeometricCenter}
\label{sec:MomentQuadrupoleGeometricCenter}

This class encapsulates the moments up to quadrupole.
The reference point for the calculation of moments is set to geometric center.

\begin{screen}
\begin{verbatim}
namespace ParticleSimulator {
    class MomentQuadrupoleGeometricCenter {
    public:
        F64    charge;    
        F64vec pos;
        F64vec dipole;
        F64mat quadrupole;
    };
}
\end{verbatim}
\end{screen}

\begin{itemize}

\item Name of the class

  \texttt{PS::MomentQuadrupoleGeometricCenter}

\item Members and their information

  \texttt{charge}: accumulated mass/charge.

  \texttt{pos}: geometric center.

  \texttt{dipole}: dipole of the particle mass or electron charge.

  \texttt{quadrupole}: quadrupole of the particle mass or electron charge.

\item Terms of use

  The class \texttt{EssentialParticleJ} (see, Sec. \ref{sec:essentialparticlej})
  has \texttt{EssentialParticleJ::getCharge} and \texttt{EssentialParticleJ::getPos}
  and returns mass/electron charge and position.
  Users can use an arbitrary name instead of \texttt{EssentialParticleJ}.

\end{itemize}
  
\subsubsubsection{PS::SEARCH\_MODE\_LONG\_SCATTER}

\subsubsubsubsection{PS::MomentMonopoleScatter}
\label{sec:MomentMonopoleScatter}

This class encapsulates the monopole moment. The reference point for the calculation of monopole is set to center of mass or center of charge.

\begin{screen}
\begin{verbatim}
namespace ParticleSimulator {
    class MomentMonopoleScatter {
    public:
        F64    mass;
        F64vec pos;
        F64ort vertex_out_;
        F64ort vertex_in_;
    };
}
\end{verbatim}
\end{screen}

\begin{itemize}
\item Name of the class
  \texttt{PS::MomentMonopoleScatter}

\item Members and their information

  \texttt{mass}: accumulated mass or electron charge.

  \texttt{pos}: center of mass or center of charge.
        
  \texttt{vertex\_out\_}: positions of vertices of the smallest rectangular parallelepiped including all grouped particles when each particle is regarded as a sphere whose radius is the returned value of EssentialParticleJ::getRSearch.
  
  \texttt{vertex\_in\_}: positions of vertices of the smallest rectangular parallelepiped including all grouped particles.

\item Terms of use

  The class \texttt{EssentialParticleJ} (see, Sec. \ref{sec:essentialparticlej}) has \texttt{EssentialParticleJ::getCharge}, \texttt{EssentialParticleJ::getPos} and \texttt{EssentialParticleJ::getRSearch} and returns mass/electron charge, position and cutoff length. Users can use an arbitrary name instead of the name of member functions of \texttt{EssentialParticleJ}.

\end{itemize}

\subsubsubsubsection{PS::MomentQuadrupoleScatter}
\label{sec:MomentQuadrupoleScatter}

This class encapsulates the monopole and quadrupole moments. The reference point for the calculation of monopole is set to center of mass or center of charge.

\begin{screen}
\begin{verbatim}
namespace ParticleSimulator {
    class MomentQuadrupoleScatter {
    public:
        F64    mass;
        F64vec pos;
        F64mat quad;
        F64ort vertex_out_;
        F64ort vertex_in_;
    };
}
\end{verbatim}
\end{screen}

\begin{itemize}
\item Name of the class
        
  \texttt{PS::MomentQuadrupoleScatter}

\item Members and their information

  \texttt{mass}: accumulated mass or electron charge.

  \texttt{pos}: center of mass or center of charge.
        
  \texttt{quad}: accumulated quadrupole.
        
  \texttt{vertex\_out\_}: positions of vertices of the smallest rectangular parallelepiped including all grouped particles when each particle is regarded as a sphere whose radius is the returned value of EssentialParticleJ::getRSearch.
  
  \texttt{vertex\_in\_}: positions of vertices of the smallest rectangular parallelepiped including all grouped particles.

\item Terms of use

  The class \texttt{EssentialParticleJ} (see, Sec. \ref{sec:essentialparticlej}) has \texttt{EssentialParticleJ::getCharge}, \texttt{EssentialParticleJ::getPos} and \texttt{EssentialParticleJ::getRSearch} and returns mass/electron charge, position and cutoff length. Users can use an arbitrary name instead of the name of member functions of \texttt{EssentialParticleJ}.

\end{itemize}

\subsubsubsection{PS::SEARCH\_MODE\_LONG\_SYMMETRY}

\subsubsubsubsection{PS::MomentMonopoleSymmetry}
\label{sec:MomentMonopoleSymmetry}

This class encapsulates the monopole moment. The reference point for the calculation of monopole is set to center of mass or center of charge.

\begin{screen}
\begin{verbatim}
namespace ParticleSimulator {
    class MomentMonopoleSymmetry {
    public:
        F64    mass;
        F64vec pos;
        F64ort vertex_out_;
        F64ort vertex_in_;
    };
}
\end{verbatim}
\end{screen}

\begin{itemize}
\item Name of the class
        
  \texttt{PS::MomentMonopoleSymmetry}

\item Members and their information

  \texttt{mass}: accumulated mass or electron charge.

  \texttt{pos}: center of mass or center of charge.
        
  \texttt{vertex\_out\_}: positions of vertices of the smallest rectangular parallelepiped including all grouped particles when each particle is regarded as a sphere whose radius is the returned value of EssentialParticleJ::getRSearch.
  
  \texttt{vertex\_in\_}: positions of vertices of the smallest rectangular parallelepiped including all grouped particles.

\item Terms of use

  The class \texttt{EssentialParticleJ} (see, Sec. \ref{sec:essentialparticlej}) has \texttt{EssentialParticleJ::getCharge}, \texttt{EssentialParticleJ::getPos} and \texttt{EssentialParticleJ::getRSearch} and returns mass/electron charge, position and cutoff length. Users can use an arbitrary name instead of the name of member functions of \texttt{EssentialParticleJ}.

\end{itemize}

\subsubsubsubsection{PS::MomentQuadrupoleSymmetry}
\label{sec:MomentQuadrupoleSymmetry}

This class encapsulates the monopole and quadrupole moments. The reference point for the calculation of monopole is set to center of mass or center of charge.

\begin{screen}
\begin{verbatim}
namespace ParticleSimulator {
    class MomentQuadrupoleSymmetry {
    public:
        F64    mass;
        F64vec pos;
        F64mat quad;
        F64ort vertex_out_;
        F64ort vertex_in_;
    };
}
\end{verbatim}
\end{screen}

\begin{itemize}

\item Name of the class

  \texttt{PS::MomentQuadrupoleSymmetry}

\item Members and their information

  \texttt{mass}: accumulated mass or electron charge.

  \texttt{pos}: center of mass or center of charge.
        
  \texttt{quad}: accumulated quadrupole.
        
  \texttt{vertex\_out\_}: positions of vertices of the smallest rectangular parallelepiped including all grouped particles when each particle is regarded as a sphere whose radius is the returned value of EssentialParticleJ::getRSearch.
  
  \texttt{vertex\_in\_}: positions of vertices of the smallest rectangular parallelepiped including all grouped particles.

\item Terms of use

  The class \texttt{EssentialParticleJ} (see, Sec. \ref{sec:essentialparticlej}) has \texttt{EssentialParticleJ::getCharge}, \texttt{EssentialParticleJ::getPos} and \texttt{EssentialParticleJ::getRSearch} and returns mass/electron charge, position and cutoff length. Users can use an arbitrary name instead of the name of member functions of \texttt{EssentialParticleJ}.

\end{itemize}

\subsubsubsection{PS::SEARCH\_MODE\_LONG\_CUTOFF}

\subsubsubsubsection{PS::MomentMonopoleCutoff}
\label{sec:MomentMonopoleCutoff}

This class encapsulates the monopole moment.
The reference point for the calculation of monopole is set to center of mass or center of charge.

\begin{screen}
\begin{verbatim}
namespace ParticleSimulator {
    class MomentMonopoleCutoff {
    public:
        F64    mass;
        F64vec pos;
    };
}
\end{verbatim}
\end{screen}

\begin{itemize}

\item Name of the class

  \texttt{PS::MomentMonopoleCutoff}

\item Members and their information

  \texttt{mass}: accumulated mass or electron charge.

  \texttt{pos}: center of mass or center of charge.

\item Terms of use

  The class \texttt{EssentialParticleJ} (see, Sec. \ref{sec:essentialparticlej})
  has \texttt{EssentialParticleJ::getCharge}, \texttt{EssentialParticleJ::getPos}
  and \texttt{EssentialParticleJ::getRSearch} and returns mass/electron charge,
  position and cutoff length. Users can use an arbitrary name instead of the name
  of member functions of \texttt{EssentialParticleJ}.

\end{itemize}

\subsubsection{Required member functions}

\subsubsubsection{Summary}

%以下ではMomentクラスを定義する際に、必要なメンバ関数を記述する。このと
%きMomentクラスのクラス名をMomとする。これは変更自由である。
Below, the required member functions of \texttt{Moment} class are described.
In this section we use the name \texttt{Mom} as a \texttt{Moment} class.

\subsubsubsection{Constructor}

\begin{screen}
\begin{verbatim}
class Mom {
public:
    Mom ();
};
\end{verbatim}
\end{screen}

\begin{itemize}
  
\item {\bf Arguments}

  None.
  
\item {\bf Returns}

  None.

\item {\bf Behaviour}

  %Momクラスのオブジェクトの初期化をする。
  Initialize the instance of \texttt{Mom} class.
  
\end{itemize}

\subsubsubsection{Mom::init}

\begin{screen}
\begin{verbatim}
class Mom {
public:
    void init();
};
\end{verbatim}
\end{screen}

\begin{itemize}
  
\item {\bf Arguments}

  None.
  
\item {\bf Returns}

  None.

\item {\bf Behaviour}

  %Momクラスのオブジェクトの初期化をする。
  Initialize the instance of \texttt{Mom} class.

\end{itemize}

\subsubsubsection{Mom::getPos}

\begin{screen}
\begin{verbatim}
class Mom {
public:
    PS::F32vec getPos();
};
\end{verbatim}
\end{screen}

\begin{itemize}

\item {\bf Arguments}

  None.
  
\item {\bf Returns}

  %PS::F32vecまたはPS::F64vec型。Momクラスのメンバ変数pos。
  \texttt{PS::F32vec} or \texttt{PS::F64vec}.
  Returns the position of a variable of class.

\end{itemize}

\subsubsubsection{Mom::getCharge}

\begin{screen}
\begin{verbatim}
class Mom {
public:
    PS::F32 getCharge() const;
};
\end{verbatim}
\end{screen}

\begin{itemize}

\item {\bf Arguments}

  None.
  
\item {\bf Returns}

  %PS::F32またはPS::F64型。Momクラスのメンバ変数mass。
  \texttt{PS::F32} or \texttt{PS::F64}.
  Returns the mass or charge of a variable of class \texttt{Mom}.

\end{itemize}

\subsubsubsection{Mom::accumulateAtLeaf}

\begin{screen}
\begin{verbatim}
class Mom {
public:
    template <class Tepj>
    void accumulateAtLeaf(const Tepj & epj);
};
\end{verbatim}
\end{screen}

\begin{itemize}

\item {\bf Arguments}

  %epj: 入力。const Tepj \&型。Tepjのオブジェクト。
  \texttt{epj}: Input. \texttt{const Tepj \&} type. The object of \texttt{Tepj}.

\item {\bf Returns}

  None.

\item {\bf Behaviour}

  %EssentialParticleJクラスのオブジェクトからモーメントを計算する。
  Accumulate the multipole moment from \texttt{EssentialParticleJ}.

\end{itemize}

\subsubsubsection{Mom::accumulate}

\begin{screen}
\begin{verbatim}
class Mom {
public:
    void accumulate(const Mom & mom);
};
\end{verbatim}
\end{screen}

\begin{itemize}

\item {\bf Arguments}

  %mom: 入力。const Mom \&型。Momクラスのオブジェクト。
  \texttt{mom}: Input. \texttt{const Mom \&} type. The instance of \texttt{Mom} class.

\item {\bf Returns}

  None.

\item {\bf Behaviour}

  %MomクラスのオブジェクトからさらにMomクラスの情報を計算する。
  Accumulate the multipole moments in \texttt{Mom} class from \texttt{Mom} class objects.

\end{itemize}

\subsubsubsection{Mom::set}

\begin{screen}
\begin{verbatim}
class Mom {
public:
    void set();
};
\end{verbatim}
\end{screen}

\begin{itemize}

\item {\bf Arguments}

  None.
  
\item {\bf Returns}

  None.

\item {\bf Behaviour}

  %上記のメンバ関数Mom::accumulateAtLeaf, Mom::accumulateではモー
  %メントの位置情報の規格化ができていない場合ので、ここで規格化する。
  Normalize the multipole moments, since the member functions
  \texttt{Mom::accumulateAtLeaf} and \texttt{Mom::accumulate} do not
  normalize the multipole moment.

\end{itemize}

\subsubsubsection{Mom::accumulateAtLeaf2}

\begin{screen}
\begin{verbatim}
class Mom {
public:
    template <class Tepj>
    void accumulateAtLeaf2(const Tepj & epj);
};
\end{verbatim}
\end{screen}

\begin{itemize}

\item {\bf Arguments}

  %epj: 入力。const Tepj \&型。Tepjのオブジェクト。
  \texttt{epj}: Input. \texttt{const Tepj \&} type. The instance of \texttt{Teps}.

\item {\bf Returns}

  None.

\item {\bf Behaviour}

  %EssentialParticleJクラスのオブジェクトからモーメントを計算する。
  Accumulate the multipole moments in \texttt{Mom} class from \texttt{EssentialParticleJ} class objects.

\end{itemize}

\subsubsubsection{Mom::accumulate2}

\begin{screen}
\begin{verbatim}
class Mom {
public:
    void accumulate(const Mom & mom);
};
\end{verbatim}
\end{screen}

\begin{itemize}

\item {\bf Arguments}

  %mom: 入力。const Mom \&型。Momクラスのオブジェクト。
  \texttt{mom}: Input. \texttt{const Mom \&} type. The instance of \texttt{Mom} class.

\item {\bf Returns}

  None.

\item {\bf Behaviour}

  %MomクラスのオブジェクトからさらにMomクラスの情報を計算する。
  Accumulate the multipole moments in \texttt{Mom} class from \texttt{Mom} class objects.

\end{itemize}

%%%%%%%%%%%%%%%%%%%%%%%%%%%%%%%%%%%%%%%%%%%%%%%%%%%%%%
\subsubsection{Required member functions for specific cases}

\subsubsubsection{Summary}

Below, the required member functions of \texttt{Moment} class for specific cases are described. In this section we use the name \texttt{Mom} as a \texttt{Moment} class.

\subsubsubsection{One of \texttt{PS::SEARCH\_MODE\_LONG\_CUTOFF}, \newline \texttt{PS::SEARCH\_MODE\_LONG\_SCATTER}, \newline \texttt{PS::SEARCH\_MODE\_LONG\_SYMMETRY} \newline is used as \texttt{PS::SEARCH\_MODE}}
%--------------
\subsubsubsection{Mom::getVertexIn}

\begin{screen}
\begin{verbatim}
class Mom {
public:
    F64ort getVertexIn();
};
\end{verbatim}
\end{screen}

\begin{itemize}

\item {\bf Arguments}

None.
  
\item {\bf Returns}

\texttt{PS::F32ort} or \texttt{PS::F64ort} types.

\item {\bf Behaviour}

  Return the positions of two vertices describing the smallest cuboid (rectangle if the space dimension is two) that contains all the particles corresponding to this \texttt{Mom} class.

\item {\bf Notes}

The positional information above must be calculated correctly in each tree cell. For this end, users must implement coordinate calculations into the member functions \texttt{accumulateAtLeaf} and \texttt{accumulate}, which are used, respectively, to calculate the moment information at the leaf cells (i.e. the tree cells at the deepest levels of a tree structure) from particles and to calculate the moment information of a parent tree cell from its child tree cells. Below, we show an example of the implementation of the member function \texttt{getVertexIn}.
  
\begin{screen}
\begin{verbatim}
class Mom {
public:
    // vertex_in_: Member variable storing the positional
    //             information of a cuboid (or rectangle)
    F64ort vertex_in_; 
    F64ort getVertexIn() const { return vertex_in_; }
    template<class Tepj>
    void accumulateAtLeaf(const Tepj & epj){ 
         // Other calculations
        (this->vertex_in_).merge(epj.getPos());
    }
    void accumulate(const Mom & mom){
        // Other calculations
        (this->vertex_in_).merge(mom.vertex_in_);
    }
};
\end{verbatim}
\end{screen}
where \texttt{merge} is one of the member functions of Orthotope type.
Users can freely change the name of this member variable.
  
\end{itemize}

%--------------
\subsubsubsection{Mom::getVertexOut}

\begin{screen}
\begin{verbatim}
class Mom {
public:
    F64ort getVertexOut();
};
\end{verbatim}
\end{screen}

\begin{itemize}

\item {\bf Arguments}

None.
  
\item {\bf Returns}

\texttt{PS::F32ort} or \texttt{PS::F64ort} type.

\item {\bf Behaviour}

Return the positions of two vertices describing the smallest cuboid (rectangle if the space dimension is two) that contains all of spheres (circles if the space dimension is two) whose centers and radii are, respectively, the positions and the returned values of \texttt{getRSearch} of particles corresponding to this \texttt{Mom} class. 

\item {\bf Notes}

As in the case of member function \texttt{getVertexIn}, the positional information above must be calculated correctly in each tree cell. For this end, users must implement coordinate calculations in the member functions \texttt{accumulateAtLeaf} and \texttt{accumulate}.Below, we show an example of the implementation of the member function \texttt{getVertexOut}.
  
\begin{screen}
\begin{verbatim}
class Mom {
public:
    // vertex_out_: Member variable storing the positional
    //              information of a cuboid (or rectangle)
    F64ort vertex_out_; 
    F64ort getVertexOut() const { return vertex_out_; }
    template<class Tepj>
    void accumulateAtLeaf(const Tepj & epj){ 
         // Other calculations
         (this->vertex_out_).merge(epj.getPos(), epj.getRSearch());
    }
    void accumulate(const Mom & mom){
        // Other calculations
        (this->vertex_out_).merge(mom.vertex_out_);
    }
};
\end{verbatim}
\end{screen}
where \texttt{merge} is a member function of Orthotope type.
Users can freely change the name of the member variable.
  
\end{itemize}