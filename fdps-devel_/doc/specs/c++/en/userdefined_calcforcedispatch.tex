
%\subsubsection{概要}
\subsubsection{Summary}

%% 関数calcForceDispatchは関数calcForceRetrieveと合わせて粒子同士の相互作
%% 用を記述するものであり、calcForceSpEp やcalcForceEpEp の代わりに相互作
%% 用の定義(節\ref{sec:overview_action}の手順0)に使うことができる。
%% calcForceSpEp やcalcForceEpEp との違いは、calcForceDispatch は複数の相
%% 互作用リストと i粒子リストを受け取ることである。これにより、GPGPU 等の
%% アクセラレータを起動する回数を削減し、実行効率を向上させる。以下、これ
%% の書き方の規定を記述する。関数calcForceDispatchの名前はGravityDispatch
%% とする。これは変更自由である。また、EssentialParitlceIクラスのクラス名
%% をEPI, EssentialParitlceJクラスのクラス名をEPJ, SuperParitlceJクラスの
%% クラス名をSPJとする。

Functor \texttt{calcForceDispatch}, paired with
functor \texttt{calcForceRetrieve}, defines the interaction between two particles.
This functor can be used  for the calculation of interactions (see,
step 0 in Sec. \ref{sec:overview_action}), instead of 
\texttt{calcForceSpEp} and \texttt{calcForceEpEp}.
The difference from \texttt{calcForceSpEp}
and \texttt{calcForceEpEp} is that \texttt{calcForceDispatch} receives
multiple interactions lists and i-particle lists. By doing so,
it reduces the numbere of kernel calls to acclerators such as
GPGPU, and improves the efficiency.
The name of the functor 
\texttt{calcForceDispatch} is \texttt{GravityDispatch}.
The name for classes \texttt{EssentialParitlceI},
\texttt{EssentialParitlceJ} and \texttt{SuperParitlceJ} are
\texttt{EPI},
\texttt{EPJ}, and
\texttt{SPI}, respectively. These names can be changed to any name
legal.

%\subsubsection{短距離力の場合}
\subsubsection{The case of short-range interactions}

\begin{lstlisting}[caption=calcForceDispatch]
class EPI;
class EPJ;
PS::S32 HydroforceDispatch(const PS::S32  tag,
                           const PS::S32  nwalk,
                           const EPI**      epi,
                           const PS::S32*  ni,
                           const EPJ**      epj,
                           const PS::S32*  nj_ep;
};
\end{lstlisting}

\begin{itemize}

%% \item {\bf 引数}

%%   tag: 入力。const PS::S32 型。tagの番号。発行されるtagの番号は0から関
%%   数PS::TreeForForce::calcForceAllandWriteBackMultiWalk()の第三引数と
%%   して設定された値から1引いた数までである。tagの番号は
%%   calcForceRetrieve()で設定するtagの番号と対応させる必要がある。

%%   nwalk: 入力。const PS::S32 型。walkの数。walkの数の最大値は

%%   PS::TreeForForce::calcForceAllandWriteBackMultiWalk()の第六引数の値である。

%%   epi: 入力。const EPI** 型。i粒子情報を持つポインタのポインタ。

%%   ni: 入力。const PS::S32*型。i粒子数のポインタ。

%%   epj: 入力。const EPJ** 型。j粒子情報を持つポインタのポインタ。
  
%%   nj\_ep: 入力。const PS::S32* 型。j粒子数のポインタ。

%% \item {\bf 返値}

%%   PS::S32型。ユーザーは正常に実行された場合は0を、エラーが起こった場合
%%   は0以外の値を返すようにする。
  
%% \item {\bf 機能}

%% epi,epjの情報をアクセラレータに送り、相互作用カーネルを発行する。
\item {\bf Arguments}

  {\tt tag}: input. Type {\tt const PS::S3}. The value should be
  non-negative and less than the value specified by the third argument
  of function\\ \texttt{PS::TreeForForce::calcForceAllandWriteBackMultiWalk()}.
   Corresponding call to {\tt
    CalcForceRetrieve()} should use the same value for {\tt tag}.

  {\tt nwalk}: input.Type {\tt const PS::S32}. The number of
  interaction lists. The value should not exceed
  the value of the  sixth argument of\\
  \texttt{PS::TreeForForce::calcForceAllandWriteBackMultiWalk()}.
  
  {\tt epi:} input. Type {\tt const EPI**}. Array of
  the array of i-particles.

  {\tt ni:} input. Type {\tt const PS::S32*}. Array
  of the numbers of i-particles.

  {\tt epj:} input. Type {\tt const EPJ**}. Array of
  the array of j-particles.
  
  {\tt nj:} input. Type {\tt const PS::S32*}. Array
  of the number of j-paricles.


\item {\bf Return value}

  Type \texttt{PS::S32}. Should return 0 upon normal completion, and
  otherwize non-zero value.
  
\item {\bf Function}

  Send {\tt epi} and {\tt epj} to the accelerator and let the
  accelerator do the interaction calculation.
  
\end{itemize}

\subsubsection{Case of long-range interaction}

\begin{lstlisting}[caption=calcForceDispatch]
class EPI;
class EPJ;
class SPJ;
PS::S32 GravityDispatch(const PS::S32   tag,
                        const PS::S32   nwalk,
                        const EPI**     epi,
                        const PS::S32*  ni,
                        const EPJ**     epj,
                        const PS::S32*  nj_ep,
                        const SPJ**     spj,
                        const PS::S32*  nj_sp);
};
\end{lstlisting}

\begin{itemize}

%% \item {\bf 引数}


%% tag: 入力。const PS::S32 型。tagの番号。発行されるtagの番号は0から関
%% 数PS::TreeForForce::calcForceAllandWriteBackMultiWalk()の第三引数と
%% して設定された値から1引いた数までである。tagの番号は
%% calcForceRetrieve()で設定するtagの番号と対応させる必要がある。


%%   nwalk: 入力。const PS::S32 型。walkの数。walkの数の最大値は
%%   PS::TreeForForce::calcForceAllandWriteBackMultiWalk()の第六引数の値
%%   である。

%%   epi: 入力。const EPI** 型。i粒子情報を持つ配列の配列。

%%   ni: 入力。const PS::S32*型。i粒子数の配列。

%%   epj: 入力。const EPJ** 型。j粒子情報を持つ配列の配列。
  
%%   nj\_ep: 入力。const PS::S32* 型。j粒子数の配列。

%%   spj: 入力。const SPJ** 型。j粒子情報を持つ配列の配列。
  
%%   nj\_sp: 入力。const PS::S32* 型。j粒子数の配列。

%% \item {\bf 返値}

%%   PS::S32型。ユーザーは正常に実行された場合は0を、エラーが起こった場合
%%   は0以外の値を返すようにする。
  
%% \item {\bf 機能}

%% epi,epj,spjの情報をアクセラレータに送り、相互作用カーネルを発行する。

\item {\bf Arguments}

  {\tt tag}: input. Type {\tt const PS::S3}. The value should be
  non-negative and less than the value specified by the third argument
  of function\\
  \texttt{PS::TreeForForce::calcForceAllandWriteBackMultiWalk()}.
   Corresponding call to {\tt
    CalcForceRetrieve()} should use the same value for {\tt tag}.

  {\tt nwalk}: input.Type {\tt const PS::S32}. The number of
  interaction lists. The value should not exceed
  the value of the  sixth argument of\\
  \texttt{PS::TreeForForce::calcForceAllandWriteBackMultiWalk()}.
  
  {\tt epi:} input. Type {\tt const EPI**}. Array of
  the array of i-particles.

  {\tt ni:} input. Type {\tt const PS::S32*}. Array
  of the numbers of i-particles.

  {\tt epj:} input. Type {\tt const EPJ**}. Array of
  the array of j-particles.
  
  {\tt nj:} input. Type {\tt const PS::S32*}. Array
  of the number of j-paricles.

  {\tt spj:} input. Type {\tt const SPJ**}. Array of
  the array of superparticles.
  
  {\tt nj\_sp:} input. Type {\tt const PS::S32*}. Array
  of the number of superparicles.


\item {\bf Return value}

  Type \texttt{PS::S32}. Should return 0 upon normal completion, and
  otherwize non-zero value.
  
\item {\bf Function}

  Send {\tt epi} and {\tt epj} to the accelerator and let the
  accelerator do the interaction calculation.

\end{itemize}

