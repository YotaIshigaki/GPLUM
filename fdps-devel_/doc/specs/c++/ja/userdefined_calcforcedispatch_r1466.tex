
\subsubsection{概要}

関数calcForceDispatchは関数calcForceRetrieveと合わせて粒子同士の相互作
用を記述するものであり、calcForceSpEp やcalcForceEpEp の代わりに相互作
用の定義(節\ref{sec:overview_action}の手順0)に使うことができる。
calcForceSpEp やcalcForceEpEp との違いは、calcForceDispatch は複数の相
互作用リストと i粒子リストを受け取ることである。これにより、GPGPU 等の
アクセラレータを起動する回数を削減し、実行効率を向上させる。相互作用リ
ストを作る方法として、粒子自体の相互作用リストを作る方法と初期に必要と
なる全粒子の情報をデバイスメモリー上に送り、粒子インデックスのみの相互
作用リストを作る方法の二通りが用意されている。以下、これの書き方の規定
を記述する。関数calcForceDispatchの名前はGravityDispatchとする。これは
変更自由である。また、EssentialParitlceIクラスのクラス名をEPI,
EssentialParitlceJクラスのクラス名をEPJ, SuperParitlceJクラスのクラス
名をSPJとする。

\subsubsection{短距離力の場合}
\subsubsubsection{粒子の相互作用リストを作る場合}
%%%%%%%%%%%%%%%%%%%%%%%%%%%%%%%%%%%%%
\begin{lstlisting}[caption=calcForceDispatch]
class EPI;
class EPJ;
PS::S32 HydroforceDispatch(const PS::S32  tag,
                           const PS::S32  nwalk,
                           const EPI**      epi,
                           const PS::S32*  ni,
                           const EPJ**      epj,
                           const PS::S32*  nj_ep;
};
\end{lstlisting}

\begin{itemize}

\item {\bf 引数}

  tag: 入力。const PS::S32 型。tagの番号。発行されるtagの番号は\\
  0から関数PS::TreeForForce::calcForceAllandWriteBackMultiWalk()の\\
  第三引数として設定された値から1引いた数までである。\\
  tagの番号はcalcForceRetrieve()で設定するtagの番号と対応させる必要がある。

  nwalk: 入力。const PS::S32 型。walkの数。walkの数の最大値は\\
  PS::TreeForForce::calcForceAllandWriteBackMultiWalk()の第六引数の値である。

  epi: 入力。const EPI** 型。i粒子情報を持つポインタのポインタ。

  ni: 入力。const PS::S32*型。i粒子数のポインタ。

  epj: 入力。const EPJ** 型。j粒子情報を持つポインタのポインタ。
  
  nj\_ep: 入力。const PS::S32* 型。j粒子数のポインタ。

\item {\bf 返値}

  PS::S32型。ユーザーは正常に実行された場合は0を、エラーが起こった場合
  は0以外の値を返すようにする。
  
\item {\bf 機能}

毎回epi,epjの相互作用リストをアクセラレータに送り、相互作用カーネルを
発行する。
  
\end{itemize}
%%%%%%%%%%%%%%%%%%%%%%%%%%%%%%%%%%%%%

\subsubsubsection{粒子のインデックスのみの相互作用リストを作る場合}
%%%%%%%%%%%%%%%%%%%%%%%%%%%%%%%%%%%%%
\begin{lstlisting}[caption=calcForceDispatch]
PS::S32 GravityDispatch(const PS::S32   tag,
                        const PS::S32   nwalk,
                        const TEpi**    iptcl,
                        const PS::S32*  ni,
                        const PS::S32** id_jptcl_ep,
                        const PS::S32*  nj_ep,
                        const TEpj*     jptcl_ep,
                        const PS::S32   n_send_ep,
                        const bool send_ptcl);
\end{lstlisting}

\begin{itemize}

\item {\bf 引数}

  tag: 入力。const PS::S32 型。tagの番号。発行されるtagの番号は\\
  0から関数PS::TreeForForce::calcForceAllandWriteBackMultiWalk()の\\
  第三引数として設定された値から1引いた数までである。\\
  tagの番号はcalcForceRetrieve()で設定するtagの番号と対応させる必要がある。

  nwalk: 入力。const PS::S32 型。walkの数。walkの数の最大値は
  PS::TreeForForce::calcForceAllandWriteBackMultiWalk()の第六引数の値
  である。

  epi: 入力。const EPI** 型。i粒子情報を持つ配列の配列。

  ni: 入力。const PS::S32*型。i粒子数の配列。

  id\_epj: 入力。const EPJ** 型。EPJのインデックスの配列の配列。
  
  nj\_ep: 入力。const PS::S32* 型。j粒子数の配列。

  epj: 入力。const EPJ* 型。j粒子情報を持つ配列。

  n\_send\_ep: 入力。const PS::S32型。デバイスに送るepjの数。

  send\_ptcl: 入力。const bool型。epjとspjの配列をデバイスに送るかを決
  める。

\item {\bf 返値}

  PS::S32型。ユーザーは正常に実行された場合は0を、エラーが起こった場合
  は0以外の値を返すようにする。
  
\item {\bf 機能}

send\_ptclがtrueの場合、epjをデバイスメモリー上に送る。毎回epiの配列と
epjのインデックスの相互作用リストをアクセラレータに送り、相互作用カー
ネルを発行する。


  
\end{itemize}
%%%%%%%%%%%%%%%%%%%%%%%%%%%%%%%%%%%%%


\subsubsection{長距離力の場合}
\subsubsubsection{粒子の相互作用リストを作る場合}
%%%%%%%%%%%%%%%%%%%%%%%%%%%%%%%%%%%%%
\begin{lstlisting}[caption=calcForceDispatch]
class EPI;
class EPJ;
class SPJ;
PS::S32 GravityDispatch(const PS::S32   tag,
                        const PS::S32   nwalk,
                        const EPI**     epi,
                        const PS::S32*  ni,
                        const EPJ**     epj,
                        const PS::S32*  nj_ep,
                        const SPJ**     spj,
                        const PS::S32*  nj_sp);
};
\end{lstlisting}

\begin{itemize}

\item {\bf 引数}

  tag: 入力。const PS::S32 型。tagの番号。発行されるtagの番号は0から関
  数PS::TreeForForce::calcForceAllandWriteBackMultiWalk()の第三引数と
  して設定された値から1引いた数までである。tagの番号は
  calcForceRetrieve()で設定するtagの番号と対応させる必要がある。

  nwalk: 入力。const PS::S32 型。walkの数。walkの数の最大値は
  PS::TreeForForce::calcForceAllandWriteBackMultiWalk()の第六引数の値
  である。

  epi: 入力。const EPI** 型。i粒子情報を持つ配列の配列。

  ni: 入力。const PS::S32*型。i粒子数の配列。

  epj: 入力。const EPJ** 型。j粒子情報を持つ配列の配列。
  
  nj\_ep: 入力。const PS::S32* 型。j粒子数の配列。

  spj: 入力。const SPJ** 型。j粒子情報を持つ配列の配列。
  
  nj\_sp: 入力。const PS::S32* 型。j粒子数の配列。

\item {\bf 返値}

  PS::S32型。ユーザーは正常に実行された場合は0を、エラーが起こった場合
  は0以外の値を返すようにする。
  
\item {\bf 機能}

epi,epj,spjの情報をアクセラレータに送り、相互作用カーネルを発行する。
  
\end{itemize}

\subsubsubsection{粒子のインデックスのみの相互作用リストを作る場合}
%%%%%%%%%%%%%%%%%%%%%%%%%%%%%%%%%%%%%
\begin{lstlisting}[caption=calcForceDispatch]
PS::S32 GravityDispatch(const PS::S32   tag,
                        const PS::S32   nwalk,
                        const TEpi**    iptcl,
                        const PS::S32*  ni,
                        const PS::S32** id_jptcl_ep,
                        const PS::S32*  nj_ep,
                        const PS::S32** id_jptcl_sp,
                        const PS::S32*  nj_sp,
                        const TEpj*     jptcl_ep,
                        const PS::S32   n_send_ep,
                        const TSpj*     jptcl_sp,
                        const PS::S32   n_send_sp,
                        const bool send_ptcl);
\end{lstlisting}

\begin{itemize}

\item {\bf 引数}
  tag: 入力。const PS::S32 型。tagの番号。発行されるtagの番号は0から関
  数PS::TreeForForce::calcForceAllandWriteBackMultiWalk()の第三引数と
  して設定された値から1引いた数までである。tagの番号は
  calcForceRetrieve()で設定するtagの番号と対応させる必要がある。

  nwalk: 入力。const PS::S32 型。walkの数。walkの数の最大値は
  PS::TreeForForce::calcForceAllandWriteBackMultiWalk()の第六引数の値
  である。

  epi: 入力。const EPI** 型。i粒子情報を持つ配列の配列。

  ni: 入力。const PS::S32*型。i粒子数の配列。

  id\_epj: 入力。const EPJ** 型。EPJのインデックスの配列の配列。
  
  nj\_ep: 入力。const PS::S32* 型。j粒子数の配列。

  id\_spj: 入力。const PS::S32** 型。SPJのインデックスの配列の配列。
  
  nj\_sp: 入力。const PS::S32* 型。j粒子数の配列。

  epj: 入力。const EPJ* 型。j粒子情報を持つ配列。

  n\_send\_ep: 入力。const PS::S32型。デバイスに送るepjの数。

  spj: 入力。const SPJ* 型。j粒子情報を持つ配列の配列。

  n\_send\_sp: 入力。const PS::S32型。デバイスに送るepjの数。

  send\_ptcl: 入力。const bool型。epjとspjの配列をデバイスに送るかを決
  める。

\item {\bf 返値}

  PS::S32型。ユーザーは正常に実行された場合は0を、エラーが起こった場合
  は0以外の値を返すようにする。
  
\item {\bf 機能}

send\_ptclがtrueの場合、epj,spjをデバイスメモリー上に送る。毎回epiの配
列とepj、spjのインデックスの相互作用リストをアクセラレータに送り、相互
作用カーネルを発行する。

\item {\bf 例}

\begin{lstlisting}[caption=calcForceDispatchExample]
PS::S32 DispatchKernelWithSPIndex(const PS::S32   tag,
                                  const PS::S32   nwalk,
                                  const TEpi**    iptcl,
                                  const PS::S32*  ni,
                                  const PS::S32** id_jptcl_ep,
                                  const PS::S32*  nj_ep,
                                  const PS::S32** id_jptcl_sp,
                                  const PS::S32*  nj_sp,
                                  const TEpj*     jptcl_ep,
                                  const PS::S32   n_send_ep,
                                  const TSpj*     jptcl_sp,
                                  const PS::S32   n_send_sp,
                                  const bool send_ptcl){
    int nj_tot = 0;
    cl_int ret;
    if(n_walk <= -1){
        for(int j=0; j<n_epj[0]; j++){
            epj_h[j].px   = jptcl_ep[j].pos.x;
            epj_h[j].py   = jptcl_ep[j].pos.y;
            epj_h[j].pz   = jptcl_ep[j].pos.z;
            epj_h[j].mass = jptcl_ep[j].mass;
        }
        for(int j=0; j<n_spj[0]; j++){
            spj_h[j].px   = jptcl_sp[j].pos.x;
            spj_h[j].py   = jptcl_sp[j].pos.y;
            spj_h[j].pz   = jptcl_sp[j].pos.z;
            spj_h[j].mass = jptcl_sp[j].getCharge();
        }
        ret = clEnqueueWriteBuffer(command_queue, epj_d,   CL_TRUE, 0, (n_send_ep)*sizeof(EpjDev),  epj_h, 0, NULL, NULL);
        ret = clEnqueueWriteBuffer(command_queue, spj_d,   CL_TRUE, 0, (n_send_sp)*sizeof(SpjDev),  spj_h, 0, NULL, NULL);
        return 0;
    }
    const float eps2 = FPGrav::eps * FPGrav::eps;
    PS::S32 ni_tot = 0;
    j_disp_h[0] = 0;
    for(int k=0; k<n_walk; k++){
        ni_tot += GetQuantizedValue(n_epi[k], NI_PIPE);
        assert(GetQuantizedValue(n_epi[k], NI_PIPE) \% NI_PIPE == 0);
        j_disp_h[k+1] = j_disp_h[k] + (n_epj[k] + n_spj[k]);
    }
    j_disp_h[n_walk+1] = j_disp_h[n_walk];
    assert(ni_tot < NI_LIMIT);
    assert(j_disp_h[n_walk] < NJ_LIMIT);
    ret = clEnqueueWriteBuffer(command_queue, j_disp_d, CL_TRUE, 0, (n_walk+2)*sizeof(int), j_disp_h, 0, NULL, NULL);
    ni_tot = 0;
    nepj_tot = 0;
    for(int iw=0; iw<n_walk; iw++){
        for(int i=0; i<n_epi[iw]; i++){
            epi_h[ni_tot].px = epi[iw][i].pos.x;
            epi_h[ni_tot].py = epi[iw][i].pos.y;
            epi_h[ni_tot].pz = epi[iw][i].pos.z;
            epi_h[ni_tot].id_walk = iw;
            ni_tot++;
        }
        for(int i=n_epi[iw]; i<GetQuantizedValue(n_epi[iw], NI_PIPE); i++){
            epi_h[ni_tot].px = 0.0;
            epi_h[ni_tot].py = 0.0;
            epi_h[ni_tot].pz = 0.0;
            epi_h[ni_tot].id_walk = iw;
            ni_tot++;
        }
        for(int j=0; j<n_epj[iw]; j++){
            id_epj_h[nepj_tot++] = id_epj[iw][j];
        }
        for(int j=0; j<n_spj[iw]; j++){
            id_spj_h[nspj_tot++] = id_spj[iw][j];
        }
    }
    ret = clEnqueueWriteBuffer(command_queue, epi_d,    CL_TRUE, 0, (ni_tot)*sizeof(EpiDev),    epi_h,    0, NULL, NULL);
    ret = clEnqueueWriteBuffer(command_queue, id_epj_d, CL_TRUE, 0, (nepj_tot)*sizeof(PS::S32), id_epj_h, 0, NULL, NULL);
    ret = clEnqueueWriteBuffer(command_queue, id_spj_d, CL_TRUE, 0, (nspj_tot)*sizeof(PS::S32), id_spj_h, 0, NULL, NULL);    
    ret = clSetKernelArg(kernel_index, 0, sizeof(cl_mem), (void*)&j_disp_d);
    ret = clSetKernelArg(kernel_index, 1, sizeof(cl_mem), (void*)&epi_d);
    ret = clSetKernelArg(kernel_index, 2, sizeof(cl_mem), (void*)&epj_d);
    ret = clSetKernelArg(kernel_index, 3, sizeof(cl_mem), (void*)&id_epj_d);
    ret = clSetKernelArg(kernel_index, 4, sizeof(cl_mem), (void*)&force_d);
    ret = clSetKernelArg(kernel_index, 5, sizeof(float),  (void*)&eps2);
    ret = clSetKernelArg(kernel_index, 6, sizeof(int),    (void*)&ni_tot);
    size_t work_size = N_THREAD_MAX;
    ret = clEnqueueNDRangeKernel(command_queue, kernel_index, 1, NULL, &work_size, NULL, 0, NULL, &event_force);
    return 0;
}
\end{lstlisting}

\end{itemize}
