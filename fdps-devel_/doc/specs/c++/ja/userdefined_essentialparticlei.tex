\subsubsection{概要}

EssentialParticleIクラスは相互作用の計算に必要なi粒子の情報を持つクラ
スであり、相互作用の定義(節\ref{sec:overview_action}の手順0)に必要とな
る。EssentialParticleIクラスはFullParticleクラス
(節\ref{sec:fullparticle})のサブセットである。FDPSは、このクラスのデー
タにアクセスする必要がある。そのため、EssentialParticleIクラスはいくつ
かのメンバ関数を持つ必要がある。以下、この節の前提、常に必要なメンバ関
数と、場合によっては必要なメンバ関数について記述する。

\subsubsection{前提}

この節の中では、EssentialParticleIクラスとしてEPIというクラスを一例と
して使う。また、FullParticleクラスの一例としてFPというクラスを使う。
EPI, FPというクラス名は変更可能である。

EPIとFPの宣言は以下の通りである。
\begin{screen}
\begin{verbatim}
class FP;
class EPI;
\end{verbatim}
\end{screen}

\subsubsection{必要なメンバ関数}

\subsubsubsection{概要}

常に必要なメンバ関数はEPI::getPosとEPI::copyfromFPである。EPI::getPos
はEPIクラスの位置情報をFDPSに読み込ませるための関数で、EPI::copyFromFP
はFPクラスの情報をEPIクラスに書きこむ関数である。これらのメンバ関数の
記述例と解説を以下に示す。

\subsubsubsection{EPI::getPos}

%%%%%%%%%%%%%%%%%%%%%%%%%%%%%%%%%%%%%%%%%%%%%%%%%%%%%%
%%%%%%%%%%%%%%%%%%%%%%%%%%%%%%%%%%%%%%%%%%%%%%%%%%%%%%
\if 0
\begin{screen}
\begin{verbatim}
class EPI {
public:
    PS::F64vec pos;
    PS::F64vec getPos() const {
        return this->pos;
    }
};
\end{verbatim}
\end{screen}

\begin{itemize}

\item {\bf 前提}
  
  EPIのメンバ変数posはある1つの粒子の位置情報。このposのデー
  タ型はPS::F64vec型。
  
\item {\bf 引数}

  なし
  
\item {\bf 返値}

  PS::F64vec型。EPIクラスの位置情報を保持したメンバ変数。
  
\item {\bf 機能}

  EPIクラスのオブジェクトの位置情報を保持したメンバ変数を返す。
  
\item {\bf 備考}

  EPIクラスのメンバ変数posの変数名は変更可能。

\end{itemize}
\fi
%%%%%%%%%%%%%%%%%%%%%%%%%%%%%%%%%%%%%%%%%%%%%%%%%%%%%%
%%%%%%%%%%%%%%%%%%%%%%%%%%%%%%%%%%%%%%%%%%%%%%%%%%%%%%

\begin{screen}
\begin{verbatim}
class EPI {
public:
    PS::F64vec getPos() const;
};
\end{verbatim}
\end{screen}

\begin{itemize}

\item {\bf 引数}

  なし
  
\item {\bf 返値}

  PS::F64vec型。EPIクラスの位置情報を保持したメンバ変数。
  
\item {\bf 機能}

  EPIクラスのオブジェクトの位置情報を保持したメンバ変数を返す。
  
\end{itemize}


\subsubsubsection{EPI::copyFromFP}

%%%%%%%%%%%%%%%%%%%%%%%%%%%%%%%%%%%%%%%%%%%%%%%%%%%%%%
%%%%%%%%%%%%%%%%%%%%%%%%%%%%%%%%%%%%%%%%%%%%%%%%%%%%%%
\if 0
\begin{screen}
\begin{verbatim}
class FP {
public:
    PS::S64    identity;
    PS::F64    mass;
    PS::F64vec position;
    PS::F64vec velocity;
    PS::F64vec acceleration;
    PS::F64    potential;
};
class EPI {
public:
    PS::S64    id;
    PS::F64vec pos;
    void copyFromFP(const FP & fp) {
        this->id  = fp.identity;
        this->pos = fp.position;
    }
};
\end{verbatim}
\end{screen}

\begin{itemize}

\item {\bf 前提}

  FPクラスのメンバ変数identity, positionとEPI
  クラスのメンバ変数id, posはそれぞれ対応する情報を持つ。

\item {\bf 引数}

  fp: 入力。const FP \&型。FPクラスの情報を持つ。
  
\item {\bf 返値}

  なし。
  
\item {\bf 機能}

  FPクラスの持つ1粒子の情報の一部をEssnetialParticleIクラス
  に書き込む。
  
\item {\bf 備考}

  FPクラスのメンバ変数の変数名、EPIクラスのメ
  ンバ変数の変数名は変更可能。メンバ関数EPI::copyFromFP
  の引数名は変更可能。EPIクラスの粒子情報はFP
  クラスの粒子情報のサブセット。対応する情報を持つメンバ変数同士のデー
  タ型が一致している必要はないが、実数型とベクトル型(または整数型とベ
  クトル型)という違いがある場合に正しく動作する保証はない。

\end{itemize}
\fi
%%%%%%%%%%%%%%%%%%%%%%%%%%%%%%%%%%%%%%%%%%%%%%%%%%%%%%
%%%%%%%%%%%%%%%%%%%%%%%%%%%%%%%%%%%%%%%%%%%%%%%%%%%%%%

\begin{screen}
\begin{verbatim}
class FP;
class EPI {
public:
    void copyFromFP(const FP & fp);
};
\end{verbatim}
\end{screen}

\begin{itemize}

\item {\bf 引数}

  fp: 入力。const FP \&型。FPクラスの情報を持つ。
  
\item {\bf 返値}

  なし。
  
\item {\bf 機能}

  FPクラスの持つ1粒子の情報の一部をEssnetialParticleIクラス
  に書き込む。
  
\end{itemize}

\subsubsection{場合によっては必要なメンバ関数}

\subsubsubsection{概要}

本節では、場合によっては必要なメンバ関数について記述する。相互作用ツリー
クラスのPS::SEARCH\_MODE型にPS::SEARCH\_MODE\_GATHERまたは
PS::SEARCH\_MODE\_SYMMETRYを用いる場合に必要となるメンバ関数ついて記述
する。

\subsubsubsection{相互作用ツリークラスのPS::SEARCH\_MODE型に\\PS::SEARCH\_MODE\_GATHERまたはPS::SEARCH\_MODE\_SYMMETRYを用いる場合}

\subsubsubsubsection{EPI::getRSearch}

%%%%%%%%%%%%%%%%%%%%%%%%%%%%%%%%%%%%%%%%%%%%%%%%%%%%%%
%%%%%%%%%%%%%%%%%%%%%%%%%%%%%%%%%%%%%%%%%%%%%%%%%%%%%%
\if 0
\begin{screen}
\begin{verbatim}
class EPI {
public:
    PS::F64 search_radius;
    PS::F64 getRSearch() const {
        return this->search_radius;
    }
};
\end{verbatim}
\end{screen}

\begin{itemize}

\item {\bf 前提}

  EPIクラスのメンバ変数search\_radiusはある1つの粒子の
  近傍粒子を探す半径の大きさ。このsearch\_radiusのデータ型はPS::F32型
  またはPS::F64型。
  
\item {\bf 引数}

  なし
  
\item {\bf 返値}

  PS::F32型またはPS::F64型。 EPIクラスの近傍粒子を探す
  半径の大きさを保持したメンバ変数。
  
\item {\bf 機能}

  EPIクラスの近傍粒子を探す半径の大きさを保持したメンバ
  変数を返す。

\item {\bf 備考}

  EPIクラスのメンバ変数search\_radiusの変数名は変更可能。
  
\end{itemize}
\fi
%%%%%%%%%%%%%%%%%%%%%%%%%%%%%%%%%%%%%%%%%%%%%%%%%%%%%%
%%%%%%%%%%%%%%%%%%%%%%%%%%%%%%%%%%%%%%%%%%%%%%%%%%%%%%

\begin{screen}
\begin{verbatim}
class EPI {
public:
    PS::F64 getRSearch() const;
};
\end{verbatim}
\end{screen}

\begin{itemize}

\item {\bf 引数}

  なし
  
\item {\bf 返値}

  PS::F32型またはPS::F64型。 EPIクラスの近傍粒子を探す
  半径の大きさを保持したメンバ変数。
  
\item {\bf 機能}

  EPIクラスの近傍粒子を探す半径の大きさを保持したメンバ
  変数を返す。

\end{itemize}
