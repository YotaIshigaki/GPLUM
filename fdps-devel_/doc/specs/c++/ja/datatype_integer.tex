\subsubsection{概要}

整数型にはPS::S32, PS::S64, PS::U32, PS::U64がある。以下、順にこれらを
記述する。

\subsubsection{PS::S32}

PS::S32は以下のように定義されている。すなわち32bitの符号付き整数である。
\begin{lstlisting}[caption=S32]
namespace ParticleSimulator {
    typedef int S32;
}
\end{lstlisting}

%ただし、GCCコンパイラとKコンパイラでのみ32bitであることが保証されている。

\subsubsection{PS::S64}

PS::S64は以下のように定義されている。すなわち64bitの符号付き整数である。
\begin{lstlisting}[caption=S64]
namespace ParticleSimulator {
    typedef long long int S64;
}
\end{lstlisting}

%ただし、GCCコンパイラとKコンパイラでのみ64bitであることが保証されている。

\subsubsection{PS::U32}

PS::U32は以下のように定義されている。すなわち32bitの符号なし整数である。
\begin{lstlisting}[caption=U32]
namespace ParticleSimulator {
    typedef unsigned int U32;
}
\end{lstlisting}

%ただし、GCCコンパイラとKコンパイラでのみ32bitであることが保証されている。

\subsubsection{PS::U64}

PS::U64は以下のように定義されている。すなわち64bitの符号なし整数である。
\begin{lstlisting}[caption=U64]
namespace ParticleSimulator {
    typedef unsigned long long int U64;
}
\end{lstlisting}

%ただし、GCCコンパイラとKコンパイラでのみ64bitであることが保証されている。

\subsubsection{PS::Count\_t}

カウント数を表す為の型。
現在はPS::U64型として定義されている。
\begin{lstlisting}[caption=Count\_t]
namespace ParticleSimulator {
    typedef U64 Count_t;
}
\end{lstlisting}
