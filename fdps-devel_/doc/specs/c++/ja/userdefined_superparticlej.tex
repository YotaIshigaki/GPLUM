\subsubsection{概要}

SuperParticleJクラスは近い粒子同士でまとまった複数の粒子を代表してまと
めた超粒子の情報を持つクラスであり、相互作用の定義
(節\ref{sec:overview_action}の手順0)に必要となる。このクラスが必要とな
るのはPS::SEARCH\_MODEに以下のいずれかを選択した場合だけである。
\begin{itemize}%[itemsep=-1ex]
\item PS::SEARCH\_MODE\_LONG
\item PS::SEARCH\_MODE\_LONG\_SCATTER
\item PS::SEARCH\_MODE\_LONG\_SYMMETRY
\item PS::SEARCH\_MODE\_LONG\_CUTOFF
\end{itemize}
このクラスのメンバ関数には、超粒子の位置情報をFDPS側とやりとりする
メンバ関数がある。また、超粒子の情報とMomentクラスの情報は対になる
ものである。従って、このクラスのメンバ関数には、Momentクラスから
このクラスへ情報を変換(またはその逆変換)するメンバ関数がある。

SuperParticleJクラスもMomentクラス同様、ある程度決っているものが多いの
で、それらについてはFDPS側で用意した。以下、既存のクラス、
SuperParticleJクラスを作るときに必要なメンバ関数、場合によっては必要な
メンバ関数について記述する。

\subsubsection{既存のクラス}

FDPSはいくつかのSuperParticleJクラスを用意している。以下、各
PS::SEARCH\_MODEに対し選ぶことのできるクラスについて記述する。
PS::SEARCH\_MODE\_LONG、PS::SEARCH\_MODE\_LONG\_SCATTER、
PS::SEARCH\_MODE\_LONG\_SYMMETRY、PS::SEARCH\_MODE\_LONG\_CUTOFF
の場合をこの順序で記述する。その他の\\PS::SEARCH\_MODEでは超粒子を必要としない。

\subsubsubsection{PS::SEARCH\_MODE\_LONG}

\subsubsubsubsection{PS::SPJMonopole}
\label{sec:SPJMonopole}

単極子までの情報を持つMomentクラスPS::MomentMonopoleと対になる
SuperParticleJクラス。以下、このクラスの概要を記述する。
\begin{screen}
\begin{verbatim}
namespace ParticleSimulator {
    class SPJMonopole {
    public:
        F64    mass;
        F64vec pos;
    };
}
\end{verbatim}
\end{screen}

\begin{itemize}
\item クラス名
  PS::SPJMonopole

\item メンバ変数とその情報

  mass: 近傍でまとめた粒子の全質量、または全電荷

  pos: 近傍でまとめた粒子の重心、または粒子電荷の重心

\item 使用条件

  MomentクラスであるPS::MomentMonopoleクラスの使用条件に準ずる。

\end{itemize}

\subsubsubsubsection{PS::SPJQuadrupole}
\label{sec:SPJQuadrupole}

単極子と四重極子を情報を持つMomentクラスPS::MomentQuadrupoleと対になる
SuperParticleJクラス。以下、このクラスの概要を記述する。
\begin{screen}
\begin{verbatim}
namespace ParticleSimulator {
    class SPJQuadrupole {
    public:
        F64    mass;
        F64vec pos;
        F64mat quad;
    };
}
\end{verbatim}
\end{screen}

\begin{itemize}
\item クラス名
  PS::SPJQuadrupole

\item メンバ変数とその情報

  mass: 近傍でまとめた粒子の全質量、または全電荷

  pos: 近傍でまとめた粒子の重心、または粒子電荷の重心

  quad: 近傍でまとめた粒子の四重極子

\item 使用条件

  MomentクラスであるPS::MomentQuadrupoleクラスの使用条件に準ずる。

\end{itemize}

\subsubsubsubsection{PS::SPJMonopoleGeometricCenter}
\label{sec:SPJMonopoleGeometricCenter}

単極子までを情報として持つ(ただしモーメント計算の際の座標系の中心は粒
子の幾何中心)MomentクラスPS::MomentMonopoleGeometricCenterと対となる
SuperParticleJクラス。以下、このクラスの概要を記述する。
\begin{screen}
\begin{verbatim}
namespace ParticleSimulator {
    class SPJMonopoleGeometricCenter {
    public:
        F64    charge;    
        F64vec pos;
    };
}
\end{verbatim}
\end{screen}

\begin{itemize}
\item クラス名
  PS::SPJMonopoleGeometricCenter

\item メンバ変数とその情報

  charge: 近傍でまとめた粒子の全質量、または全電荷

  pos: 近傍でまとめた粒子の幾何中心

\item 使用条件

  PS::MomentMonopoleGeometricCenterの使用条件に準ずる。

\end{itemize}

\subsubsubsubsection{PS::SPJDipoleGeometricCenter}
\label{sec:SPJDipoleGeometricCenter}

双極子までを情報として持つ(ただしモーメント計算の際の座標系の中心は粒
子の幾何中心)MomentクラスPS::MomentDipoleGeometricCenterと対となる
SuperParticleJクラス。以下、このクラスの概要を記述する。
\begin{screen}
\begin{verbatim}
namespace ParticleSimulator {
    class SPJDipoleGeometricCenter {
    public:
        F64    charge;    
        F64vec pos;
        F64vec dipole;
    };
}
\end{verbatim}
\end{screen}

\begin{itemize}
\item クラス名
  PS::SPJDipoleGeometricCenter

\item メンバ変数とその情報

  charge: 近傍でまとめた粒子の全質量、または全電荷

  pos: 近傍でまとめた粒子の幾何中心

  dipole: 粒子の質量または電荷の双極子

\item 使用条件

  PS::MomentDipoleGeometricCenterの使用条件に準ずる。

\end{itemize}

\subsubsubsubsection{PS::SPJQuadrupoleGeometricCenter}
\label{sec:SPJQuadrupoleGeometricCenter}

四重極子までを情報として持つ(ただしモーメント計算の際の座標系の中心は
粒子の幾何中心)MomentクラスPS::MomentQuadrupoleGeometricCenterと対とな
るSuperParticleJクラス。以下、このクラスの概要を記述する。
\begin{screen}
\begin{verbatim}
namespace ParticleSimulator {
    class SPJQuadrupoleGeometricCenter {
    public:
        F64    charge;    
        F64vec pos;
        F64vec dipole;
        F64mat quadrupole;
    };
}
\end{verbatim}
\end{screen}

\begin{itemize}
\item クラス名
  PS::SPJQuadrupoleGeometricCenter

\item メンバ変数とその情報

  charge: 近傍でまとめた粒子の全質量、または全電荷

  pos: 近傍でまとめた粒子の幾何中心

  dipole: 粒子の質量または電荷の双極子

  quadrupole: 粒子の質量または電荷の四重極子

\item 使用条件

  PS::MomentQuadrupoleGeometricCenterの使用条件に準ずる。

\end{itemize}
  
\subsubsubsection{PS::SEARCH\_MODE\_LONG\_SCATTER}

\subsubsubsubsection{PS::SPJMonopoleScatter}
\label{sec:SPJMonopoleScatter}

単極子までを情報として持つMomentクラスPS::MomentMonopoleScatterと
対となるSuperParticleJクラス。以下、このクラスの概要を記述する。
\begin{screen}
\begin{verbatim}
namespace ParticleSimulator {
    class SPJMonopoleScatter {
    public:
        F64    mass;
        F64vec pos;
    };
}
\end{verbatim}
\end{screen}

\begin{itemize}
\item クラス名
  PS::SPJMonopoleScatter

\item メンバ変数とその情報

  mass: 近傍でまとめた粒子の全質量、または全電荷

  pos: 近傍でまとめた粒子の重心、または粒子電荷の重心

\item 使用条件

  PS::MomentMonopoleScatterの使用条件に準ずる。

\end{itemize}

\subsubsubsubsection{PS::SPJQuadrupoleScatter}
\label{sec:SPJQuadrupoleScatter}

単極子と四重極子を情報として持つMomentクラスPS::MomentQuadrupoleScatterと
対となるSuperParticleJクラス。以下、このクラスの概要を記述する。
\begin{screen}
\begin{verbatim}
namespace ParticleSimulator {
    class SPJQuadrupolepoleScatter {
    public:
        F64    mass;
        F64vec pos;
        F64mat quad;
    };
}
\end{verbatim}
\end{screen}

\begin{itemize}
\item クラス名
  PS::SPJQuadrupoleScatter

\item メンバ変数とその情報

  mass: 近傍でまとめた粒子の全質量、または全電荷

  pos: 近傍でまとめた粒子の重心、または粒子電荷の重心
        
  quad: 近傍でまとめた粒子の四重極子

\item 使用条件

  PS::MomentQuadrupoleScatterの使用条件に準ずる。

\end{itemize}

\subsubsubsection{PS::SEARCH\_MODE\_LONG\_SYMMETRY}

\subsubsubsubsection{PS::SPJMonopoleSymmetry}
\label{sec:SPJMonopoleSymmetry}

単極子までを情報として持つMomentクラスPS::MomentMonopoleSymmetryと
対となるSuperParticleJクラス。以下、このクラスの概要を記述する。
\begin{screen}
\begin{verbatim}
namespace ParticleSimulator {
    class SPJMonopoleSymmetry {
    public:
        F64    mass;
        F64vec pos;
    };
}
\end{verbatim}
\end{screen}

\begin{itemize}
\item クラス名
  PS::SPJMonopoleSymmetry

\item メンバ変数とその情報

  mass: 近傍でまとめた粒子の全質量、または全電荷

  pos: 近傍でまとめた粒子の重心、または粒子電荷の重心

\item 使用条件

  PS::MomentMonopoleSymmetryの使用条件に準ずる。

\end{itemize}

\subsubsubsubsection{PS::SPJQuadrupoleSymmetry}
\label{sec:SPJQuadrupoleSymmetry}

単極子と四重極子を情報として持つMomentクラスPS::MomentQuadrupoleSymmetryと
対となるSuperParticleJクラス。以下、このクラスの概要を記述する。
\begin{screen}
\begin{verbatim}
namespace ParticleSimulator {
    class SPJQuadrupolepoleSymmetry {
    public:
        F64    mass;
        F64vec pos;
        F64mat quad;
    };
}
\end{verbatim}
\end{screen}

\begin{itemize}
\item クラス名
  PS::SPJQuadrupoleSymmetry

\item メンバ変数とその情報

  mass: 近傍でまとめた粒子の全質量、または全電荷

  pos: 近傍でまとめた粒子の重心、または粒子電荷の重心
        
  quad: 近傍でまとめた粒子の四重極子

\item 使用条件

  PS::MomentQuadrupoleSymmetryの使用条件に準ずる。

\end{itemize}

\subsubsubsection{PS::SEARCH\_MODE\_LONG\_CUTOFF}

\subsubsubsubsection{PS::SPJMonopoleCutoff}
\label{sec:SPJMonopoleCutoff}

単極子までを情報として持つMomentクラスPS::MomentMonopoleCutoffと
対となるSuperParticleJクラス。以下、このクラスの概要を記述する。
\begin{screen}
\begin{verbatim}
namespace ParticleSimulator {
    class SPJMonopoleCutoff {
    public:
        F64    mass;
        F64vec pos;
    };
}
\end{verbatim}
\end{screen}

\begin{itemize}
\item クラス名
  PS::SPJMonopoleCutoff

\item メンバ変数とその情報

  mass: 近傍でまとめた粒子の全質量、または全電荷

  pos: 近傍でまとめた粒子の重心、または粒子電荷の重心

\item 使用条件

  PS::MomentMonopoleCutoffの使用条件に準ずる。

\end{itemize}

\subsubsection{必要なメンバ関数}

\subsubsubsection{概要}

以下ではSuperParticleJクラスを作る際に必要なメンバ関数を記述する。この
ときSuperParticleJクラスのクラス名をSPJとする。これは変更自由である。

\subsubsubsection{SPJ::getPos}

\if 0
\begin{screen}
\begin{verbatim}
class SPJ {
public:
    PS::F64vec pos;
    PS::F64vec getPos() const {
        return this->pos;
    }
};
\end{verbatim}
\end{screen}

\begin{itemize}

\item {\bf 前提}
  
  SPJのメンバ変数posはある1つの超粒子の位置情報。このposのデータ型は
  PS::F32vecまたはPS::F64vec型。
  
\item {\bf 引数}

  なし
  
\item {\bf 返値}

  PS::F32vec型またはPS::F64vec型。SPJクラスの位置情報を保持したメンバ
  変数。
  
\item {\bf 機能}

  SPJクラスの位置情報を保持したメンバ変数を返す。
  
\item {\bf 備考}

  SPJクラスのメンバ変数posの変数名は変更可能。

\end{itemize}
\fi

\begin{screen}
\begin{verbatim}
class SPJ {
public:
    PS::F64vec getPos() const;
};
\end{verbatim}
\end{screen}

\begin{itemize}

\item {\bf 引数}

  なし
  
\item {\bf 返値}

  PS::F32vec型またはPS::F64vec型。SPJクラスの位置情報を保持したメンバ
  変数。
  
\item {\bf 機能}

  SPJクラスの位置情報を保持したメンバ変数を返す。
  
\end{itemize}

\subsubsubsection{SPJ::setPos}

\if 0
\begin{screen}
\begin{verbatim}
class SPJ {
public:
    PS::F64vec pos;
    void setPos(const PS::F64vec pos_new) {
        this->pos = pos_new;
    }
};
\end{verbatim}
\end{screen}

\begin{itemize}

\item {\bf 前提}
  
  SPJクラスのメンバ変数posは1つの粒子の位置情報。このposのデータ型は
  PS::F32vecまたはPS::F64vec。

\item {\bf 引数}

  pos\_new: 入力。const PS::F32vecまたはconst PS::F64vec型。FDPS側で修
  正した粒子の位置情報。

\item {\bf 返値}

  なし。
  
\item {\bf 機能}

  FDPSが修正した粒子の位置情報をSPJクラスの位置情報に書き込む。

\item {\bf 備考}

  SPJクラスのメンバ変数posの変数名は変更可能。メンバ関数SPJ::setPosの
  引数名pos\_newは変更可能。posとpos\_newのデータ型が異なる場合の動作
  は保証しない。

\end{itemize}
\fi

\begin{screen}
\begin{verbatim}
class SPJ {
public:
    void setPos(const PS::F64vec pos_new);
};
\end{verbatim}
\end{screen}

\begin{itemize}

\item {\bf 引数}

  pos\_new: 入力。const PS::F32vecまたはconst PS::F64vec型。FDPS側で修
  正した粒子の位置情報。

\item {\bf 返値}

  なし。
  
\item {\bf 機能}

  FDPSが修正した粒子の位置情報をSPJクラスの位置情報に書き込む。

\end{itemize}


\subsubsubsection{SPJ::copyFromMoment}

\if 0
\begin{screen}
\begin{verbatim}
class Mom {
public:
    PS::F32    mass;
    PS::F32vec pos;
}
class SPJ {
public:
    PS::F32    mass;
    PS::F32vec pos;
    void copyFromMoment(const Mom & mom) {
        mass = mom.mass;
        pos  = mom.pos;
    }
};
\end{verbatim}
\end{screen}

\begin{itemize}

\item {\bf 前提}

  なし
  
\item {\bf 引数}

  mom: 入力。const Mom \&型。Momにはユーザー定義またはFDPS側で用意した
  Momentクラスが入る。

\item {\bf 返値}

  なし。
  
\item {\bf 機能}

  Momクラスの情報をSPJクラスにコピーする。

\item {\bf 備考}

  Momクラスのクラス名は変更可能。MomクラスとSPJクラスのメンバ変数名は
  変更可能。メンバ関数SPJ::copyFromMomentの引数名は変更可能。

\end{itemize}
\fi

\begin{screen}
\begin{verbatim}
class Mom;
class SPJ {
public:
    void copyFromMoment(const Mom & mom);
};
\end{verbatim}
\end{screen}

\begin{itemize}

\item {\bf 引数}

  mom: 入力。const Mom \&型。Momにはユーザー定義またはFDPS側で用意した
  Momentクラスが入る。

\item {\bf 返値}

  なし。
  
\item {\bf 機能}

  Momクラスの情報をSPJクラスにコピーする。

\end{itemize}

\subsubsubsection{SPJ::convertToMoment}

\if 0
\begin{screen}
\begin{verbatim}
class Mom {
public:
    PS::F32    mass;
    PS::F32vec pos;
    Mom(const PS::F32 m,
        const PS::F32vec & p) {
        mass = m;
        pos  = p;
    }
}
class SPJ {
public:
    PS::F32    mass;
    PS::F32vec pos;
    Mom convertToMoment() const {
        return Mom(mass, pos);
    }
};
\end{verbatim}
\end{screen}

\begin{itemize}

\item {\bf 前提}

  なし
  
\item {\bf 引数}

  なし

\item {\bf 返値}

  Mom型。Momクラスのコンストラクタ。
  
\item {\bf 機能}

  Momクラスのコンストラクタを返す。

\item {\bf 備考}

  Momクラスのクラス名は変更可能。MomクラスとSPJクラスのメンバ変数名は
  変更可能。メンバ関数SPJ::copyFromMomentの引数名は変更可能。メンバ関
  数SPJ::convertToMomentで使用されるMomクラスのコンストラクタが定義さ
  れている必要がある。

\end{itemize}
\fi

\begin{screen}
\begin{verbatim}
class Mom;
class SPJ {
public:
    Mom convertToMoment() const;
};
\end{verbatim}
\end{screen}

\begin{itemize}

\item {\bf 引数}

  なし

\item {\bf 返値}

  Mom型。Momクラスのコンストラクタ。
  
\item {\bf 機能}
  
  超粒子をモーメントに変換し、その変換したものをMomクラスのコンストラク
  タを返す。

\end{itemize}

\subsubsubsection{SPJ::clear}

\if 0
\begin{screen}
\begin{verbatim}
class SPJ {
public:
    PS::F32    mass;
    PS::F32vec pos;
    void clear() {
        mass = 0.0;
        pos  = 0.0;
    }
};
\end{verbatim}
\end{screen}

\begin{itemize}

\item {\bf 前提}

  なし
  
\item {\bf 引数}

  なし

\item {\bf 返値}

  なし
  
\item {\bf 機能}

  SPJクラスのオブジェクトの情報をクリアする。

\item {\bf 備考}

  メンバ変数名は変更可能。

\end{itemize}
\fi

\begin{screen}
\begin{verbatim}
class SPJ {
public:
    void clear();
};
\end{verbatim}
\end{screen}

\begin{itemize}

\item {\bf 引数}

  なし

\item {\bf 返値}

  なし
  
\item {\bf 機能}

  SPJクラスのオブジェクトの情報をクリアする。

\end{itemize}


\subsubsection{場合によっては必要なメンバ関数}

\subsubsubsection{概要}

本節では、場合によっては必要なメンバ関数について記述する。

\subsubsubsection{LET交換時に粒子データをシリアライズして送る場合}
\label{sec:SPJ:serialize}

LET交換時に粒子データをシリアライズして送る場合には、メンバ関数に
SPJ::packとSPJ::unpackを用意する必要がある。以下にそれぞれの規定を記述
する。

%%%%%%%%%%%%%%%%%%%%%%%%%%%%%%%%%%%%%%%%%%%%%%%%%%%%%%

\subsubsubsubsection{SPJ::pack}

\begin{screen}
\begin{verbatim}
class SPJ {
public:
    static PS::S32 pack(const PS::S32 n_ptcl, const SPJ *ptcl[], char *buf, 
                        size_t & packed_size, const size_t max_buf_size);
};
\end{verbatim}
\end{screen}

\begin{itemize}

\item {\bf 引数}

  n\_ptcl: LET交換時に送る粒子の数。
  

  ptcl: 送る粒子へのポインタの配列。

  buf: 送信バッファーの先頭アドレス。
  
  packed\_size: ユーザーがバッファーへ書き込むサイズ。単位はバイト。
  
  max\_buf\_size: 送信バッファーの書き込み可能な領域のサイズ。単位はバイト。

\item {\bf 返値}

  PS::S32型。packed\_sizeがmax\_buf\_sizeを超えた場合は-1を返す。それ
  以外の場合は0を返す。
  
\item {\bf 機能}

 LET交換時に送信する粒子をシリアライズし、送信バッファーに書き込む。

\end{itemize}

%%%%%%%%%%%%%%%%%%%%%%%%%%%%%%%%%%%%%%%%%%%%%%%%%%%%%%

\subsubsubsubsection{SPJ::unPack}

\begin{screen}
\begin{verbatim}
class SPJ {
public:
    static void unPack(const PS::S32 n_ptcl, SPJ[], const char *buf);
};
\end{verbatim}
\end{screen}

\begin{itemize}

\item {\bf 引数}

  n\_ptcl: LET交換時に受け取る粒子の数。
  
  ptcl: 受け取る粒子の配列。
  
  buf: 受信バッファーの先頭アドレス。

\item {\bf 返値}

  なし。

\item {\bf 機能}

 LET交換時に受信する粒子をデシリアライズし、粒子配列に書き込む。

\end{itemize}
