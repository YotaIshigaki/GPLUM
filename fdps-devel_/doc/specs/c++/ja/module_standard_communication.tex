本節では、通信用データクラス{\tt CommInfo}クラスと{\tt Comm}クラスにつ
いて記述する。これらのクラスはノード間通信のための情報の保持や実際の通
信を行うモジュールである。{\tt CommInfo}クラスは1つのMPIのコミュニケー
タを持ち、そのメンバ関数はそのコミュニケータの元で動作する.{\tt
ParticleSystem}クラス、{\tt TreeForForce}クラス、{\tt DomainInfo}クラ
スはそれぞれ通信用データクラスをもっており,ユーザーは自分で作成した
{\tt CommInfo}クラスのオブジェクトをそれらのクラスのオブジェクトに与え
ることで、任意のコミュニケータ上で相互作用計算をさせることができる.

{\tt Comm}クラスはコミュニケータに{\tt MPI\_COMM\_WORLD}を指定した場合
のラッパークラスとなっている.{\tt Comm}クラスはシングルトンパターンで
管理されており、ユーザーはオブジェクトの生成は必要としない.

ここでは{\tt CommInfo}クラスと{\tt Comm}クラスのAPIを記述する。

%\subsubsubsection{CommInfoクラス}

%%%%%%%%%%%%%%%%%%%%%%%%%%%%%%%%%%%%%%%%%%%%%%%%%%%%%%%%%%%%%%%%%
%\subsubsubsubsection{オブジェクトの生成}
\subsubsubsection{オブジェクトの生成}
{\tt CommInfo}クラスは以下のように宣言されている。
\begin{lstlisting}[caption=CommInfo0]
namespace ParticleSimulator {
    class CommInfo;
}
\end{lstlisting}

{\tt CommInfo}クラスのオブジェクトの生成は以下のように行う。ここでは
{\tt comm\_info}というオブジェクトを生成している。
\begin{screen}
\begin{verbatim}
PS::CommInfo comm_info;
\end{verbatim}
\end{screen}


%%%%%%%%%%%%%%%%%%%%%%%%%%%%%%%%%%%%%%%%%%%%%%%%%%%%%%%%%%%%%%%%%
\subsubsubsection{API}

{\tt CommInfo}クラスのAPIの宣言は以下のようになっている。
\begin{lstlisting}[caption=CommunicationInformation]
namespace ParticleSimulator {
    class CommInfo{
    public:
        void setCommunicator(const MPI_Comm & comm);
        CommInfo create(const int n, const int rank[]);
        CommInfo split(int color, int key);
        S32 getRank();
        S32 getNumberOfProc();
        S32 getRankMultiDim(const S32 id);
        S32 getNumberOfProcMultiDim(const S32 id);
        bool synchronizeConditionalBranchAND
                    (const bool local);
        bool synchronizeConditionalBranchOR
                    (const bool local);
        template<class T>
        T getMinValue(const T val);
        template<class Tfloat, class Tint>
        void getMinValue(const Tfloat f_in,
                                const Tint i_in,
                                Tfloat & f_out,
                                Tint & i_out);
        template<class T>
        T getMaxValue(const T val);
        template<class Tfloat, class Tint>
        void getMaxValue(const Tfloat f_in,
                                const Tint i_in,
                                Tfloat & f_out,
                                Tint & i_out );
        template<class T>
        T getSum(const T val);
        template<class T>
        void broadcast(T * val,
                              const S32 n,
                              const S32 src=0);
        MPI_Comm getCommunicator();
    };
}
\end{lstlisting}


{\tt Comm}クラスのAPIの宣言は以下のようになっている。
\begin{lstlisting}[caption=Communication]
namespace ParticleSimulator {
    class Comm{
    public:
        static S32 getRank();
        static S32 getNumberOfProc();
        static S32 getRankMultiDim(const S32 id);
        static S32 getNumberOfProcMultiDim(const S32 id);
        static bool synchronizeConditionalBranchAND
                    (const bool local);
        static bool synchronizeConditionalBranchOR
                    (const bool local);
        template<class T>
        static T getMinValue(const T val);
        template<class Tfloat, class Tint>
        static void getMinValue(const Tfloat f_in,
                                const Tint i_in,
                                Tfloat & f_out,
                                Tint & i_out);
        template<class T>
        static T getMaxValue(const T val);
        template<class Tfloat, class Tint>
        static void getMaxValue(const Tfloat f_in,
                                const Tint i_in,
                                Tfloat & f_out,
                                Tint & i_out );
        template<class T>
        static T getSum(const T val);
        template<class T>
        static void broadcast(T * val,
                              const S32 n,
                              const S32 src=0);
    };
}
\end{lstlisting}

%%%%%%%%%%%%%%%%%%%%%%%%%%%%%%%%%%%%%%%%%%%%%%%%%%%%%%
\subsubsubsubsection{初期設定}

%%%%%%%%%%%%%%%%%%%%%%%%%%%%%%%%%%%%%%%%%%%
\subsubsubsubsubsection{コンストラクタ}

\begin{screen}
\begin{verbatim}
PS::CommInfo PS::CommInfo::CommInfo(const MPI_Comm & comm = MPI_COMM_NULL);
\end{verbatim}
\end{screen}

\begin{itemize}

\item {\bf 引数}

comm: 入力.{\tt const MPI\_Comm \&} 型.

\item {\bf 返値}

CommInfo型.生成されたコミュニケータ.

\item {\bf 機能}

{\tt CommInfo}クラスのオブジェクトを生成する。デフォルトのコミュニケー
タは{\tt MPI\_COMM\_NULL}である.

\end{itemize}

\subsubsubsubsection{PS::CommInfo::setCommunicator}
\begin{screen}
\begin{verbatim}
void PS::CommInfo::setCommunicator(const MPI_Comm & comm = MPI_COMM_WORLD);
\end{verbatim}
\end{screen}

\begin{itemize}

\item {\bf 引数}

comm: 入力.{\tt const MPI\_Comm \&} 型.

\item {\bf 返値}

なし

\item {\bf 機能}

{\tt CommInfo}クラスのオブジェクトにMPIコミュニケータを設定する.デフォ
ルトは{\tt MPI\_COMM\_WORLD}である.

\end{itemize}

\subsubsubsubsubsection{PS::CommInfo::create}
\begin{screen}
\begin{verbatim}
CommInfo create(const int n, const int rank[]) const;
\end{verbatim}
\end{screen}

\begin{itemize}

\item {\bf 引数}

n: 入力.int型.生成されるコミュニケータに所属するプロセスの数.

rank: 入力.int型の配列.生成されるコミュニケータに所属するプロセスのランクの配列.

\item {\bf 返値}

CommInfo型.生成されたコミュニケータ.

\item {\bf 機能}

呼び出しもとの{\tt CommInfo}クラスのもつMPIコミュニケータから新たなコ
ミュニケータを作成する.配列rank[]で表せられるプロセスが所属するコミュ
ニケータを作成し返す.

\end{itemize}

\subsubsubsubsubsection{PS::CommInfo::split}
\begin{screen}
\begin{verbatim}
CommInfo split(int color, int key) const;
\end{verbatim}
\end{screen}

\begin{itemize}

\item {\bf 引数}

color: 入力.int型.
key: 入力.int型.

\item {\bf 返値}

CommInfo型.スプリットされたコミュニケータを返す.

\item {\bf 機能}

呼び出しもとの{\tt CommInfo}クラスのもつMPIコミュニケータを分割する.
同じcolorのプロセスは同一コミュニケータに所属し、keyの小さいものから順
にそのコミュニケータでのランクが割り振られる.

\end{itemize}


%%%%%%%%%%%%%%%%%%%%%%%%%%%%%%%%%%%%%%%%%%%%%%%%%%%%%%
\subsubsubsubsection{情報取得}

\subsubsubsubsubsection{PS::CommInfo::getRank}

\begin{screen}
\begin{verbatim}
PS::S32 PS::CommInfo::getRank();
\end{verbatim}
\end{screen}

\begin{itemize}

\item{{\bf 引数}}

なし。

\item{{\bf 返り値}}

{PS::S32}型。コミュニケータ内でのランクを返す。

\end{itemize}

\subsubsubsubsubsection{PS::CommInfo::getNumberOfProc}

\begin{screen}
\begin{verbatim}
PS::S32 PS::CommInfo::getNumberOfProc();
\end{verbatim}
\end{screen}

\begin{itemize}

\item{{\bf 引数}}

なし。

\item{{\bf 返り値}}

PS::S32型。全プロセス数を返す。

\end{itemize}

\subsubsubsubsubsection{PS::CommInfo::getRankMultiDim}

\begin{screen}
\begin{verbatim}
PS::S32 PS::CommInfo::getRankMultiDim(const PS::S32 id);
\end{verbatim}
\end{screen}

\begin{itemize}

\item{{\bf 引数}}

id: 入力。const PS::S32型。軸の番号。x軸:0, y軸:1, z軸:2。

\item{{\bf 返り値}}

PS::S32型。id番目の軸でのランクを返す。2次元の場合、id=2は1を返す。

\end{itemize}

\subsubsubsubsubsection{PS::CommInfo::getNumberOfProcMultiDim}

\begin{screen}
\begin{verbatim}
PS::S32 PS::CommInfo::getNumberOfProcMultiDim(const PS::S32 id);
\end{verbatim}
\end{screen}

\begin{itemize}

\item{{\bf 引数}}

id: 入力。const PS::S32型。軸の番号。x軸:0, y軸:1, z軸:2。

\item{{\bf 返り値}}

PS::S32型。id番目の軸のプロセス数を返す。2次元の場合、id=2は1を返す。

\end{itemize}

\subsubsubsubsubsection{PS::CommInfo::synchronizeConditionalBranchAND}

\begin{screen}
\begin{verbatim}
bool PS::CommInfo::synchronizeConditionalBranchAND(const bool local)
\end{verbatim}
\end{screen}

\begin{itemize}

\item{{\bf 引数}}

local: 入力。const bool型。

\item{{\bf 返り値}}

bool型。全プロセスでlocalの論理積を取り、結果を返す。

\end{itemize}

\subsubsubsubsubsection{PS::CommInfo::synchronizeConditionalBranchOR}

\begin{screen}
\begin{verbatim}
bool PS::CommInfo::synchronizeConditionalBranchOR(const bool local);
\end{verbatim}
\end{screen}

\begin{itemize}

\item{{\bf 引数}}

local: 入力。const bool型。

\item{{\bf 返り値}}

bool型。全プロセスでlocalの論理和を取り、結果を返す。

\end{itemize}

\subsubsubsubsubsection{PS::CommInfo::getMinValue}

\begin{screen}
\begin{verbatim}
template <class T>
T PS::CommInfo::getMinValue(const T val);
\end{verbatim}
\end{screen}

\begin{itemize}

\item{{\bf 引数}}

val: 入力。const T型。

\item{{\bf 返り値}}

T型。全プロセスでvalの最小値を取り、結果を返す。

\end{itemize}

\begin{screen}
\begin{verbatim}
template <class Tfloat, class Tint>
void PS::CommInfo::getMinValue(const Tfloat f_in,
                               const Tint i_in,
                               Tfloat & f_out,
                               Tint & i_out);
\end{verbatim}
\end{screen}

\begin{itemize}

\item{{\bf 引数}}

f\_in: 入力。const Tfloat型。

i\_in: 入力。const Tint型。

f\_out: 出力。Tfloat型。全プロセスでf\_inの最小値を取
り、結果を返す。

i\_out: 出力。Tint型。f\_outに伴うIDを返す。

\item{{\bf 返り値}}

なし。

\end{itemize}

\subsubsubsubsubsection{PS::CommInfo::getMaxValue}

\begin{screen}
\begin{verbatim}
template <class T>
static T PS::CommInfo::getMaxValue(const T val);
\end{verbatim}
\end{screen}

\begin{itemize}

\item{{\bf 引数}}

val: 入力。const T型。

\item{{\bf 返り値}}

T型。全プロセスでvalの最大値を取り、結果を返す。

\end{itemize}

\begin{screen}
\begin{verbatim}
template <class Tfloat, class Tint>
void PS::CommInfo::getMaxValue(const Tfloat f_in,
                               const Tint i_in,
                               Tfloat & f_out,
                               Tint & i_out);
\end{verbatim}
\end{screen}

\begin{itemize}

\item{{\bf 引数}}

f\_in: 入力。const Tfloat型。

i\_in: 入力。{const Tint}型。

{f\_out}: 出力。{Tfloat}型。全プロセスで{f\_in}の最大値を取
り、結果を返す。

{i\_out}: 出力。{Tint}型。{f\_out}に伴うIDを返す。

\item{{\bf 返り値}}

なし。

\end{itemize}

\subsubsubsubsubsection{PS::CommInfo::getSum}

\begin{screen}
\begin{verbatim}
template <class T>
T PS::Comm::getSum(const T val);
\end{verbatim}
\end{screen}

\begin{itemize}

\item{{\bf 引数}}

{val}: 入力。{const T}型。

\item{{\bf 返り値}}

{T}型。全プロセスで{val}の総和を取り、結果を返す。

\end{itemize}

\subsubsubsubsubsection{PS::CommInfo::broadcast}

\begin{screen}
\begin{verbatim}
template <class T>
void PS::CommInfo::broadcast(T * val,
                             const PS::S32 n,
                             const PS::S32 src=0);
\end{verbatim}
\end{screen}

\begin{itemize}

\item{{\bf 引数}}

val: 入力。T *型。

n: 入力。const PS::S32型。T型変数の数。

src: 入力。const PS::S32型。放送するプロセスランク。デフォルトのランク
は0。

\item{{\bf 返り値}}

なし。

\item{{\bf 機能}}

プロセスランクsrcのプロセスがn個のT型変数を全プロセスに放送する。

\end{itemize}


%%%%%%%%%%%%%%%%%%%%%%%%%%%%%%%%%%%%%%%%%%%%%%%%%%%%%%%%%%%%%%%%%
%%%%%%%%%%%%%%%%%%%%%%%%%%%%%%%%%%%%%%%%%%%%%%%%%%%%%%%%%%%%%%%%%

%\subsubsubsection{Commクラス}

%%%%%%%%%%%%%%%%%%%%%%%%%%%%%%%%%%%%%%%%%%%%%%%%%%%%%%%%%%%%%%%%%
%\subsubsubsubsection{CommクラスのAPI}



\subsubsubsubsubsection{PS::Comm::getRank}

\begin{screen}
\begin{verbatim}
static PS::S32 PS::Comm::getRank();
\end{verbatim}
\end{screen}

\begin{itemize}

\item{{\bf 引数}}

なし。

\item{{\bf 返り値}}

{PS::S32}型。全プロセス中でのランクを返す。

\end{itemize}

\subsubsubsubsubsection{PS::Comm::getNumberOfProc}

\begin{screen}
\begin{verbatim}
static PS::S32 PS::Comm::getNumberOfProc();
\end{verbatim}
\end{screen}

\begin{itemize}

\item{{\bf 引数}}

なし。

\item{{\bf 返り値}}

PS::S32型。全プロセス数を返す。

\end{itemize}

\subsubsubsubsubsection{PS::Comm::getRankMultiDim}

\begin{screen}
\begin{verbatim}
static PS::S32 PS::Comm::getRankMultiDim(const PS::S32 id);
\end{verbatim}
\end{screen}

\begin{itemize}

\item{{\bf 引数}}

id: 入力。const PS::S32型。軸の番号。x軸:0, y軸:1, z軸:2。

\item{{\bf 返り値}}

PS::S32型。id番目の軸でのランクを返す。2次元の場合、id=2は1を返す。

\end{itemize}

\subsubsubsubsubsection{PS::Comm::getNumberOfProcMultiDim}

\begin{screen}
\begin{verbatim}
static PS::S32 PS::Comm::getNumberOfProcMultiDim(const PS::S32 id);
\end{verbatim}
\end{screen}

\begin{itemize}

\item{{\bf 引数}}

id: 入力。const PS::S32型。軸の番号。x軸:0, y軸:1, z軸:2。

\item{{\bf 返り値}}

PS::S32型。id番目の軸のプロセス数を返す。2次元の場合、id=2は1を返す。

\end{itemize}

\subsubsubsubsubsection{PS::Comm::synchronizeConditionalBranchAND}

\begin{screen}
\begin{verbatim}
static bool PS::Comm::synchronizeConditionalBranchAND(const bool local)
\end{verbatim}
\end{screen}

\begin{itemize}

\item{{\bf 引数}}

local: 入力。const bool型。

\item{{\bf 返り値}}

bool型。全プロセスでlocalの論理積を取り、結果を返す。

\end{itemize}

\subsubsubsubsubsection{PS::Comm::synchronizeConditionalBranchOR}

\begin{screen}
\begin{verbatim}
static bool PS::Comm::synchronizeConditionalBranchOR(const bool local);
\end{verbatim}
\end{screen}

\begin{itemize}

\item{{\bf 引数}}

local: 入力。const bool型。

\item{{\bf 返り値}}

bool型。全プロセスでlocalの論理和を取り、結果を返す。

\end{itemize}

\subsubsubsubsubsection{PS::Comm::getMinValue}

\begin{screen}
\begin{verbatim}
template <class T>
static T PS::Comm::getMinValue(const T val);
\end{verbatim}
\end{screen}

\begin{itemize}

\item{{\bf 引数}}

val: 入力。const T型。

\item{{\bf 返り値}}

T型。全プロセスでvalの最小値を取り、結果を返す。

\end{itemize}

\begin{screen}
\begin{verbatim}
template <class Tfloat, class Tint>
static void PS::Comm::getMinValue(const Tfloat f_in,
                                  const Tint i_in,
                                  Tfloat & f_out,
                                  Tint & i_out);
\end{verbatim}
\end{screen}

\begin{itemize}

\item{{\bf 引数}}

f\_in: 入力。const Tfloat型。

i\_in: 入力。const Tint型。

f\_out: 出力。Tfloat型。全プロセスでf\_inの最小値を取
り、結果を返す。

i\_out: 出力。Tint型。f\_outに伴うIDを返す。

\item{{\bf 返り値}}

なし。

\end{itemize}

\subsubsubsubsubsection{PS::Comm::getMaxValue}

\begin{screen}
\begin{verbatim}
template <class T>
static T PS::Comm::getMaxValue(const T val);
\end{verbatim}
\end{screen}

\begin{itemize}

\item{{\bf 引数}}

val: 入力。const T型。

\item{{\bf 返り値}}

T型。全プロセスでvalの最大値を取り、結果を返す。

\end{itemize}

\begin{screen}
\begin{verbatim}
template <class Tfloat, class Tint>
static void PS::Comm::getMaxValue(const Tfloat f_in,
                                  const Tint i_in,
                                  Tfloat & f_out,
                                  Tint & i_out);
\end{verbatim}
\end{screen}

\begin{itemize}

\item{{\bf 引数}}

f\_in: 入力。const Tfloat型。

i\_in: 入力。{const Tint}型。

{f\_out}: 出力。{Tfloat}型。全プロセスで{f\_in}の最大値を取
り、結果を返す。

{i\_out}: 出力。{Tint}型。{f\_out}に伴うIDを返す。

\item{{\bf 返り値}}

なし。

\end{itemize}

\subsubsubsubsubsection{PS::Comm::getSum}

\begin{screen}
\begin{verbatim}
template <class T>
static T PS::Comm::getSum(const T val);
\end{verbatim}
\end{screen}

\begin{itemize}

\item{{\bf 引数}}

{val}: 入力。{const T}型。

\item{{\bf 返り値}}

{T}型。全プロセスで{val}の総和を取り、結果を返す。

\end{itemize}

\subsubsubsubsubsection{PS::Comm::broadcast}

\begin{screen}
\begin{verbatim}
template <class T>
static void PS::Comm::broadcast(T * val,
                                const PS::S32 n,
                                const PS::S32 src=0);
\end{verbatim}
\end{screen}

\begin{itemize}

\item{{\bf 引数}}

val: 入力。T *型。

n: 入力。const PS::S32型。T型変数の数。

src: 入力。const PS::S32型。放送するプロセスランク。デフォルトのランク
は0。

\item{{\bf 返り値}}

なし。

\item{{\bf 機能}}

プロセスランクsrcのプロセスがn個のT型変数を全プロセスに放送する。

\end{itemize}


