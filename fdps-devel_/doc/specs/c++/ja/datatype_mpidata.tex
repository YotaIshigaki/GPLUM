\subsubsection{概要}

本節ではFDPSで定義されているMPIデータ型について記述する。

\subsubsection{PS::GetDataType$<$S32$>$()}

PS::S32に対応するMPIデータ型である。

\subsubsection{PS::GetDataType$<$S64$>$()}

PS::S64に対応するMPIデータ型である。

\subsubsection{PS::GetDataType$<$U32$>$()}

PS::U32に対応するMPIデータ型である。

\subsubsection{PS::GetDataType$<$U64$>$()}

PS::U64に対応するMPIデータ型である。

\subsubsection{PS::GetDataType$<$F32$>$()}

PS::F32に対応するMPIデータ型である。

\subsubsection{PS::GetDataType$<$F64$>$()}

PS::F64に対応するMPIデータ型である。

\subsubsection{PS::MPI\_F32VEC}

PS::F32vecに対応するMPIデータ型である。PS::F32vecは、2次元直交座標系を
扱っている場合には2次元ベクトル、3次元直交座標系を扱っている場合には3
次元ベクトルである。

\subsubsection{PS::MPI\_F64VEC}

PS::F64vecに対応するMPIデータ型である。PS::F64vecは、2次元直交座標系を
扱っている場合には2次元ベクトル、3次元直交座標系を扱っている場合には3
次元ベクトルである。
