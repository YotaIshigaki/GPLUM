
\subsubsection{概要}

関数calcForceRetrieveは関数calcForceDispatchで行った相互作用の結果を回
収する関数である。以下、これの書き方の規定を記述する。関数
calcForceRetrieveの名前はGravityRetrieveとする。これは変更自由である。
また、Forceクラスのクラス名をResultとする。

\begin{lstlisting}[caption=calcForceDispatch]
class EPI;
class EPJ;
class Result;
PS::S32 GravityRetrieve(const PS::S32  tag,
                        const PS::S32  nwalk,
                        const PS::S32  ni     [],
                        Result         result [][]);
};
\end{lstlisting}

\begin{itemize}

\item {\bf 引数}


  tag: 入力。const PS::S32 型。tagの番号。対応する calcForceDispatch の
  tag 番号と一致させる必要がある。

  nwalk: 入力。const PS::S32 型。walkの数。対応する calcForceDispatch
  に与えた nwalk の値と一致させる必要がある。

  ni: 入力。const PS::S32*型。i粒子数の配列。
  
  result: 出力。Result *型。i粒子の相互作用結果を返す配列の配列。

\item {\bf 返値}

  PS::S32型。ユーザーは正常に実行された場合は0を、エラーが起こった場合
  は0以外の値を返すようにする。
  

\item {\bf 機能}

同じtag番号を持つ関数calcForceDispatchで行った相互作用の結果を回収する。

\end{itemize}

