FDPSは任意の粒子シミュレーションコードの開発を支援するフレームワークで
ある。FDPSユーザ(以下、ユーザ)が1粒子の持つ情報と粒子間の相互作用の形
などをFDPSコードジェネレータ(以下、コードジェネレータ)に入力すると、粒
子シミュレーション用のライブラリ(PSライブラリ、以後PSL)が出力される。ユー
ザはPSLを基に粒子シミュレーションコードを書くことができる。PSLは、任意
の粒子と相互作用する粒子群の探査と、それらの相互作用の計算を、あらゆる
階層の並列化(マルチプロセス、マルチスレッド、SIMD演算)を用いて高速に行
う関数群を提供する。PSL使用の最大の利点は、ユーザが粒子シミュレーション
コードを高速化を意識せずに作成できることである。

PSLはC++で記述されている。ユーザはC++を用いて粒子シミュレーションコード
を記述することが推奨される。

PSLは2粒子間相互作用の計算のみサポートする。3粒子以上の相互作用はサポー
トしない。また、独立時間刻み法もサポートしない。ただし、近傍粒子を返す
APIを用意するので、ユーザがP$^3$T法を用いて独立時間刻み法を実装すること
は可能である。

本文書では、PSLについて記述する。粒子間相互作用の形の定義の仕方などは必
要がない限り記述しない。コードジェネレータについては全く触れない。本文
書の構成は以下の通りである。\ref{sec:algorithm}節では、PSLで実行される
ことについて簡潔に記述する。\ref{sec:api}節では、PSL のAPIを示す。
\ref{sec:sample}節では、PSLを使用した粒子シミュレーションコードの例を示
す。最後に\ref{sec:roadmap}節では、FDPS作成のロードマップを記述する。
